\documentclass[twocolumn]{scrartcl}
%\documentclass{scrartcl}

\usepackage[utf8]{inputenc}
\usepackage[T1]{fontenc}
\usepackage{lmodern}
\usepackage[ngerman]{babel}
\usepackage{amsmath}

\usepackage{makeidx}
\makeindex

\usepackage{fancyvrb}
\usepackage{fvextra}

\usepackage[sc]{mathpazo} % or option osf
\usepackage{newpxmath}
\usepackage[output-decimal-marker={,}]{siunitx}

\usepackage[comma,authoryear]{natbib}
\bibliographystyle{natdin}

\usepackage{adjustbox}
\usepackage[a4paper, margin=1mm, includefoot, footskip=15pt]{geometry}
\usepackage{afterpage}

\usepackage{pgfplotstable}
\usepackage{tikz}
\usepackage{array}
\usepackage{arydshln}

\usepackage[pdftitle={Kriterien und Hilfen bei der Baumartenwahl}
, pdfauthor={Georg Kindermann}
, pdfsubject={Waldbau, Waldwachstum}
, pdfkeywords={Waldbau, Waldwachstum, Wald, Forst}
, pdflang={de-AT-1996}
, hidelinks
, pdfpagemode=None]{hyperref}

\nonfrenchspacing
\sloppy
\usepackage{breqn}
\usepackage{enumitem}
%\usepackage{rotating}
\usepackage{pdflscape}

\title{Kriterien und Hilfen bei der Baumartenwahl}
\author{Georg Kindermann}

\newlength{\eX}
\settoheight{\eX}{X}
\newcommand*{\torte}[1]{\begin{tikzpicture}[baseline={([yshift=2pt]current bounding box.south)},line width=1pt]%
    \ifnum0=#1 \else
    \ifnum1=#1 \draw (0,0) circle (.5\eX); \else
    \draw (0,0) circle (.5\eX);
    \fill (0,0) -- (90:.5\eX) arc (90:90-(#1-1)*72:.5\eX) -- cycle;
    \fi\fi
  \end{tikzpicture}}

\newcommand*{\hst}[1]{\begin{tikzpicture}[baseline={([yshift=2pt]current bounding box.south)},line width=1pt]
    %\path (0,0) (1em,1.0em);
    \ifnum4=#1\draw (.5\eX,1\eX) -- (0\eX,0\eX) -- (1\eX,0\eX) -- cycle;\else
    \ifnum3=#1\draw[fill] (.3\eX,1\eX) -- (.15\eX,.5\eX) -- (.85\eX,.5\eX) -- (.7\eX,1\eX) -- cycle;
    \draw (.15\eX,.5\eX) -- (0\eX,0\eX) -- (1\eX,0\eX) -- (.85\eX,.5\eX) -- cycle;
    \else
    \ifnum2=#1\draw (.3\eX,1\eX) -- (.15\eX,.5\eX) -- (.85\eX,.5\eX) -- (.7\eX,1\eX) -- cycle;
    \draw[fill] (.15\eX,.5\eX) -- (0\eX,0\eX) -- (1\eX,0\eX) -- (.85\eX,.5\eX) -- cycle;
    \else
    \ifnum1=#1\draw (0\eX,.2\eX) -- (1\eX,.2\eX) {[rounded corners=2pt] -- (1\eX,1\eX) -- (.5\eX,.5\eX)} -- (0\eX,.5\eX) -- cycle;
    \else
    \ifnum0=#1\draw[fill] (0,0) -- (1\eX,0) -- (1\eX,.5\eX) {[rounded corners=2pt] -- (.5\eX,.2\eX)} -- (0,.5\eX) -- cycle;
    \fi\fi\fi\fi\fi
  \end{tikzpicture}
}

\listfiles
\begin{document}

\twocolumn[
  \begin{@twocolumnfalse}
    \maketitle
    \begin{abstract}

    \end{abstract}
  \end{@twocolumnfalse}
]

\tableofcontents

\section{Einleitung}

Die Baumartenwahl ist eine der weitreichendsten waldbaulichen
Entscheidungen und kann selbst in Mischwäldern nach der
Verjüngungsphase nur bedingt verändert werden. Durch die Vielzahl an
Kriterien, welche zum Großteil, zum Zeitpunkt der Entscheidung,
quantitativ kaum fassbar sind, können die Baumarten nur untereinander
verglichen und nach subjektiven Schwerpunkten abgewogen werden. Auf
Standorten, die nur wenigen Baumarten zusagen, ist die Entscheidung
einfach, solange das vorliegen eines Zwangsstandortes erkannt wird und
die Baumartenwahl auf heimische Baumarten beschränkt ist. Ähnlich ist
es bei einer Beschränkung auf Baumarten die sich natürlich einstellen
oder jene der potentiell natürlichen Vegetation, insbesondere in
Regionen in denen die Artenvielfalt in der letzten Eiszeit drastisch
reduziert wurde. Wobei auch hier die potentiell natürliche Vegetation
eines Standortes erst einmal richtig erkannt werden muss. Hingegen
zunächst alle Baumarten, die eventuell standortstauglich sein könnten,
in Betracht zu ziehen, wird oft schon daran scheitern, dass man einige
davon nicht oder nur mit erheblichem Aufwand erhalten wird können und
von vielen deren waldbauliches Verhalten, auf dem vorliegenden
Standort, wenig bis gar nicht bekannt ist.

Um Wissen in diese Richtung zu erweitern und sich oder seinen
Nachfolgern dadurch in der Zukunft weitere Möglichkeiten zu
erschließen, können auf keinen Flächen durchaus auch neue Baumarten
erprobt werden. Dabei muss das Risiko des Versagens der Baumart
bewusst in Kauf genommen und deren Ausmaß auf Kleinflächen beschränkt
bleiben. Flächen auf denen verschiedene Baumarten unmittelbar
verglichen werden können finden sich in botanischen Gärten, Arboreten
und vereinzelt auch auf eigens dafür angelegten Versuchsflächen, wobei
bei ersteren in der Regel die Bäume in keiner Bestandessituation,
sondern als Solitäre, anzutreffen sind und den Bäumen oft besondere
Pflege (z.\,B.\ Bewässerung) zukommt. Eine erste Auswertung der
Höhenwuchsleistung, verschiedener Baumarten, am selben Standort, wurde
von \cite{mayer1970anbauversuch} gemacht. Von Seiten des
Bundesforschungszentrums für Wald wurde 2012 ein Forschugsantrag zur
ersten Anlage einiger solcher Versuchsflächen beim Ministerium
eingereicht. Dieses Vorhaben konnte 2017 auf einer Fläche am Wechsel
($47,4999^{\circ}$\,Nord, $15,9741^{\circ}$\,Ost) auf 1340\,m mit
31~Baumarten begonnen werden. 2019 wurde in Matzen (Niederösterreich)
ein weiterer Versuch mit 15~Baumarten angelegt. 2022 wurde in
Kronsdorf (Oberösterreich) von der Landwirschaftskammer unter der
Leitung von Christoph Jasser eine Fläche mit 44~Baumarten
angelegt. Diese Versuchswälder werden als
\emph{Site--Index--Benchmark--} bzw.\
\emph{Wuchsleistungsvergleichsbestände}, \emph{Klimaforschungswald}
oder \emph{Waldlabor} bezeichnet. Es bleibt zu Hoffen, dass dieser
Trend nicht nur fortgesetzt sondern auch verstärkt wird, um das
Möglichkeisspekturm, bei der Baumartenwahl, aufzuzeigen und vermehrt
von der subjektiven in Richtung objektive Entscheidung leiten zu
können. Das dieser Vergleich nicht nur auf Baumarten baumarten
beschränkt, sondern auch auf verschiedene Herkünfte auszudehnen ist,
zeigt Abbildung~\ref{fig:fichtenHerkuenfte} aus
\citet[S.~86]{hegi1906IllustrierteFloraBd1} auf der drei verschiedenen
Fichtenherkünfte mit verschiedenen Wuchsformen und Wuchsleistungen zu
sehen sind. Da die Publikation vom Jahr
\citeyear{hegi1906IllustrierteFloraBd1} ist, wurden die drei Bäume
bereits um 1800 gepflanzt und zeigen sehr anschaulich deren
Unterschiede auf. Ähnlich anschauliches ist auch von Versuchsflächen
mit möglichst vielen verschiedenen Baumarten und Herkünften zu
erwarten.

\begin{figure}[htbp]
  \centering
  \includegraphics[width=.95\columnwidth]{./pic/fichtenHerkuenfte.jpg}
  \caption{Drei verschiedene Wuchsformen von Fichte an dem gleichen
    Standort (les Monts nördlich von le Locle, Kanton Neuenburg,
    Schweiz). Phot. Kreisförster
    Pillichody. \citep[S.~86]{hegi1906IllustrierteFloraBd1}}
  \label{fig:fichtenHerkuenfte}
\end{figure}

\section{Kriterien}
\label{sec:kriterien}

Die folgenden Kriterien, die bei der Bauartenwahl eine Rolle spielen
können, stellen eine subjektive Zusammenstellung dar und müssen bei
Bedarf erweitert werden.

\subsection{Standort}
\label{ssec:standort}

Die standörtliche Tauglichkeit ist wohl der erste Punkt, der, bei der
Baumartenwahl, erfüllt sein muss. Eine Beschränkung auf Baumarten, die
in der potentiell natürlichen Waldgesellschaft vorkommen würden, ist
allerdings nicht immer nötig. Baumarten des Vorbestandes und der
näheren Umgebung mit vergleichbarem Standort sind
Standortstauglich. Da der Standort z.B. durch den Klimawandel,
Stickstoff-- und Schadstoffeinträge laufend verändert wird, können
standortstaugliche Baumarten des Vorbesatndes, im Folgebestand
plötzlich nicht mehr standortstauglich sein. Dies kann so weit gehen,
dass der Standort für eine andere Landutzungsform (Grasland, Acker)
geeignetere wird. In vielen Fällen zeigt sich die Tauglichkeit oder
Untauglichkeit einer Baumart für einen Standort bereits in der
Verjüngungsphase. Wobei alleine der Ausfall einer Baumart in der
Verjüngung, insbesondere wenn nur ein Jahr betrachtet wurde, noch kein
ausreichender Hinweis für die \emph{standörtliche} Untauglichkeit der
Baumart darstellt. Dabei kann z.\,B.\ schlecht oder zur falschen Zeit
gepflanzt worden sein, die Witterung in diesem einen Jahr ungünstig
gewesen sein, Tiere, Pilze oder andere Pflanzen die jungen Bäume zum
absterben gebracht haben.

Der Begriff Standort darf sich allerdings nicht darf beschränken, ob
eine Baumart auf einen Standort wachsen kann oder nicht. Der Standort
bestimmt auch die Wuchsleistung und die Gefährdung (Nassschneelage --
Schneebruch, Durchwurzelungstiefe, Windgeschwindigkeit -- Windwurf,
Windbruch) und damit wie häufig welche Pflegemaßnahmen durchgeführt
werden sollten.

\subsection{Ertrag}
\label{ssec:ertrag}

Standortstaugliche Baumarten können sich in der Ertragsleistung
beträchtlich unterscheiden. Durch dieses Kriterium wird die Anzahl zur
Auswahl stehenden Baumarten oft entscheidend reduziert und führte
aufgrund der Reinertragslehre oft zum dominieren einer Baumart in
ganzen Regionen. So wichtig die Ertragsleistung auch ist, mag es
durchaus gerechtfertigt sein die zweit--, dritt-- oder
viertleistungsfähigste Baumart zu wählen bzw.\ die Bedeutung dieses
Kriteriums gering zu gewichten, zumal die erwartete Ertragsleistung,
selbst wenn sich dies nur auf die Zuwachsleistung bezieht, nur recht
unsicher geschätzt werden kann. Nicht nur die Zuwachsleistung, sondern
auch das gewünschte \emph{Zielsortiment} beeinflussen die
Baumartenwahl. Ein Waldbesitzer, der seinen Brennholzbedarf decken
will, wird am selben Standort meist andere Baumarten wählen als einer
mit dem Ziel Wertholzproduktion.

\subsection{Pflegeaufwand}
\label{ssec:pflegeaufwand}

Baumarten unterscheiden sich hinsichtlich ihres nötigen
Pflegeaufwandes. Selbst wenn jemand den Ertrags-- und
Qualitätskriterien wenig Bedeutung beimisst, ist sind Pflegeeingriffe
(Durchforstungen) dringend zu empfehlen um die Bestandesstabilität
(Windwurf, Schneebruch, Insekten, Pilze) nicht zu gefährden. Neben der
Baumart entscheiden auch die Ziele (z.\,B.\ Produktion von
Qualitätsholz), der Standort (Wuchsleistung der Baumart auf diesem
Standort) und eine eventuelle Mischung (entfernen von einigen
konkurrenzstärkeren Nachbar einer anderen Baumart um die
konkurrenzschwächere Baumart zu erhalten) den Pflegeaufwand. Allgemein
erhöht sich der Pflegeaufwand hinsichtlich Frequenz und Umfang mit
Zunahme der Wuchsleistung.

\subsection{Risiko}
\label{ssec:risiko}

Baumarten ohne Ausfallsrisiko gibt es nicht. Aber es gibt Baumarten
die, auf bestimmten \emph{Standorten}, ein wesentlich höheres
Ausfallsrisiko haben als andere. Auch die Nachbarbestände können das
Risiko mitbestimmen indem sie Deckungsschutz bieten oder auch
nicht. Dieser Umstand wurde von
\cite{wagner1923DerBlendersaumschlagUndSeinSystem} ausführlich
dargestellt und als Lösung der Blendersaumschlag (Streifenweise
Schlage die sich, z.\,B.\ entgegen die Hauptwindrichtung, im laufe der
Zeit aneinander reihen) vorgeschlagen. Aber nicht nur der unmittelbare
Nachbarbestand, sonder die Baumartenvereilung einer ganzen Region,
kann das Ausfallsrisiko einer Baumart mitbestimmen indem sich z.\,B.\
Insektenkalamitäten von diesen auf deren Nachbarn übertragen
können. Ob eine Baumart im Bestand die gleiche oder eine andere
Baumart als Nachbarn hat, beeinflusst ebenfalls das
Ausfallsrisiko. Steht diese neben einer Baumart mit hohem
Wasserverbrach, kann diese Mischbaumart das Ausfallsrisiko der
Nachbarsbaumart in Trockenzeiten erhöhen.

\subsection{Bestandestyp}
\label{ssec:bestandestyp}

Ob man einen \emph{Hochwald}, \emph{Mittelwald} oder \emph{Niederwald}
anstrebt, beeinflusst das Spektrum der zur Auswahl stehenden
Baumarten. So können im Niederwald ausschließlich Baumarten mit
Stockausschlag oder Wurzelbrut Verwendung finden. Der Bestandestyp
kann in vielen Fällen frei gewählt werden. Auf bestimmten Standorten
zeigt kann einer dieser Typen Vorteile gegenüber den anderen
Zeigen. Beispielsweise ist die Verjüngung auf trockenen Standorten im
Niederwald meist einfacher als in Hochwald.

Die Baumartenwahl kann in \emph{Reinbeständen} anders als in
\emph{Mischbeständen} ausfallen wobei hier die \emph{Mischungsform}
(Einzel, Trupp, Gruppe, Horst, Fläche, Reihe, Streifen) und die
Baumartenanteile die Auswahl mitbestimmen. Im Mischbestand müssen die
Baumarten zusammenpassen und sollten eine Funktion in der Mischung
übernehmen.

Die \emph{Struktur} (Einschitig, Mehrschichtig, Stufig; Regelmäßig,
Zufällig, Geklumpt) und ob ein Bestand gleich-- oder ungleichaltirg
sein soll, beeinflussen die Baumartenwahl. In mehrschichtigen
Beständen müssen die Baumarten der unteren Schichten mit dem dort
geringeren Lichtangebot auskommen.

Auch die Entscheidung zwischen Natur-- oder Kunstverjüngung bzw.\ eine
Kombination von beiden beeinflusst die zur Auswahl stehenden
Baumarten.

\subsection{Schutzwirkung}
\label{ssec:schutz}

Baumarten unterscheiden sich hinsichtlich dem Rückhaltevermögen von
Schnee und beeinflussen damit deren Schutzwirkung gegen
Lawinen. Ähnliches, gilt auch für den Wasserrückhalt, welcher
Auswirklungen auf den Hochwasserschutz aber auch auf die
Gleichmäßigkeit der Wasserspende in Quelschutzwäldern
hat. Verschiedenen Baumarten unterscheiden sich hinsichtlich ihrer
Wirkungen auf das Biotop bzw.\ Habitat und leisten unterschiedliches
für den Naturschutz und die Biodiversität. Dies ist insbesondere in
Wechselwirkung zu den restlichen Baumarten der umliegenden Bestände zu
betrachten.

\subsection{Erholungswirkung}
\label{ssec:erholung}

Insbesondere in stadtnahen Wäldern haben Landschaftspflege und
Erholungswirkung hohe Bedeutung. Diese werden in der Regeln kaum von
einer Baumart eines Bestand, sondern im Zusammenwirken aller Bestände
mit deren Baumarten bestimmt.

\section{Hilfen}
\label{sec:hilfen}

Für die Baumartenwahl sind die waldbaulichen Eingenschaften und
Ansprüche der Baumarten entscheidend. Diese sind beispielsweise in den
Waldbaulehrbüchern von
\citet{mayer1992Waldbau,burschel2003Waldbau,Dengler2020Waldbau,tschermak1950Waldbau,rittershofer2006Waldbau,rubner1960Waldbau,koestler1950Waldbau,bauer1962WaldbauAlsWissenschaft}
oder in den Baumartenbeschreibungen von
\citet{eth2002MitteleuropaeischeWaldbaumarten,leibundgut1984Waldbaeume,ec2016baumartenatlas}
oder auch in Floren von
\citet{fischer2008Exkursionsflora,hegi1906IllustrierteFloraBd1,oberdorfer2001Exkursionsflora,rothmaler2021Exkursionsflora,schmeil2019Exkursionsflora,fitschen2017Gehoelzflora,hieke1989Dendrologie,mayr1906FremdlaendischeWaldUndParkbaeumeFuerEuropa,stimm2014EnyklopedieDerHolzgewaechse,schuett1993LexikonDerForstbotanik}
enthalten.

Ökogramme
Verbreitungskarten
Wuchsleistungsschätzung
Karten: Geologie, Boden, Niederschlag, Temperatur, Vegetation

\begin{table*}[htbp!]
  \setlength{\tabcolsep}{1pt}
  \centering
  \pgfplotstabletypeset[col sep=comma
  , text indicator="
  , string type
  , after row=\cdashline{3-38}
  %, every head row/.style={before row=\hline}
  %, every last row/.style={after row={}}
  %, every last column/.style={column type=c}
  , every column/.style={column type=c:,preproc cell content/.style={@cell content = \torte{##1}}}
  , display columns/0/.style={column type=l,column name=\rotatebox{-90}{Waldgesellschaft}, preproc cell content/.style={@cell content = ##1}}
  , display columns/1/.style={column type=c|,preproc cell content/.style={@cell content = \hst{##1}}}
  , assign column name/.style={/pgfplots/table/column name={\multicolumn{1}{c}{\rotatebox{90}{#1}}}}
  , columns/Weisskiefer/.style={column name=Weißkiefer}
  , every row no 4/.style={before row=\hline}
  , every row no 12/.style={before row=\hline}
  , every row no 22/.style={before row=\hline}
  ]{./tabs/gesselschaftsanschluss.csv}
  \caption{Natürlicher Gesellschaftsanschluß mit Vergesellschaftung der Baumarten in Wäldern des Ostalpenraumes \citep{mayer1992Waldbau}}
  \footnotesize{\hst{0}\dots Auwald
    , \hst{1}\dots Planar--Kolin --400\,m
    , \hst{2}\dots Submontan 200--900\,m
    , \hst{3}\dots Montan 500--1800\,m
    , \hst{4}\dots Subalpin 1200--2300\,m\\
    \tikz \draw (0,0) rectangle (1em,1em);\dots fehlend
    , \torte{1}\dots sporadisch
    , \torte{2}\dots spärlich
    , \torte{3}\dots gering
    , \torte{4}\dots mittel
    , \torte{5}\dots reichlich
    , \torte{6}\dots vorherschend
  }
  \label{tab:b}
\end{table*}

\begin{landscape}
  \begin{figure}[hp]
    \centering
    \includegraphics[height=.9\textheight]{./pic/waelderDesOstalpenraumes.pdf}
    \caption{Wälder des Ostalpenraumes \citep{mayer1977KarteWaldtypen}}
    \label{fig:waelder}
  \end{figure}
\end{landscape}


%\addcontentsline{toc}{section}{Stichwortverzeichnis}
%\printindex

\addcontentsline{toc}{section}{Literatur}
\bibliography{literature}

%Autor: Georg Kindermann

\end{document}
