\documentclass[twocolumn]{scrartcl}
%\documentclass{scrartcl}

\usepackage[utf8]{inputenc}
\usepackage[T1]{fontenc}
\usepackage{lmodern}
\usepackage[ngerman]{babel}
\usepackage{amsmath}

\usepackage{makeidx}
\makeindex

\usepackage{fancyvrb}
\usepackage{fvextra}

\usepackage[sc]{mathpazo} % or option osf
\usepackage{newpxmath}
\usepackage[output-decimal-marker={,}]{siunitx}

\usepackage[comma,authoryear]{natbib}
\bibliographystyle{natdin}

\usepackage{adjustbox}
\usepackage[a4paper, margin=1mm, includefoot, footskip=15pt]{geometry}
\usepackage{afterpage}

\usepackage{pgfplotstable}
\usepackage{tikz}
\usepackage{array}
\usepackage{arydshln}

\usepackage[pdftitle={Kriterien und Hilfen bei der Baumartenwahl}
, pdfauthor={Georg Kindermann}
, pdfsubject={Waldbau, Waldwachstum}
, pdfkeywords={Waldbau, Waldwachstum, Wald, Forst}
, pdflang={de-AT-1996}
, hidelinks
, pdfpagemode=None]{hyperref}

\nonfrenchspacing
\sloppy
\usepackage{breqn}
\usepackage{enumitem}
%\usepackage{rotating}
\usepackage{pdflscape}

\usepackage{pdfpages}

\title{Kriterien und Hilfen bei der Baumartenwahl}
\author{Georg Kindermann}

\newlength{\eX}
\settoheight{\eX}{X}
\newcommand*{\torte}[1]{\begin{tikzpicture}[baseline={([yshift=2pt]current bounding box.south)},line width=1pt]%
    \ifnum0=#1 \else
    \ifnum1=#1 \draw (0,0) circle (.5\eX); \else
    \draw (0,0) circle (.5\eX);
    \fill (0,0) -- (90:.5\eX) arc (90:90-(#1-1)*72:.5\eX) -- cycle;
    \fi\fi
  \end{tikzpicture}}

\newcommand*{\hst}[1]{\begin{tikzpicture}[baseline={([yshift=2pt]current bounding box.south)},line width=1pt]
    %\path (0,0) (1em,1.0em);
    \ifnum4=#1\draw (.5\eX,1\eX) -- (0\eX,0\eX) -- (1\eX,0\eX) -- cycle;\else
    \ifnum3=#1\draw[fill] (.3\eX,1\eX) -- (.15\eX,.5\eX) -- (.85\eX,.5\eX) -- (.7\eX,1\eX) -- cycle;
    \draw (.15\eX,.5\eX) -- (0\eX,0\eX) -- (1\eX,0\eX) -- (.85\eX,.5\eX) -- cycle;
    \else
    \ifnum2=#1\draw (.3\eX,1\eX) -- (.15\eX,.5\eX) -- (.85\eX,.5\eX) -- (.7\eX,1\eX) -- cycle;
    \draw[fill] (.15\eX,.5\eX) -- (0\eX,0\eX) -- (1\eX,0\eX) -- (.85\eX,.5\eX) -- cycle;
    \else
    \ifnum1=#1\draw (0\eX,.2\eX) -- (1\eX,.2\eX) {[rounded corners=2pt] -- (1\eX,1\eX) -- (.5\eX,.5\eX)} -- (0\eX,.5\eX) -- cycle;
    \else
    \ifnum0=#1\draw[fill] (0,0) -- (1\eX,0) -- (1\eX,.5\eX) {[rounded corners=2pt] -- (.5\eX,.2\eX)} -- (0,.5\eX) -- cycle;
    \fi\fi\fi\fi\fi
  \end{tikzpicture}
}

\newcommand*{\rating}[1]{\begin{tikzpicture}[baseline={([yshift=2pt]current bounding box.south)},line width=1pt]
    \ifnum0=#1\tikz \draw[rounded corners=2pt] (0,0) rectangle (1em,1em);\else
    \ifnum1=#1\begin{tikzpicture}
      \draw[rounded corners=2pt] (0,0) rectangle (1em,1em);
      \fill[black, radius=.1em] (.5em,.5em) circle;
    \end{tikzpicture} \else
    \ifnum2=#1\begin{tikzpicture}
      \draw[rounded corners=2pt] (0,0) rectangle (1em,1em);
      \fill[black, rounded corners=2pt] (.2em,.2em) rectangle (.8em,.8em);
    \end{tikzpicture} \else
    \ifnum3=#1\tikz \fill[black,rounded corners=2pt] (0,0) rectangle (1em,1em);
    \fi\fi\fi\fi
  \end{tikzpicture}
}

\newcommand{\ratingB}[2]{\begin{tikzpicture}[baseline={([yshift=2pt]current bounding box.south)},line width=1pt,x=1em,y=1em]
    \begin{scope}
      \clip[rounded corners=2pt] (0,0) rectangle (1,1);
      \fill[black] (0,0) rectangle (1,#1/#2);
    \end{scope}
    \draw[rounded corners=2pt] (0,0) rectangle (1,1);
  \end{tikzpicture}%
}

%\usepackage{tabularray} %Maybe in future
\usepackage{xtab}
\newcommand{\ratingC}[2]{\begin{tikzpicture}[baseline={([yshift=2pt]current bounding box.south)},line width=1pt,x=1em,y=1em]
    \begin{scope}
      \clip[rounded corners=2pt] (-.5,-.5) rectangle (.5,.5);
      \draw[fill] (0,0) -- (90-#1/100*360:1) arc (90-#1/100*360:90-#2/100*360:1) -- cycle;
    \end{scope}
    \draw[rounded corners=2pt] (-.5,-.5) rectangle (.5,.5);
  \end{tikzpicture}%
}

\newcommand*{\jaNein}[1]{\begin{tikzpicture}[baseline={([yshift=2pt]current bounding box.south)},line width=1pt,x=1em,y=1em]%
    \ifnum0=#1 \draw[rounded corners=2pt] (-.5,-.5) rectangle (.5,.5); \else
    \ifnum1=#1 \draw[fill,rounded corners=2pt] (-.5,-.5) rectangle (.5,.5);
    \fi\fi
  \end{tikzpicture}}

\newcommand{\jaNeinP}[2]{\begin{tikzpicture}[baseline={([yshift=2pt]current bounding box.south)},line width=1pt,x=1em,y=1em]
    \begin{scope}
      \clip[rounded corners=2pt] (-.5,-.5) rectangle (.5,.5);
      \ifnum1=#1\fill (0,-.5) rectangle (.5,.5);\fi
      \ifnum1=#2\fill (0,-.5) rectangle (-.5,.5);\fi
      \draw[white,line width=.2pt] (0,-.5) -- (0,.5);
      \draw[densely dotted,line width=.2pt] (0,-.5) -- (0,.5);
    \end{scope}
    \draw[rounded corners=2pt] (-.5,-.5) rectangle (.5,.5);
    \ifnum-1=#1\fill[white] (0,-.5em-1pt) rectangle (.5em+1pt,.5em+1pt);\fi
    \ifnum-1=#2\fill[white] (0,-.5em-1pt) rectangle (-.5em-1pt,.5em+1pt);\fi
  \end{tikzpicture}%
}

\newcommand{\jaNeinQ}[4]{\begin{tikzpicture}[baseline={([yshift=2pt]current bounding box.south)},line width=1pt,x=1em,y=1em]
    \begin{scope}
      \clip[rounded corners=2pt] (-.5,-.5) rectangle (.5,.5);
      \ifnum1=#1\fill (0,0) rectangle (.5,.5);\fi
      \ifnum1=#2\fill (0,0) rectangle (.5,-.5);\fi
      \ifnum1=#3\fill (0,0) rectangle (-.5,-.5);\fi
      \ifnum1=#4\fill (0,0) rectangle (-.5,.5);\fi
      \draw[white,line width=.2pt] (0,-.5) -- (0,.5);
      \draw[white,line width=.2pt] (-.5,0) -- (.5,0);
      \draw[densely dotted,line width=.2pt] (0,-.5) -- (0,.5);
      \draw[densely dotted,line width=.2pt] (-.5,0) -- (.5,0);
    \end{scope}
    \draw[rounded corners=2pt] (-.5,-.5) rectangle (.5,.5);
    \ifnum-1=#1\fill[white] (0,0) rectangle (.5em+1pt,.5em+1pt);\fi
    \ifnum-1=#2\fill[white] (0,0) rectangle (.5em+1pt,-.5em-1pt);\fi
    \ifnum-1=#3\fill[white] (0,0) rectangle (-.5em-1pt,-.5em-1pt);\fi
    \ifnum-1=#4\fill[white] (0,0) rectangle (-.5em-1pt,.5em+1pt);\fi
  \end{tikzpicture}%
}

\newcommand{\typ}[2]{\begin{tikzpicture}[baseline={([yshift=2pt]current bounding box.south)},line width=1pt,x=1em,y=1em]
    \ifnum1=#1 %Nadel
    \ifnum0=#2\draw (0,-.5) -- (0,-.3) -- (-.3,-.3) -- (0,.5) -- (.3,-.3) -- (0,-.3);
    \else \draw[fill] (0,-.5) -- (0,-.3) -- (-.3,-.3) -- (0,.5) -- (.3,-.3) -- (0,-.3);
    \fi\fi
    \ifnum2=#1 %Laub
    \ifnum0=#2 \draw (0,-.5) -- (0,-.3); \draw[radius=0.4] (0,.1) circle;
    \else \draw (0,-.5) -- (0,-.3); \draw[fill,radius=0.4] (0,.1) circle;
    \fi\fi
    \ifnum3=#1 %Gingko
    \draw (0,-.5) -- (0,-.3);
    \draw (0,-.3) -- (180:.5) arc (180:0:.5) -- cycle;
    \fi
    \ifnum4=#1 %Kletter
    \ifnum0=#2 \draw[rounded corners=2pt,line width=.5pt] (0,-.5) -- (-.2,-.25) -- (0,0)  -- (.2,.25) -- (0,.5);
    \else \draw[rounded corners=2pt,line width=2pt] (0,-.5) -- (-.2,-.25) -- (0,0)  -- (.2,.25) -- (0,.5);
    \fi\fi
    \ifnum5=#1 %Palme
    \draw (0,-.5) -- (0,.5) {[rounded corners=2pt] -- (-.2,.5) -- (-.5,.3) (-.5,.3) -- (0,.4) (0,.5) -- (.2,.5) -- (.5,.3) (.5,.3) --(0,.4)};
    \fi
    \ifnum6=#1 %Kaktus
    \draw[rounded corners=2pt] (-.2,-.5) -- (-.1,.5) -- (.1,.5) -- (.2,-.5) -- cycle;
    \fi
  \end{tikzpicture}%
}


\setlength{\tabcolsep}{1pt}

\listfiles
\begin{document}

\twocolumn[
  \begin{@twocolumnfalse}
    \maketitle
    \begin{abstract}
      Die Baumartenwahl ist eine der weitreichendsten waldbaulichen
      Entscheidungen. Dieser Artikel listet Kriterien auf und gibt ein paar
      Hilfen, die bei der Auswahl und Entscheidung berücksichtigt werden können.
    \end{abstract}
  \end{@twocolumnfalse}
]

\tableofcontents

\section{Einleitung}

Die Baumartenwahl ist eine der weitreichendsten waldbaulichen
Entscheidungen und kann, selbst in Mischwäldern, nach der
Verjüngungsphase nur bedingt verändert werden. Durch die Vielzahl an
Kriterien, welche zum Großteil, zum Zeitpunkt der Entscheidung,
quantitativ kaum fassbar sind, können die Baumarten, meist nur nach
subjektiven Schwerpunkten, gegeneinander abgewogen werden. Auf
Standorten, die nur wenigen Baumarten zusagen, ist die Entscheidung
einfacher, solange das Vorliegen eines Zwangsstandortes erkannt wird
und die Baumartenwahl auf heimische Baumarten beschränkt ist. Ähnlich
ist es bei einer Beschränkung auf Baumarten die sich natürlich
einstellen oder jene der potenziellen natürlichen Vegetation,
insbesondere in Regionen, in denen die Artenvielfalt durch die letzten
Eiszeiten drastisch reduziert wurde. Wobei auch hier, die potenzielle
natürliche Vegetation, eines Standortes, erst einmal richtig erkannt
werden muss. Hingegen, zunächst alle Baumarten, die eventuell
standortstauglich sein könnten, in Betracht zu ziehen, wird oft schon
daran scheitern, dass man einige davon nicht oder nur mit erheblichem
Aufwand erhalten wird können und von vielen deren waldbauliches
Verhalten, auf dem vorliegenden Standort, wenig bis gar nicht bekannt
sein wird.

Um Wissen in diese Richtung zu erweitern und sich oder seinen Nachfolgern, in
der Zukunft, weitere Möglichkeiten zu erschließen, können auf keinen Flächen
durchaus auch neue Baumarten erprobt werden. Dabei muss das Risiko, des
Versagens der Baumart, bewusst in Kauf genommen und deren Ausmaß auf
Kleinflächen beschränkt bleiben. Flächen auf denen verschiedene Baumarten
unmittelbar verglichen werden können, finden sich in botanischen Gärten,
Arboreten und vereinzelt auch auf eigens dafür angelegten Versuchsflächen, wobei
bei ersteren in der Regel die Bäume in keiner Bestandessituation, sondern als
Solitäre, anzutreffen sind und den Bäumen oft besondere Pflege (z.\,B.\
Bewässerung) zukommt. Die Erste, mir bekannte, Auswertung der
Höhenwuchsleistung, verschiedener Baumarten, am selben Standort, wurde von
\cite{mayer1970anbauversuch} gemacht. Von der bayerischen Landesanstalt für Wald
und Forstwirtschaft (LWF) wurde 2012 der internationale Baumartenversuch KliP18
initiiert, an dem die Universität für Bodenkultur in Bruckneudorf eine Fläche
einrichtete. Weitere Flächen dieses Versuchs liegen in Großostheim, Schmellenhof
und Oldisleben (Deutschland) und Mutrux (Schweiz) und haben ihren Schwerpunkt in
der Erprobung von nichtheimischen Baumarten. Vonseiten des
Bundesforschungszentrums für Wald wurde 2012 ein Forschungsantrag zur ersten
Anlage einiger solcher Versuchsflächen beim Ministerium eingereicht. Dieses
Vorhaben konnte 2017 auf einer Fläche am Wechsel ($47,4999^{\circ}$\,Nord,
$15,9741^{\circ}$\,Ost) auf 1340\,m mit 31~Baumarten begonnen werden. 2019 wurde
in Matzen (Niederösterreich) ein weiterer Versuch mit 15~Baumarten angelegt.
2022 wurde in Kronsdorf (Oberösterreich) von der Landwirtschaftskammer unter der
Leitung von Christoph Jasser eine Fläche mit 44~Baumarten angelegt. Diese
Versuchswälder werden als \emph{Site--Index--Benchmark--Bestand} bzw.\
\emph{Wuchsleistungsvergleichsbestände}, \emph{Klimaforschungswald} oder
\emph{Waldlabor} bezeichnet. Es bleibt zu hoffen, dass dieser Trend nicht nur
fortgesetzt, sondern auch verstärkt wird, um das Möglichkeisspekturm, bei der
Baumartenwahl, aufzuzeigen und vermehrt von der subjektiven, in Richtung
objektive Entscheidung, leiten zu können. Bei solchen Versuchsflächen ist strikt
darauf zu achten, dass diese \emph{praxisüblich} behandelt werden. Wenn
beispielsweise diese Bestände mit wirtschaftlich unvertretbaren Aufwand
bewässert werden würden und somit das Aufkommen von Baumarten ermöglicht wird,
die in der Praxis versagen werden, wäre damit der Praxis nicht geholfen. Dass
dieser Vergleich nicht nur auf Baumarten beschränkt, sondern auch auf
verschiedene Herkünfte auszudehnen ist, zeigt
Abbildung~\ref{fig:fichtenHerkuenfte}
\citep[S.~86]{hegi1906IllustrierteFloraBd1} auf, die drei verschiedenen
Fichtenherkünfte, mit verschiedenen Wuchsformen und Wuchsleistungen, zeigt. Da
die Publikation vom Jahr \citeyear{hegi1906IllustrierteFloraBd1} stammt, wurden
die drei Bäume bereits um 1800 gepflanzt und zeigen sehr anschaulich deren
Unterschiede auf. Ähnlich anschauliches ist auch von Versuchsflächen mit
möglichst vielen verschiedenen Baumarten und Herkünften zu erwarten.

\begin{figure}[htbp]
  \centering
  \includegraphics[width=.95\columnwidth]{./pic/fichtenHerkuenfte.jpg}
  \caption{Drei verschiedene Wuchsformen von Fichte am gleichen
    Standort (les Monts nördlich von le Locle, Kanton Neuenburg,
    Schweiz). Phot. Kreisförster
    Pillichody. \citep[S.~86]{hegi1906IllustrierteFloraBd1}}
  \label{fig:fichtenHerkuenfte}
\end{figure}

\section{Kriterien}
\label{sec:kriterien}

Die folgenden Kriterien, die bei der Baumartenwahl eine Rolle spielen können,
stellen eine subjektive Zusammenstellung dar und müssen bei Bedarf erweitert
oder reduziert werden. Eine Übersicht der Beurteilungskriterien kann nach
\citet[Bd.1, S.108]{bauer1962WaldbauAlsWissenschaft} folgendermaßen aussehen:
\begin{enumerate}
\item Biologisch--ökologischer Komplex
  \begin{enumerate}
  \item Biologische Komponente
    \begin{enumerate}
    \item Standortsansprüche
    \item Schattenfestigkeit
    \item Fortpflanzung
    \item Jugendwachstum
    \item Gefährdungen
    \end{enumerate}
  \item Ökologische Komponente
    \begin{enumerate}
    \item Rückwirkung auf den Boden
    \item Soziologisches Verhalten
    \item Vorwuchseigenschaften
    \item Eignung zum Überhalt
    \item Häufigkeit der Samenjahre
    \end{enumerate}
  \end{enumerate}
\item Ökonomischer Komplex
  \begin{enumerate}
  \item Vegetative Leistung
    \begin{enumerate}
    \item Zuwachs dgz
    \item Nutzholztüchtigkeit
    \item Wertholztüchtigkeit
    \item Produktionszeitraum
    \end{enumerate}
  \item Wirtschaftliche Leistung
    \begin{enumerate}
    \item Gemeinwirtschaftliche Komponente
      \begin{enumerate}
      \item Technischer Wert der Rohholzsorten
      \item Marktwirtschaftlicher Wert
      \end{enumerate}
    \item Privatwirtschaftliche Komponente
      \begin{enumerate}
      \item Massenproduktion
      \item Wertholzzucht
      \item Geldertrag
      \end{enumerate}
    \end{enumerate}
  \end{enumerate}
\end{enumerate}


\subsection{Standort}
\label{ssec:standort}

Die standörtliche Tauglichkeit ist wohl der erste Punkt, der, bei der
Baumartenwahl, erfüllt sein muss. Eine Beschränkung auf Baumarten, die in der
potenziell natürlichen Waldgesellschaft, bzw.\ aller Sukzessionsstadien bis zu
ihrem erreichen, vorkommen würden, ist allerdings nicht immer nötig. Baumarten
des Vorbestandes und der näheren Umgebung, auf vergleichbarem Standort, sind
standortstauglich. Da der Standort z.\,B.\ durch den Klimawandel, Stickstoff-- und
Schadstoffeintrag laufend verändert wird, können standortstaugliche Baumarten
des Vorbestandes, im Folgebestand plötzlich nicht mehr standortstauglich sein.
Dies kann so weit gehen, dass der Standort für eine andere Landnutzungsform
(Grasland, Acker) geeigneterer werden kann. In vielen Fällen zeigt sich die
Tauglichkeit oder Untauglichkeit einer Baumart für einen Standort bereits in der
Verjüngungsphase. Wobei alleine der Ausfall einer Baumart in der Verjüngung,
insbesondere wenn nur ein Jahr betrachtet wurde, noch kein ausreichender Hinweis
für die \emph{standörtliche} Untauglichkeit der Baumart darstellt. Dabei kann
z.\,B.\ schlecht oder zur falschen Zeit gepflanzt worden sein, die Witterung in
diesem einen Jahr ungünstig gewesen sein, Tiere, Pilze oder andere Pflanzen die
jungen Bäume zum Absterben gebracht haben.

Wenn eine Baumart gefunden wurde, die auf dem Standort des Bestandes
wachsen kann, ist damit die Baumartenwahl keinesfalls abgeschlossen!
Der Standort bestimmt auch die Wuchsleistung und die Gefährdung
(Nassschneelage -- Schneebruch, Durchwurzelungstiefe,
Windgeschwindigkeit -- Windwurf, Windbruch) und damit wie häufig
welche Pflegemaßnahmen durchgeführt werden sollten.

Durch den Klimawandel verändert sich der Standort relativ
rasch. Leider reagieren bereits etablierte Bäume bei einer
Standortsveränderung anders als Bäume die immer bei derartigen
Standortsverhältnissen stocken. D.\,h.\ die dortigen Beobachtungen können
nur bedingt übertragen werden \cite{yue2022SiteIndex}. Sobald die
Geschwindigkeit der Standortsveränderung abnimmt, oder zum Erliegen
kommt, werden die Bedingungen von Standorten, die dem erwarteten Endzustand
heute schon aufweisen, besser übertragbar sein. Die Phase der
Veränderung ist damit besonders schwierig einzuschätzen, da eben die
Beobachtungen von anderen Standorten nur eingeschränkt übertragbar
sind und zusätzlich die Baumarten den Standortsbedingungen am Anfang
und während der \emph{gesamten} Veränderung und falls es zu einem
Endzustand kommt, auch diesem, nicht nur gewachsen sein müssen, sondern
die Forderungen, die an den Wald gestellt werden, möglichst gut
erfüllen sollen.

\subsection{Ertrag}
\label{ssec:ertrag}

Standortstaugliche Baumarten können sich in der Ertragsleistung
beträchtlich unterscheiden. Durch dieses Kriterium wird die Anzahl zur
Auswahl stehenden Baumarten oft entscheidend reduziert und führte
aufgrund der Reinertragslehre oft zum Dominieren einer Baumart in
ganzen Regionen. So wichtig die Ertragsleistung auch ist, mag es
durchaus gerechtfertigt sein die zweit--, dritt-- oder
viertleistungsfähigste Baumart zu wählen bzw.\ die Bedeutung dieses
Kriteriums gering zu gewichten, zumal die erwartete Ertragsleistung,
selbst wenn sich diese nur auf die Zuwachsleistung bezieht, recht
unsicher geschätzt werden kann. Nicht nur die Zuwachsleistung, sondern
auch das gewünschte \emph{Zielsortiment} beeinflussen die
Baumartenwahl. Ein Waldbesitzer, der seinen Brennholzbedarf decken
will, wird am selben Standort meist andere Baumarten wählen als einer
dessen Ziel die Wertholzproduktion ist.

\subsection{Pflegeaufwand}
\label{ssec:pflegeaufwand}

Baumarten unterscheiden sich hinsichtlich ihres nötigen
Pflegeaufwandes. Selbst wenn jemand den Ertrags-- und
Qualitätskriterien wenig Bedeutung beimisst, sind Pflegeeingriffe
(Durchforstungen) dringend zu empfehlen, um die Bestandesstabilität
(Windwurf, Schneebruch, Insekten, Pilze) nicht zu gefährden. Neben der
Baumart entscheiden auch die Ziele (z.\,B.\ Produktion von
Qualitätsholz), der Standort (Wuchsleistung der Baumart auf diesem
Standort) und eine eventuelle Mischung (entfernen von
konkurrenzstärkeren Nachbar einer anderen Baumart um die
konkurrenzschwächere Baumart zu erhalten) den Pflegeaufwand. Allgemein
erhöht sich der Pflegeaufwand, hinsichtlich Frequenz und Umfang, mit
Zunahme der Wuchsleistung und hinsichtlich Komplexität, mit Zunahme
des Qualitätsanspruches an das zu erntende Holz, der
Ungleichaltrigkeit und der Baumartenmischung.

\subsection{Fruchtfolge}
\label{ssec:Fruchtfolge}

\cite{jentsch1911fruchtwechsel} warf die Frage auf, ob ein Fruchtwechsel auch in
der Forstwirtschaft Sinn mache. Was von \cite{sieber1919Holzartenwechsel} und
\cite{fabricius1924Holzartenwechsel} nochmals aufgegriffen, aber nie eindeutig
beantwortet, wurde. Allerdings beobachtete \cite{simak1951Baumartenwechsel},
dass es in natürlichen Plenterwäldern durchaus zu einem gegenseitigen Wechsel
des Standortes, zwischen den Baumarten, kommt. Wenn der Aufwand des
Baumartenwechsels, gleich oder gar geringer ist gegenüber dem Beibehalten der
bisherigen Baumart(en) und andere Kriterien ebenfalls nicht dagegen sprechen,
stellt er eine einladende Möglichkeit dar. Eventuell ist eine
\glqq{}Kahlfläche\grqq{}, mit entsprechender Schlagvegetation und ein paar
Jahren Schlagruhe bereits als Fruchtfolge einzustufen. Ob ein Fruchtwechsel, in
welcher Form auch immer, sinnvoll oder schädlich ist, wird vom Standort und den
beteiligen Baum-- und Pflanzenarten abhängen.

Baumarten, die in der Lage sind, den Wurzelraum zu vergrößern, wie dies
bei Tanne, Lärche, Traubeneiche oder Schwarzerle \cite[S.~114, 130,
150, 180]{koestler1969WurzelnDerWaldbaeume} beobachtet wurde, sollten
in der Lage sein, bisher unerreichbare Nährstoffe oder im Boden
gespeichertes Wasser zu erschließen. Wenn damit ein Mangel verringert werden kann, sollte dies zu einer Steigerung der Wuchsleistung führen.
Auch scheint es möglich auf Standorten mit wenig Stickstoff,
durch die Beimischung von stickstofffixierenden Baum-- und
Straucharten, wie etwas Robinie, Erle, Geweihbaum, Gleditschie,
Gelbholz, Schnurbaum, Goldregen, Sanddorn, Ölweide, Blasenstrauch,
Ginster oder andern, zum Teil nicht verholzenden, Leguminosen, die
Zuwachsleistung zu erhöhen. Diese Effekte sind, zumindest teilweise,
auch zu erwarten, wenn diese Standortsverbesserungen von Arten des Vor-- oder Zwischenbestands herbeigeführt werden.

\subsection{Risiko}
\label{ssec:risiko}

Baumarten ohne Ausfallrisiko gibt es nicht. Aber es gibt Baumarten
die, auf bestimmten \emph{Standorten}, ein wesentlich höheres
Ausfallrisiko haben als andere. Auch die Nachbarbestände können das
Risiko mitbestimmen, indem sie Deckungsschutz bieten oder auch nicht.
Der Deckungsschutz wurde von
\cite{wagner1923DerBlendersaumschlagUndSeinSystem} ausführlich
dargestellt und als Lösung der Blendersaumschlag (Streifenweise
Schläge die sich, z.\,B.\ entgegen die Hauptwindrichtung, im Laufe der
Zeit aneinander reihen) vorgeschlagen. Aber nicht nur der unmittelbare
Nachbarbestand, sondern die Baumartenverteilung einer ganzen Region,
kann das Ausfallrisiko einer Baumart mitbestimmen, indem sich z.\,B.\
Krankheiten oder Insektenkalamitäten von diesen auf deren Nachbarn
übertragen. Ob eine Baumart im Bestand die gleiche oder eine
andere Baumart als Nachbarn hat, beeinflusst ebenfalls das
Ausfallrisiko. Steht diese neben einer Baumart mit hohem
Wasserverbrauch, kann diese Mischbaumart das Ausfallrisiko der
Nachbarbaumart in Trockenzeiten erhöhen. Das Risiko ist somit keine
Konstante, selbst bei einer bestimmten Baumart auf einem konkreten
Standort. Bestandesdichte und geplante Umtriebszeit, stellen weitere
wesentliche Einflussgrößen für das Bestandesrisiko dar. Dabei gibt es
Baumarten, die durch ihren Wachstumsverlauf kurze Umtriebszeiten eher
ermöglichen (Pionierbaumarten) als andere (Klimaxbaumarten). Bei
ungewissen, massiven Standortsveränderungen, hilft die Wahl einer
kürzere Umtriebszeit, die Ungewissheit ein wenig einzuschränken, da
sich in kürzeren Zeiträumen der Standort weniger stark verändern kann
als in längeren. Längere Umtriebszeiten verringern dafür den Aufwand,
da die Frequenz der Endnutzung mit anschließender Verjüngungsphase
länger ist. Gleichzeitig beeinflusst die Umtriebszeit auch die
Wuchsleistung der Bestände. Abweichungen von der zuwachsoptimalen
Umtriebszeit führen sowohl bei Verkürzung als auch bei Verlängerung zu
Zuwachseinbußen, wobei die Kurve der durchschnittlichen Zuwachsleistung über dem Erntealter sehr flach verläuft und damit die Zuwachsverluste innerhalb eines weiten Bereichs von Erntealtern, gegenüber dem zuwachsidealen Erntealter, gering ausfallen werden.

\subsection{Bestandestyp}
\label{ssec:bestandestyp}

Ob man einen \emph{Hochwald}, \emph{Mittelwald} oder \emph{Niederwald}
anstrebt, beeinflusst das Spektrum der zur Auswahl stehenden
Baumarten. So können im Niederwald ausschließlich Baumarten mit
Stockausschlag oder Wurzelbrut Verwendung finden. Der Bestandestyp
kann in vielen Fällen frei gewählt werden. Auf bestimmten Standorten
kann einer dieser Typen Vorteile gegenüber den anderen
Zeigen. Beispielsweise ist die Verjüngung auf trockenen Standorten im
Niederwald meist einfacher als im Hochwald.

Die Baumartenwahl ist für \emph{Reinbeständen} einfacher als für
\emph{Mischbeständen}. Wobei Reinbestände keinesfalls bedeutet, das eine ganze
Region nur von einer Baumart dominiert wird. Hier wechseln die Baumarten von
Bestand zu Bestand und bilden so ein Baumartenmosaik. In Mischbeständen
bestimmen die \emph{Mischungsform} (Einzel, Trupp, Gruppe, Horst, Fläche, Reihe,
Streifen) und die Baumartenanteile, die Baumartenauswahl mit. Im Mischbestand
müssen die Baumarten zusammenpassen und sollten eine Funktion in der Mischung
übernehmen.

Die \emph{Struktur} (Einschichtig, Mehrschichtig, Stufig; Regelmäßig,
Zufällig, Geklumpt) und ob ein Bestand gleich-- oder ungleichaltirg
sein soll, beeinflusst ebenfalls die Baumartenwahl. In mehrschichtigen
Beständen müssen die Baumarten der unteren Schichten mit dem dort
geringeren Lichtangebot auskommen.

Auch die Entscheidung zwischen Natur-- oder Kunstverjüngung, bzw.\
eine Kombination von beiden, beeinflusst die zur Auswahl stehenden
Baumarten.

\subsection{Schutzwirkung}
\label{ssec:schutz}

Baumarten unterscheiden sich hinsichtlich dem Rückhaltevermögen von
Schnee und beeinflussen damit deren Schutzwirkung gegen
Lawinen. Ähnliches, gilt auch für den Wasserrückhalt, welcher
Auswirkungen auf den Hochwasserschutz aber auch auf die
Gleichmäßigkeit der Wasserspende in Quelschutzwäldern hat. In
Quelschutzwäldern spielen auch die nichtfreisetzung von Stickstoff und
ein möglichst geringe Deposition von Luftschadtoffen eine Rolle.
Verschiedenen Baumarten unterscheiden sich hinsichtlich ihrer
Wirkungen auf das Biotop bzw.\ Habitat und leisten unterschiedliches
für den Naturschutz und die Biodiversität. Dies ist insbesondere in
Wechselwirkung zu den restlichen Baumarten, der umliegenden Bestände,
zu betrachten.

\subsection{Erholungswirkung}
\label{ssec:erholung}

Insbesondere in stadtnahen Wäldern haben Landschaftspflege und
Erholungswirkung eine hohe Bedeutung. Diese werden in der Regel kaum von
einer Baumart eines Bestandes, sondern im Zusammenwirken aller Bestände
mit deren Baumarten und deren Bestandesstruktur bestimmt.

\section{Hilfen}
\label{sec:hilfen}

\subsection{Baumartenbeschreibung}
\label{sec:baBeschreibung}

Für die Baumartenwahl sind die waldbaulichen Eingenschaften und
Ansprüche der Baumarten entscheidend. Diese sind beispielsweise in den
Waldbaulehrbüchern von
\citet{mayer1992Waldbau,burschel2003Waldbau,Dengler2020Waldbau,tschermak1950Waldbau,rittershofer2006Waldbau,rubner1960Waldbau,koestler1950Waldbau,bauer1962WaldbauAlsWissenschaft}
oder in den Baumartenbeschreibungen von
\citet{eth2002MitteleuropaeischeWaldbaumarten,leibundgut1984Waldbaeume,ec2016baumartenatlas,hieke1989Dendrologie,mayr1906FremdlaendischeWaldUndParkbaeumeFuerEuropa,stimm2014EnyklopedieDerHolzgewaechse,schuett1993LexikonDerForstbotanik,fva2021Artensteckbrief,hempel1890BaeumeUStraeucher}
oder auch in Floren von
\citet{fischer2008Exkursionsflora,hegi1906IllustrierteFloraBd1,oberdorfer2001Exkursionsflora,rothmaler2021Exkursionsflora,schmeil2019Exkursionsflora,fitschen2017Gehoelzflora}
enthalten. Ein Beispiel einer kurzen Baumartenbeschreibung zeigt
Tab.~\ref{tab:baCharakter}.

Ein erster Entwurf einer längeren Baumartenliste samt ein paar Charakteristika
zeigt Tabelle~\ref{tab:BaumartenCharakteristik}.\footnote{Dazu werden gerne
Hinweise zu Ergänzungen und Korrekturen entgegengenommen.} Es ist zu beachten,
dass derzeit nicht alle dort aufgelisteten Baumarten nach dem Forstgesetz als
Baumart betrachtet werden und somit rechtlich keinen Wald bilden können.
Verboten sind diese Baumarten jedoch nicht. Ihr Anteil darf, solange keine
Sondergenehmigung vorliegt, jedoch nicht so hoch sein, dass der verbleibende
Bestand nicht mehr in der Lage ist, nach dem Forstgesetz, einen Wald zu bilden.
Umgekehrt sollten solche Baumarten, beispielsweise bei einer Ackeraufforstung,
die Landnutzungsform von Landwirtschaft auf Wald rechtlich nicht verändern
können. Verbotene Baum-- und Straucharten gibt es laut österreichischem
Forstgesetz zurzeit nicht. Von den restlichen Pflanzen ist mir derzeit nur das
burgenländische Ragweed--Bekämpfungsgesetz bekannt in dem Grundeigentümer
verpflichtet werden Ragweed zu entfernen. Dennoch gibt es auch bei Baumarten
hinweise auf invasive Arten, die sich selbständig relativ leicht verbreiten und
teilweise in der Lage sind, den Standort zu verändern und die derzeit dort
vorhandene Vegetation zu verdrängen. Umgekehrt ist aber gerade in Regionen, in
denen die Verjüngung schwierig ist und aufgrund einer geringen Wuchsleistung,
die Bereitschaft, in Verjüngungsmaßnahmen zu investiert, eingeschränkt ist, die
Eigenschaft einer natürlichen Verjüngung sowie die Fähigkeit zur Steigerung der
Wuchsleistung, durch Stickstofffixierung, oft willkommen. Das Abwiegen des
möglichen Schadens, durch Einbringen einer fremdländischen Baumart, gegenüber
dem möglichen Nutzen ist schwierig. Regweed verursacht Kosten und belastet viele
Menschen, umgekehrt werden die wenigsten in Europa auf Mais, Kartoffeln oder
Tomaten aus lokalem Anbau verzichten wollen. Aber auch Walnuss (Mittelasien),
Marille/Aprikose (Armenien, China, Indien), die Rosskastanie (Balkanhalbinsel)
oder Robinie/Akazie (Nordamerika) sind in Mitteleuropa nicht heimisch. Bei
fremdländischen, aber auch standortsfremden heimischen Baumarten, die in einer
Region noch nicht anzutreffen sind, wäre es ratsam Experten vor einer geplanten
Einbringung einzubinden. Bei, auf einem Standort, bereits etablierten Baumarten,
einer Region, scheint diese zurückhaltende Vorsicht nicht mehr so nötig zu sein,
insbesondere, wenn ein nachträgliches dauerhaftes Ausrotten dieser Art,
aussichtslos erscheint.

\subsection{Verbreitungskarten}
\label{sec:verbreitungskarten}

Das Verbreitungsgebiet einer Baumart zeigt, wo diese derzeit angetroffen werden
kann. Das die Standortsverhälnisse, außerhalb dieses Gebiets, eine weitere
Ausbreitung verhindern, kann \emph{nicht} automatisch angenommen werden, da eben
die Verbreitung neben dem Standort auch von der Konkurrenz anderer Baumarten
abhängt. Zusätzlich fluktuieren die möglichen Verbreitungsgrenzen einer Baumart
laufend (Standort aber auch Standortsanspruch der Baumart ändert sich) und es
ist je nach Baumart mit beträchtlichen Zeiträumen zu rechnen bis diese auch von
der jeweiligen Baumart tatsächlich bestockt sind. In
Abbildung~\ref{fig:tannerVerbreitungskarte} ist beispielsweise das derzeitige
Verbreitungsgebiet der Tanne dargestellt.

\begin{figure}[htbp]
  \centering
  \includegraphics[width=.95\columnwidth]{./pic/tannerVerbreitungskarte.jpg}
  \caption{Verbreitungskarte der Tanne (Quelle: Wikipedia/Euforgen)}
  \label{fig:tannerVerbreitungskarte}
\end{figure}

Genauso wie für die derzeitigen Verbreitungsgebiete, wurden auch Karten für
zukünftige Baumartenverbreitungen, bei angenommen Klimaszenarien, gezeichnet.
Eine Differenzierung hinsichtlich standortstauglich und anbauwürdig wird in der
Regel leider nicht gemacht. Diese Unterscheidung wäre jedoch für die Praxis
eminent wichtig. So mag auf einen Standort beispielsweise die Flaumeiche
anbaufähig sein und in 100~Jahren eine Höhe von 15\,m erreichen. Wenn am
gleichen Standort die Traubeneiche in 100~Jahren eine Höhe von 25\,m erreichen
kann, wird die Flaumeiche dort nur in Ausnahmefällen anbauwürdig sein. Meist
wird auch nicht berücksichtigt, dass einige Baumarten nicht auf jenen Standorten
angetroffen werden, die ihnen am besten zusagen würden, sondern auf jene, wo sie
aufgrund der Konkurrenzsituation verdrängt wurden
\citep{lian2022BaVerbreitungOpt}. Damit zeigen solche Karten gelegentlich
Regionen, die der Baumart gut zusagen würden, fälschlich als ungeeignet an.
Umgekehrt werden Regionen als geeignet ausgeschieden, die weder als anbauwürdig
noch als standortstauglich gelten können.

Deutlich mehr Information für die Baumartenwahl geben Karten der
natürlichen Waldgesellschaft (Abb.~\ref{fig:waelderDesOstalpenraumes})
mit dazugehöriger Tabelle (Tab.~\ref{tab:waldgesellschaften}) die
angibt, in welcher Höhenstufe diese Gesellschaft zu finden ist und mit
welchen Anteilen bestimmte Baumarten vertreten sind. Darin ist zu
erkennen, dass in der subalpinen Höhenstufe von Natur aus entweder
Zirbe oder Lärche oder Fichte vorherrschend und weitere Baumarten nur
in geringem Ausmaß beigemischt sind. Dies setzt sich in der montanen
Stufe mit dem montanen Fichtenwald fort und geht dann in eine Mischung
aus Fichte, Tanne und Buche, in die von Buche dominierten Wälder,
über. In der planar kollinen Zone sind Mischungen häufiger vertreten,
doch auch hier dominiert, auf etlichen Standorten, entweder
Traubeneiche, Weißkiefer oder Schwarzkiefer.

Es gibt auch Karten, welche die geschätzte Wuchsleistung zeigen
(Abb.~\ref{fig:Eichenbonitaet}). Dabei muss angemerkt werden, dass diese nur
einen groben Anhaltspunkt geben können und von der, von kleinstandörtlichen
Gegebenheiten geprägten Bestandesbonität, beträchtlich abweichen kann. Diese
Karten sind derzeit relativ grob und differenzieren beispielsweise selten
zwischen verschiedenen Eichenarten. Auch hier sind Karten, welche die
Wuchsleistung bei einem veränderten Klima darstellen, vorhanden.

\begin{figure}[htbp]
  \centering
  \includegraphics[width=.95\columnwidth]{./pic/quSp_t0_p1.png}
  \caption{Höhe [m] der höchsten 100 Bäume je Hektar eines Bestandes im
    Alter 100 (Oberhöhenbonität von Eiche)
    \citep[S.~17]{kindermann2021Eiche}}
  \label{fig:Eichenbonitaet}
\end{figure}

Neben diesen waldspezifischen Karten geben auch Karten zur Geologie,
Boden, Niederschlag oder Temperatur wichtige Hinweise und können oft
in Schulatlanten oder im Internet gefunden werden.

\subsection{Ökogramme}
\label{sec:oekogramme}

Ökogramme stellen auf den Achsen Bodenfeuchte (dürr--nass) und
Bodenreaktion (sauer--basisch) das Vorkommen einer Baumart dar. Dabei
wird zwischen der physiologischen Amplitude, physiologisches Optimum
und Herrschaftsbereich der Baumart, in der Regel für eine bestimmte
\emph{Höhenstufe} (planar--subalpin), unterschieden
(Abb.\ref{fig:oekogramm}). De facto werden also 3~Dimensionen,
Feuchte, Bodenreaktion und Temperatur, dargestellt.

\begin{figure}[htbp]
  \centering
\begin{tikzpicture}
  \draw [line width=2pt] (0,0) -- (8,0) -- (8,8) -- (0,8) -- cycle;
  \draw (0,7.5) node[anchor=east] {dürr};
  \draw (0,4) node[anchor=east] {frisch};
  \draw (0,0.5) node[anchor=east] {nass};
  \draw (0.5,0) node[anchor=north,align=center] {sehr\\sauer};
  \draw (4,0) node[anchor=north,align=center] {mässig\\sauer};
  \draw (7.5,0) node[anchor=north] {basisch};
  \draw[line width=2pt,dashed] (0,7.5) -- (8,7.5);
  \draw[line width=2pt,dashed] (0,1.5) {[rounded corners=1cm] -- (1,.5)} -- (8,.5);
  %\draw[pattern=north west lines]
  \fill[gray!70] (8,2.5) {[rounded corners=1cm] -- (1.5,2.5) -- (1.5,5.5)} -- (8,5.5);
  \fill[gray!170] (8,7.3) {[rounded corners=.3cm] -- (0.3,7.3) -- (0,7)} {[rounded corners=.8cm] -- (0,4.5) -- (1.5,4.5) -- (1.5,6) -- (8,6.8)};
  \draw[line width=3pt,fill=gray,fill opacity=0.2] (8,2) -- (1,2) {[rounded corners=1cm] -- (0,2)} {[rounded corners=.3cm] -- (0,7) -- (0.3,7.3)} -- (8,7.3) -- cycle;
  \draw [line width=2pt] (4,4) circle (.1);
\end{tikzpicture}
  \caption{Beispiel eines Ökogramms}
  \footnotesize{
    \tikz \draw[line width=2pt,dashed,rounded corners=2] (0,0) rectangle (1em,1em);~Grenze waldfähiger Standorte,
    \tikz \draw[line width=3pt,fill=gray,fill opacity=0.2,rounded corners=2] (0,0) rectangle (1em,1em);~Physiologische Amplitude,
    \tikz \fill[gray!80,rounded corners=2] (0,0) rectangle (1em,1em);~Physiologisches Optimum,
    \tikz \fill[gray!160,rounded corners=2] (0,0) rectangle (1em,1em);~Herrschaftsbereich der Baumart (ökologisches Optimum / Niesche)
  }
  \label{fig:oekogramm}
\end{figure}

Ähnlich den Ökogrammen sind \emph{Klimadiagramme}, mit den Achsen
Temperatur und Niederschlag, in welche, meist als Punktwolke,
Beobachtungen einer Baumart dargestellt werden. Zwischen
physiologischem Optimum und ökologischer Nische wird dort meist nicht
unterschieden, da dies eben alleine über Niederschlag und Temperatur
kaum differenzierbar sein wird.

Eine Erweiterung auf zusätzliche, wesentliche Standortsdimensionen stellen die
\emph{Zeigerwerte nach Ellenberg} dar \citep{ellenberg2010vegetation}. Mit
diesen kann ein Standort, aufgrund der auf ihm wachsenden Pflanzen,
charakterisiert werden. D.h.\ es wird in der Regel das ökologische Optimum,
nicht jedoch das physiologische Optimum oder Spektrum dargestellt. Beschrieben
werden Licht, Temperatur, Kontinentalität, Feuchtigkeit, Bodenreaktion,
Stickstoff, Salz und Schwermetallresistenz. Weiters wird die Lebensform und die
Blattausdauer (immergrün, sommergrün, \dots) angegeben. Für Österreich wurden
diese Zeigerwerte von \cite{karrer1992Zeigerwerte} angepasst. Ähnlich sind auch
die Zeigerwerte nach \cite{landolt2010floarIndicative} für die Schweiz,
\cite{ehrendorfer1971NaturgechichteWiens} für Wien oder
\cite{tichy2023zeigerwerte} mit einer Zusammenstellung für Europa.

Basierend auf diesen Zeigerwerten wurde von
\cite{ninemetsValladares2006TolleranceToShadeDroughtAndWaterlogging}
nicht das häufigste Vorkommen, sondern die Grenze des Vorkommens,
hinsichtlich Trockenheit--, Überschwemmung-- und Schattentoleranz, von mehr als
800~Baum-- und Straucharten, aufbereitet
(Tabelle:~\ref{tab:trockentolleranz}). Ein Wert von 1 bedeutet
geringe, einer von 5 hohe Toleranz. Dazu sei aber angemerkt, dass auch hier der Standort die Reihung modifizieren kann. So wird auf einem trockenen Standort ohne Grundwasseranschluss, bei öfter wiederkehrenden aber schwachen Niederschlägen eine flachwurzelnde Baumart eher zurechtkommen als eine tiefwurzelnde, bei seltenen, dafür aber ausgiebigeren Niederschlagen hingegen, wird die tiefwurzelnde besser als die flachwurzelnde zurechtkommen.

\begin{table}[htbp!]
\centering
\begin{tabular}{l S[
  table-number-alignment = center,
  separate-uncertainty = true,
  table-figures-uncertainty = 1,
  table-figures-decimal = 2,
  table-figures-integer = 1
  ] S[
  table-number-alignment = center,
  separate-uncertainty = true,
  table-figures-uncertainty = 1,
  table-figures-decimal = 2,
  table-figures-integer = 1
  ]}
  & {Trockentoleranz} & {Schattentoleranz}\\
  \hline
  Gleditschie   & 4.98 \pm 0.02 & 1.61 \pm 0.2  \\
  Schwarzkiefer & 4.38 \pm 0.47 & 2.1  \pm 0.43 \\
  Weißkiefer    & 4.34 \pm 0.47 & 1.67 \pm 0.33 \\
  Zerreiche     & 4.29 \pm 0.21 & 2.55 \pm 0.11 \\
  Robine        & 4.11 \pm 0.65 & 1.72 \pm 0.25 \\[.3em]
  Flaumeiche    & 4.1  \pm 0.25 & 2.31 \pm 0.22 \\
  Edelkastanie  & 3.46 \pm 0.18 & 3.15 \pm 0.23 \\
  Traubeneiche  & 3.02 \pm 0.15 & 2.73 \pm 0.27 \\
  Zirbe         & 3.01 \pm 0.43 & 2.87 \pm 0.3  \\
  Nuss          & 2.98 \pm 0.22 & 2.27 \pm 0.24 \\[.3em]
  Stieleiche    & 2.95 \pm 0.31 & 2.45 \pm 0.28 \\
  Roteiche      & 2.88 \pm 0.12 & 2.75 \pm 0.18 \\
  Aspe          & 2.85 \pm 0.25 & 2.22 \pm 0.07 \\
  Winterlinde   & 2.75 \pm 0.15 & 4.18 \pm 0.16 \\
  Bergahorn     & 2.75 \pm 0.16 & 3.73 \pm 0.21 \\[.3em]
  Spitzahorn    & 2.73 \pm 0.16 & 4.2  \pm 0.37 \\
  Hainbuche     & 2.66 \pm 0.16 & 3.97 \pm 0.12 \\
  Kirsche       & 2.66 \pm 0.22 & 3.33 \pm 0.33 \\
  Douglasie     & 2.62 \pm 0.41 & 2.78 \pm 0.18 \\
  Esche         & 2.5  \pm 0.25 & 2.66 \pm 0.13 \\[.3em]
  Buche         & 2.4  \pm 0.43 & 4.56 \pm 0.11 \\
  Küstentanne   & 2.33 \pm 0.33 & 4.01 \pm 0.19 \\
  Lärche        & 2.31 \pm 0.55 & 1.46 \pm 0.29 \\
  Schwarzerle   & 2.22 \pm 0.66 & 2.71 \pm 0.5  \\
  Schwarzpappel & 2.2  \pm 0.38 & 2.46 \pm 0.09 \\[.3em]
  Hängebirke    & 1.85 \pm 0.21 & 2.03 \pm 0.09 \\
  Tanne         & 1.81 \pm 0.28 & 4.6  \pm 0.06 \\
  Fichte        & 1.75 \pm 0.41 & 4.45 \pm 0.5  \\
  Moorbirke     & 1.27 \pm 0.18 & 1.85 \pm 0.07
\end{tabular}
\caption{Trocken-- und Schattentoleranz nach \cite{ninemetsValladares2006TolleranceToShadeDroughtAndWaterlogging}.}
\label{tab:trockentolleranz}
\end{table}


\subsection{Gefährdungen}
\label{sec:gefaerdungen}

Die Dispositionierung, für bestimmte Gefahren, hängt von vielen
Faktoren ab, die oft als zufällig betrachtet werden. Es gibt immer
wieder Bestandessituationen, bei denen die Wahrscheinlichkeit, dass
ein Schaden eintritt, sehr groß ist, und dennoch wird kein Schaden
beobachtet. Umgekehrt gibt es welche, die als recht sicher gelten und
geschädigt werden. Angaben zur Gefährdung sind deshalb immer nur
\emph{Wahrscheinlichkeiten} und keine Prognosen. So zeigt etwa
Tab.\ref{tab:gefaehrdungen}) von Wellenstein in
\citet[S.~228]{speidel1972PlanungImForstbetrieb}, Produktionsgefahren
und Risikowahrscheinlichkeiten bei verschiedenen Baumarten.

\begin{table}[htbp!]
  \setlength{\tabcolsep}{1pt}
  \centering
  \pgfplotstabletypeset[col sep=comma
  , comment chars=\#
  , text indicator="
  , string type
  , every column/.style={column type=c,preproc cell content/.style={@cell content = \rating{##1}}}
    , display columns/0/.style={column type=l,column name=\rotatebox{-90}{}, preproc cell content/.style={@cell content = ##1}}
  , assign column name/.style={/pgfplots/table/column name={\multicolumn{1}{c}{\rotatebox{90}{#1}}}}
  ]{./tabs/gefaehrdung.csv}
  \caption{Produktionsgefahren und Risikowahrscheinlichkeiten bei verschiedenen
    Baumarten nach Wellenstein in \cite{speidel1972PlanungImForstbetrieb}}
  \footnotesize{\rating{0}\dots keine Gefährdung,
    \rating{1}\dots wirtschaftl.\ unbedeutende Gefährdung,
    \rating{2}\dots w.\ bedeutende Gefährdung,
    \rating{3}\dots w.\ sehr bedeutende Gefährdung}
  \label{tab:gefaehrdungen}
\end{table}



%https://klimadatenportal.lgl-bw.de/viewer/client/index.html


%\addcontentsline{toc}{section}{Stichwortverzeichnis}
%\printindex

\addcontentsline{toc}{section}{Literatur}
\bibliography{literature}

\clearpage % To flush out all floats

\begin{table*}[htbp!]
  \setlength{\tabcolsep}{1pt}
  %\centering
  \pgfplotstabletypeset[col sep=comma
  , text indicator="
  , string type
  , every column/.style={column type=c,preproc cell content/.style={@cell content = \ratingB{##1}{2}}}
  , my colIterator/.style={every col no #1/.style={column type=c,preproc cell content/.style={@cell content = \ratingB{####1}{3}}}}
  , my colIterator/.list={1,2,11,16}
  , my colIteratorB/.style={every col no #1/.style={column type=c,preproc cell content/.style={@cell content = \ratingB{####1}{4}}}}
  , my colIteratorB/.list={10,15}
  , display columns/5/.style={column type=c|,preproc cell content/.style={@cell content = \ratingB{##1}{4}}}
  , display columns/9/.style={column type=c||,preproc cell content/.style={@cell content = \ratingB{##1}{3}}}
  , display columns/19/.style={column type=c,preproc cell content/.style={@cell content = \ratingB{\the\numexpr ##1 - 410 \relax}{340}}}
  , display columns/0/.style={column type=l,column name=\rotatebox{-90}{}, preproc cell content/.style={@cell content = ##1}}
  , assign column name/.style={/pgfplots/table/column name={ \multicolumn{1}{c}{\rotatebox{90}{#1}}}}
  , every row no 0/.append style={before row={
      & \multicolumn{5}{c|}{Anspruch} & \multicolumn{4}{c||}{Empfindl.}
      & \multicolumn{9}{c}{Eigenschaft} \\
     }}
  ]{./tabs/baCharakter.csv}
  \caption{Ansprüche und Eigenschaften heimischer Wirtschaftsbaumarten nach \citet[Bd.1, S.183f]{bauer1962WaldbauAlsWissenschaft}}
  \footnotesize{\emph{Anspruch, Empfindlichkeit, Eigenschaft:} \ratingB{0}{3}\dots gering, \ratingB{1}{3}\dots mäßig, \ratingB{2}{3}\dots hoch, \ratingB{3}{3}\dots sehr hoch\\
    \emph{Verhalten zu Nachbarn:} \ratingB{0}{2}\dots unverträglich, \ratingB{1}{2}\dots störend, \ratingB{2}{2}\dots verträglich
  }  
  \label{tab:baCharakter}
\end{table*}

\afterpage{%
%  \clearpage% To flush out all floats
\begin{landscape}
  \begin{figure}[ht]
    \centering
    \includegraphics[height=.9\textheight]{./pic/waelderDesOstalpenraumes.pdf}
    \caption{Wälder des Ostalpenraumes \citep{mayer1977KarteWaldtypen}}
    \label{fig:waelderDesOstalpenraumes}
  \end{figure}
\end{landscape}
}

\begin{table*}[htbp!]
  \setlength{\tabcolsep}{1pt}
  \centering
  \pgfplotstabletypeset[col sep=comma
  , text indicator="
  , string type
  , after row=\cdashline{3-38}
  %, every head row/.style={before row=\hline}
  %, every last row/.style={after row={}}
  %, every last column/.style={column type=c}
  , every column/.style={column type=c:,preproc cell content/.style={@cell content = \torte{##1}}}
  , display columns/0/.style={column type=l,column name=\rotatebox{-90}{Waldgesellschaft}, preproc cell content/.style={@cell content = ##1}}
  , display columns/1/.style={column type=c|,preproc cell content/.style={@cell content = \hst{##1}}}
  , assign column name/.style={/pgfplots/table/column name={\multicolumn{1}{c}{\rotatebox{90}{#1}}}}
  , columns/Weisskiefer/.style={column name=Weißkiefer}
  , every row no 4/.style={before row=\hline}
  , every row no 12/.style={before row=\hline}
  , every row no 22/.style={before row=\hline}
  ]{./tabs/gesselschaftsanschluss.csv}
  \caption{Natürlicher Gesellschaftsanschluß mit Vergesellschaftung der Baumarten in Wäldern des Ostalpenraumes \citep{mayer1992Waldbau}}
  \footnotesize{\hst{0}\dots Auwald,
    \hst{1}\dots Planar--Kolin --400\,m,
    \hst{2}\dots Submontan 200--900\,m,
    \hst{3}\dots Montan 500--1800\,m,
    \hst{4}\dots Subalpin 1200--2300\,m\\
    \tikz \draw (0,0) rectangle (1em,1em);\dots fehlend,
    \torte{1}\dots sporadisch,
    \torte{2}\dots spärlich,
    \torte{3}\dots gering,
    \torte{4}\dots mittel,
    \torte{5}\dots reichlich,
    \torte{6}\dots vorherschend
  }
  \label{tab:waldgesellschaften}
\end{table*}

%\begin{table*}[htbp!]
%  \setlength{\tabcolsep}{1pt}
%  %\centering
%  \pgfplotstabletypeset[col sep=comma
%  , text indicator="
%  , string type
%  ]{./tabs/baEigenschaften.csv}
%  \caption{Eigenschaften}
%  \footnotesize{\emph{Eigenschaft:} \ratingB{0}{3}\dots gering, \ratingB{1}{3}\dots mäßig, \ratingB{2}{3}\dots hoch, \ratingB{3}{3}\dots sehr hoch\\
%  }  
%  \label{tab:baEigenschaften}
%\end{table*}

%\begin{table*}[htbp!]
%  \setlength{\tabcolsep}{1pt}
%  %\centering
%  \pgfplotstabletypeset[col sep=comma
%  , text indicator="
%  , string type
%  ]{./tabs/baAnspruch.csv}
%  \caption{Ansprüche}
%  \footnotesize{\emph{Anspruch:} \ratingB{0}{3}\dots gering, \ratingB{1}{3}\dots mäßig, \ratingB{2}{3}\dots hoch, \ratingB{3}{3}\dots sehr hoch\\
%  }  
%  \label{tab:baAnsprueche}
%\end{table*}

%\begin{table*}[htbp!]
%  \setlength{\tabcolsep}{1pt}
%  %\centering
%  \pgfplotstabletypeset[col sep=comma
%  , text indicator="
%  , string type
%  ]{./tabs/baEmpfindlichkeiten.csv}
%  \caption{Empfindlichkeiten}
%  \footnotesize{\emph{Empfindlichkeit:} \ratingB{0}{3}\dots gering, \ratingB{1}{3}\dots mäßig, \ratingB{2}{3}\dots hoch, \ratingB{3}{3}\dots sehr hoch\\
%  }  
%  \label{tab:baEmpfindlichkeiten}
%\end{table*}

\newpage

% \includepdf[pages=-,pagecommand={},width=\textwidth]{test.pdf}

\tablecaption{Charakteristika einiger Baumarten}
\label{tab:BaumartenCharakteristik}
\tablehead{Code & Art Sci & Kurz & Art & t & H & h & A & X & D & T & K & W & L & R & N & F & Ü & G & Z & U & a & w & d & x & M & V & F & S & I & B & P\\ \hline \noalign{\vskip .1em}}
\tabletail{\hline Code & Art Sci & Kurz & Art & t & H & h & A & X & D & T & K & W & L & R & N & F & Ü & G & Z & U & a & w & d & x & M & V & F & S & I & B & P\\}
\tablelasttail{Code & Art Sci & Kurz & Art & t & H & h & A & X & D & T & K & W & L & R & N & F & Ü & G & Z & U & a & w & d & x & M & V & F & S & I & B & P\\}

\afterpage{%
  \onecolumn
  \begin{landscape}
    \begin{xtabular}{llllcccccc@{$\cdot$}cccccc@{$\cdot$}ccc@{$\cdot$}cccccc@{$\cdot$}cccc@{$\cdot$}ccc}
      \input{./rtab/SpTab.tex}\\
      \hline
    \end{xtabular}\\
    \footnotesize{
      Code\dots Baumartencode in Anlehnung an EN~13556\\
      Art Sci\dots Wissenschaftlicher Name\\
      Kurz\dots Baumartencode in Anlehnung an DIN~4076\\
      Art\dots Deutscher Name\\
      t\dots Typ: \typ{1}{0}\dots Nadel sommergrün,
      \typ{1}{1}\dots Nadel immergrün,
      \typ{2}{0}\dots Laub sommergrün,
      \typ{2}{1}\dots Laub immergrün,
      \typ{3}{0}\dots Gingko
      \typ{4}{0}\dots Kletterpflanze sommergrün,
      \typ{4}{1}\dots Kletterpflanze immergrün,
      \typ{5}{0}\dots Palme,
      \typ{6}{0}\dots Kaktus\\
      H\dots Höhe die auf guten Standorten erreicht werden kann: \ratingC{0}{25}\dots 15\,m, \ratingC{0}{50}\dots 30\,m, \ratingC{0}{75}\dots 45\,m, \ratingC{0}{100}\dots 60\,m oder mehr\\
      h\dots Höhe die im Alter~15 auf guten Standorten erreicht werden kann: \ratingC{0}{25}\dots 4\,m, \ratingC{0}{50}\dots 8\,m, \ratingC{0}{75}\dots 12\,m, \ratingC{0}{100}\dots 16\,m oder mehr\\
      A\dots Erreichbares Alter: \ratingC{0}{25}\dots 100~Jahre, \ratingC{0}{50}\dots 200~Jahre, \ratingC{0}{75}\dots 300~Jahre, \ratingC{0}{100}\dots 400~Jahre oder mehr\\
      X\dots Niesche: \ratingC{0}{0}\dots Pionier, \ratingC{0}{50}\dots Extramstandort, \ratingC{0}{100}\dots Konkurrenzstark\\
      D\dots Durchwurzelungstiefe: \ratingC{0}{33}\dots 1\,m, \ratingC{0}{67}\dots 2\,m, \ratingC{0}{100}\dots 3\,m oder mehr\\
      T\dots Temperatur: \ratingC{0}{0}\dots Sehr Kälteempfindlich, \ratingC{0}{100}\dots Geringe Kälteempfindlichkeit\\
      K\dots Kontinentalität: \ratingC{0}{0}\dots Verträgt nur geringe Temperaturschwankung (ozeanisch), \ratingC{0}{100}\dots Verträgt größe Temperaturschwankungen (kontinental)\\
      W\dots Wasser: \ratingC{0}{20}\dots Verträgt Stauwasser, \ratingC{40}{60}\dots frisch (nicht trocken und nicht nass), \ratingC{80}{100}\dots Verträgt Trockenheit\\
      L\dots Licht:  \ratingC{0}{0}\dots Benötigt viel Licht, \ratingC{0}{100}\dots Erträgt starke Beschattung\\
      R\dots Bodenreaktion: \ratingC{0}{10}\dots Stark saure Böden (silikat), \ratingC{40}{60}\dots neutrale Böden, \ratingC{80}{100}\dots Sehr basische Böden (kalk)\\
      N\dots Nährstoffbedarf: \ratingC{0}{0}\dots Sehr hoch, \ratingC{0}{100}\dots Sehr Gering\\
      F\dots \jaNeinQ{0}{0}{0}{1}\dots Stickstofffixierung,
      \jaNeinQ{0}{0}{1}{0}\dots Windbestäubt,
      \jaNeinQ{1}{0}{0}{0}\dots Salzresistent,
      \jaNeinQ{0}{1}{0}{0}\dots Schwermetallresistent\\
      Ü\dots Überdauerungszeitraum von Samen: \ratingC{0}{25}\dots 25~Jahre, \ratingC{0}{50}\dots 50~Jahre, \ratingC{0}{75}\dots 75~Jahre, \ratingC{0}{100}\dots 100~Jahre oder mehr\\
      G\dots Als Nahrung verwendet / Giftig : \jaNeinP{0}{1}\dots Als Nahrung verwendet. Dies bedeuted \textbf{nicht}, das diese Pflanze ohne Gefahr essbar ist.
      \jaNeinP{1}{0}\dots Giftig. Ungiftig wurde hier nicht vergeben! D.h.\ keine Angabe bedeuted \textbf{nicht}, das diese Pflanze ohne Gefahr essbar ist.\\
      Z\dots Planbarkeit: \ratingC{0}{10}\dots Gering --
      Wahrscheinlichkeit für frühzeitigen Ausfall, \ratingC{0}{90}\dots Hoch\\
      U\dots Streuumsatz: \ratingC{0}{0}\dots 5~Jahre oder mehr, \ratingC{0}{100}\dots 1~Jahr oder weniger\\
      a\dots Stockausschlag: \ratingC{0}{0}\dots Nein, \ratingC{0}{100}\dots Ja\\
      w\dots Wurzelbrut: \ratingC{0}{0}\dots Nein, \ratingC{0}{100}\dots Ja\\
      d\dots Holzichte: \ratingC{0}{0}\dots Darrgewicht 350 kg/m$^3$ oder weniger, \ratingC{0}{100}\dots Darrgewicht 800 kg/m$^3$ oder mehr\\
      x\dots Dauerhaftigkeit des unbehandelten Kern--Holzes in Anlehnung an EN 350-2: \ratingC{0}{0}\dots geringe Dauerhaftigkeit, \ratingC{0}{100}\dots Sehr Dauerhaft\\
      M\dots Mäuse: \ratingC{0}{0}\dots geringe Gefährdung, \ratingC{0}{100}\dots hohe Gefährdung\\
      V\dots Verbiss: \ratingC{0}{0}\dots geringe Gefährdung, \ratingC{0}{100}\dots hohe Gefährdung\\
      F\dots Fegen/Schlagen: \ratingC{0}{0}\dots geringe Gefährdung, \ratingC{0}{100}\dots hohe Gefährdung\\
      S\dots Schälen: \ratingC{0}{0}\dots geringe Gefährdung, \ratingC{0}{100}\dots hohe Gefährdung\\
      I\dots Windwurf: \ratingC{0}{0}\dots geringe Gefährdung, \ratingC{0}{100}\dots hohe Gefährdung\\
      B\dots Schneebruch: \ratingC{0}{0}\dots geringe Gefährdung, \ratingC{0}{100}\dots hohe Gefährdung\\
      P\dots Pilze: \ratingC{0}{0}\dots geringe Gefährdung, \ratingC{0}{100}\dots hohe Gefährdung\\
    }
  \end{landscape}
  \twocolumn
}

  
%Autor: Georg Kindermann

\end{document}
