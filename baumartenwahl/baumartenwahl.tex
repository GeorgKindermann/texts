\documentclass[twocolumn]{scrartcl}
%\documentclass{scrartcl}

\usepackage[utf8]{inputenc}
\usepackage[T1]{fontenc}
\usepackage{lmodern}
\usepackage[ngerman]{babel}
\usepackage{amsmath}

\usepackage{makeidx}
\makeindex

\usepackage{fancyvrb}
\usepackage{fvextra}

\usepackage[sc]{mathpazo} % or option osf
\usepackage{newpxmath}
\usepackage[output-decimal-marker={,}]{siunitx}

\usepackage[comma,authoryear]{natbib}
\bibliographystyle{natdin}

\usepackage{adjustbox}
\usepackage[a4paper, margin=1mm, includefoot, footskip=15pt]{geometry}
\usepackage{afterpage}

\usepackage[pdftitle={Kriterien und Hilfen bei der Baumartenwahl}
, pdfauthor={Georg Kindermann}
, pdfsubject={Waldbau, Waldwachstum}
, pdfkeywords={Waldbau, Waldwachstum, Wald, Forst}
, pdflang={de-AT-1996}
, hidelinks
, pdfpagemode=None]{hyperref}

\nonfrenchspacing
\sloppy
\usepackage{breqn}
\usepackage{enumitem}
%\usepackage{rotating}
\usepackage{pdflscape}

\title{Kriterien und Hilfen bei der Baumartenwahl}
\author{Georg Kindermann}

\listfiles
\begin{document}

\twocolumn[
  \begin{@twocolumnfalse}
    \maketitle
    \begin{abstract}

    \end{abstract}
  \end{@twocolumnfalse}
]

\tableofcontents

\section{Einleitung}

\section{Kriterien}
\label{sec:kriterien}

\subsection{Standort}
\label{ssec:standort}

Die standörtliche Tauglichkeit ist wohl der erste Punkt, der, bei der
Baumartenwahl, erfüllt sein muss. Eine Beschränkung auf Baumarten, die
in der potentiell natürlichen Waldgesellschaft vorkommen würden, ist
allerdings nicht nötig.

Der Standort beeinflusst auch welche Plegemaßnahmen durchführbar sind.

\subsection{Ertrag}
\label{ssec:ertrag}

\subsection{Pflegeaufwand}
\label{ssec:pflegeaufwand}

\subsection{Risiko}
\label{ssec:risiko}

\subsection{Bestandestyp}
\label{ssec:bestandestyp}

Hochwald, Mittelwald, Niederwald.
Reinbestand, Mischbestand.
Einschichtig, Mehrschichtig, Gleichalt Ungleichaltrig.
Naturverjüngung / Kunstverjüngung

\subsection{Umgebende Bestände}
\label{ssec:nachbarschaft}


\subsection{Schutzwirkung}
\label{ssec:schutz}

Lawine
Biotop, Habitat, Naturschutz, Biodiversität.


\subsection{Erholungswirkung}
\label{ssec:erholung}

Landschaftspflege


\section{Hilfen}
\label{sec:hilfen}


%\addcontentsline{toc}{section}{Stichwortverzeichnis}
%\printindex

\addcontentsline{toc}{section}{Literatur}
\bibliography{literature}

%Autor: Georg Kindermann

\end{document}
