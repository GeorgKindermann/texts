\documentclass[twocolumn]{scrartcl}

\usepackage[utf8]{inputenc}
\usepackage[T1]{fontenc}
\usepackage{lmodern}
\usepackage[ngerman]{babel}
\usepackage{amsmath}

%\usepackage[xindy]{imakeidx}
%\makeindex[program=xindy, columns=3, title=Stichwortverzeichnis, intoc]
\usepackage{makeidx}
\makeindex

\usepackage{fancyvrb}
\usepackage{fvextra}

\usepackage[sc]{mathpazo} % or option osf
\usepackage{newpxmath}
\usepackage[output-decimal-marker={,}]{siunitx}

\usepackage[comma,authoryear]{natbib}
\bibliographystyle{natdin}

\usepackage{adjustbox}
\usepackage[a4paper, margin=1mm, includefoot, footskip=15pt]{geometry}
\usepackage{afterpage}

%\usepackage{tikz}
%\usetikzlibrary{calc}
%\usetikzlibrary{shapes.geometric}
%\usetikzlibrary{arrows.meta}

%\usepackage[inline]{enumitem}

\usepackage[pdftitle={Waldwachstum}, pdfauthor={Georg Kindermann}
, pdfsubject={Waldwachstum, Baumwachstum}
, pdfkeywords={Waldwachstum, Waldbau, Wald, Forst}
, pdflang={de-AT-1996}
, hidelinks
, pdfpagemode=None]{hyperref}

\nonfrenchspacing
\sloppy

\title{Waldwachstum}
\author{Georg Kindermann}

\begin{document}

\twocolumn[
  \begin{@twocolumnfalse}
    \maketitle
    \begin{abstract}
      Die Wuchsleistung von Wäldern wird von den \emph{Standortseigenschaften}
      (Boden, Wetter, Topografie) den beteiligten \emph{Baumarten} sowie anderer
      Pflanzen-- und Tierarten, deren \emph{gegenseitige Beeinflussung}
      (Bestandesdichte/Konkurrenz um Licht, Wasser, Nährstoffe; Begünstigend
      durch Erschließung von Bodenhorizonten, Stickstoffixierung) und den
      \emph{Baummerkmalen} (Durchmesser, Höhe, Kronenlänge, Alter, Veranlagung,
      Schäden, Krankheiten) bestimmt. Standortseigenschaften können modifiziert
      werden (Veränderung physikalischer und chemischer Bodeneigenschaften:
      befahren, düngen, Depositionen, Baumartenwahl -- Durchwurzelungsenergie,
      Streueigenschaften $\rightarrow$ Bodenlebewesen; Schlagrand Nord/Süd; mit
      Vorwald Temperaturextreme für empfindliche Verjüngung verringern). An
      Baumverbreitungsgrenzen (kalt, heiß, trocken, überflutet, nährstoffarm,
      Salzböden) ist die Zahl der möglichen Baumarten sowie deren
      Zuwachsleistung gering. Auf gleichem Standort kann die Zuwachsleistung
      verschiedener Baumarten stark voneinander abweichen. In Wirtschaftswäldern
      bildet die Relation von \emph{Wertzuwachs} zu \emph{Schadrisiko} und
      \emph{Pflegeaufwand} die Basis der \emph{Baumartenwahl}.

      Geringe Stammzahlen bei Aufforstungen, Stammzahlreduktionen bei
      Naturverjüngung sowie Durchforstungen führen zu einer verringerten
      \emph{Bestandesdichte}. Mit geringer Bestandesdichte nimmt der
      Durchmesserzuwachs und die Jahrringbreite zu, nimmt der Höhenzuwachs bei
      vielen Laubbäumen ab (bei Nadelbäumen bleibt er annähernd gleich), bleibt
      der Volumszuwachs des Bestandes über weite Bereiche annähernd gleich,
      nimmt die Aststärke zu, nimmt die Kronenlänge zu, nimmer der
      Kronendurchmesser zu, verlagert sich der Durchmesserzuwachs verstärkt am
      Stammfuß an, womit sich die Stammform von \glqq Walzenförmig\grqq{} in
      Richtung Kegelförmig verändert, erfolgt die Samen-- bzw.\ Fruchtbildung
      bei geringerem Alter, stellt sich eher Verjüngung oder (verdämmende)
      Bodenvegetation ein. Die Wuchsreaktion bei Verringerung der
      Bestandesdichte wird stark von der Kronendimmension bestimmt, welche
      wiederum durch die Bestandesdichte in der Vergangenheit beeinflusst wird.
      Windwurf-- oder Schneebruchrisiko sind unmittelbar nach Durchforstungen
      erhöht (Reduzierung der kollektiven Bestandesstabilität). Langfristig
      reduzieren Durchforstungen diese \emph{Riskiken}.

      Umtriebszeit.

      Baumartenmischung
    \end{abstract}
  \end{@twocolumnfalse}
]

\tableofcontents

\section{Einleitung}

Eine knappe Darstellung mir wichtig erscheinender Grundlagen des Wachstums von
Bäumen, Beständen und Betriebsklassen (Zusammenfassung von Beständen eines
Forstbetriebes). Ausführlichere bzw.\ historisch interessante Arbeiten zum
Waldwachstum sind:
\cite{guttenberg1885Wachstumsgesetze,guttenberg1912ZuwachslehreInHandbuchDerForstwissenschaft,vanselow1941Zuwachslehre,weck1948,wiedemann1951Ertragskunde,assmann1961Waldertraskunde,erteld1966Waldertragslehre,mitscherlich1975WaldWachstumUmwelt,kramer1988Waldwachstumslehre,wenk1990Waldertragslehre,Gadow1999Modelling,Gadov2003Waldstruktur,pretzsch2002Grundlagen,Pretzsch2003Modellierung,Vanclay2006,Weiskittel2011,Pretzsch2019}.
Hinweise auf Fehler oder Ergänzungen werden gerne entgegengenommen.

\section{Standort -- Bonität}\index{Bonität}

Die \emph{Wuchsleistung} eines Bestandes wird über die \emph{Bonität}
beschrieben. Die Bonität kann beispielsweise als (\textbullet)~Mittelhöhe der
100~stärksten Bäume oder als (\textbullet)~durchschnittlicher Gesamtzuwachs
(\{Holzvorrat im Wald + alle bisherigen Entnahmen bzw. Mortalität\} dividiert
durch Alter) im Alter~100 angegeben sein. Der Standort gibt die
Rahmenbedingungen des Wachstums vor und verändert sich laufend.
Standortsbedingungen wie Temperatur, Strahlung, Niederschlag, CO$_2$
Konzentration, Exposition, Hangneigung, Hanglage, Relief, Bodentyp, Bodenart,
Nähr-- oder Hemmstoffe sind kaum durch den einzelnen Waldbesitzer beeinflussbar.
Künstliche Bewässerung wird im Wald höchstens in der Verjüngungsphase
praktiziert und Düngung findet kaum statt. Insbesondere der Boden wird durch die
Bewirtschaftung beeinflusst. Durch (\textbullet)~befahren kann er verdichtet
werden und damit der Wasser und Lufthaushalt verändert, bei der
(\textbullet)~Ernte können Nährstoffe entzogen oder umverteilt werden, durch die
(\textbullet)~Baumartenwahl, Bestandesdichte und --struktur wird die
Durchwurzelungstiefe und --intensität, die Blattfläche, die Streumenge, der
Streuabbau, die Deposition, der Wasserverbrauch, \dots und damit der Wasser--
und Nährstoffhaushalt, die Aktivität und Zusammensetzung der Bodenlebewesen, die
Mykorrhiza, die Humusschicht, der pH--Wert, \dots beeinflusst. Insbesondere
durch stickstoffbindende Pflanzen (Leguminosen, Klee, Lupinie, Erle, Robinie,
\dots) kann die Wuchsleistung auf Standorten mit Stickstoffmangel verbessert
werden. Laufend gibt es Gesteinsverwitterung, Austräge (Erosion) und Einträge
(Deposition). Einträge können Nährstoffe aber auch Schadstoffe beinhalten.

Verschiedene Baumarten haben bei gleichem Standort meist unterschiedliche
Wuchsleistungen. Unter \emph{Wuchsleistung}\index{Wuchsleistung} versteht man
die Menge an Holz, die auf einer bestimmten Fläche (meist Hektar), in einem
bestimmten Zeitraum, zugewachsen ist. Neben Standort und Baumart beeinflussen
insbesondere der \hyperref[sec:AlterUndErntezeitpunkt]{Zeitraum} (Alter,
Umtriebszeit, Erntealter) und die
\hyperref[sec:StammzahlUndBestandesdichte]{Bestandesdichte} die Wuchsleistung.
Die Wuchsleistung wird üblicherweise in Vorratsfestmeter [Vfm/ha/Jahr],
Erntefestmeter (Vfm - Ernteverluste, meist 20-30\,\%) [Efm/ha/Jahr], Biomasse
[tBM/ha/Jahr] oder Kohlenstoff [tC/ha/Jahr] ausgedrückt. Bei zugewachsenem Holz
wird üblicherweise zwischen Schaftholz (Stamm), Derbholz (Stamm und Äste stärker
als z.\,B.\ 7\,cm Durchmesser) oder Gesamtbiomasse (Stamm, Äste, Blätter/Nadeln,
Rinde, Früchte, Wurzeln) unterschieden. Wenn Kosten (Verjüngung, Ernte, \dots)
und Verkaufspreise bekannt sind, kann die Wuchsleistung in eine monetäre
\emph{Ertragsleistung}\index{Ertragsleistung} [Euro/ha/Jahr] umgerechnet werden.

Die Herleitung der Wuchsleistung aus den vielfältigen Standortsbedingungen, die
oft schwierig oder nur mit großen Unsicherheiten erhoben werden können, ist
komplex und meist ungenau. Dennoch ist sie die einige Methode um die
Auswirkungen einer Standortsveränderung (z.\,B.\ Erhöhung der Temperatur um 3°C,
Verringerung der Niederschlagsmenge um 30\,\%) abschätzen zu können. Mit
Standortseigenschaften können erwartete Wuchsleistungen verschiedener Baumarten
im Vorhinein abgeschätzt und verglichen werden. Wächst eine Baumart bereits seit
einigen Jahren auf einem bestimmten Standort in einem (Rein)Bestand, kann deren
Wuchsleistung aufgrund des erzielten Bestandeszuwachses oder den erreichten
Baumgrößen klassifiziert werden. Für gleichaltrige Reinbestände wird derzeit oft
die Mittelhöhe der 100 stärksten Bäume (=\emph{Oberhöhe}\index{Oberhöhe})
verwendet. Mit dieser Oberhöhe und dem Bestandesalter wird eine erwartete Höhe,
bei einem Referenzalter (meist 100 Jahre), bestimmt. Die erwartete Oberhöhe im
Referenzalter wird als \emph{Oberhöhenbonität}\index{Oberhöhenbonität}
bezeichnet. Neben den Bäumen selbst können auch andere Pflanzen zur
Standortsbeschreibung und damit zur Abschätzung der zu erwartenden Wuchsleistung
verwendet werden. Mit den Pflanzen kann auf einen natürlichen Waldtyp und damit
auf eine naturnahe Baumartenzusammensetzung geschlossen werden. Mischbaumarten
können die Wuchsleistung und Gefährdungen gegenseitig sowohl positiv als auch
negativ beeinflussen.

\section{Höhe}

Der einzelne Baum wird im Laufe seines Lebens, solange kein Schaden (Verbiss,
Wipfelbruch) eintritt, stetig höher. Bei geneigt wachsenden Bäumen kann zwischen
Höhe und Schaft-- bzw.\ Stammlänge unterschieden werden. Bei manchen Baumarten
ist besonders im höheren Alter, durch die Abrundung der Krone, ein Endtrieb
bzw.\ die Wipfelknospe nicht zu identifizieren, was die Bestimmung der Baumhöhe
und deren Zuwachs erschwert. Bei manchen Baumarten kann der Höhenzuwachs der
letzten Jahre von außen, anhand von Astquirlen rekonstruiert werden.

Der \emph{Austriebszeitpunkt} und der anschließende Zuwachszeitraum ist abhängig
von Standort, Witterung, Baumart und Herkunft und kann über Schädigungen durch
Früh-- oder Spätfrösten mitentscheiden. Manche Bäume stellen innerhalb eines
Jahres ihren Höhenzuwachs ein, um ihn im selben Jahr wieder fortzusetzen
(Johannistrieb). Der Höhenzuwachs eines Jahres wird vom Standort, der Witterung
desselben als auch des Vorjahres, der Baumart, dem Alter bzw.\ der Baumgröße,
der Konkurrenzsituation sowie Schädigungen und Krankheiten beeinflusst. Neben
dem Höhenwachstum kann das Längenwachstum der Seitentriebe (Äste) bestimmt
werden.

Die Höhenentwicklung wird durch die benachbarten Bäume beeinflusst. Durch
\emph{Überschirmung} (Verjüngung unter Altbestand) wird die Höhenentwicklung
stark verzögert. Durch starke Einengung der Krone (hohe \emph{Bestandesdiche},
seitliche Konkurrenz) aber auch durch Fehlen jeglicher Konkurrenz kann es zu
einer Verringerung des Höhenzuwachses kommen. Unter forstlich üblichen
Bestandesdichten wird der Höhenzuwachs von herrschenden Bäumen relativ wenig
durch die Bestandesdichte verändert.

Die Höhenentwicklung eines einzelnen Baumes kann durch wiederholte
Baumhöhenmessungen, mit entsprechenden zeitlichem Abstand zwischen den Messungen
oder durch \emph{Stammanalysen}\index{Stammanalyse} erfasst werden. Bei
Stammanalysen werden bei gefällten Bäumen die Jahrringbreiten, meist in mehreren
Richtungen, in verschiedenen Baumhöhen gemessen und damit auch die Anzahl der
Jahrringe erfasst. Durch Auftragen von Höhe und Jahrringanzahl kann auf die
Höhenentwicklung rückgeschlossen werden. Da die Stammscheiben, an denen die
Jahrringe gezählt bzw.\ gemessen werden, meist nicht exakt in der Höhe der
damaligen Wipfelknopspe geworben werden, kommt es zu systematischen Differenzen
zwischen der wirklichen Baumhöhe und der Höhe, in der die Stammscheibe liegt.
Diese können eliminiert werden, wenn die Positionen der Wipfelknospen z.\,B.\
anhand von sichtbaren Asquirlen rekonstruierbar sind oder vom Alter als
durchschnittliche Abweichung ein halbes Jahr abgezogen wird. Auch können
Höhenentwicklungen kürzerer Zeiträume z.\,B.\ durch Rückmessung jüngerer
Jahrestrieblängen (Internodien) mithilfe von Astquirlen oder Knospenspuren bzw.\
Triebbasisnarben (Nodien),  von verschiedenen Bäumen oder Beständen der gleichen
Baumart, auf vergleichbaren Standorten, die zusammen eine lange
Altersentwicklung abdecken, zu einer Höhenentwicklung über dem Alter zusammen
gefasst werden (\emph{Wuchsreihe}\index{Wuchsreihe}, Abb.~\ref{fig:wuchsreihe}).
Dabei sollten sich die Altersabschnitte überlappen, um im Überlappungsbereich
die synchrone Entwicklung, der verschiedenen Bäume, überprüfen zu können.
Aufgrund der fehlenden Überprüfbarkeit einer synchronen Entwicklung von
verschiedenen Bäumen, Beständen oder Wäldern sind Beobachtungen zu lediglich
einem Zeitpunkt meist ungeeignet um eine Wuchsreihe zu entwickeln. Durch
Johannestriebe kann es zu Fehlern bei der Höhenzuwachsermittlung mittels
Astquirlen oder Triebbasisnarben (Knospenspuren) und durch Scheinjahrringe oder
ausgefallener Jahrringe zu Fehlern bei er Altersbestimmung kommen.

\begin{figure}[htbp]
  \centering
  \includegraphics[width=.95\columnwidth]{./pic/wuchsreiheHoehe}
  \caption{Ableitung der Höhenentwicklung über dem Alter mittels Wuchsreihe}
  \label{fig:wuchsreiheHoehe}
\end{figure}

In Abbildung~\ref{fig:hoeheAlter} ist eine idealisierte (keine jährlichen
Zuwachsschwankungen, keine Schädigungen) Höhenentwicklung über dem Alter in
Schwarz aufgetragen (Höhenkurve). Zusätzlich wird noch der \emph{laufende
Höhenzuwachs}\index{Zuwachs!laufender}, also der Höhenzuwachs der im jeweiligen
Alter erfolgt, und der \emph{durchschnittliche
Höhenzuwachs}\index{Zuwachs!durchschnittlicher} (Höhe dividiert durch Alter)
gezeigt. Zunächst nimmt der laufende Zuwachs stetig zu und die Höhenkurve wird
immer steiler, bis der maximale laufende Zuwachs erreicht wird. Zu diesem
Zeitpunkt hat die Höhenkurve einen Wendepunkt und verläuft ab da immer flacher.
Auch wenn nun der laufende Höhenzuwachs zurückgeht, liegt er noch einige Zeit
über dem durchschnittlichen Höhenzuwachs. Zu dem Zeitpunkt, wenn laufender und
durchschnittlicher Höhenzuwachs identisch sind und das Alter größer Null ist,
erreicht der durchschnittliche Höhenzuwachs sein Maximum. Zu diesem Zeitpunkt
geht die Tangente der Höhenkurve (Strich-Punkt-Punkt Linie) durch den Ursprung.
Ab diesem Zeitpunkt geht der durchschnittliche Höhenzuwachs zurück.

Neben der Darstellung der Zuwächse über dem Alter können diese auch über der
Höhe aufgetragen werden. Wenn über die Bonität der Höhenwachstumsgang bekannt
ist, kann bei herrschenden Bäumen alleine durch Messen ihrer Höhe auf das
\emph{ideelle Baumalter} (Alter, dass ein Baum dieser Höhe ohne Überschirmung
bei üblichen Bestandesdichten hat) und auf den zu Erwartenden laufenden
Höhenzuwachs herrschender Bäume geschlossen werden. Bei Bäumen deren
Entwicklung z.\,B.\ durch Überschirmung gehemmt wurde, passt deren Alter und
Höhe nicht mit jenen, die sich ständig ungehindert entwickeln konnten, überein.
Bei Beseitigung dieser Wuchshemmungsfaktoren, haben diese bisher unterdrückten
Bäume nicht die jährlichen Höhenzuwächse wie gleich alte, sondern in etwa wie
gleich große, ungehindert gewachsenen Bäume. Demnach eignen sich neben der
Standorts-- und Konkurrenzsituation die Baumdimensionen Höhe, Stammdurchmesser
und Kronengröße besser zum Abschätzen der laufenden Zuwächse als das Baumalter.

\begin{figure}[htbp]
  \centering
  \includegraphics[width=.95\columnwidth]{./pic/hoehenzuwachs}
  \caption{Höhenentwicklung über dem Alter}
  \label{fig:hoeheAlter}
\end{figure}

Die Höhenentwicklung kann auch für einen ganzen Bestand, genauso wie für den
Einzelbaum beschrieben werden. Dabei muss zunächst definiert werden, wie und für
welches Kollektiv die Bestandeshöhe bestimmt wird. Beispielsweise kann das
arithmetische Mittel aller Höhen, das mit dem Baumvolumen oder der Kreisfläche
gewichtete Höhenmittel (\emph{Loreysche Mittelhöhe}) oder nur für die
100~stärksten bzw.\ höchsten Bäume je Hektar eine Höhe errechnet werden. Im Zuge
der Bestandesentwicklung scheiden laufend Bäume aus und können damit die
Mittelhöhe des verbleibenden Bestandes beeinflussen. Scheiden kleine aus,
steigt, scheiden große aus fällt das Mittel. Zur Bestimmung der Bonität wird oft
die \emph{Oberhöhe}\index{Oberhöhe} verwendet, welche z.\,B.\ aus den
100~stärksten Bäumen je Hektar bestimmt wird. Ein Ziel von Oberhöhendefinitionen
war, das sie durch Eingriffe möglichst wenig verändert werden. Scheiden nur
kleinen Bäume aus, die nicht zum Kollektiv der Oberhöhenstämme gehören, ändert
sich diese Höhe nicht. Wiewohl Bestandesdichteänderungen immer noch einen
Einfluss auf den Höhenzuwachs der verbleibenden Bäume in den folgenden Jahren
haben können. Durch Eingriffe in der vorherrschenden Schicht (Hochdurchforstung,
Zielstärkennutzung) und der Entnahme von Bäumen des Oberhöhenkollektivs kann die
Oberhöhe und damit auch die Oberhöhenbonität verändert werden ohne dass sich die
reale Bonität verändert hat.

Meist steigern Pionier-- bzw.\ Lichtbaumarten (z.\,B.\ Lärche, Kiefer, Birke) in
ihrer Jugend sehr rasch ihren Höhenzuwachs und erreichen sehr früh einen
Höchstwert und fallen danach schnell ab, was zu einer deutlich gekrümmten
Höhenentwicklung führt (Frühkulminierer). Klimax-- bzw.\ Schatten ertragende
Baumarten (z.\,B. Buche, Tanne) steigern nur allmählich ihren Höhenzuwachs und
erreichen einen deutlich niedrigeren Höchstwert wesentlich später als Pioniere,
fallen aber langsamer ab, was zu einer gestreckten Höhenentwicklung führt
(Spätkulminierer) (Abb.~\ref{fig:hoeheFrueMittelSpaet}). Angelehnt an diese
Beobachtung wurde behauptet, dass Bäume, die in der Jugend überschirm waren und
dadurch ein langsames Jugendwachstum hatten, ein gestreckteres Höhenwachstum
haben und dadurch höhere Endhöhen erreichen können, als unüberschirm erwachsene,
was sich meines Wissens bis jetzt in realen Wachstumsversuchen nicht bestätigt
hat.

\begin{figure}[htbp]
  \centering
  \includegraphics[width=.95\columnwidth]{./pic/hoehenzuwachsFrueMittelSpaet}
  \caption{Höhenentwicklung früh, mittel und spät kulminierender Baumarten}
  \label{fig:hoeheFrueMittelSpaet}
\end{figure}

Die gleiche Baumart kulminiert auf Standorten mit hoher Bonität früher und auf
höherem Niveau als auf Standorten mit niederer Bonität
(Abb.~\ref{fig:hoeheAlterBonitaeten}). Kann eine früh kulminierende Lichtbaumart
einen Standort schlecht nutzen, hat also eine niedere Bonität und eine spät
kulminierende Schattbaumart eine hohe Bonität, wird der Vorsprung des
Höhenzuwachses, sofern überhaupt einer vorhanden ist, der Lichtbaumart schon
recht bald von der Schattbaumart wettgemacht. Sobald Schattbaumarten im selben
Bestand höher sind als Lichtbaumarten und die Bestandesdichte nicht extrem
niedrig ist, führt dies meist zum Ausfall der Lichtbaumart.

\begin{figure}[htbp]
  \centering
  \includegraphics[width=.95\columnwidth]{./pic/hoehenzuwachsBonitaet}
  \caption{Höhenentwicklung bei verschiedenen Bonitäten}
  \label{fig:hoeheAlterBonitaeten}
\end{figure}


\section{Durchmesser}

Ähnlich wie die Höhe, entwickelt sich auch der Baumdurchmesser bzw.\ dessen
\emph{Kreisfläche}\index{Kreisfläche} (Querschnittsfläche des Stammes, die bei
überwiegend einseitiger Belastung eher elliptisch ist, aber vereinfacht oft als
Kreis betrachtet wird) im Laufe der Zeit mit zunächst kleinen, sich bis zu einem
bestimmten Alter steigernden, Zuwächsen, welche danach wieder abnehmen. Dieser
Zuwachsgang ist abhängig von Baumart, Standort und Bestandesdichte. Baumarten
deren Höhenzuwachs früh kulminiert zeigen meist auch beim Durchmesserzuwachs
eine frühe Kulmination. Die Kulmination des Durchmesserzuwachses ist in der
Regel früher als die des Kreisflächenzuwachses, da der Kreisflächenzuwachs
sowohl von der Jahrringbreite als auch vom Durchmesser, auf dem sich der
Jahrring anlegt, abhängt. Der Durchmesserzuwachs legt sich entlang des gesamten
Stammes sowie den Ästen an. Dabei werden Jahrring an Jahrring von innen
(\emph{Mark}\index{Mark}: Zentrum der Stammscheibe um den sich die Jahrringe
anlegen) nach außen (\emph{Kambium}\index{Kambium}: Wachsstumsschicht zwischen
Holz und Rinde) und meist in geringerem Ausmaß Rindenschichten in umgekehrter
Schichtung (außen alt, innen jung) angelegt. Es ist bei uns üblich den
Durchmesser in 1,3\,m Höhe zu messen (meist keine oder wenig Verdickung durch
den Wurzelanlauf, kein Bücken beim Messen). Dieser Durchmesser wird als
\emph{Brusthöhendurchmesser}\index{Brusthöhendurchmesser|see{BHD}}
(BHD)\index{BHD (Brusthöhendurchmesser)} bezeichnet. Von Bäumen die noch nicht
1,3\,m hoch sind, kann kein BHD gemessen werden. Alternativ kann der
\emph{Wurzelhalsdurchmesser}\index{Wurzelhalsdurchmesser} (Übergang zwischen
Stamm und Wurzel) gemessen werden.

Der Durchmesserzuwachs erfolgt ungleichförmig innerhalb eines Jahres. In unseren
Breiten wird er im Winter eingestellt und beginnt je nach Witterung und Baumart
im Frühling. Bei \emph{ringporigern}\index{ringporig} Baumarten (z.\,B.\ Eiche,
Esche, Ulme, Robinie) wird zunächst helles, weitluminges, poren-- bzw.\
gefäßreiches Frühholz zur Leitung von Wasser mit geringer Holzdichte gebildet.
Im Laufe des Jahres wird das Holz dunkler, englumig, porenarm und dichter bis
der Zuwachs beendet wird. Bei \emph{zerstreutporigen}\index{zerstreutporig}
Baumarten (z.\,B.\ Ahorn, Hainbuche, Weide, Linde, Birke) sind die Poren bzw.\
Gefäße recht regelmäßig sowohl im Früh-- als auch im Spätholz vertreten und die
Jahrringgrezen weniger gut zu erkennen. Dazwischen gibt es noch die
\emph{halbringporigen}\index{halbringporig} Baumarten (z.\,B.\ Nuss, Kirsche).
Buche wurde früher den zerstreut-- und heute den halbringporigen zugeordnet.
Nadelhölzer haben keine Poren und werden daher keinem dieser Typen zugeordnet.
Ihre Jahrringgrenzen sind meist gut erkennbar.

Durch Wuchsunterbrechungen innerhalb der Vegetationszeit (Trockenheit,
Insektenfraß der Blätter) können \emph{Scheinjahrringe}\index{Scheinjahrringe}
bzw.\ \emph{Doppeljahrringe}\index{Doppeljahrringe} entstehen. In sehr
zuwachsschwachen Jahren können Jahrringe an Teilen des Stammes ausfallen. Beides
erschwert die Zuordnung der Jahrringe und deren Zuwächse auf bestimmte Jahre
sowie die Altersbestimmung.

Wird der Durchmesser zu den Zeitpunkten $t_0$ und $t_1$ gemessen, kann im
\emph{Zuwachszeitraum} $t_1 - t_0$ meist ein Durchmesserzuwachs $d_1 - d_0$
bzw.\ ein \emph{Kreisflächenzuwachs}\index{Kreisflächenzuwachs} $(d_1^2 -
d_0^2)\cdot \pi/4$ beobachtet werden. Dieser Zuwachs wird als \emph{periodischer
Zuwachs} bezeichnet. Wenn möglich, sollten die Messungen außerhalb des
Zeitraumes, in dem Zuwachs stattfindet, also nicht während der
\emph{Vegetationsperiode}\index{Vegetationsperiode} erfolgen und damit die
Periodenlänge in vollen Jahren angebbar sein. Wird der Zuwachs durch die
Periodenlänge dividiert, erhält man den durchschnittlichen jährlichen Zuwachs in
der Periode. Wird der jährliche Kreisflächenzuwachs aller Bäume eines Bestandes
durch die Bestandesfläche dividiert, erhält man den, in der Beobachtungsperiode
durchschnittlich pro Jahr geleisteten,
\emph{Grundflächenzuwachs}\index{Grundflächenzuwachs} [m²/ha/Jahr].

Kurz-- und mittelfristig kann insbesondere die Bestandesdichte mittels
Durchforstungen abgesenkt und damit in der Regel der Durchmesserzuwachs der
verbleibenden Bäume gesteigert werden. Bezogen auf den
Grundlfächenzuwachs/ha/Jahr, also die Zuwachsleistung je Fläche, gibt es eine
\emph{zuwachsoptimale Bestandesdichte}.
Abbildung~\ref{fig:zuwachsBestockungsgrad} zeigt mögliche Verläufe zwischen
natürlichem Bestockungsgrad und relativer Zuwachsleistung. Der \emph{natürliche
Bestockungsgrad}\index{Bestockungsgrad!natürlicher} gibt das Verhältnis von
aktueller zu maximal möglicher Bestandesdichte, in der Regel ausgedrückt in m²
Grundfläche je ha, an. Beim
\emph{Ertragstafelbestockungsgrad}\index{Bestockungsgrad!Ertragstafel--} wird
die Relation zur Grundflächenangabe einer Ertragstafel, und damit die Abweichung
zu der dort empfohlenen Bestandesdichte, angegeben. Die relative Zuwachsleistung
gibt das Verhältnis des Zuwachses bei maximaler Bestandesdichte zu Zuwachs bei
aktueller Bestandesdichte an, wobei meist der Grundflächenzuwachs/ha/Jahr,
Volumszuwachs/ha/Jahr oder Biomassenzuwachs/ha/Jahr betrachtet wird. Es gibt
Bestandessituationen, bei denen jede Absenkung der Bestandesdichte zu einem
Zuwachsverlust führt (gestrichelte Linie). Bei anderen hingegen kommt es bei
einer Absenkung der Bestandesdichte, zunächst zu einer Zuwachszunahme, um erst
danach abzunehmen (durchgezogene Linie). In beiden Situationen gibt es eine
zuwachsoptimale Bestandesdichte. In der einen Situation ist dies bei der maximal
möglichen, in der anderen bei einer geringeren Bestandesichte. Kleine Bäume mit
gut entwickelten Kronen auf zuwachsstarken Standorten zeigen eher ein Optimum
bei geringeren Dichten. Hingegen ältere Bestände mit großen Bäumen, Bestände mit
kurzen Kronen auf zuwachsschwachen Standorten haben den größten Zuwachs eher bei
der maximalen Bestandesdichte. Die Baumgröße hat einen wesentlicheren Einfluss
auf Zuwachs und Zuwachsreaktion als das Alter. Auch die Baumart hat neben
Bestandesstruktur, Baumartenmischung, \dots einen Einfluss auf die
Zuwachsreaktion bei verschiedenen Bestandesdichten. Bei sehr geringen
Bestandesdichten gibt es einen Punkt, ab dem eine weitere Dichtereduktion zu
keiner Zuwachssteigerung des Einzelbaumes führt, was auf Bestandesebene einen
linearen Zusammenhang zwischen Bestandesdichte und Zuwachs bedingt.

\begin{figure}[htbp]
  \centering
  \includegraphics[width=.95\columnwidth]{./pic/zuwachsBestockungsgrad}
  \caption{Abhängigkeit der Zuwachsleistung vom natürlichen Bestockungsgrad}
  \label{fig:zuwachsBestockungsgrad}
\end{figure}

Einen Bestand laufend auf der zuwachsoptimalen Bestandesdichte zu halten ist
praktisch nicht möglich. Vielmehr wird im Zuge von Durchforstungseingriffen die
Bestandesdichte soweit abgesenkt, wie es ein \emph{akzeptierter Zuwachsverlust}
erlaubt. In Abb.~\ref{fig:zuwachsBestockungsgrad} wurde angenommen, dass
3\,\%~Zuwachsverlust gegenüber dem \emph{vollbestockten Bestand} akzeptiert
werden. Dieser tolerierte Zuwachsverlust wurde durch eine horizontale graue
strichlierte Linie angedeutet. Beim Bestand mit Scheitelpunkt kann auf ca.\
60\,\% der Bestandesdichte abgesenkt werden, beim anderen nur auf ca.\ 80\,\%.
Bei dem Bestand mit Scheitelpunkt hätte man aber bei einer Bestandesdichte von
60\,\%, nicht 3\,\%, sondern mehr als 3\,\% Zuwachsverlust, wenn anstatt auf den
Zuwachs bei \emph{maximaler} Bestandesdichte, auf den Zuwachs bei
\emph{zuwachsoptimaler} Bestandesdichte bezogen wird. Bezogen auf das Optimum
wäre in dem Bereich zwischen den beiden weißen Punkten mit maximal 3\,\%
Zuwachsverlust zu rechnen.

Die Beziehung Bestandesdichte--Zuwachs ist nicht statisch, sondern verändert
sich im Lauf der Zeit. Die Art der Veränderung wird unter anderem auch durch die
Bestandesdichte mitbestimmt (Abb.~\ref{fig:zuwachsveraenderungBestockungsgrad}).
Wird in einem Jungbestand, dessen laufender Zuwachs noch nicht kulminiert hat,
in der Aussgangssituation eine Bestandesdichte nahe dem Zuwachsoptimum gewählt,
also bei relativ hoher Bestandesdichte, nimmt der Durchmesser der einzelnen
Bäume langsamer zu als bei einem Bestand, bei dem der Bestockungsgrad stärker
abgesenkt wurde. Der unmittelbare Zuwachs ist zwar bei dem Bestand mit hoher
Dichte höher als bei jenem mit geringer. Dadurch dass die Durchmesser der
einzelnen Bäume des Bestandes mit geringer Dichte schneller zunehmen, nimmt auch
deren Zuwachsleistung schneller zu als beim Bestand mit höherer Dichte, sodass
der Bestand mit geringerer Dichte nach wenigen Jahren eine höhere
Zuwachsleistung hat als jener mit hoher Dichte
(\emph{Wuchsbeschleunigungseffekt}\index{Wuchsbeschleunigungseffekt}).

\begin{figure}[htbp]
  \centering
  \includegraphics[width=.95\columnwidth]{./pic/zuwachsveraenderungBestockungsgrad}
  \caption{Veränderung der Beziehung Zuwachs Bestandesdichte}
  \label{fig:zuwachsveraenderungBestockungsgrad}
\end{figure}

Die Beziehung Bestandesdichte--Zuwachs spielt sich auf der Einzelbaumebene ab
und dort ist eine nahezu stufenlose Absenkung der Bestandesdichte nicht möglich.
Dort kann nur mindestens einen Nachbarbaum entnommen werden und damit sinkt die
Bestandesdichte stufenweise wie dies in Abb.~\ref{fig:bestandesdichteEinzelbaum}
schematisch angedeutet ist. Links oben (A) ist ein gleichmäßiger Dreiecksverband
als Ausgangssituation dargestellt. Wenn ein Nachbar entnommen wird (B)
vergrößert sich die Standfläche der Nachbarbäume um 1/6 d.\,h. wenn der
Bestockungsgrad eins war reduziert er sich auf 6/7 ($\sim 0.86$). Die dabei
entstehende \emph{Standfläche} ist \emph{asymmetrisch}, was in der Regel zu
Zuwachseinbußen, asymmetrischen Kronen, schiefem Wuchs und damit zu
Reaktionsholz führt. Dieses Ausweichen der Krone in frei werdende Bereiche kann
dazu führen, dass auch Bäume, die nicht unmittelbar an die entstandene Lücke
anschließen, ihre Standfläche ein wenig vergrößern können.

Man könnte in dem vereinfachten Beispiel zusätzlich den gegenüberliegenden
Nachbarn entnehmen, womit der Standflächenschwerpunkt wieder im Zentrum liegt.
Die Form wäre aber stark abweichend von der Idealform des Kreises. Ein
Kompromiss wäre die Entnahme von drei Nachbarn, wie es in
Abb.~\ref{fig:bestandesdichteEinzelbaum} links unten (C) dargestellt ist, wobei
sich die ursprüngliche Standfläche um 50\,\% erhöht. Eine wirklich
zufriedenstellende Verteilung und Standflächenform erreicht man nur, wenn
6~Nachbarn entnommen werden (D), was die Standfläche der verbleibenden Bäume
verdreifacht.

\begin{figure}[htbp]
  \centering
  \includegraphics[width=.95\columnwidth]{./pic/bestandesdichteEinzelbaum}
  \caption{Veränderung der Bestandesdichte durch Entnahme von Nachbarbäumen}
  \label{fig:bestandesdichteEinzelbaum}
\end{figure}

Die möglichen Bestandesdichtestufen sind in Abb.~\ref{fig:bestockungsgradstufen}
dargestellt wobei eine Verringerung des Zuwachses, aufgrund ungünstiger
Standflächen-- und Kronenformen sowie der, bei der Entnahme entstandenen,
Bestandeslücken, nicht berücksichtigt ist. Dabei spiel es eine Rolle wie hoch
die Bestandesdichte vor dem Eingriff war. Oben (A, B) wurde angenommen das die
maximal mögliche, unten (C, D) das 80\,\% der maximal möglichen Bestandesdichte
vor dem Eingriff geherrscht hat. Im linken Teil der Abbildung (A, C) ist die
Situation des Dreicksverbandes wie in Abb.~\ref{fig:bestandesdichteEinzelbaum}.A
links oben dargestellt und im rechten Teil der
Abb.~\ref{fig:bestockungsgradstufen} (B, D), wenn um einen Zentrumsbaum nur noch
drei Nachbarn vorhanden sind (Abb.~\ref{fig:bestandesdichteEinzelbaum}.C links
unten), dargestellt. Schlussendlich geht es bei einer Durchforstung meist um die
Entscheidung 0--6 ausgewählte Konkurrenten zu entnehmen. Eine Entnahme von
Bäumen die derzeit noch keine unmittelbaren Nachbarn sind, wird kurzfristig noch
keinen Effekt zeigen. Langfristig, insbesondere wenn erst nach längerer Zeit
oder gar keine weiteren Durchforstungen geplant ist, kann auch die Entnahme von
zukünftigen Konkurrenten erwogen werden. Da Durchforstungen neben dem Zuwachs
auch weitere Faktoren wie beispielsweise die Bestandesstabilität oder die
Holzqualität beeinflussen, müssen auch diese in den Entscheidungsfindungsprozess
einfließen.

\begin{figure}[htbp]
  \centering
  \includegraphics[width=.95\columnwidth]{./pic/bestockungsgradstufen}
  \caption{Bestockungsgradstufen}
  \label{fig:bestockungsgradstufen}
\end{figure}

Bei der Betrachtung von Zuwachs und Bestandesdichte darf nicht übersehen werden,
dass mit der Bestandesdichte auch das Kronenvolumen der Bäume gesteuert wird und
damit die zukünftige Zuwachsleistung bzw.\ die zukünftige Form der Beziehung
zwischen Bestandesdichte und Zuwachs bestimmt wird, ob es eine Kurve mit oder
ohne Scheitel ist und ab welche Bestandesdichte mit bedeutendem Zuwachsrückgang
zu rechnen ist. Der Baumdurchmesser hat einen großen Einfluss auf die
Zuwachsleistung und auf die Beziehung Zuwachs--Bestandesdichte. Zunächst steigt
die Zuwachsleistung mit Zunahme des BHD's. Eine Reduktion der Bestandesdichte
führt auf Bestandesebene kurzfristig zu einer Reduktion der Zuwachsleistung.
Hingegen nimmt der Zuwachs der verbleibenden Einzelbäume mit einer Reduktion der
Bestandesdichte zu. Wobei es in den ersten Jahren zu einem Freistellungsschock
kommen kann und beispielsweise Schattenblätter erst allmählich durch
Lichtblätter ersetzt werden müssen. Mit der schnelleren Durchmesserzunahme geht
eine schnellere Zunahme der Zuwachsleistung Hand in Hand
(\emph{Wuchsbeschleunigungseffekt}\index{Wuchsbeschleunigungseffekt}). Ab einem
gewissen Baumdurchmesser nimmt jedoch die Zuwachsleistung wieder ab. Zu Beginn
sind geringe Bestandesdichten gut, da sie den BHD--Zuwachs steigern und die
Kronen schneller an Volumen zunehmen. Jedoch ab einem bestimmten BHD, ab einer
bestimmen Kronegrösse sollten diese nicht zu schnell steigen, da ansonsten die
Zuwachsleistung rascher wieder absinkt. Hier hilft zum einen eine hohe
Bestandesdichte als auch die Entnahme der stärkeren Bäume
(\emph{Plenterdurchforstung}\index{Durchforstung!Plenter--}). Dies gilt aber nur
solange dadurch zum einen nicht die Baumkronen zu stark verkleinert und von der
Veranlagung starkwüchsige Bäume zugunsten schwachwüchsiger entnommen werden. Die
\emph{gestaffelte Durchforstung}\index{Durchforstung!gestaffelte} nutzt diese
Zusammenhänge ideal aus und führt in der Jugend starke Eingriffe in der
herrschenden Schicht, geringen niederdurchforstungsartigen Entnahmen im
mittleren Alter, sowie keine geplanten Entnahmen ab etwa der halben
Umtriebszeit, durch.

Da Altbestände weniger auf Druchforstungen reagieren können (geringerer
Höhenzuwachs), bzw.\ dort Entnahmen, aufgrund der Größe der entnommenen Bäume,
wesentlich größere Lücken hinterlassen und es länger dauert bis diese
geschlossen werden können, ist es üblich ab etwa der Hälfte der geplanten
Umtriebszeit keine weiteren Durchforstungen zu machen. Bei der letzten
Durchforstung sollte die Baumverteilung möglichst einem gleichmäßigen
Dreiecksverband, mit sechseckigen Standflächen, nahe kommen und die
Bestandesdichte so weit abgesenkt werden, dass bei Erreichen des Erntealters die
maximal mögliche Bestandesdichte noch nicht erreicht wird. Eine Bestandesdichte
nahe der maximal Möglichen und eine Baumverteilung wie in
Abb.~\ref{fig:bestandesdichteEinzelbaum} links unten (C) ist vor der letzten
Durchforstung zu begrüßen, da hier nach der Entnahme von drei Nachbarn ein
Dreiecksverband entsteht und der natürliche Bestockungsgrad auf ca.\ 0.5
abgesenkt wird. Dieser starke Eingriff führt zwar auf den ersten Blick zu einem
Zuwachsverlust, erlaubt es jedoch den verbleibenden Bäumen relativ große, vitale
Kronen zu entwickeln, die bei gleicher Bestandesdichte höhere Zuwächse erbringt
als Bäume mit unterentwickelten Kronen. Nur eine geringe Bestandesdichte nach
der letzten Durchforstung gewährleistet, dass die maximal mögliche
Bestandesdichte nicht zu früh erreicht wird und es nicht zu konkurrenzbedingter
Mortalität kommt. Eingriffe knapp vor Erreichen der geplanten Umtriebszeit
können beabsichtigt sein um z.\,B.\ eine Naturverjüngung unter Schirm oder am
Bestandessaum einzuleiten.

In dem Zeitraum kurz nach einem Eingriff kommt es so gut wie immer zu einem
Zuwachsrückgang, da sich die Blätter und Nadeln der verbleibenden Bäume erst an
die geänderte Situation anpassen müssen und Zeit benötigt wird, um die
entstandenen Lücken zu schließen. Auch sind Bäume eines realen Bestandes nie
gleich groß. Die weit verbreitete Regel: >>Entnahme der 2--3 stärksten Bedränger
eines Z(ukunfts)--Baumes<< kann für Auslesedurchforstungen empfohlen werden.
Bäume die keine bzw.\ nur eine geringe Konkurrenz ausüben und bei einer Entnahme
nicht Kostendeckend sind oder sonstige nachteilige Wirkungen ausüben, werden
nicht entnommen.

Der Zuwachs ist entlang des Stammes verschieden. Bei freistehenden Bäume ist der
Durchmesserzuwachs am Stammfuß am größten und nimmt nach obenhin ab
(Abb.~\ref{fig:freistellung} rechts). Bei Bäumen im Bestand, unter seitlicher
Konkurrenz, nimmt der Durchmesserzuwachs vom Stammfuß aufwärts zunächst ab, bis
ein Minimum erreicht wird, um danach wieder zuzunehmen, bis ein Maximum nahe des
Wipfels erreicht wird (Abb.~\ref{fig:freistellung} links - vor der Freistellung.
Abb.~\ref{fig:stammanalyse}). Wird ein Baum freigestellt so verlagert er
allmählich den Zuwachs vermehrt in Richtung Stammfuß
(Abb.~\ref{fig:freistellung} links - nach der Freistellung). Diese Verlagerung
könnte zum Teil durch Veränderungen der mechanischen Belastungen des Stammes
verursacht werden. Die Verschiebung der Zuwächse in Richtung Stammfuß bei
Freistellung ist bei Buche und Eiche weniger stark ausgeprägt als bei Fichte,
Tanne Kiefer oder Lärche. Zu einer ähnlichen Verlagerung Richtung Stammfuß kommt
es auch bei Düngung sowie bei mitherrschenden und unterdrückten Bäumen in
Trockenjahren.

\begin{figure}[htbp]
  \centering
  \includegraphics[width=.95\columnwidth]{./pic/freistellung}
  \caption{Zuwachsverteilung nach Freistellung (links) und eines Solitärs (rechts)}
  \label{fig:freistellung}
\end{figure}

Neben dem Durchmesserzuwachs kann auch die Verteilung des Kreisflächenzuwachses
entlang des Stammes betrachtet werden. Dieser ist sowohl von der Jahrringbreite
als auch dem Durchmesser, an dem der Jahrring angelegt wird, bestimmt. Bei
Bäumen, die bei sehr hohen Bestandesdichten wachsen, kann selbst der
Kreisflächenzuwachs in Richtung Wipfel leicht zunehmen
(Abb.~\ref{fig:stammanalyse}).

\begin{figure}[htbp]
  \centering
  \includegraphics[width=.95\columnwidth]{./pic/stammanalyse}
  \caption{Stammanalyse: Zuwachsverteilung entlang des Stammes eines älter
  werdenden Baumes}
  \footnotesize{Ganz links (Schwarz): Stammaufriss mit gleicher Skalierung für Durchmesser und Höhe. Daneben: Entwicklung des Stammquerschnitts im Laufe der Zeit. $\Delta$d (Durchmesserzuwach) und $\Delta$g (Keisflächenzuwachs) entlang der Stammachse eines älter werdenden Baumes.}
  \label{fig:stammanalyse}
\end{figure}

Die Verteilung der Durchmesserzuwächse entlang des Stammes bestimmt die
Schaftform. Eine einfache Funktion zur Beschreibung der Schaftform stellt die
\emph{Schaftkurvenfunktion}\index{Schaftkurve} $d_x = c_0 \cdot (h - x)^{c_1}$
dar. Mit ihr kann der Schaftdurchmesser $d_x$ in der Höhe $x$ annähernd
errechnet werden. Dabei kann der Verlauf durch eine Position am Stamm gezwungen
werden, sodass z.\,B.\ der errechnete Wert in Höhe des BHD's exakt mit diesem
übereinstimmt und zusätzlich das Integral der Kurve das Schaftholzvolumen
ergibt. Je nach Fragestellung könnte es jedoch wichtiger sein, nicht einen
bestimmten Durchmesser exakt reproduzieren zu können, sondern die Abweichung zu
möglichst allen bzw.\ den meisten Durchmessern gering zu halten. Zusätzlich
könnte eine gewisse Abweichung zwischen beobachteten und über die Schaftkurve
errechnetem Volumen, etwa aufgrund von Ignorieren des Wurzelanlaufes, akzeptiert
werden (Abb.~\ref{fig:schaftkurve}).

\begin{figure}[htbp]
  \centering
  \includegraphics[width=.95\columnwidth]{./pic/schaftkurve}
  \caption{Schaftkurve}
  \label{fig:schaftkurve}
\end{figure}

Die Entwicklung des BHD'S der Kreisfläche, der Höhe und des Volumens des in
Abb.~\ref{fig:stammanalyse} gezeigten Baumes ist in
Abb.~\ref{fig:stammanalyseDGHV} zu sehen. Der laufende \emph{Durchmesserzuwachs}
(BHD) erreicht seinen größten Wert kurz nach dem Zeitpunkt bei dem der Baum die
Höhe von 1,3\,m erreicht hat. Da aber in den ersten Jahren der BHD--Zuwachs null
war, liegt der Zeitpunkt mit dem größten durchschnittlichen BHD--Zuwachs etwas
nach dem Zeitpunkt, bei den die Höhe von 1,3\,m erreicht wird. Der laufende
\emph{Kreisflächenzuwachs} stieg bis zum Alter~30 an und fiel danach bis zum
Alter~60 langsam um im Alter~60 abrupt abzufallen und danach relativ konstant zu
bleiben. Der \emph{Höhenzuwachs} blieb bei diesem Baum im Alter~15 bis 40~Jahre
annähernd konstant und fiel danach abrupt ab um in weiterer Folge allmählich
weiter abzusinken. Der laufende \emph{Volumszuwachs} ist, aufgrund der geringen
Baumgröße, in den ersten Jahren sehr gering und steigt ab dem Alter~15 annähernd
linear bis zum Alter~35 an. Danach hat das Muster große Ähnlichkeit mit dem des
Kreisflächenzuwachses. Der Kulminationszeitpunkt des durchschnittlichen
Zuwachses ist meist beim BHD, am frühesten, gefolgt von Höhe, Kreisfläche und
wird abgeschlossen vom Volumen.

\begin{figure}[htbp]
  \centering
  \includegraphics[width=.95\columnwidth]{./pic/stammanalyseDGHV}
  \caption{Jählriche Entwicklung von BHD, Höhe, Kreisfläche in BHD--Höhe und
  Schaftvolumen des in Abb.~\ref{fig:stammanalyse} gezeigten Baumes.}
  \label{fig:stammanalyseDGHV}
\end{figure}

\section{Volumen und Biomasse}

Üblicherweise wird unter Volumen das \emph{Schaftvolumen} (Stamm) oder das
\emph{Derbholzvolumen} (bei uns in der Regel Stamm und Äste ab 7\,cm
Durchmesser) gemeint. Im weiteren Sinn kann aber auch Astvolumen und
Wurzelvolumen und bei Betrachtung von Biomasse zusätzlich noch Blatt-- bzw.\
Nadelmasse sowie Früchte beinhaltet sein.

Die Veränderung des Schaftvolumens wird durch den Durchmesser-- und Höhenzuwachs
bestimmt, wobei die Größe des Durchmesserzuwachses entlang der Stammachse
variabel ist. Bäume mit großer Standfläche und geringer Konkurrenz legen
vermehrt im unteren Stammbereich, Bäume mit kleiner Standfläche vermehrt im
oberen Stammbereich ihren Holzzuwachs an. Bäume die bei hohen Bestandesdichten
wachsen haben vollholzigere Stämme (Durchmesserabnahme nach oben hin ist
gering). Dies hat z.B. Vorteile beim Einschnitt im Sägewerk, da beim Besäumen
(wegschneiden der Waldkante) weniger Verschnitt anfällt. Umgekehrt sind
abholzige Bäume meist stabiler gegen Windwurf und Schneebruch.

Multipliziert man eine Stammquerschnittsfläche mit der Baumhöhe erhält man ein
Walzenvolumen. Wird dieses Walzenvolumen mit der dazugehörigen
\emph{Schaftholz--Formzahl}\index{Formzahl!Schaftholz--} multipliziert, erhält
man das Schaftvolumen. Zur Berechnung des Derbholzvolumens wird mit der, dem
entsprechenden Durchmesser zugehörigen,
\emph{Derbholz--Formzahl}\index{Formzahl!Derbholz--} multipliziert.

In Abb.~\ref{fig:stammanalyseDfz}--oben sind die Durchmesserentwicklungen von
BHD (Messhöhe 1,3\,m), d$_{03}$ (Messhöhe 0,3 $\times$ Baumhöhe), d$_f$
(Durchmesser dessen Querschnittsfläche $\times$ Baumhöhe das Schaftholzvolumen
ergibt) und d$_{00e}$ (errechnetter Durchmesser in Bodenhöhe, hier Schaftform
mit der Funktion $d_x = c_0 \cdot (h - x)^{c_1}$ angepasst und Durchmesser in
Höhe 0 errechnet) sowie deren relative Messhöhe am Stamm dargestellt. Da der BHD
erst für Bäume ab einer Höhe von 1,3\,m Werte über Null annehmen kann, können
mit ihm keine Volumina für kleine Bäume bestimmt werden. In
Abb.~\ref{fig:stammanalyseDfz}--unten sind der h/d--Wert, welche ein Maß für die
Stabilität des Baumes ist sowie die Entwicklung der Formzahl über dem Alter
dargestellt.

\begin{figure}[htbp]
  \centering
  \includegraphics[width=.95\columnwidth]{./pic/stammanalyseDfz}
  \caption{Verschiedene Durchmesser des in Abb.\,\ref{fig:stammanalyse}
  dargestellten Baumes}
  \label{fig:stammanalyseDfz}
\end{figure}

Die Veränderung der Formzahl über der Zeit gibt einen Hinweis, ob der Stamm
vollholziger oder abholziger wird. Vereinfacht wird der Zuwachs des einzelnen
Baumes mit der Zunahme von Baumhöhe und BHD sowie der Veränderung der Formzahl
beschrieben. Die Veränderung der Formzahl wird durch den Unterschied des
Durchmesserzuwachses, auf den sich die Formzahl bezieht (BHD, $d_{03}$, \dots)
und dem Durchmesserzuwachs entlang des Stammes, sowie durch den Höhenzuwachs
bedingt. Damit wird die Formzahl von der relativen Veränderung des Durchmessers
an einer bestimmten Stelle am Stamms recht stark mitbestimmt. Außer im Fall des
Durchmessers am Stammfuß verändert sich im Lauf der Zeit entweder die relative
oder die absolute Lage dieses Bezugsdurchmessers.

Durch ausgleich der Schaftform mit der Funktion $d_x = c_0 \cdot (h - x)^{c_1}$
lässt sich der Durchmesser $d_x$ für einen Baum mit der Höhen $h$ in der Höhe
$x$ annähernd bestimmen. Der Koeffizient $c_1$ beschreibt die Form des Stammes.
Die Funktion kann zwar so bestimmt werden, dass sie einen bestimmten Durchmesser
(z.\,B.\ BHD) und das Volumen ohne Abweichung wiedergeben kann. Damit bekommt
aber wieder ein Durchmeser in einer bestimmten Höhe relativ großen Einfluß auf
den Formexponenten. Ein Ausgleich mit dem Ziel über einen Großteil der
Schaftlänge nahe an der realen Schaftkurve zu liegen dürfte eher in der Lage
sein eine Veränderung der Sachftform beschreiben zu können. Dabei dürfte die
Bedingung, das die Schaftkurve ein mit der realen Schaftform identes Volumen
hat, wenig Einfluss haben. Abb.~\ref{fig:stammanalyseForm} zeigt, dass die
Stammform bis zum Alter~50 immer vollholziger wurde, danach ein wenig an
Vollholzigkeit verlor und anschließend die Form kaum verändert hat. Wenn die
Schaftformkurve durch den BHD gezwungen wurde, scheint es, dass der Stamm ab dem
Alter~50 laufend an Vollholzigkeit einbüßte, was aber in dem Fall eher auf dem
Umstand zurückzuführen ist, dass der BHD immer stärker vom Wurzelanlauf
beeinflusst wurde und dieser größere Jahrringbreiten hatte als der Überwiegende
Teil des Stammes.

\begin{figure}[htbp]
  \centering
  \includegraphics[width=.95\columnwidth]{./pic/stammanalyseForm}
  \caption{Schaftformveränderung}
  \label{fig:stammanalyseForm}
\end{figure}

Bäume mit ständig großer Standfläche haben große Kronen mit vielen Ästen und
großer Blatt-- bzw.\ Nadelmasse sowie großer Wurzelmasse und auch entsprechend
höhere Zuwächse. Große Ausdurchmesser verursachen geringer Holzqualität.
Andererseits sind großkronige Bäume in der regel vitaler und können
beispielsweise einen Borkenkäferbefall besser abwehren.

Um auf eine Flächenbezogene Zuwachsleistung zu kommen, kann entweder für den
Einzelbaum dessen Zuwachs auf seine Standfläche, oder der Zuwachs aller Bäume
eines Bestandes auf die Bestandesfläche, bezogen werden. Dabei zeigt sich
ähnlich wie beim Durchmesserzuwachs, dass es einen Scheitelpunkt der
Zuwachsleistung über der Bestandesdichte geben kann, bei dem die aktuelle
Zuwachsleistung am größten ausfällt. Aber auch hier gilt, dass die aktuelle
Bestandesdichte das zukünftige Zuwachspotential beeinflusst, und eine aktuell
geringere Zuwachsleistung bei geringer Bestandesdichte zu einem späteren
Zeitpunkt höhere Zuwachsleistungen hervorrufen kann.

Zur Bestimmung von Biomassen wird meist das Holzvolumen mit einer
\emph{Holzdichte}\index{Holzdichte} multipliziert. Da Holz in Abhängigkeit der
Holzfeuchte sein Volumen verändert, muss sich die verwendete Holzdichte auf das
entsprechende Holzvolumen beziehen. Wurde das Holzvolumen im absolut trockenem
Zustand (darrtrocken u=0\%) wird die
\emph{Darrdichte}\index{Darrdichte|see{Holzdichte}}\index{Holzdichte!Darrdichte}
verwendet. Wurde das Holzvolumen waldfrisch (Holzfeuchtigkeit u>30\%) bestimmt
wird die
\emph{Raumdichte}\index{Raumdichte|see{Holzdichte}}\index{Holzdichte!Raumdichte}
verwendet.
\emph{Rohdichten}\index{Rohdichte|see{Holzdichte}}\index{Holzdichte!Rohdichte}
werden für bestimmte Holzfeuchtigkeiten angegeben und sind bei u=0\% identisch
mit der Darrdichte und bei ca.\ u>30\% identisch mit der Raumdichte. Die
Holzdichte wird unter anderem von der Jahrringbreite beeinflusst (Abb.\
\ref{fig:holzdichteJahrringbreite}). Fichten von wärmeren Standorten (höhere
Wassersaugspannungen in den Holzzellen) haben bei gleiche Jahrringbreite höhere
Holzdichten als Bäume von kälteren Standorten.

\begin{figure}[htbp]
  \centering
  \includegraphics[width=.95\columnwidth]{./pic/holzdichteJahrringbreite}
  \caption{Zusammenhang Holzdichte und Jahrringbreite}
  \label{fig:holzdichteJahrringbreite}
\end{figure}


\section{Alter und Erntezeitpunkt}
\label{sec:AlterUndErntezeitpunkt}

Die Baumgröße hat in der Regel einen größeren Einfluss auf den laufenden Zuwachs
als das Alter. Zur Bestimmung einer Zuwachsleistung wird durch die Zeitspanne,
in der der Zuwachs erfolgte, dividiert. Eine gebräuchliche Kennzahl ist der
durchschnittliche Gesamtzuwachs (DGZ). Hierbei wird der bis zu einem bestimmten
Alter geleistete Gesamtzuwachs (= stehender Vorrat + Summe des bisher
ausgeschiedenen Vorrats) je Hektar durch das Alter dividiert. Wenn man den DGZ
für jedes Alter bestimmt, findet man ein Alter, bei dem dieser Wert am größten
ist. Bei dem Ziel der Zuwachsmaximierung wäre der ideale Erntezeitpunkt jenes
Alter bei dem der DGZ ein Maximum erreicht. Da der Kurvenverlauf zu seiner
Kulmination relativ flach verläuft, führt eine Verschiebung diese Zeitpunks um
einige Jahre bis Jahrzehnte nur zu geringen Verringerungen der
durchschnittlichen Zuwachsleistung.

Der Verlauf des DGZ über dem Alter wird von den durchgeführten Durchforstungen
mitbestimmt. Geringe Bestandesdichten führen zu einer rascheren Zunahme des
BHD's und damit zu einem früheren Kulminationszeitpunkt des DGZ. In der Jugend
stark durchforstete Bestände füllen den ihnen zur Verfügung gestellten Raum
relativ früh mit ihren Kronen aus, haben aber durch die geringe Stammzahl auch
eine geringe kokurrenzbedingte Mortalität. Es kommt somit auch im höheren Alter
zu geringen Ausfällen und damit nur zu wenigen unproduktiven Bestandeslücken die
im Alter relativ lange Zeiträume benötigen, um sich wieder zu schließen. Damit
fällt die Zuwachsleistung mit zunehmendem Alter relativ gering ab und erlaubt
den Erntezeitpunkt relativ lange nach hinten zu verschieben ohne mit größeren
Zuwachseinbußen rechnen zu müssen.

Auf Betriebsebene ergibt sich durch die Größe der unterschiedlich alten Bestände
eine Altersverteilung der Betriebsfläche. Bei Annahme von gleicher Wüchsigkeit
und gleicher Baumartenzusammensetzung der einzelnen Bestände wäre eine
Gleichverteilung der Bestandesflächen je Altersklasse bis zum Alter der
geplanten Umtriebszeit anzustreben (Normalwaldmodell) um jedes Jahr die gleiche
Nutzungsmenge bei gleichbleibendem Zuwachs und Vorrat erhalten zu können. Bei
Beständen mit unterschiedlicher Bonität und unterschiedlicher
Baumartenzusammensetzung ergeben sich meist unterschiedliche Zielumtriebszeiten.
Falls die Fläche der Wälder je Zielumtriebszeit zu klein sind um eine eigene
Betriebsklasse zu bilden, muss entweder die Erntemenge und Nutzungsfläche in
einzelnen Jahren variieren oder es muss flexibel mit der Zielumtriebszeit
umgegangen werden.

\section{Zufallsnutzung}

Bei geplanten Vor-- und Endnutzungen sowie Untriebszeiten sollten
Zufallsnutzungen (ungeplante Ausfälle), sofern diese einigermaßen bekannt sind,
berücksichtigt werden. In der Regel ist es nicht das Ziel das Niveau der
zufälligen Ausfälle möglichst niedrig zu halten. Wer das Ziel hat möglichst hohe
Holzmengen zu ernten, kann dies mit einer Baumart mit einer hohen
Zuwachsleistung aber auch höherem Ausfallsrisiko eventuell besser erreichen als
mit einer schwachwüchsigen Baumart mit sehr geringem Ausfallsrisiko.
Zufallsnutzungen verlangen oft eine rasche Aufarbeitung um Folgeschäden
abzuwenden. Damit binden sie Arbeitskräfte und können dazu führen das geplante
Durchforstungen und Endnutzungen aufgeschoben werden, was wiederum Auswirkungen
auf deren Leistung hat. Am einfachen Beispiel der geplanten Umtriebszeit führt
eine Zufallsnutzung zum einen zu einer verfrühten Nutzung und kann durch die
Bindung der Arbeitskraft bei den verbleibenden Beständen zu einer verspäteten
Nutzung führen. Die Integration dieser Einflussfaktoren in der Planung wird
dadurch Erschwert, dass zum Einen das Niveau der Zufallsnutzungen nicht exakt
bekannt ist und dass dieses zwischen einzelnen Jahren stark variieren kann.

\section{Stammzahl und Bestandesdichte}
\label{sec:StammzahlUndBestandesdichte}

Es gibt eine maximale Stammzahl, Grundfläche, Volumen bzw.\ Biomasse je
Flächeneinheit. Diese Obergrenze ist abhängig von der Baumart, Baumgröße sowie
Standortseigenschaften. Auf Einzelbaumebene ist dies der minimale Wuchsraum
bzw.\ die minimale Standfläche die ein Baum zum Überleben benötigt. Wird ein
Baum größer, nimmt auch diese minimale Standfläche zu. Oft wird der Zusammenhang
zwischen Stammzahl und Stammdurchmesser, wenn beide logarithmiert werden, als
linear angenommen (Abb.~\ref{fig:bestandesdichteDN}). Dabei wird in der Regel
der BHD als Stammdurchmesser verwendet, was insbesondere bei kleinen Bäumen
problematisch wird. Augenfällig wird dies bei Bäumen die eine Höhe von 1,3\,m
noch nicht erreicht haben, damit einen BHD von 0 haben und ihnen damit eine
hypothetisch mögliche Stammzahl von unendlich unterstellt wird, was in Realität
selbstverständlich nicht möglich ist. Wird hingegen der Durchmesser dort
angenommen, wo der Samen zu keimen beginnt, wird diese Problematik zumindest
abgemildert. Eine andere (bessere) Möglichkeit besteht darin die maximal
mögliche Grundfläche in Abhängigkeit von der (Ober)--Höhe zu beschreiben, wobei
beim Logarithmieren beider der Zusammenhang meist als linearer angenommen wird
(Abb.~\ref{fig:bestandesdichteHG}).

\begin{figure}[htbp]
  \centering
  \includegraphics[width=.95\columnwidth]{./pic/bestandesdichteDN}
  \caption{Maximale Stammzahl bei erreichtem Stammdurchmesser}
  \label{fig:bestandesdichteDN}
\end{figure}

\begin{figure}[htbp]
  \centering
  \includegraphics[width=.95\columnwidth]{./pic/bestandesdichteHG}
  \caption{Maximale Grundfläche bei erreichte (Ober)--Höhe}
  \label{fig:bestandesdichteHG}
\end{figure}

Die Gleichmäßigkeit der Baumverteilung bzw.\ die möglichst kreisförmige
Standfläche bzw.\ Kronenprojektiosfläche, welche zur Erzielung hoher
Bestandesdichten dienlich sein dürfte, kann nur mit sehr starken Eingriffen, als
nur in Ausnahmen, über die Bestandesentwicklung ständig erhalten werden. Eine
Entwicklung ständig in der Nähe der maximal möglichen Bestandesdichte ist nur
mit inhomogenen Beständen denkbar. Auf Baumebene kann eben nur ein ganzer
Nachbar entnommen werden und damit die Bestandesdichte nur stufenweise, aber
nicht kontinuierlich, abgesenkt werden (Abb.~\ref{fig:bestandesdichteDN}). Dabei
wird die Bestandesentwicklung einen Einfluss auf die maximale Stammzahl bei
erreichtem Durchmesser haben. Ein Bestand der mit 10\,000~Bäumen die maximale
Bestandesdichte erreicht und dann Stufenweise immer wieder das neue Maximum
erreicht, wird bei einer maximalen Stammzahl von 1500~Bäumen einen geringeren
Durchmesser aufweisen als ein Bestand der mit 1500~Bäumen begründet wurde und
das erste Mal seine Maximaldichte erreicht.

Eine Kombination von Durchmesser, Höhe, Volumen, Kroendimensionen,
Bestandesstruktur, Standort, \dots könnte zur Bestimmung der maximalen
Bestandesdichte (Stammzahl, Grundfläche, Volumen, Biomasse) nötig sein. Da die
Bestimmung der Standfläche für Einzelbäume noch nicht eindeutig gelöst ist,
werden üblicherweise Bestände oder Teile dieser, mit bekannter Fläche
betrachtet. Summen wie Stammzahl, Grundfläche oder Volumen lassen sich für diese
Flächen berechnen. Wie die dazugehörigen (Mittel)--Werte von Durchmesser und
Höhe zu bestimmen sind ist zwar klar definiert, aber ob beispielsweise der
Grundflächenmittelstamm, also das quadratische Mittel des Durchmessers, oder das
Mittel aus $d^\text{Geradenanstieg}$ also beispielsweise für einen Anstieg von
$-1.6$ mit $\left(\frac{\sum d^{1.6}}{n}\right)^{1/1.6}$ bestimmt besser
geeignet wäre, ist erst in wenigen Arbeiten untersucht worden. Generell kann
gesagt werden, dass Bestände aus möglichst ähnlichen, gleichmäßig verteilten
Bäumen für die Ermittlung maximal möglicher Bestandesdichten unproblematischer
verwendet werden können.

\section{Baumartenmischung}

Die Nachbarbäume beeinflussen das Wachstum des betrachteten Baumes. Es ist zu
erwarten das dabei auch die Baumart der Nachbarbäume eine Rolle spiel. Der
einfachste Fall liegt vor, wenn untersuchter Baum und Nachbarbäume derselben
Baumart angehören. Komplexer wird es, wenn sich diese unterscheiden, wobei nicht
nur 2, sondern auch mehrere Baumarten beteiligt sein können und unterschiedlich
miteinander interagieren. Neben der Baumart spiel nach wie vor auch deren
Baumgrößen, Bestandesdichten und damit deren Konkurrenzirkung eine Rolle und
dies wird durch Standortsfaktoren überprägt. Eine große Bedeutung hat dabei wie
die Baumarten gemischt sind, ob sie sich lediglich entlang einer Bestandesgrenze
beeinflussen bis hin zur Einzelmischung.

Mischungen von verschiedenen Baumarten können die Zuwachsleistung steigern aber
auch verringern. Generell kann man die Produktivität von Beständen aus
Lichtbaumarten (Baumarten die Beschattung kaum ertragen), durch Einbringen von
Schattbaumarten (Baumarten die Beschattung gut ertragen, aber auch ohne
Beschattung gut wachsen) im Unterbestand steigern. Für die Beurteilung, ob ein
konkreter Mischbestand aus zwei Baumarten eine höhere oder eine geringere
Zuwachsleistung als zwei Reinbestände der gleichen Baumarten hat sollte
zumindest der Anteil der beiden Baumarten im Mischbestand bestimmbar sein sowie
deren Mischungsform (einzeln, Trupp, Gruppe, Horst, Reihe, Streifen, flächig)
und deren Veränderung im Lauf der Zeit angegeben werden. Die Behandlung des
Mischbestands und der beiden Reinbestände sollte ähnlich gestaltet sein, also
z.B.\ das Ziel haben die höchst mögliche Gesammtwuchsleistung zu erzielen. Wobei
neben der Zuwachsleistung beispielsweise die erzeile Holzqualität oder der
Bewirtschaftungsaufwand von Interesse sein dürften.

\renewcommand{\indexname}{Stichwortverzeichnis}
\addcontentsline{toc}{section}{Stichwortverzeichnis}
\printindex

\addcontentsline{toc}{section}{Literatur}
\bibliography{literature}

%Autor: Georg Kindermann

\end{document}