\documentclass[twocolumn]{scrartcl}
%\documentclass{scrartcl}

\usepackage[utf8]{inputenc}
\usepackage[T1]{fontenc}
\usepackage{lmodern}
\usepackage[ngerman]{babel}
\usepackage{amsmath}

\usepackage{makeidx}
\makeindex

\usepackage{fancyvrb}
\usepackage{fvextra}

\usepackage[sc]{mathpazo} % or option osf
\usepackage{newpxmath}
\usepackage[output-decimal-marker={,}]{siunitx}

\usepackage[comma,authoryear]{natbib}
\bibliographystyle{natdin}

\usepackage{adjustbox}
\usepackage[a4paper, margin=1mm, includefoot, footskip=15pt]{geometry}
\usepackage{afterpage}

\usepackage{tikz}
\usetikzlibrary{calc}
\usetikzlibrary{shapes.geometric}
\usetikzlibrary{arrows.meta}

\usepackage[pdftitle={Waldwachstum}
, pdfauthor={Georg Kindermann}
, pdfsubject={Waldwachstum, Baumwachstum}
, pdfkeywords={Waldwachstum, Waldbau, Wald, Forst}
, pdflang={de-AT-1996}
, hidelinks
, pdfpagemode=None]{hyperref}

\nonfrenchspacing
\sloppy

\title{Waldwachstum}
\author{Georg Kindermann}

\begin{document}

\twocolumn[
  \begin{@twocolumnfalse}
    \maketitle
%    \begin{abstract}
%      Zusammenfassung
%    \end{abstract}
  \end{@twocolumnfalse}
]

\tableofcontents

\section{Einleitung}

Eine knappe Darstellung mir wichtig erscheinender Grundlagen des Wachstums von
Bäumen, Beständen und Betriebsklassen (Zusammenfassung von Beständen eines
Forstbetriebes). Ausführlichere bzw.\ historisch interessante Arbeiten zum
Waldwachstum sind:
\cite{guttenberg1885Wachstumsgesetze,guttenberg1912ZuwachslehreInHandbuchDerForstwissenschaft,vanselow1941Zuwachslehre,weck1948,wiedemann1951Ertragskunde,assmann1961Waldertraskunde,erteld1966Waldertragslehre,mitscherlich1975WaldWachstumUmwelt,kramer1988Waldwachstumslehre,wenk1990Waldertragslehre,pretzsch2002Grundlagen,Pretzsch2003Modellierung,Vanclay2006,Weiskittel2011,Pretzsch2019}.

\section{Standort -- Bonität}

Der Standort gibt die Rahmenbedingungen des Wachstums vor und verändert sich
laufend. Standortsbedingungen wie Temperatur, Strahlung, Niederschlag, CO$_2$
Konzentration oder Bodentyp sind kaum durch den einzelnen Waldbesitzer
beeinflussbar. Künstliche Bewässerung wird im Wald höchstens in der
Verjüngungsphase praktiziert und Düngung findet kaum statt. Insbesondere der
Boden wird durch die Bewirtschaftung beeinflusst. Durch (*) befahren kann er
verdichtet werden und damit der Wasser und Lufthaushalt verändert, bei der (*)
Ernte können Nährstoffe entzogen oder umverteilt werden, durch die (*)
Baumartenwahl und Bestandesdichte wird die Durchwurzelungstiefe und Intensität,
die Streumenge, der Streuabbau und damit der Wasserhaushalt und
Nährstoffkreislauf beeinflusst. Insbesondere durch stickstoffbindende Pflanzen
(Leguminosen, Klee, Lupinie, Erle, Robinie, \dots) kann die Wuchsleistung auf
Standorten mit Stickstoffmangel verbessert werden. Laufend gibt es
Gesteinsverwitterung, Austräge (Erosion) und Einträge (Deposition). Einträge
können Nährstoffe aber auch Schadstoffe beinhalten.

Verschiedene Baumarten haben bei gleichem Standort meist unterschiedliche
Wuchsleistungen. Die Herleitung der Wuchsleistung aus den Standortsbedingungen
ist recht schwierig und meist ungenau. Dennoch ist sie die einige Methode um die
Auswirkungen einer Standortsveränderung (z.\,B.\ Erhöhung der Temperatur um 3°C,
Verringerung der Niederschlagsmenge um 30\,\%) abschätzen zu können. Mit
Standortseigenschaften können erwartete Wuchsleistungen verschiedener Baumarten
im Vorhinein abgeschätzt und verglichen werden. Wächst eine Baumart bereits seit
einigen Jahren auf einem bestimmten Standort, wird deren Wuchsleistung aufgrund
der erreichten Baumgröße klassifiziert. Für gleichaltrige Reinbestände wird
derzeit die Mittelhöhe der 100 stärksten Bäume (=Oberhöhe) verwendet. Mit dieser
Oberhöhe und dem Bestandesalter wird eine erwartete Höhe, bei einem
Referenzalter (meist 100 Jahre), bestimmt. Die erwartete Oberhöhe im
Referenzalter wird als Oberhöhenbonität bezeichnet. Neben den Bäumen selbst
können auch andere Pflanzen zur Standortsbeschreibung und damit zur Abschätzung
der zu erwartenden Wuchsleistung verwendet werden. Mit den Pflanzen kann auf
einen natürlichen Waldtyp und damit auf eine naturnahe Baumartenzusammensetzung
geschlossen werden. Mischbaumarten können die Wuchsleistung und Gefährdungen
gegenseitig sowohl positiv als auch negativ beeinflussen.

Die Wuchsleistung wird üblicherweise in Vorratsfestmeter, Erntefestmeter oder
Biomasse ausgedrückt. Wenn Kosten (Verjüngung, Ernte, \dots) und Verkaufspreise
bekannt sind, kann die Wuchsleistung in eine monetäre Ertragsleistung
umgerechnet werden.

\section{Höhe}

Der einzelne Baum wird im Laufe seines Lebens, solange kein Schaden (Verbiss,
Wipfelbruch) eintritt, stetig höher. Die Höhenentwicklung wird durch die
benachbarten Bäume beeinflusst. Durch Überschirmung (Verjüngung unter
Altbestand) wird die Höhenentwicklung stark verzögert. Durch starke Einengung
der Krone (hohe Bestandesdiche, seitliche Konkurrenz) aber auch durch Fehlen
jeglicher Konkurrenz kann es zu einer Verringerung des Höhenzuwachses kommen.
Unter forstlich üblichen Bestandesdichten wird der Höhenzuwachs von herrschenden
Bäumen relativ wenig verändert. Die Höhenentwicklung eines einzelnen Baumes kann
durch wiederholte Baumhöhenmessungen oder durch Stammanalysen erfasst werden.
Bei Stammanalysen werden bei gefällten Bäumen die Jahrringbreiten, meist in
mehreren Richtungen, in verschiedenen Baumhöhen gemessen und damit auch die
Anzahl der Jahrringe erfasst. Durch Auftragen von Höhe und Jahrringanzahl kann
auf die Höhenentwicklung rückgeschlossen werden. Da die Stammscheiben meist
nicht exakt in der Höhe der damaligen Wipfelknopspe geworben werden, kommt es zu
systematischen Differenzen zwischen der wirklichen Baumhöhe und der Höhe, in der
die Stammscheibe geworben wurde. Diese können eliminiert werden, wenn die
Positionen der Wipfelknospen z.\,B.\ anhand von sichtbaren Asquirlen
rekonstruiert werden können.

In Abbildung~\ref{fig:hoeheAlter} ist eine idealisierte (keine jährlichen
Zuwachsschwankungen, keine Schädigungen) Höhenentwicklung über dem Alter in
Schwarz aufgetragen (Höhenkurve). Zusätzlich wird noch der laufende
Höhenzuwachs, also der Höhenzuwachs der im jeweiligen Alter erfolgt, und der
durchschnittliche Höhenzuwachs (Höhe durch Alter) gezeigt. Zunächst nimmt der
laufende Zuwachs ständig zu und die Höhenkurve wird immer steiler, bis der
maximale laufende Zuwachs erreicht wird. Zu diesem Zeitpunkt hat die Höhenkurve
einen Wendepunkt und verläuft ab da immer flacher. Auch wenn nun der laufende
Höhenzuwachs zurückgeht, liegt er noch einige Zeit über dem durchschnittlichen
Höhenzuwachs. Zu dem Zeitpunkt, wenn laufender und durchschnittlicher
Höhenzuwachs identisch sind, erreicht der durchschnittliche Höhenzuwachs sein
Maximum. Zu diesem Zeitpunkt geht die Tangente der Höhenkurve durch den
Ursprung. Ab diesem Zeitpunkt geht der durchschnittliche Höhenzuwachs zurück.

Neben der Darstellung der Zuwächse über dem Alter können diese auch über der
Höhe aufgetragen werden. Wenn über die Bonität der Höhenwachstumsgang bekannt
ist, kann bei herrschenden Bäumen alleine durch Messen ihrer Höhe auf das
ideelle Baumalter und auf den zu Erwartenden laufenden Höhenzuwachs geschlossen
werden. Bei Bäumen deren Entwicklung z.\,B.\ durch Überschirmung gehemmt wurde,
passt deren Alter und Höhe nicht mit jenen, die sich ständig ungehindert
entwickeln konnten, überein. Bei Beseitigung dieser Wuchshemmung, haben diese
Bäume nicht die jährlichen Höhenzuwächse wie gleich alte, aber in etwa wie
gleich große ungehindert gewachsenen Bäume. Demnach eignen sich neben der
Standorts-- und Konkurrenzsituation die Baumdimensionen Höhe, Stammdurchmesser
und Kronengröße besser zum abschätzen der laufenden Zuwächse als das Baumalter.

\begin{figure}[htbp]
  \centering
  \includegraphics[width=.95\columnwidth]{./pic/hoehenzuwachs}
  \caption{Höhenentwicklung über dem Alter}
  \label{fig:hoeheAlter}
\end{figure}

Die Höhenentwicklung kann auch für einen ganzen Bestand, genauso wie für den
Einzelbaum beschrieben werden. Dabei muss zunächst definiert werden, wie und für
welches Kollektiv die Bestandeshöhe bestimmt wird. Beispielsweise kann das
arithmetische Mittel aller Höhen, das mit dem Baumvolumen oder der Kreisfläche
gewichtete Höhenmittel (Loreysche Mittelhöhe) oder nur für die 100~stärksten
bzw.\ höchsten Bäume je Hektar eine Höhe errechnet werden. Im Zuge der
Bestandesentwicklung scheiden laufend Bäume aus und können damit die Höhe des
verbleibenden Bestandes beeinflussen. Scheiden kleine aus, steigt, scheiden
große aus fällt das Mittel. Zur Bestimmung der Bonität wird oft die Oberhöhe
verwendet, welche z.\,B.\ aus den 100~stärksten Bäumen je Hektar bestimmt wird.
Ein Ziel von Oberhöhendefinitionen war, das sie durch Eingriffe möglichst wenig
verändert werden. Scheiden nur kleinen Bäume aus, die nicht zum Kollektiv der
Oberhöhenstämme gehören, ändert sich diese Höhe nicht. Wiewohl
Bestandesichteänderungen immer noch einen Einfluss auf den folgenden
Höhenzuwachs haben können.

\section{Durchmesser}

Ähnlich wie die Höhe entwickelt sich auch der Baumdurchmesser bzw.\ dessen
Kreisfläche im Laufe der Zeit. Auf den Durchmesserzuwachs hat jedoch die
Durchmessergröße eine wesentlicheren Einfluss als das Alter. Es ist bei uns
üblich den Durchmesser in 1,3\,m Höhe zu messen (Wurzelanlauf, kein Bücken).
Dieser Durchmesser wird als Brusthöhendurchmesser (BHD) bezeichnet. Von Bäume
die eine Höhe 1,3\,m noch nicht erreicht haben, kann daher kein BHD gemessen
werden. Dort wird gelegentlich der Wurzelhalsdurchmesser gemessen. Der
Durchmesserzuwachs wird stark von der Konkurrenzsituation, der Bestandesdichte
geprägt. Wird der Kreisflächenzuwachs $(d_1^2 - d_0^2)\cdot \pi/4$, wobei $d_0$
der Durchmesser zum Zeitpunkt null und $d_1$ zum Zeitpunkt eins ist, von allen
Bäumen eines Bestandes durch die Bestandesfläche und Zuwachsperiodenlänge
dividiert, erhält man die Grundflächenzuwachsleistung [m²/ha/Jahr]. Diese
Zuwachsleistung ist abhängig von Baumart, Baumdimensionen, Alter, Standort und
Bestandesdichte, wobei es eine zuwachsoptimale Bestandesdichte geben kann.

Abbildung~\ref{fig:zuwachsBestockungsgrad} zeigt mögliche Verläufe zwischen
natürlichem Bestockungsgrad und relativer Zuwachsleistung. Der natürliche
Bestockungsgrad gibt das Verhältnis von aktueller zu maximal möglicher
Bestandesdichte, in der Regel ausgedrückt in m² Grundfläche je ha, an. Die
relative Zuwachsleistung gibt das Verhältnis des Zuwachses bei maximaler
Bestandesdichte zu Zuwachs bei aktueller Bestandesdichte an, wobei meist der
Grundflächenzuwachs/ha/Jahr, Volumnszuwachs/ha/Jahr oder
Biomassenzuwachs/ha/Jahr betrachtet wird. Es gibt Bestandessituationen, bei
denen eine Absenkung der Bestandesdichte zu einem Zuwachsverlust führt
(gestrichelte Linie). Bei anderen kommt es bei einer Absenkung der
Bestandesdichte, zunächst zu einer Zuwachszunahme, um erst danach abzunehmen
(durchgezogene Linie). Bei diesen gibt es eine zuwachsoptimale Bestandesdichte.
Junge Bäume mit gut entwickelten Kronen auf zuwachsstarken Standorten zeigen
eher ein Optimum als ältere mit kurzen Kronen auf zuwachsschwachen. Insbesondere
die Baumart hat neben Bestandesstruktur, Baumartenmischung, \dots einen Einfluss
auf die Zuwachsreaktion auf verschiedene Bestandesdichten.

\begin{figure}[htbp]
  \centering
  \includegraphics[width=.95\columnwidth]{./pic/zuwachsBestockungsgrad}
  \caption{Abhängigkeit der Zuwachsleistung vom natürlichen Bestockungsgrad}
  \label{fig:zuwachsBestockungsgrad}
\end{figure}

Einen Bestand laufend auf der zuwachsoptimalen Bestandesdichte zu halten ist
praktisch nicht möglich. Vielmehr wird im Zuge von Durchforstungseingriffen die
Bestandesdichte soweit abgesenkt, wie es ein akzeptierter Zuwachsverlust
erlaubt. In Abb.~\ref{fig:zuwachsBestockungsgrad} wurde angenommen, dass 3\,\%
Zuwachsverlust gegenüber dem Vollbestockten Bestand akzeptiert werden und durch
eine horizontale graue strichlierte Linie angedeutet. Beim Bestand mit
Zuwachsoptimum kann auf ca.\ 60\,\% der Bestandesdichte abgesenkt werden, beim
anderen nur auf ca.\ 80\,\%. Bei dem Bestand mit Zuwachsoptimum hätte man aber,
bezogen auf das Optimum nicht 3\,\%, sondern mehr als 3\,\% Zuwachsverlust.
Bezogen auf das Optimum wäre in dem Bereich zwischen den beiden weißen Punkten
mit maximal 3\,\% Zuwachsverlust zu rechnen.

Die Beziehung Bestandesdichte--Zuwachs ist nicht statisch, sondern verändert sich im Lauf der Zeit. Die Art der Veränderung wird unter anderem auch durch die Bestandesdichte mitbestimmt.

\begin{figure}[htbp]
  \centering
  \includegraphics[width=.95\columnwidth]{./pic/zuwachsveraenderungBestockungsgrad}
  \caption{Veränderung der Beziehung Zuwachs Bestandesdichte}
  \label{fig:zuwachsveraenderungBestockungsgrad}
\end{figure}

Die Beziehung Bestandesdichte--Zuwachs spielt sich auf der Einzelbaumebene ab
und dort ist eine stufenlose Variation der Bestandesdichte nicht möglich. Dort
kann ich mindestens einen Nachbarbaum entnehmen und damit sinkt die
Bestandesdichte stufenweise wie dies in Abb.~\ref{fig:bestandesdichteEinzelbaum}
schematisch angedeutet ist. Links oben ist ein gleichmäßiger Dreiecksverband als
Ausgangssituation dargestellt. Wenn ein Nachbar entnommen wird vergrößert sich
die Standfläche der Nachbarbäume um 1/6 d.\,h. wenn der Bestockungsgrad eins war
reduziert er sich auf 6/7 ($\sim 0.86$). Die dabei entstehende Standfläche ist
asymmetrisch, was in der Regel zu Zuwachseinbußen, asymmetrischen Kronen,
schiefem Wuchs und damit zu Reaktionsholz führt. Es könnte in dem vereinfachten
Beispiel den gegenüberliegenden Nachbarn entnehmen und könnte den
Standflächenschwerpunkt wieder in Zentrum bringen. Die Form wäre aber stark
abweichend von der Idealform des Kreises. Ein Kompromiss wäre die Entnahme von
drei Nachbarn, wie es in Abb.~\ref{fig:bestandesdichteEinzelbaum} links unten
dargestellt ist, wobei sich die ursprüngliche Standfläche um 50\,\% erhöht. Eine
wirklich zufriedenstellende Verteilung und Standflächenform erreicht man nur,
wenn 6~Nachbarn entnommen werden, was die Standfläche verdreifacht.

\begin{figure}[htbp]
  \centering
  \includegraphics[width=.95\columnwidth]{./pic/bestandesdichteEinzelbaum}
  \caption{Veränderung der Bestandesdichte durch Entnahme von Nachbarbäumen}
  \label{fig:bestandesdichteEinzelbaum}
\end{figure}

Die möglichen Bestandesdichtestufen sind in Abb.~\ref{fig:bestockungsgradstufen}
dargestellt wobei eine Verringerung des Zuwachses aufgrund ungünstiger
Standflächen-- und Kronenformen nicht berücksichtigt sind. Dabei spiel es eine
Rolle wie hoch die Bestandesdichte vor dem Eingriff war. Oben wurde angenommen
das die maximal mögliche, unten nur 80\,\% der maximal möglichen Bestandesdichte
vor dem Eingriff geherrscht hat. Im linken Teil der Abbildung ist die Situation
des Dreicksverbandes wie in Abb.~\ref{fig:bestandesdichteEinzelbaum} links oben
dargestellt und im rechten Teil der Abb.~\ref{fig:bestockungsgradstufen}, wenn
um einen Zentrumsbaum nur noch drei Nachbarn vorhanden sind
(Abb.~\ref{fig:bestandesdichteEinzelbaum} links unten), dargestellt.

\begin{figure}[htbp]
  \centering
  \includegraphics[width=.95\columnwidth]{./pic/bestockungsgradstufen}
  \caption{Bestockungsgradstufen}
  \label{fig:bestockungsgradstufen}
\end{figure}

Bei der Betrachtung von Zuwachs und Bestandesdichte darf nicht übersehen werden,
dass mit der Bestandesdichte auch das Kronenvolumen der Bäume gesteuert wird und
damit die Zuwachsleistung bzw.\ die zukünftige Form der Kurve zwischen
Bestandesdichte und Zuwachs bestimmt wird, ob es eine Kurve mit oder ohne
Optimum ist. Auch hat der BHD des Baumes einen großen Einfluss zum einen auf die
Zuwachsleistung, zum anderen auf die Beziehung Zuwachs--Bestandesdichte.
Zunächst steigt die Zuwachsleistung mit Zunahme des BHD's. Mit einer Reduktion
der Bestandesdichte steigt der Durchmesser des BHD's schneller an und führt
damit zu größeren Zuwächsen (=Wuchsbescheunigung). Ab einem gewissen BHD nimmt
die Zuwachsleistung wieder ab. Zu Beginn sind geringe Bestandesdichte gut, da
sie den BHD--Zuwachs steigern, die Kronen schneller an Volumen zunehmen. Jedoch
ab einem bestimmten BHD sollte dieser nicht zu schnell steigen, da ansonsten die
Zuwachsleistung wieder sinkt. Hier Hilft zum einen eine hohe Bestandesdichte als
auch die Entnahme der stärkeren Bäume (Plenterdurchforstung). Dies gilt aber nur
solange dadurch zum einen nicht die Baumkronen zu stark verkleinert und
starkwüchsige Bäume zugunsten schwachwüchsiger entnommen werden.

Da Altbestände weniger auf Druchforstungen reagieren können (geringerer
Höhenzuwachs), bzw.\ dort Entnahmen, aufgrund der Größe der entnommenen Bäume,
wesentlich größere Lücken hinterlassen und es länger dauert bis diese
geschlossen werden können, ist es üblich ab etwa der Hälfte der geplanten
Umtriebszeit keine weiteren Durchforstungen zu machen. Bei der letzten
Durchforstung sollte die Baumverteilung möglichst einem gleichmäßigen
Dreiecksverband, mit sechseckigen Standflächen, nahe kommen und die
Bestandesdichte so weit abgesenkt werden, dass bei Erreichen des Erntealters in
etwa die maximal mögliche Bestandesdichte erreicht wird. Eine Bestandesdichte
nahe der maximal Möglichen und eine Baumverteilung wie in
Abb.~\ref{fig:bestandesdichteEinzelbaum} links unten ist vor der letzten
Durchforstung zu begrüßen, da hier nach der Entnahme von drei Nachbarn ein
Dreiecksverband entsteht und der natürliche Bestockungsgrad auf ca.\ 0.5
abgesenkt wird. Dieser stake Eingriff führt zwar auf den ersten Blick zu einem
Zuwachsverlust, erlaubt es jedoch den Verbleibenden Bäumen relativ große, vitale
Kronen zu entwickeln, die bei gleicher Bestandesdichte höhere Zuwächse erbringt
als unterentwickelte Kronen. Auch kann nur mit starken Reduktion der
Bestandesdichte gewährleistet werden, dass die maximal mögliche Bestandesdichte
nicht zu früh erreicht wird und es nicht zu konkurrenzbedingter Mortalität bzw.\
vorzeitigen Eingriffen, vor erreichen der geplanten Umtriebszeit, kommt.

Zum Zeitpunkt des Eingriffs kommt es so gut wie immer zu einem Zuwachsrückgang,
da sich die Blätter und Nadeln der verbleibenden Bäume erst and die geänderte
Situation anpassen müssen. Auch sind Bäume eines realen Bestandes nie gleich
groß. Die weit verbreitete Regel: >>Entnahme der 2--3 größten Bedränger eines
Z(ukunfts)--Baumes<< kann für Auslesedurchforstungen empfohlen werden. Bäume die
keine bzw.\ nur eine geringe Konkurrenz ausüben und bei einer Entnahme nicht
Kostendeckend sind oder sonstige nachteilige Wirkungen ausüben, werden
nicht entnommen.

\section{Volumen}

\section{Alter}

\section{Erntezeitpunkt}

\section{Zufallsnutzung}

\section{Stammzahl}
\section{Bestandesdichte}

\section{Baumartenmischung}

\addcontentsline{toc}{section}{Literatur}
\bibliography{literature}

%Autor: Georg Kindermann

\end{document}