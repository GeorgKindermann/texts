\documentclass[twocolumn]{scrartcl}
%\documentclass{scrartcl}

\usepackage[utf8]{inputenc}
\usepackage[T1]{fontenc}
\usepackage{lmodern}
\usepackage[ngerman]{babel}
\usepackage{amsmath}

\usepackage{makeidx}
\makeindex

\usepackage{fancyvrb}
\usepackage{fvextra}

% \usepackage{palatino}
%
% \usepackage{newpxtext,newpxmath}
%
\usepackage[sc]{mathpazo} % or option osf
\usepackage{newpxmath}
\usepackage[output-decimal-marker={,}]{siunitx}

\usepackage[comma,authoryear]{natbib}
\bibliographystyle{natdin}

\usepackage{adjustbox}
\usepackage[a4paper, margin=1mm, includefoot, footskip=15pt]{geometry}
%\usepackage[a4paper, margin=20mm, includefoot, footskip=15pt]{geometry}
\usepackage{afterpage}

\usepackage[pdftitle={Misch- oder Reinbestand?}
, pdfauthor={Georg Kindermann}
, pdfsubject={Waldbau, Waldwachstum}
, pdfkeywords={Waldbau, Waldwachstum, Wald, Forst}
, pdflang={de-AT-1996}
, hidelinks
, pdfpagemode=None]{hyperref}

\nonfrenchspacing
\sloppy
\usepackage{breqn}
\usepackage{enumitem}
%\usepackage{rotating}
\usepackage{pdflscape}

\title{Misch-- oder Reinbestand?}
\author{Georg Kindermann}

\listfiles
\begin{document}

\twocolumn[
  \begin{@twocolumnfalse}
    \maketitle
    \begin{abstract}
      Weder Misch-- noch Reinbestand sollen erzwungen werden. Wenn
      sich natürlich eine Verjüngung aus mehreren Baumarten einstellt,
      soll diese zwar in Richtung der gewünschten Bestockung geführt,
      dabei aber nicht, mit übertriebenem Eifer, alle anderen
      Baumarten ausgemerzt werden. Wenn künstlich gemischt verjüngt
      wird, müssen die beteiligten Baumarten, in der Mischung,
      bestimmte Aufgaben erfüllen. Andernfalls sollte auf der
      Schlagfläche die Kunstverjüngung nur mit einer Baumart
      durchgeführt werden und die Baumartenmischung auf regionaler
      Ebene, durch ein Mosaik von Reinbeständen, unterschiedlicher
      Baumarten, gefunden werden.
    \end{abstract}
  \end{@twocolumnfalse}
]

\tableofcontents

\section{Einleitung}

Die Frage, ob Misch-- oder Reinbestand, muss, \emph{allgemein}
gestellt, \emph{offen bleiben}. Wenn in einer Region beide Typen
möglich sind, ist es in der Regel wünschenswert, beide auch real
verwirklicht zu sehen. Hingegen auf \emph{Bestandesebene} muss bei der
Bestandeszielformulierung, der Weg zu diesem Ziel, klar vorgegeben
werden und somit auch die Entscheidung, für oder gegen Misch-- oder
Reinbestand, gefällt werden. Um diese Entscheidung zu unterstützen,
werden hier Beobachtungen und mit Argumenten untermauerte Empfehlungen
zusammen getragen. Zusätzlich soll die Vielfalt von Rein-- aber
insbesondere von Mischwäldern verdeutlicht und empfehlenswerte Formen
herausgearbeitet werden.

\section{Charakterisierung von Misch-- und Reinbestand}
\label{sec:charakterisierung}

Um sich für oder gegen etwas entscheiden zu können, müssen die beiden
Typen klar voneinander trennbar sein. Bei Misch-- und Reinbeständen
handelt es sich jedoch nicht um zwei gegenüberliegende Pole ohne
dazwischenliegende Übergänge, sondern um ein Kontinuum das von
Monokulturen mit ausschließlich einer vertretenen Baumart hin zu
Wäldern die aus so vielen Baumarten bestehen, dass jede Baumart nur
mit wenigen Exemplaren vertreten ist. Als Merkmal dienen die Anzahl an
Baumarten, deren Anteile sowie deren Größe und Verteilung.

\subsection{Definition}
\label{ssec:definition}

Zunächst soll die Definition von Rein-- und und Mischbestand nach
\cite{bruenig1980WaldbaulicheTerminologi} wiedergegeben werden.
\begin{description}
\item[Mischbestand:\index{Deffinition!Mischbestand}] Ein Bestand aus
  zwei oder mehr Baumarten, die jede einzeln und besonders im
  Zusammenwirken die Bestandesökologie signifikant beeinflussen. Die
  Bauarten können stamm--, trupp--, gruppen-- oder kleinhorstweise
  miteinander gemischt sein. Größere Horste verlieren den
  Mischbestandscharakter; Gegensatz Reinbestand.
\item[Reinbestand:\index{Deffinition!Reinbestand}] Bestand, der aus
  einer Baumart (mindestens 90\,\%) gebildet wird. Vorkommen anderer
  eingemischter Baumarten, die keinen gewichtigen ökologischen
  Einfluss ausüben, verändert nicht den Charakter als Reinbestand.
\end{description}

Es wird, neben allen möglichen Varianten der Mischung, auch beim
Reinbestand eine Beimischung von bis zu 10\,\% von anderen Baumarten
toleriert. Und dabei stellt sich bereits die Frage auf was sich diese
10\,\% beziehen. Meist wird dabei wohl ein Anteil an der
Bestandesfläche gemeint sein. Allerdings wie lässt sich der Anteil an
der Bestandesfläche für eine Baumart bestimmen? In der Praxis der
Forsteinrichtung wird dieser gutachtlich, etwa über den
Schirmflächenanteil, geschätzt. Gelegentlich wird auch der Anteil am
Volumen, an der Biomasse, oder, da dies oft einfacher zu bestimmen
ist, der Grundflächenanteil verwendet. Auch für den
Wuchsleistungsvergleich zwischen Misch-- und Reinbeständen muss die
Frage, wie groß der \emph{Anteil} der Baumarten im Mischbestand ist,
realitätsnah beantwortet werden können und wird sich wohl auf den
Ressourcenverbrauch beziehen müssen.

\subsection{Anzahl an Baumarten}
\label{ssec:anzahlBaumarten}

Laut Definition bedarf es für einen Mischbestand \emph{mindestens
  zwei} Baumarten wobei keine einen Anteil von 90\,\% erreichen oder
überschreiten darf. Ein Mindestanteil einer Baumart ist insofern
gefordert, da sie die Bestandesökologie signifikant beeinflussen
sollte. Da dies in den seltensten Fällen beantwortbar sein wird, kann
als Untergrenze, das in der Forsteinrichtung übliche 1/10, als
Richtwert empfohlen werden. Bei Naturverjüngung ergeben sich die
Baumarten und deren Anteile zu Beginn von selbst und werden im Zuge
der Mischwuchsregulierung zum Bestockungsziel geführt. Aber auch bei
Naturverjüngung kann die Baumartenzusammensetzung künstlich mittels
Ausbessern ergänzt werden. Bei Kunstverjüngung muss die Anzahl an
Baumarten vorgegeben werden. Wobei die Anzahl von der Bestandesgröße
abhängig sein kann. Da ein Charakteristikum des Bestandes, dessen
Einheitlichkeit ist, ist der Größe nach oben hin eine Grenze gesetzt
und damit auch gleichzeitig der Wechsel von Baumartenm innerhalb eines
Bestandes aufgrund von standörtlichen Unterschieden, solange diese
nicht kleinflächig sind, aufgrund der geforderten Einheitlichkeit,
zumindest eingeschränkt.

\cite[S.~218--219]{wiedemann1951Ertragskunde} fordert für
Mischkulturen \frqq eine klare Planung, welche \emph{Aufgaben} jede
Baumart zu erfüllen hat, und wieweit jede unter den gegebenen
Standortsverhältnissen begünstigt werden soll, was meist nur bei
Beschränkung der Mischung auf \emph{2--3~Arten} möglich ist. Bei
Buntmischungen von 5 oder noch mehr Baumarten können die besonderen
Bedürfnisse der einzelnen Arten nicht beachtet werden. Diese
Mischungen stellen hohe Ansprüche an Läuterung und Durchforstung,
welche oft unerfüllbar bleiben, mit der Folge, dass mehrere Baumarten
spätestens im Dickungsalter absterben oder vollständig untertauchen
und dass die vorwüchsigen Holzarten keine genügende Stammzahl zur
Astreinigung unter sich selbst haben.\flqq{}

Hier sei ausdrücklich darauf hingewiesen das sich diese Empfehlung von
zwei bis maximal drei Baumarten auf den Bestand und nicht auf die
Region bezieht. Aber auch auf regionaler Ebene wird es gelegentlich
schwer fallen mehrere wirtschaftlich optimale Baumarten zu finden.
Dies schließt jedoch nicht aus bzw.\ ist es wünschenswert, dass
vereinzelt weitere Baumarten, im Bestand vertreten sind, von denen
ausgehend, bei Bedarf, deren Anteil im Folgebestand, über
Naturverjüngung, kostengünstig erhöht werden kann.

Eine seltene Ausnahme bilden Site-Index--Benchmark Beispielsbestände
auf denen möglichst viele Baumarten, horstweise, so dass im Endbesatnd
etwa 10--12 Bäume der jeweiligen Baumart Platz haben, gemischt,
gepflanzt werden. In diesen Beständen kann die Wuchsleistung
verschiedenster Baumarten am selben Standort verglichen werden, wie
dies beispielsweise \cite{mayer1970anbauversuch} zeigten, aber auch
direckt vor Ort ein vergleichendes Urteil zwischen den verschiedenen
Baumarten gefällt werden. Charakteristisch für diese Bestände ist,
dass viele Baumarten versagen und oft nicht einmal eine Handvoll als
empfehlenswert für diesen Standort übrig bleibt. Aber eben diese
sollen eine Anregung für weitere, in einer Region weniger übliche,
aber möglicherweise vielversprechende, Baumarten geben und somit
mithelfen die Baumartenvielfalt zu steigern. Um eine kleine
Vorstellung über mögliche Baumarten einer Region zu bekommen wurden im
Jahr~2017 am Wechsel ($47,4999^{\circ}$ Breite, $15,9741^{\circ}$
Länge) auf 1340\,m 31~Baumarten, und im Jahr~2019 der
Klimaforschungswald in der Nähe von Matzen mit 15~Baumarten, auf
Versuchsflächen gepflanzt. Ein weiterer Ausbau dieses Versuchsnetztes
könnte die sonst üblichen Möglichkeiten bei der Baumartenwahl
erweitern und auf eine sicherere Grundlage stellen.

\subsection{Mischungsart}
\label{ssec:mischugnsart}

Die Mischungsart gibt an, wie sehr die Baumarten untereinander
gemischt bzw.\ wie stark sie in den Mischbeständen voneinander
getrennt aufwachsen. Auch wenn es für die Frage Misch-- oder
Reinbestand nicht wesentlich scheint, wie der Mischbestand aufgebaut
ist, soll dargestellt werden, dass es innerhalb der Gruppe der
Mischbestände fließende Übergänge in Richtung Reinbestand gibt. Nicht
nur über die \emph{Anteile} der beteiligten Baumarten, von nicht
vertreten bis zu mit gleichen Anteilen vertreten, kann der Übergang
vom Rein-- zum Mischbestand beobachtet werden, sondern auch über die
\emph{Mischungsform} von einzeln bis hin zu horst-- und flächenweiser
Mischung. Die Mischungsform kann sich im Lauf der Bestandesentwicklung
ändern. Eine truppweise Mischung im Jungbestand kann im Altbestand
eine Einzelmischung ergeben.

\subsubsection{Mischungsform}
\label{sssec:mischungsform}

Die Mischungsform beschreibt die Ausdehnung und Verteilung der
Baumarten:

\begin{description}
\item[Einzel:] Buntmischung. Baumweise Verteilung. Baum ist nicht von gleicher Art umgeben.
\item[Trupp:] Einige, weniger als etwa 5~Bäume in Baumholzalter. Durchmesser maximal halbe Baumlänge des Altbestandes.
\item[Gruppe:] Durchmesser maximal Baumlänge des Altbestandes.
\item[Horst:] Kleinfläche mit teilweise ökologischer und wirtschaftlicher Selbständigkeit die aber waldbaulich nicht selbständig behandelt werden kann. Etwa 0,1--0,5 Hektar.
\item[Fläche:] Fläche die waldbaulich selbständig behandelt wird.
\item[Reihe:] Wechsel der Baumart von Reihe zu Reihe.
\item[Streifen:] Mehrere Reihen der selben Baumart.
\end{description}

Auch wenn in der Natur die Verjüngung immer wieder in Form von
Einzelmischungen beobachtet werden kann, wird diese, aufgrund der
dadurch in der Regeln entstehenden Mehraufwände, kaum für die
Kunstverjüngung empfohlen. Reihen-- oder Streifenmischung ist
arbeitstechnisch bei der Kunstverjüngung auf größeren Kahlflächen
relativ einfach zu realisieren. Bei kleinflächigen Nutzungen sowie bei
vorübergehenden Fehlstellen in der Verjüngung kann Trupp--, Gruppen--
oder Horstweisemischung sowohl durch Natur-- als auch durch
Kunstverjüngung entstehen. Auf Kahlflächen kann auch die
\emph{Clusterpflanzungen} in Form von \emph{Nesterpflanzung} (auf ca.\
1\,m$^2$ werden etwa 20~Bäume gepflanzt) oder \emph{Trupppflanzung}
(je Trupp werden ca.\ 20~Bäume mit einem Abstand von ca.\ 1\,m
zueinander gelegentlich auch mit 2~Baumarten gepflanzt) erfolgen,
wobei die Nester bzw.\ Trupps zueinander den Abstand der Bäume des
Endbestands, oder einen Bruchteil davon, haben. Die dazwischenliegende
Bestandesfläche wird der Naturverjüngung überlassen.

Obwohl von der Einzelmischung in der Verjüngungsphase generell
abgeraten wird, gibt es doch Ausnahmen, in denen sie sinnvoll
eingesetzt werden kann.
\begin{itemize}
\item Geeignete Schattbaumarten für Einzelmischung von der
  Verjüngungsphase bis in den Altbestand sind in Mitteleuropa
  selten. Je nach Standort teilweise geeignet wären Ficht, Tanne und
  Buche.
\item Bei einer Mischung von Schatt-- und Lichbaumarten muss die
  Lichtbaumart vorwüchsig sein oder die Schattbaumart später
  eingebracht werden.
\item Mischungen von Lichtbaumarten, da es selten zu Entmischungen
  kommt.
\item Der Pflanzverband wurde so weit gewählt dass die
  konkurrenzstärkere Baumart die konkurrenzschwächere, während der
  Bestandesentwicklung, nicht verdrängen kann. Dabei sollte die
  dadurch bedingte geringe Selektionsmöglichkeit bei Durchforstungen
  und Astreinigung und deren Auswirkungen auf die Qualität beachtet
  werden.
\item Wenn die Funktion am besten bei Einzlmischung erfüllt wird
  (Vorwald, Schaftreinigung), wobei dann eine Bauart eine dienende
  Funktion übernimmt.
\item Bei Zeitmischungen: Eine Baumart wird früher als die andere
  geerntet und / oder später als eine andere Verjüngt (Vorwald,
  Voranbau, Unterbau). Es ist die früher hiebsreife Art einzeln mit
  geringen Anteilen beizumischen, solange keine Lücken im Bestand
  verbleiben sollen.
\end{itemize}
\frqq Die Einzelmischung in gleichaltrigen Bestand stellt Ansprüche an
die einzelnen Holzarten, welche ihrer Natur in der größten Mehrzahl
der Fälle zuwider sind, und diese bei der großen Verschiedenheit in
ihren Ansprüchen an den Standort, den Lichtgenuß und den
Entwickelungsraum im gleichförmigen Schlusse unserer Bestände nicht
erfüllen \emph{können}\flqq{}
\citep[S.~147]{gayer1886DerGemischteWald}. Gayer sieht im gemischten
Wald überwiegend Vorteile, betont aber dass die Mischung in
hinreichend großen Horsten, welche im Idealfall ungleichaltrig sind,
erfolgen muss. Einen Hinweis was unter hinreichend zu verstehen ist
findet man auf Seite 78/79, wo für Eiche mindestens 1/2~ha gefordert
wird. Anscheinend wurden viele dieser wichtigen Hinweise zu wenig
beachtet, denn schon sein Nachfolger berichtet: \frqq Gayers Methode
hat, da die Bestände für ihre Verjüngung nicht erzogen wurden,
vielfach ganz versagt, vielfach nur Stückwerk ergeben; der langsame
Verjüngungsgang hat schwere Nachteile für die Rentabilität gebracht
\dots was heute, kurz nach seiner Begründung, als kleingruppenweiser
oder stammweiser Mischbestand erscheint, wird im kritischen Alter des
Stangenholzes ohne fortgesetzte Hilfe und Kostenaufwand wieder reiner
Bestand werden\flqq{}
\citep[S.~547]{mayr1909WaldbauAufNaturgesetzlicherGrundlage}. Er
schlägt daher als besser Alternative den \emph{Kleinbestandswald} mit
Größen von 0,3 bis 3~Hektar vor. Jeder Kleinbestand besteht aus einer
\emph{anderen} Baumart, ist selbst aber ein Reinbestand. Damit
entsteht wohl ein Bestandesmuster und Mischungen die dem, welches
Gayer vorgeschlagen hatte, recht ähnlich sein wird. Er vermeidet aber,
dadurch, dass er die Bezeichnung kleine Reinbestände anstatt
horstweiser Mischbestände verwendet, das fataler weise als
Mischbestand ein in der Verjüngung stammweiser gemischter Bestand
gemeint wird. Einen anderen Weg beschreibt
\cite{wagner1923DerBlendersaumschlagUndSeinSystem} mit dem
\emph{Blendersaumschlag}, wo durch Aneinanderreihung langgestreckter,
also saumweiser Schläge, die nutzungstechnische Vorteile bringen,
streifenweise eine Baumartenmsichung, wen möglich über
Naturverjüngung, zu erzielen.

Baumarten können in bestandesbildende und nicht bestandesbildende
Baumarten unterteilt werden, wobei letzteren gelegentlich unterstellt
wird dass sie für Reinbestände ungeeignet und ausschließlich in
Mischbeständen vertreten sein können. Nicht bestandesbildende
Baumarten sind gegenüber bestandesbildenenden konkurrenzschwach und
kommen damit natürlich entweder auf kleinräumigen Sonderstandorten
beschränkt rein vor (z.B. Schwarzerle) oder sind durch, für sie in der
Bestandesenwicklung günstige Zufälle, vereinzelt im Bestand vertreten
(z.B. Bergahorn im Buchenbestand). Bei der Bestandesbegründung ist es
sehr wohl möglich Reinbestände auch mit nicht bestandesbildende
Baumarten zu schaffen. In der Regel geht jedoch die Eigenschaft
konkurrenzschwach mit zuwachsschwach Hand in Hand und deshalb werden,
solange diese Zuwachsschwäche nicht durch den Holzpreis aufgewogen
wird, diese Baumarten auf kleine Anteile beschränkt bleiben.

\subsubsection{Struktur}
\label{sssec:struktur}

Sowohl in Rein-- als auch in Mischbeständen kann durch Veränderung der
\emph{Altersstruktur} ein Übergang vom einschichtigen, gleichaltrig
zum ungleichaltirgien, mehrschichtigen Bestand verwirklicht
werden. Auch hier kann, wie bei der Baumartenmischung, die
Alterskohorte von einzeln bis zu horstweise gemischt realisiert
sein. Sowohl in Rein-- als auch in Mischbeständen kann das
Abtriebsalter bzw.\ die \emph{Zieldimension} in weiten Bereichen
gewählt werden und steht im Wechselspiel zur durchschnittlichen
\emph{Bestandesdichte} bzw.\ der Eingriffsstärke und --frequenz,
welche ebenfalls in weiten Bereichen gewählt werden kann. Auch können
die Bäume des Bestandes in regelmäßigen Abständen zueinander oder in
Gruppen angeordnet sein, wobei regelmäßige Abstände zu bevorzugen sind,
da sie meist symmetrische Kronen, eine gleichmäßige Lastverteilung und
damit weniger Reaktionsholz hervorbringen. Gruppen entstehen meist
aufgrund der Betriebsart (Stockausschlag im Niederwald) oder aufgrund
der Auslese von seltenen guten Stammqualitäten
(\cite{kato1969Buchendurchforstung,kato1988Gruppendurchforstung})
wobei sich letztere als wenig Erfolgreich herausstellte.

Die Mischung kann auch über vertikale Struktur modifiziert werden. So
können beispielsweise alle Baumarten zu gleichen Teilen in der
Oberschicht in der herrschenden Baumklasse vertreten, oder eine Baumart
ausschließlich in der Unterschicht, in der unterdrückten Baumklasse,
vertreten sein. Diese Höhendifferenzierung geht meist Hand in Hand mit
einer Durchmesserdifferinzierung.

\begin{description}
\item[Vertikale Struktur] (Abb.~\ref{fig:vertikaleStruktur})
  \begin{description}
  \item[Einschichtig:] Kronenraum des Bestandes besteht aus einer Schicht.
  \item[Mehrschichtig:] Kronenraum des Bestandes besteht aus mehreren übereinanderliegenden Schichten.
  \item[Stufig:] Kronenraum des Bestandes besteht aus mehreren nebeneinanderliegenden Schichten.
  \end{description}
\item[Horizontale Struktur]~
  \begin{description}
  \item[Regelmäßig:] Die Bäume stehen möglichst weit von einander entfernt.
  \item[Zuällig:] Es gibt sowohl weit, mittel als auch eng beieinander stehende Bäume.
  \item[Geklumpt:] auch Rottenstruktur. Bäume stehen gedrängt in Gruppen.
  \end{description}
\end{description}

\begin{figure}[htbp]
  \centering
  \begin{minipage}{0.9\linewidth}
    \begin{minipage}{0.3\linewidth}
      \includegraphics[width=\linewidth]{./pic/lichtNebenSchattenBa.pdf}
      \textit{Einschichtig}
    \end{minipage}\hfill
    \begin{minipage}{0.65\linewidth}
      \includegraphics[width=\linewidth]{./pic/lichtUeberSchattenBa.pdf}
      \null\hfill\textit{Zweischichtig}
    \end{minipage}
  \end{minipage}
  \begin{minipage}{0.9\linewidth}
    \centering
    \includegraphics[width=\linewidth]{./pic/stufig.pdf}
    \textit{Stufig}
  \end{minipage}
  \caption{Vertikale Struktur}
  \label{fig:vertikaleStruktur}
\end{figure}

\begin{figure}[htbp]
  \centering
  \includegraphics[width=.8\linewidth]{./pic/regelmaessig.pdf}\\
  \textit{Regelmäßig}\\[.5em]
  \includegraphics[width=.8\linewidth]{./pic/zufaellig.pdf}\\
  \textit{Zufällig}\\[.5em]
  \includegraphics[width=.8\linewidth]{./pic/geklumpt.pdf}\\
  \textit{Geklumpt}
  \caption{Horizontale Struktur}
  \label{fig:horizontaletruktur}
\end{figure}

Zu einer Mischung in zwei Bestandesschichten kommt es oft bei
Vergesellschaftung einer Haupt-- und einer dienenden Baumart, also
beispielsweise Eiche (haupt) und Hainbuche (dienend im Unterstand)
oder Tanne (haupt) und Erle (dienend im Vorwald). Zu einer stufigen
Struktur kommt es wenn die Baumarten räumlich getrennt sind und eine
unterschiedliche Höhenentwicklung haben oder unterschiedlich alt sind.

\subsection{Baumartenanteile}
\label{ssec:Baumartenanteile}

Die Anteile der Baumarten an der Bestandesfäche sind aufgrund der
Menge der von ihnen \emph{aufgenommenen Standortsressourcen} zu
bestimmen. Daher ist der Anteil nicht zu einem Zeitpunkt sondern für
einen \emph{Zeitraum} zu bestimmen. Dieser Zeitraum ist idealer weise
die Umtriebszeit mit der \emph{höchsten} durchschnittlichen
\emph{Gesamtertragsleistung}. Die Aufnahme der Standortsressourcen:
\begin{itemize}
\item Lichtenergie
\item Wasser
\item Bodennährstoffe
\item CO$_2$
\end{itemize}
kann in der Regel nicht direkt für die beteiligten Baumarten bestimmt
werden. Die zugewachsene Biomasse sollte allerdings in recht guter
Relation zu ihr stehen. Aber auch die zugewachsene Gesamtbiomasse
(Wurzel, Stamm, Äste, Blätter, Früchte) lässt sich schwer beobachten
und somit wird man sich auf Ableitungen aus Brusthöhendurchmesser
(BHD) und Schaftlänge oder aus sektionsweisen Durchmessermessungen
stützen müssen. Aber auch der Wirkungsgrad der Ressourcenausnutzung
wird nicht nur zwischen verschiedenen Baumarten sondern auch zwischen
verschiedenen Baumaltern und Dimensionen unterschiedlich sein. Man
braucht also einen \emph{Referenzwert}, der für einen konkreten
Standort, in der Regel charakterisiert mittels Oberhöhenbonität und
Ertragsniveau und einen Baum mit bestimmten Charakteristiken (Alter,
BHD, Höhe, Kronenlänge, Blattmasse), angibt, wie groß \emph{die
Fläche} sein muss, um den beobachteten \emph{Zuwachs} in einem
Reinbestand, gebildet von identen Bäumen, zu erzielen. Bei
Reinbeständen, aus annähernd gleichen Bäumen, sollte die Summe dieser
Einzelbaumflächen der Bestandesfläche, solange es keine unproduktiven
Lücken oder Baumschädigungen gibt, entsprechen. Im heterogenen
Reinbestand oder im Mischbestand kann über die Relation Summe der
Referenzflächen zu Bestandesfläche abgeleitet werden ob die
Produktivität höher oder niedriger als im homogenen Reinbestand ist.

An dieser Stelle stell sich die berechtigte Frage, ob nicht die
Anteilsbestimmung nach den Wertzuwächsen erfolgen sollte, da ja im
Wirschaftswald meist nicht ein Mehrzuwachs an Biomasse, sondern ein
Mehrzuwachs an Erträgen von Bedeutung ist. Das Problem dabei ist, dass
die Flächenanteile aufgrund des Resourchenverbrauches und nicht
aufgrund eines Wertzwuwachses erfolgen muss. Die Bewertung kann erst
anschließend erfolgen. Dazu bedarf es neben dem Refernzwert des
Flächenbedarfs zusätzlich auch einen des Wert--, Schaftholz--,
Derbholzzuwachses oder einer anderen Zielgröße, der auf dieser Fläche
erfolgt. Wobei insbesondere zur Beschreibung des Wertzuwachses weitere
Baumkriterien angegeben werden müssten. Entsprechend anzupassen ist
allerdings die Umtriebszeit bei der der gewählte Zielwert kulminiert.

Zu der Frage, wie groß der Anteil einer Baumart im Revier werden kann,
schrieb \cite[S.~233]{wiedemann1951Ertragskunde} sich dabei auf
fremdländische Baumarten beziehend: \frqq Trotz der 50jährigen
Erprobung sind wir \dots bei diesen ausländischen Holzarten nie davor
sicher, daß irgendwelche Krankheiten oder Schädlinge eingeschleppt
werden, die schweren Schaden anrichten können. Daher sollte keine
einzige ausländische Holzart in irgendeinem Revier in so großen
Flächen angebaut werden, daß ihr Ausfall die Ertragsfähigkeit des
Reviers entscheidend schädigt, die Grenze liegt vielleicht bei
20~Prozent der Revierfläche. Gewisse, oft sogar sehr große Einbußen
durch Insekten und andere Waldkrankheiten nehmen wir ja auch bei
unseren einheimischen Holzarten als selbstverständlich hin.\flqq{}
Zwar ohne konkrete Zahlen und auf die Fichte konzentriert, beschreibt
\cite[S.~394]{assmann1961Waldertraskunde} die Problematik: \frqq Weit
ernster und gewichtiger als die Folgen echter Bodendegradation, die
nach Ausmaß und zeitlichem Fortschreiten noch exakt zu untersuchen
bleibt, sind die Gefährdungen der Betriebssicherheit und der
nachhaltigen Erträge bei reinem Fichtenanbau auf Großflächen
anzusehen, welche auf zoopathogene, phytopathogene und klimatische
Schadwirkungen zurückgehen. Man vergegenwärtige sich, was geschehen
würde, wenn eines Tages in unsern Fichtenbeständen eine Pilzkrankheit
oder ein Schädling von so fataler Wirkung auftreten sollte, wie etwa
Graphium ulmi oder so schwer zu bekämpfen wie Phylloxera vastatrix,
die Reblaus! Schon solche Überlegungen sollten uns vor zu ausgedehntem
Reinanbau der Fichte zurückhalten.\flqq{} Dass der großräumige
\emph{Totalausfall} einer Baumart durchaus nichts ungewöhnliches ist,
zeigen, neben dem erwähnten Ulmensterben, auch Blasenrost oder
Eschentriebsterben. Da es nicht immer der Fall sein wird, aus einer
große Anzahl, von wirtschaftlich vielversprechenden Baumarten,
auswählen zu können, würde ich, soweit möglich, empfehlen, den
regionalen Anteil, einer Baumart, von 1/3 nicht zu überschreiten.

\section{Unterschiede}
\label{sec:unterschiede}

Eine Entscheidung, für oder gegen Misch-- oder Reinbestand, wird
darauf beruhen, dass in einem konkreten Bestand von dem einen Vorteile
gegenüber dem anderen erwartet werden. Darum werden einige, bisher
beobachtete, Unterschiede der beiden aufgezählt. Dabei ist zu beachten
dass es kaum Vorteile gibt die immer und überall zutreffen und es oft
vorkommt, dass die Vorteile des einen, auf einem anderen Standort, zu
Nachteilen werden.

\subsection{Zuwachsleistung}
\label{ssec:zuwachs}

Bei der Baumartenwahl, und auch bei der Entscheidung zwischen Misch
oder Reinbestand, ist die erwartete Ertragsleistung ein wesentliche
Punkt im Wirtschaftswald. Der Ertrag hängt unter anderem von der
Holzmarktlage, der Stammqualität und --dimmension, dem Ausfallsrisiko
und dem zugewachsenen verwertbaren Holzvolumen ab.

Untersuchungen zum Wachstum von Mischbeständen sind Zusammenfassend
schon seit längerer Zeit in Lehrbüchern z.B.\
\cite{wiedemann1951Ertragskunde}, \cite{assmann1961Waldertraskunde}
oder \cite{mitscherlich1978WaldWachstumUmwelt} sowie in Fachartikeln
z.B.\ \cite{baader1942WasLeistetDerMischbestand},
\cite{kennel1965MischbestandFichteBuche},
\cite{mitscherlich196566ReinUndMischbestand},
\cite{guericke2001mischungBuLae}, \cite{pretzsch2003mischwald},
\cite{bauhus2004mischung} oder \cite{kelty2006mischwald}
publiziert. Generell kann man die Produktivität von Beständen mit
Lichtbaumarten, durch einbringen von Schattbaumarten im Unterbestand
steigern. Die Produktivität von Beständen mit Schattbaumarten kann man
nur dann steigern, wenn die derzeit verwendete bestandesbildende
Schattbaumart, eine geringere Produktivität hat, als diejenige, die
man beimischen will. Wobei es dann in Hinblick auf die
Gesamtwuchsleistung zielführender erscheint, die schlechtwüchsigere
mit der besserwüchsigen zu ersetzen, als mit ihr zu mischen. Eine
Zeitmischung zwischen Pionier-- und Klimaxbaumarten kann ebenfalls zu
einer Steigerung der Wuchsleistung führen, wobei ein, zu erwartender,
höherer Aufwand für Pflege und Ernte dem Mehrzuwachs
gegenüberzustellen wäre. Auch scheint es möglich auf Standorten mit
wenig Stickstoff durch die Beimischung von stickstofffixierenden
Baum-- und Straucharten, wie etwas Robinie, Erle, Geweihbaum,
Gletitschie, Gelbholz, Schnurbaum, Goldregen, Sanddorn, Ölweide,
Blasenstrauch, Ginster oder andern zum Teil nicht verholzenden
Leguminosen, die Zuwachsleistung zu erhöhen.

\cite{pretzsch2017mischwald} zeigen, dass einige beobachtete
Mischbestände zweier Baumarten mehr leisten sollen, als zwei, in Summe
gleich große, Reinbestände. Diese Aussage bedeutet aber keinesfalls,
dass der Mischbestand je Fläche und Zeit mehr produziert, als der
Reinbestand der produktiveren, der beiden verglichenen,
Mischbaumarten. Z.B. sei der laufende Zuwachs:\\
Fichte: 10\,fm/ha/Jahr\\
Buche: 8\,fm/ha/Jahr\\
50\,\%Fichte + 50\,\% Buche: 9.3\,fm/ha/Jahr\\
In dem fiktiven Beispiel produziert der Mischbestand aus Fichte und
Buche mehr als die Erwarteten 9\,fm, aber weniger als Fichte
rein. Ähnliches gilt auch für die Biomasseproduktion.
\cite{sterba2018struktur} führen den Effekt der Mehrleistung
hauptsächlich auf die unterschiedliche Struktur der Bestände und
weniger auf die Baumartenmischung zurück. Gleichzeitig gibt es aber
auch Beobachtungen, das der Mischbestand weniger leistet als zwei
gleich große Reinbestände.
 
\cite{vospernik2021grundflaechenzuwachs} konnte auf die
Zuwachsleistung positive als auch negative Mischungen feststellen. So
wurden Zuwachssteigerungen der Fichte bei Beimischung von Tanne oder
Lärche, Zuwachsverluste bei der Mischung mit Buche beobachtet. Bei
\cite{kindermann2018siteIndex} zeigte die Fichte bei Beimischung von
Schwarzerle eine Steigerung der Oberhöhenbonität, bei Beimischung von
Esche und Lärche eine gering, bei Birke und Tanne eine stärkere, bei
Buche eine noch stärkere und bei Beimischung von Weißkiefer die größte
Reduzierung der Oberhöhenbonität. Dabei wird aber darauf hingewiesen,
dass diese Bonitätsveränderung durch die beigemsichten Baumarten,
durch die Etablierung von Mischbeständen auf anderen Standorten als
Reinbestände und durch, von Reinbeständen abweichende Behandlung der
Baumarten in den Mischbestände, hervorgerufen werden kann.

\cite[S.~259]{muench1928KlimarassenDerDouglasie} berichtet, dass
zwischen Douglasienparzellen Buchen als Trennungsreihen gepflanzt
wurden und sich deren Höhenwuchs nach dem der benachbarten
Douglasienherkunft und deren Höhe, richtete. Dies wird im Waldbau
gelegentlich als treibende Baumart beschrieben. Teilweise kann dieser
Effekt auch durch die Bestandesdichte bedingt sein.
\cite{boehmerle1903BestandesdichteUndBestandeshoehe},
\cite{schmied1928UeberDenEinflussDerBestandesdichteAufDieBestandeshoeheInJuengerenBuchenbestaenden},
\cite[S.~75--76]{wiedemann1951Ertragskunde} oder
\cite[S.~46--47]{assmann1961Waldertraskunde} berichten von einem, bis
zu einem bestimmten Grad mit der Zunahme der Bestandesdichte,
zunehmenden Höhenzuwachs.
\cite{zundel1960ErtragskundlicheUntersuchungenInZweialtrigenBestaendenNordwuerttembergs}
berichtet, dass ein Kiefernaltbestand, dessen Höhenwachstum schon
recht gering war, zu dem Zeitpunkt, als der Unterbestand in seine
Krone einzuwachsen begann, plötzlich wieder den Höhenzuwachs seigern
konnte.

Das \emph{Hauptproblem} bei derartigen Untersuchungen ist die
\emph{Aufteilung der Bestandesfläche} auf die beteiligten
Mischbaumarten, die Beurteilung ob der \emph{Standort} der
verglichenen Rein-- und Mischbestände wirklich \emph{gleich} ist und
die Problematik ob die \emph{Bäume} in den Beständen
\emph{vergleichbar} sind (gleiches Alter, gleiche Dimension, gleiche
Blattmasse, gleiches Durchwurzelungsvolumen, \dots).

\subsection{Qualität}
\label{ssec:qualitaet}

Bei Holzverkauf entscheidet neben der Menge, die Qualität den
Preis. Mischungen zwischen dienender- und Hauptbaumart werden oft
aufgrund einer erwarteten Steigerung der Holzqualität (Eiche,
Hainbuche) begründet. Umgekehrt gibt es aber auch Beobachtungen das
die Qualität durch die Mischung leidet. So beschreibt
\cite{tiebel2016qualitaet} bei der Mischung von Fichte und Buche, dass
die Astreinigung und die Ausbildung gerader Schäfte im Horst
begünstigt ist, nicht jedoch in der Kontaktzone zwischen den
Arten. \cite{rais2019MischungQualitaet} beschreibt, dass besonders die
Mischung der Buche mit den Lichtbaumarten Eiche oder Kiefer zu
signifikanten Qualitätsminderungen bzw.\ Festigkeitsverlusten
führt. Buchen in Reinbeständen besaßen die höchsten
Steifigkeiten. Nach \cite{pretzsch2016holzqualitaetMischRein} halten
sich die Beobachtung von Festigkeitszu-- und --abnahme zwischen
Misch-- und Reinbestand die Waage, die Asitgkeit war in Mischbeständen
höher, die Holzdichte bleibt unverändert. Reaktionsholz war in
Mischbeständen höher und wurde meist durch exzentrische Kronen,
Schiefstand oder unregelmäßige Jahrringbreiten verursacht. Ursache für
den Qualitätsverlust soll die Heterogenität der Bestandesstruktur sein,
die in Misch-- und ungleichaltrigen Beständen höher ist als in
gleichaltrigen Reinbeständen.

\subsection{Stabilität}
\label{ssec:stabilitaet}

Forstliche Planung kann nur dann zielführend sein, wenn die Wälder das
gesteckte Ziel erreichen können. Eine Grundvoraussetzung dafür ist
deren Stabilität. Wesentliche stabilitätsbeeinflussende Faktoren sind
die Bestandesbehandlung (Durchforstungsstärke und Häufigkeit), die
Umtriebszeit, der Standort der die Wuchsleistung (z.B.\ Einfluss der
Baumhöhe auf Windwurfrisiko), die Durchwurzelungsintensität (z.B.\ oft
weniger auf Pseudogley), die Exponiertheit (Windstärke) aber auch
Gefährdung durch Trockenheit oder Überflutung, Frost oder Hitze. Diese
Faktoren kommen bei verschiedenen Baumarten unterschiedlich zur
Ausprägung. Leider leisten stabile Baumart nicht immer gleichzeitig
auch den höchsten Ertrag. So liegt es nahe stabile Baumarten mit
ertragreichen zu mischen um beides gleichzeitig zu erreichen.

\cite{schmidtVogt1987Sturmstabilitaet} zeigte, dass Fichten Tannen
Mischbestände gegenüber \emph{Windwurf stabiler} sind als reine
Fichtenbestände. \cite{molisch1937allelopathie} zeigte, dass sich
Pflanzen nicht nur gegenseitig Ressourcen entziehen können, sondern
auch durch Abgabe von chemischen Verbindungen Wechselbeziehungen
auslösen können, was allgemein als \emph{Allelopathie} bezeichnet
wird.  Die Mischung von Baumarten mit einem unterschiedlichem
Wasserhaushaltsregime kann den Stress einer Baumart erhöhen. So
zeigten \cite{schume2004wasserFichteBuche}, dass die Beimischung von
Buche zur Fichte den \emph{Wasserstress} der Fichte
\emph{steigert}. Nach \cite{nothdurft2020mischbestand} wurde die
Resistenz der Fichte durch die Beimischung der Buche reduziert und bei
der Mischung von Eiche und Weißkiefer wurde bei beiden sowohl deren
Produktivität als auch deren Resistenz gegenüber dem Reinbestand
verringert. \cite{thurm2016mischungDougBuStress} beobachteten, dass
auf einem Standort mit Rotbuche -- Douglasie Mischung ein
Trockenstress für die Douglasie in der Mischung geringer war als im
Reinbestand, hingegen litt die Rotbuche in der Mischung stärker unter
dem Trockenstress als in Reinbestand.

\cite{sylvie2021mischwaldInsekten} zeigten, dass eine
Baumartenmischung das \emph{Borkenkäferrisiko} für anfällige Baumarten
(Fichte, Lärche) reduzieren konnte. Gleichzeitig wurde aber das Risiko
für weniger attraktive Baumarten (Kiefer) erhöht. Krankheitserreger
mit obligatorischem \emph{Wirtswechsel}, beispielsweise
Kieferndrehrost zwischen Kiefer und Pappel, können sich in
Mischbeständen besser verbreiten.

In Zeiten, möglicherweise drastischer Standortsveränderungen, ist man
bestrebt das \emph{Risiko, des Versagens} von Baumarten, zu
\emph{reduzieren}. Neben abiotischen Faktoren (Trockenheit) können
auch biotische (Krankheiten, Wild, Rüsselkäfer, Mäuse, \dots)
bestimmte Baumarten ausfallen lassen. In einer ähnlichen Situation ist
man, wenn man neue, in der Region noch nicht oder kaum vertretene
Baumarten, verwenden will. Eine in der Natur zu beobachtende Lösung
besteht darin, dass mehrere Baumarten, in der Verjüngung, in Form von
kleinflächiger-- oder Einzelmischung, eine Chance gegeben wird und
sich im Laufe der Bestandesentwicklung die konkurrenzstärkste
durchsetzt. Bei einer künstlichen Bestandesbegründung ergibt sich das
Problem, dass die Verjüngung, aufgrund dessen das man ja damit rechnet
dass eine Baumart ausfällt, so dicht sein muss, wie sie ohne weiterer
Baumart wäre, was Kosten verursacht. Auch ist es Möglich, dass eine
Baumart zwar in der Jugend konkurrenzstark ist und die restlichen
Baumarten verdrängt, dann aber dennoch frühzeitig ausfällt, oder das
diese Baumart nicht das Bestandesziel darstellt und mit Eingriffen das
Überleben der anderen Baumarten erst ermöglicht wird. Dadurch, dass
der Zeitpunkt, wann eine Baumart ausfällt, im Vorhinein nicht bekannt
ist, ist die Planung dieser Zeitmischung, bei der die früher zu
entnehmende Baumart, zum Zeitpunkt der Entnahme, einzeln beigemischt
sein sollte, um keine bleibenden Lücken zu hinterlassen, nicht möglich
und wird und wenn überhaupt, dann nur zufällig, den gewünschten
Effekt, dass die verbleibenden Baumarten, den \emph{freiwerdenenden
  Raum übernehmen} und nutzen, zeigen. Die dafür nötige Einzlmischung
kann nicht über die gesamte Bestandesentwicklung aufrecht erhalten
werden. Wenn die ausgefallene oder im absterben begriffene Baumart
geerntet werden soll, ist dies bei einer Einzelmischung in der Regel
mehr Arbeit und kann zu Verletzungen des verbleibenden Bestandes
führen. Die dabei entstehenden Lücken können das Risiko des
verbleibenden Bestandes z.B.\ hinsichtlich Windwurf oder Sonnenbrand
erhöhen. Eine durch die Lücke erhöhte Wasserreisebildung kann die
Qualität mindern. Bei größeren Lücken kann eine erwünschte Verjüngung
z.B.\ durch vermehrte Schneeablagerung erschwert werden oder nötige
Verbisschutzmaßnahmen, durch die Kleinflächigkeit, unverhältnismäßig
teuer werden. Auch ist es möglich, dass erst durch die beigemische
Baumart die Steressituation der anderen Baumart derart gesteigert
wird, das sie ausfällt. Die horstweise oder kleinbestandesweise
Trennung der Baumarten vermeidet viele dieser Nachteile, kann aber den
freiwerdenden Raum, einer ausfallenden Baumart, nicht, oder nur in
sehr eingeschränktem Ausmaß, übernehmen. Das Ausfallsrisiko einer
Baumart kann, durch die Verwendung mehrerer Baumarten, in dem Ausmaß
gespreitet werden, in dem damit der Flächenanteil der Baumart
reduziert wird. Die verschiedenen Baumarten sollten, für diesen Zweck,
räumlich getrennt aufwachsen.

Die Streuung des Risikos, durch die Verwendung mehrerer Baumarten, ist
zu begrüßen. Dabei ist zu erwarten dass der gewünschte Effekt bei der
Wahl von z.B.\ Stieleich und Traubeneiche geringer, als bei der Wahl
von Schwarzkiefer und Traubeneiche, sein wird. Die Auswahl mehrerer
Baumarten, die annähernd dieselbe Leistung erbringen, wird,
insbesondere wenn die Auswahl auf die derzeitigen Hauptbaumarten
beschränkt ist, schwer fallen. Dies wird zur Folge haben, dass
entweder nur eine Baumart gewählt wird, mit der Folge dass das Risiko
nicht gestreut wird, oder dass Leistungsverluste in Kauf genommen
werden. Selbst in Zeiten, in denen keine langfristige Klimaveränderung
unterstellt wurde, war die Baumartenwahl des Folgebestandes
schwierig. So warf etwa \cite{jentsch1911fruchtwechsel} die Frage auf,
ob ein Fruchtwechsel auch in der Forstwirtschaft Sinn mache. Was von
\cite{sieber1919Holzartenwechsel} und
\cite{fabricius1924Holzartenwechsel} nochmals aufgegriffen, aber nie
eindeutig beantwortet, wurde. Allerdings beobachtete
\cite{simak1951Baumartenwechsel} dass es in natürlichen Plenterwäldern
durchaus zu einem gegenseitigen Wechsel des Standortes, zwischen den
Baumarten, kommt. Umso schwerer ist die Baumartenwahl bei sich
ändernden Standortsverhältnissen, insbesondere wenn, von der möglichen
Größe der Veränderung, nur ein relativ weiter Rahmen bekannt ist. Wenn
mehrere Baumarten ausgewählt wurden, ist die Entscheidung, ob diese
besser in Reinbeständen oder in Mischbeständen stocken sollen, und wie
groß deren Anteile sein sollen, individuell zu beantworten. Dabei
spielt, neben den ausgewählten Baumarten, der Standort, die Größe und
Qualität der bereitstellbaren Arbeitsleistung von Seiten des Besitzers
sowie dessen voraussichtlichen Nachbesitzern aber auch
Berücksichtigung von Naturschutzinteressen und gesetzlichen
Rahmenbedingungen, ob das Produkt nachgefragt (werden) wird als auch
die Bestände der gesamten umgebenden Region und deren zu erwartende
Bewirtschaftung, eine Rolle.

\subsection{Biodiversität}
\label{ssec:biodiversitaet}

Auch die Biodiversität sollte berücksichtigt werden, wobei
\cite{heinrichs2019Reinbestand} oder \cite{schall2017Reinbestand} zu
dem Ergebnis kommen, das gleichaltrige Reinbestände, verschiedener
Baumarten, auf regionaler Ebene oft eine höher Biodiversität aufweisen
als ungleichaltrige Mischbestände.  Auch wenn Reinbestände überwiegend
von einer Baumart gebildet werden, ist bei einer Aneinanderreihung von
Reinbeständen, mit verschiedenen Baumarten, dennoch eine gewisse
\emph{Artenvielfalt} in einer Region zu erzielen. Wobei zu erwarten
ist, dass der Charakter beispielsweise zwischen Fichten-- und
Buchenreinbeständen unterschiedlicher sein wird, als zwischen
Fichten--Buchen--Mischbeständen. Am meisten Variation ist zu erwarten
wenn sowohl Fichten-- und Buchenreinbeständen als auch
Fichten--Buchen--Mischbestände vertreten sind. Meist stehen nicht nur
zwei, sondern wesentlich mehr Baumarten für einen Standort zur Auswahl
und es gibt unzählige Abwandlungen hinsichtlich Struktur und
Mischungsform. Diese Fülle an Möglichkeiten wird im
\emph{Wirtschaftswald}, in der Regel durch das Ziel, \emph{Erträge} zu
erbringen, eingeschränkt.

\subsection{Bewirtschaftung}
\label{ssec:bewirtschaftung}

In der Regel ist die Bewirtschaftung in Mischbeständen komplexer als
in Reinbeständen.  Daher muss, selbst bei gut ausgebildeten
Fachkräften, häufiger mit Fehlentscheidungen oder auch mit Opfern
zugunsten der Mischung gerechnet werden, welche zu Ertragseinbußen und
zur Erhöhung eines Risikos führen können. Wenn Zuwachssteigerungen
durch Mischungen möglich sind, sind diese den damit möglicherweise
höheren Aufwendungen bei Verjüngung, Pflege und Ernte
gegenüberzustellen. So sind die größten Synergien zwischen Baumarten
dann zu erwarten, wenn die eine Baumart stets nur andre Baumarten als
Nachbar hat, also Einzelmischung vorliegt. Bei einer Mischung von
Herrschender und dienender Baumart (z.B.\ Stieleiche mit Hainbuche)
ist dies meist gegeben. Bei Mischungen von zwei herrschenden Baumarten
wird von der Einzelmischung üblicher weise abgeraten und die
horstweise Mischung empfohlen. Damit liegen aber somit zwei
Reinbestände, mit einem, mehr oder weniger langen, gemeinsamen
Bestandesrand, vor.

\cite{knoke2007mischwald} betonet, dass bei der Bewirtschaftung
insbesondere die Risiken zu senken sind. Dies kann durch Erhaltung
oder Steigerung der \emph{Flexibilität} erreicht werden. Eine
Möglichkeit wäre die Verkürzung der Produktionszeit, wie beim
Schnellwuchsbetrieb nach Gehrhardt, eine Andere die Verwendung
mehrerer möglichst diverser Baumarten. Diese Diversifizierung kann
sich auf die Preise (entwickeln sich parallel, unabhängig oder
gegenläufig) oder die Risiken (Wind, Schnee, Insekten, Pilze)
beziehen. Günstig sind Mischungen von Baumarten deren sich Preise
unabhängig oder gegenläufig entwickeln und mit Risiken die sich nicht
überschneiden. Bei einer großflächigen Mischung, d.h.\ kleine
Reinbestände, aus Fichte und Buche soll das Risiko eines Verlustes von
30\,\% im Fichtenbetrieb auf 22\,\% in Fichten Buchen Betrieb
absinken. Gleichzeitig sinkt aber auch die Ertragschancen im
Mischbetrieb gegenüber dem Fichtenbetrieb. Das geringere Risiko wird
mit einer Reduktion eines erzielbaren Ertrages erkauft.
\cite{knoke2007mischwaldB} erwartet sich eine ökonomische
Überlegenheit von der kleinflächigen Mischung (25\,x\,40\,m) von
Fichten und Buchen gegenüber großflächigen Mischungen, was wohl auf
die, hier von Mischungsanteil der Baumarten unabhängige, Annahme, das
bei der kleinflächigen nur 20\,\% bei der großflächigen hingegen
40\,\% der Fichten ausfallen werden, zurückzuführen ist. Dies führt
dazu, dass der Ertrag der kleinflächigen Mischung, bei einem hohen
Fichtenanteil, jenen der großflächigen übersteigt. Wie im
Abschnitt~\ref{ssec:stabilitaet} beschrieben, kann eine Mischung die
Stabilität sowohl steigern aber auch herabsetzen.

\section{Schlussfolgerungen}
\label{sec:schlussfolgerungen}

Weder Misch-- noch Reinbestand sollen erzwungen werden. Wenn sich
natürlich Verjüngung aus mehreren Baumarten einstellt, soll diese zwar
in Richtung der gewünschten Bestockung geführt, dabei aber nicht, mit
übertriebenem Eifer, alle anderen Baumarten ausgemerzt werden. Oft
scheint es sogar angebracht der natürlichen Entmischung
entgegenzuwirken und nicht, der in der jeweiligen Entwicklungsphase
konkurrenzstärksten Baumart, die Überhand gewinnen zu lassen, solange
dies nicht dem Bewirtschaftungsziel entspricht. In Naturverjüngung,
mit Einzelmischung, ist es ausreichend die gewünschten Baumarten im
Zuge von Pflegemaßnahmen von den stärksten Bedrängern zu befreien
anstatt, mit hohem Aufwand, alle nicht erwünschten Baumarten zu
entfernen.

Insbesondere wenn auf künstlichem Weg eine gemischt Verjüngt
angestrebt wird, müssen die an der Mischung beteiligten Baumarten,
bestimmte \emph{Aufgaben} erfüllen. Z.B.\ übernimmt im Eichen
Hainbuchenwald hauptsächlich die Eiche die Wertleistung und die
Hainbuche hilft als dienende Baumart dieses Ziel, in Form von
Schaftreinigung, zu erreichen. Als Aufgaben können Erhöhung der
Wuchs-- bzw.\ Wertleistung, Verbesserung der Stoffkeisläufe durch
unterschiedliche Wurzelenergie und Verbesserung des Streuumsatzes,
Erhöhung der Stabilität (z.B.\ gegen Sturm), Verminderung der
Gefährdung (z.B.\ durch Insekten, Pilze oder Feuer), \dots den
Baumarten zugeschrieben werden. Dass diese Aufgaben von den gewählten
Baumarten tatsächlich erfüllt werden kann, sollte in der Praxis öfter
beobachtet worden und idealerweise durch Messungen quantifizierbar
sein. Mögliche Nachteile, die durch die Mischung entstehenden können,
sowie deren Abwägung mit den Vorteilen, darf nicht unterblieben.

Da jede Baumart eine Aufgabe erfüllen muß und ohnedies meist nur
wenige optimale Baumarten für einen Standort existieren, sollten nur
\emph{zwei bis höchsten drei Baumarten} einen Mischbestand
bilden. Dies schließt nicht aus bzw.\ ist es wünschenswert, dass
vereinzelt \emph{weitere Baumarten}, im Bestand vertreten sind. Damit
wird die Möglichkeit geschaffen, diese eingesprengten Baumarten, im
Folgebestand über Naturverjüngung horstweise einzubringen.

Eine Mischung muß nicht über die gesmmte Bestandesentwicklung in
gleichen Anteilen bestehen. Verschiebungen von einer zur anderen
Baumart bis hin zu \emph{Zeitmischungen}, also eine Baumart wird
früher als die andere geerntet und / oder eine Baumart wird später als
eine andere Verjüngt (Vorwald, Voranbau, Unterbau), sind
möglich. Wesentlich ist, das die Entwicklung des Mischbestandes im
Vorhinein \emph{geplant} wird. Diese Entwicklung sollte sich zum
überwiegenden Teil aus \emph{natürlichen Entwicklungsprossesen}
ergeben, welche durch moderate Eingriffe, zur Erreichung der
Zielbestockung, modifiziert werden. Dies erfordert eine gewisse
Ausbildung bzw.\ Erfahrung des Planenden.

In Mischbeständen spielt die \emph{Mischungsform} (Einzel, Trupp,
Gruppe, Horst, Fläche, Reihe, Streifen) eine entscheidende Rolle, die
neben den beteiligten Baumarten und deren Aufgaben auch durch das
Alter und die Bestandesstruktur beeinflusst wird. So kann eine trupp--
bis gruppenweise Mischung in der Jugend, im Alter in eine
Einzelmischung übergehen. Damit die Hainbuche im Unterstand ihre
Funktion der Schaftreinigung bei den Eichen übernehmen kann, müssen
die beiden möglichst einzelstammweise gemischt sein. Betrachtet man
hingegen im Altbestand den Kronenraum sind lediglich Eichen die
Nachbarn von Eichen. Solange es die Funktion nicht erfordert, sind in
der Verjüngungsphase Einzelmischungen zu vermeiden, da es insbesondere
wenn konkurrenzstarke Schatbaumarten beteiligt sind, im laufe der Zeit
zu konkurrenzbedingter Entmischung kommt. Je Schlagfläche mit jeweils
einer Baumart zu verjüngen, wird in vielen Fällen eine praktikable
Lösung darstellen. Die Größe der Schlagfläche wird sich in der Regel
an die Besitzgröße und das typische Nutzungsverhalten anpassen. Dies
wird, zur oft zielführendsten Baumartenmischung nicht auf Bestandes--,
sondern auf regionaler Ebene, durch ein \emph{Mosaik von Reinbeständen
  unterschiedlicher Baumarten}, führen. Diese Reinbestände sollten
nicht zu groß sein, um im Folgebestand die Möglichkeit zu haben,
diesen zum Großteil mit den Baumarten der Nachbarbestände natürlich zu
verjüngen und damit einen \emph{Baumartenwechsel} zu realisieren. Je
nach beteiligter Baumart sollte damit der überwiegende Teil der
Bestandesfläche nicht weiter als 30--70\,m vom Bestandesrand entfernt
liegen, was bei quadratischer Form 1/3--2\,Hektar entspricht.

Wie groß die \emph{Anteile der Baumarten} im Bestand bzw.\ in der
Region sein sollen, kann nicht pauschal beantwortet
werden. Reinbestände sind auf Grund ihrer geringen Größe weniger
problematisch. Sobald jedoch eine ganze Region von einer, durch die
Bewirtschaftung geförderten, Baumart dominiert wird, muss mit einer
Erhöhung des Ausfallsrisikos gerechnet werden. Daher sollte nach
Möglichkeit der Anteil einer Baumart in einer Region 1/3 nicht
überschreiten, also mindestens drei in etwa gleich häufig vertretene
Baumarten haben, wobei der Begriff Region nicht zu groß (500 ha Wald)
gefasst und gesondert nach Höhenstufen betrachtet werden muss. Dieses
1/3 darf natürlich in Regionen, die von Natur aus von einer Baumart
dominiert werden, überschritten werden.

Dieses Thema kann nie generell beantwortet werden, sondern muss von
Bestand zu Bestand entscheiden werden. Genau so wie es keine
universelle Baumart und kein universelles Waldbausystem gibt, darf
auch keine Baumart, kein System kategorisch im Vorhinein
ausgeschlossen werden. Von Einzelmischung bis Reinbestand, von
Ungleichaltrig zu Gleichaltrig, von einschichtig zu mehrschichtig, vom
Kurzumtrieb bis zur natürlichen Mortalität, von der Einzelstammnutzung
zum Kahlschlag, von der Naturverjüngung zur Containerflanzung
fremdländischer Hybriden, von der außer Nutzung Stellung bis hin zur
Produktion von Furnierholz sollte zunächst alles als Möglichkeit in
Betracht genommen werden dürfen.

%\addcontentsline{toc}{section}{Stichwortverzeichnis}
%\printindex

\addcontentsline{toc}{section}{Literatur}
\bibliography{literature}

%Autor: Georg Kindermann

\end{document}
