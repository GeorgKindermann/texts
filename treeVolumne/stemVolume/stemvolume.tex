\documentclass[twocolumn]{scrartcl}

\usepackage[utf8]{inputenc}
\usepackage[T1]{fontenc}
\usepackage{lmodern}
\usepackage[english]{babel}

\usepackage[sc]{mathpazo} % or option osf
\usepackage{newpxmath}

\usepackage[comma,authoryear]{natbib}
\bibliographystyle{apalike}

\usepackage{amsmath}
\usepackage{enumitem}
\usepackage{adjustbox}

\usepackage{siunitx}
\usepackage{etoolbox}
\renewcommand{\bfseries}{\fontseries{b}\selectfont}
\robustify\bfseries
\newrobustcmd{\B}{\bfseries}

\usepackage{graphicx}
\usepackage[a4paper, margin=1mm, includefoot, footskip=15pt]{geometry}

\usepackage[pdftitle={Estimating the stem volume of a tree by using section wise measured diameters}
, pdfauthor={Georg Kindermann}
, pdfsubject={Tree, Stem, Volume, Forestry, Baum, Schaft, Volumen, Forestwirtschaft}
, pdfkeywords={Tree, Stem, Volume, Forestry, Baum, Schaft, Volumen, Forestwirtschaft}
, pdflang={en}
, hidelinks
, pdfpagemode={UseNone}]{hyperref}

\nonfrenchspacing
\sloppy

\title{Estimating the stem volume of a tree by using section wise measured diameters}
\author{Georg Kindermann}

\begin{document}

\twocolumn[
  \begin{@twocolumnfalse}
    \maketitle
    \begin{abstract}
      Tree volumes are typically not measured directly. Some inventories just
      measure the diameter in 1.3\,m height, measure or estimate the tree height
      and use a form factor to estimate stem volume. Those form factors are
      based on relations between diameter (and height, crown length, \dots) and
      estimated tree volumes. Those tree volumes typical are estimated by using
      section wise measured stem diameters. Some methods to estimate tree volume
      using section wise diameters are compared. From this investigation the
      methods interpolating between the measured diameters and building the
      integral of those areas along the stem could be recommended. From the
      interpolation methods the one which are monotonic and non overshooting
      showed good results. Of the interpolation methods compared, the method
      developed by Steffen in 1990 showed the best performance.
    \end{abstract}
  \end{@twocolumnfalse}
]

\tableofcontents

\section{Introduction}

The volume of a tree could be distinguished in the parts below ground (roots)
and those above ground (stem, branches, twigs, needles, leaves). Typical those
volumes are not measured directly. By using laser scanners diameters along the
stem could be measured. Still in most cases only the diameter in 1.3\,m height
(diameter at breast height -- DBH) is measured. Sometimes tree height is
measured in addition, sometimes it is estimated. Sometimes also the height of
the crown base. With DBH, height and height of the crown base a form factor is
used to calculate the stem volume. The form factor was estimated by using the
relation between DBH, height and volume from trees where diameters along the
stem (section wise) have been measured. The volume was estimated by assuming
that the stem has the shape of a circle and calculating the stem area at the
position of the measured diameter with $d^2 \pi/4$ and multiplying this area
with the length of the section where this diameter was in the middle. This
method was e.\,g.\ described by
\cite{huber1839volumen,kruenitz1781encyclopaedie,kaestner1766mathematische}.

In order to get some information about the real shape of the stem, sometimes two
cross-sectional diameters are measured. Their arithmetic mean will represent the
stem area in a good way while their geometric mean would be theoretical better
\citep{Tischendorf1925holzmassenermittlung}. When diameters are measured
orthogonal as stem disc radii the arithmetic mean of the basal areas of the
individual radii is close to the true basal area of the stem
\citep{Siostrzonek1958grundflaechenzuwachs}.

Nowadays, it is possible to interpolate between the measured diameters and
calculate the integral along the stem. The simplest interpolation between
measured diameters will be linear, but also splines have been used since decades
(see fig.~\ref{fig:vMethods}). As the real volume is not known the different
methods could not be ranked by comparing how near they come to the real volume.
Instead, it is possible to leave one measured diameter away and let the
interpolation method estimate the diameter on this now missing position.

\begin{figure}[htbp]
  \centering
  \includegraphics[width=.95\columnwidth]{./pic/vMethods}
  \caption{Different methods to estimate stem volume.}
  \label{fig:vMethods}
\end{figure}

There are different definitions of stem volume. Here the stem is the
part of the tree which goes from the ground continuously to the top. There is
only one stem per tree. The stem includes the bark. Even if many older stems are
hollow, this volume which was wood in the past but does not exist now anymore is
here still counted to the stem volume. The height where a tree was or will be
cut plays no role when estimating the stem volume. Here it is considered that
trees are standing on flat land. If it is standing on a slope the tree height is
typically measured starting from the slope side, but this point might move over
time e.\,g.\ when the diameter is growing. So the point should be there where
the root and the shoot come together when the tree was a seedling what will be
approximately at the height of half of the stem.


\section{Data}

The dataset provided by
\cite{Didion2024sectionWiseStemDiametersText,Didion2024sectionWiseStemDiametersData}
has been used. The dataset contains 15\,684 measured spruce stems (Picea abies)
which have been selected. The DBH ranges form 0.6\,cm to 100.8\,cm, height
ranges form 1.6\,m to 48.6\,m. Most observations are in the range of DBH
10--50\,cm and height 5--35\,m (see fig.~\ref{fig:dhScatter} and
\ref{fig:dhCumObs}).

\begin{figure}[htbp]
  \centering
  \includegraphics[width=.95\columnwidth]{./pic/dhScatter}
  \caption{DBH and Height scatter plot of measured spruce.}
  \label{fig:dhScatter}
\end{figure}

\begin{figure}[htbp]
  \centering
  \includegraphics[width=.95\columnwidth]{./pic/dhCumObs}
  \caption{Cumulative observations along DBH and Height.}
  \label{fig:dhCumObs}
\end{figure}

The stem diameter was measured starting at the height of 1\,m in sections of
2\,m as long as the diameter was larger than 7\,cm 1\,m above the measurement
position. If there was a final segment larger than 7\,cm but shorter than 2\,m
the diameter of this segment was measure at half of its length. The diameter of
the final segment smaller than 7\,cm was measured in its middle. In addition,
the DBH (Diameter at Breast Height) at 1.3\,m height and for some trees the
diameter at 0.65\,m height was measured. The number of measured diameters is
given in tab.~\ref{tab:nDiamAtHeight}.

\begin{table}[htbp]
  \centering
  \begin{tabular}{l *{7}{c@{~~}}}
    h & 0.65 & 1 & 1.3 & 3 & 5 & 7 & 9 \\
    n & 2964 & 15\,177 & 15\,684 & 14\,651 & 13\,930 & 13\,016 & 12\,026\\
    \hline
    h & 11 & 13 & 15 & 17 & 19 & 21 & 23 \\
    n & 11\,037 & 10\,025 & 8\,939 & 7\,812 & 6\,634 & 5\,604 & 4\,617\\
    \hline
    h & 25 & 27 & 29 & 31 & 33 & 35 & 37\\
    n & 3\,649 & 2\,748 & 1\,886 & 1\,045 & 475 & 164 & 58\\
    \hline
    h & 39 & 41 & 43 & 45 & Coarse & d=7 & DTop \\
    n & 24 & 8 & 4 & 1 & 10\,063 & 15\,417 & 13\,079\\
  \end{tabular}
  \caption{Measured diameters at different tree heights.}
  \label{tab:nDiamAtHeight}
  \raggedright
  \footnotesize{h .. Height im [m]\\
  n .. Number of measured diameters.\\
  Coarse .. Diameter of the last coarse segment of the stem\\
  d=7 .. Height where diameter is 7\,cm\\
  DTop .. Diameter of stem top with a dimater lower than 7\,cm}
\end{table}


\section{Methods}

As the traditional method needs the diameter in the middle of each segment,
those have been measured. For methods which are interpolating between the
measurements at least the diameters at the ground and at the tree top are needed
in addition. As the tree top does not have a diameter of 0\,cm an assumption
needs to be made how wide the diameter at the tree top is. For the used data it
is not relevant how large the diameter for small trees is as they are not
present there. Anyhow, it was assumed that the diameter is 0.1\,cm at height
0\,m and is increasing up to 0.7\,cm following the relation: $d_t = 0.1 + 0.6
\cdot \tanh(15 \cdot h^{1.7})$ where the diameter at the tree top ($d_t$) is
in~cm and the tree height ($h$) is in~m (see fig. \ref{fig:dAtTop}).

\begin{figure}[htbp]
  \centering
  \includegraphics[width=.95\columnwidth]{./pic/dAtTop}
  \caption{Assumed dependence between tree height and diameter at the top of the tree.}
  \label{fig:dAtTop}
\end{figure}

The tree diameter at the ground was estimated creating a regression of the form
$d = c_0 + c_1 \dot (x - h)^{c_2}$ where $d$ is the diameter at height $h$.
$c_0$, $c_1$, $c_2$ are coefficients which are estimated for each tree
individually and $x$ is the highest used height. For $x$ a height of 5\,m was
selected and all measured diameters until the height of 5\,m have been used to
estimate the regression coefficients. By using this function the diameter at
height~0 could be estimated. For some interpolation methods it is useful to have
an addition point behind the last point of the interpolation range. So a
hypothetical diameter at height -1\,m was also estimated with this equation. For
2\,828 trees $d_0$ and $d_{-1m}$ could be estimated by using the measured
diameters up to a height of 5\,m. 40 tree individual estimated $d_0$ where
larger than $2\cdot$DBH and 412 $d_{-1m}$ where larger than $3\cdot$DBH and have
been replaced with an estimate based only on DBH and tree height. For those and
the remaining trees where $d_0$ and $d_{-1m}$ could not be estimated on a tree
individual basis it was estimated with eq.~\ref{eq:d0} and \ref{eq:dm1}.
\begin{align}
  d_{a} & = \begin{cases}
    DBH + 1.3 & \text{if } h \geq 1.3\,m\\
    h + d_t   & \text{otherwise}
\end{cases}\\
\label{eq:d0}
d_0 & = d_{a} \cdot (1 + e^{c_0 + c_1\cdot \ln(h) + c_2\cdot d_{a}})\\
\label{eq:dm1}
d_{-1m} & = d_{a} \cdot (1 + e^{c_3 + c_4\cdot \ln(h) + c_5\cdot d_{a}})
\end{align}

Where $d_{a}$ is an auxiliary diameter, $DBH$ diameter in 1.3\,m height in cm,
$h$ the tree height in m, $d_t$ the diameter at the tree top, $d_0$ diameter in
0\,m height, $d_{-1m}$ hypothetical diameter 1\,m below ground and $c_0$ to
$c_5$ coefficients.

For each tree, one measured diameter was omitted, and for this position, a
diameter was calculated using various interpolation methods. This was done for
each single measured diameter as long as interpolation was possible for this
position. Interpolation is not possible for the first and last observation. The
estimated diameter was compared with the observed. The interpolation methods
which have been compared are: % given in tab.~\ref{tab:interpolMethods}:

%\begin{table}[htbp]
\begin{description}[nolistsep]
  \item[A] Linear
  \item[B] B-Spline second, third and fourth order
  \item[C] Finite Difference (cubic interpolation, no tension parameter)
  \item[D] Cubic cardinal splines, with tension parameter 0.1, 0.5, 0.9 and 1
  \item[E] Fritsch Carlson \citep{fritschCarlson1980interpolation}
  \item[F] Fritsch Butland \citep{fritschButland1984interpolation}
  \item[G] Steffen \citep{steffen1990interpolation}
  \item[H] Akima \citep{akima1970interpolation}
  \item[I] cubic spline keeping first and second derivatives continuous
  \item[J] Hyman89 \citep{Dougherty1989interpolation}
  \item[K] C2MP -- adjust first derivatives according to the simplest monotonicity preserving scheme \citep{huynh1993interpolation}
  \item[L] C2MP2 -- adjust first derivatives according to the extended monotonicity preserving scheme \citep{huynh1993interpolation}
  \item[M] Huyn-Rational -- the first derivatives are estimated through a rational function which guarantees monotonicity \citep{huynh1993interpolation}
  \item[N] Van-Albada -- Van Albada type of approximation for the first derivatives \citep{huynh1993interpolation}
  \item[O] Van-Leer -- Van Leer type of approximation for the first derivatives \citep{huynh1993interpolation}
  \item[P] Brodlie -- similar to Fritch Butland but takes into account the non-uniformity of the knots
  \item[Q] Hyman Non Negative
  \item[R] Bessel
  \item[S] Fukasawa
  \item[T] Quadratic Lagrange Interpolation
\end{description}
%\caption{Methods used for interpolation.}
%\label{tab:interpolMethods}
%\end{table}

After finding a good method to interpolate between measured diameters the
estimated stem volumes using
\begin{itemize}[nolistsep]
  \item Traditional sectional step method using the diameter in the middle of the section
  \item Linear interpolation
  \item Cubic spline
  \item Selected interpolation method
\end{itemize}
are calculated. Differences of the volumes between those methods will be shown.

All calculations have been made with the programming language Julia 1.11.3
\citep{Julia-2017}. The interpolations have been done with the packages
Interpolations v0.15.1 \citep{juliaMathInterpolations}, Dierckx v0.5.4
\citep{juliaDierckx} and PPInterpolation v0.7.3 \citep{juliaPPInterpolation}.
Integrals are calculated using the package Integrals v4.5.0
\citep{juliaIntegrals} with the method QuadGKJL \citep{laurie1997calculation}.
Regressions are estimated with the package GLM v1.9.0 \citep{juliaGLM} and
nonlinear with LsqFit v0.15.0 \citep{juliaLsqFit}. For data handling DataFrames
v1.7.0 \citep{juliaDataframes} and CSV v0.10.15 \citep{juliaCsv} have been used.
For plotting and formatting Plots v1.40.9 \citep{juliaPlots}, LaTeXStrings
v1.4.0 \citep{juliaLaTeXStrings}, Format v1.3.7 \citep{juliaFormat} and for
trend lines Loess v0.7.2 \citep{juliaLoess} have been used. For statistics
StatsBase v0.34.4 \citep{juliaStatsBase} and Statistics 1.11.3
\citep{Julia-2017} have been used.


\section{Results}

To estimate the diameter in 0\,m height ($d_0$) using eq.~\ref{eq:d0} and the
hypothetical diameter 1\,m below ground ($d_{-1m}$) using eq.~\ref{eq:dm1} the
coefficients $c_0$ to $c_5$ have been estimate as:

\begin{tabular}{l S[table-format=2.6] S[table-format=1.6] S[table-format=2.6] S[table-format=2.6]}
      & {Coef.} &{Std. Error}  &  {Lower 95\%} &   {Upper 95\%}\\
\hline
$c_0$ & -5.87166  & 0.306584 & -6.47255 & -5.27077 \\
$c_1$ &  1.48349  & 0.100722 &  1.28608 &  1.68090 \\
$c_2$ & -0.008991 & 0.001367 & -0.01167 & -0.00631 \\[.4em]
$c_3$ & -5.39029  & 0.321219 & -6.01986 & -4.76071 \\
$c_4$ &  1.64376  & 0.106548 &  1.43493 &  1.85259 \\
$c_5$ & -0.012548 & 0.001531 & -0.01555 & -0.00954
\end{tabular}

The residuals along prediction, $d_a$, tree height and height\,/\,$d_a$ are
shown for $d_0$ in fig.~\ref{fig:d0Resid} and for $d_{-1m}$ in
fig.~\ref{fig:dm1Resid} which show no trend.

\begin{figure}[htbp]
  \centering
  \includegraphics[width=.95\columnwidth]{./pic/d0Resid}
  \caption{Residual plots of $d_0$ model.}
  \label{fig:d0Resid}
  \footnotesize{Residual .. (predicted $d_0$ - "observed" $d_0) / d_a - 1$}
\end{figure}

\begin{figure}[htbp]
  \centering
  \includegraphics[width=.95\columnwidth]{./pic/dm1Resid}
  \caption{Residual plots of $d_{-1m}$ model.}
  \label{fig:dm1Resid}
  \footnotesize{Residual .. (predicted $d_{-1m}$ - "observed" $d_{-1m}) / d_a - 1$}
\end{figure}

The number of observations between measured and estimated diameters at different
tree heights is given in tab.~\ref{tab:nInterpol}. Regular measured Coarse
diameters higher than 27\,m are not shown as the number of observations is
declining, and most methods show the same behavior as shown at the lower
diameters. The last segment of coarse wood, the diameter of 7\,cm and the
diameter of the top are also given.

\begin{table*}[htbp]
\begin{adjustbox}{width=\linewidth}
\small
\begin{tabular*}{\linewidth}{l | *{19}{c@{~~}}}
  h & 0.65 & 1 & 1.3 & 3 & 5 & 7 & 9 & 11 & 13 & 15 & 17 & 19 & 21 & 23 & 25 & 27 & Coarse & d=7 & DTop\\
  \cline{1-20}
  n & 2\,960 & 2\,960 & 14\,527 & 13\,807 & 12\,895 & 11\,909 & 10\,940 & 9\,959 & 8\,890 & 7\,781 & 6\,616 & 5\,596 & 4\,615 & 3\,647 & 2\,746 & 1\,884 & 9\,824 & 7\,986 & 12\,519
\end{tabular*}
\end{adjustbox}
\caption{Number of interpolated diameters at different tree heights.}
\label{tab:nInterpol}
\end{table*}

The median between the estimated and the observed diameters for different
heights and methods are shown in tab.~\ref{tab:difMeadInterpol}. The linear
method (A) is overestimating in the concave sections on the stem basis and
overestimating in the convex sections. Many methods find quite good estimates in
the higher stem sections and show median differences lower than 0.03\,cm. Really
strong in this region is Method G having 6~times the lowest median difference.
In addition, it has the lowest median difference at DTop and in the sections of
lower height the differences reach a maximum of 0.18\,cm. Methods I, J, K, Q and
R belong 4~times to the methods of the lowest differences. Method E has with
$-0.09$\,cm the lowest maximum median difference, but it underestimates most of
the time. Up to a height of 3\,m method E has 2~times the lowest difference.

\begin{table*}[htbp]
  \begin{adjustbox}{width=1.95\columnwidth}
  \small
  \begin{tabular*}{\linewidth}{l |@{~~~} *{19}{S[detect-weight,mode=text,table-format=1.2]}}
h & {0.65} & {1} & {1.3} & {3} & {5} & {7} & {9} & {11} & {13} & {15} & {17} & {19} & {21} & {23} & {25} & {27} & {Coarse} & {d=7} & {DTop}\\
    \cline{1-20}
A & 0.51 & 0.39 & 0.30 & 0.31 & \B 0.00 & -0.05 & -0.10 & -0.10 & -0.10 & -0.15 & -0.15 & -0.15 & -0.20 & -0.20 & -0.25 & -0.25 & \B 0.00 & -0.14 & -0.15 \\
B2 & -0.16 & 0.13 & -0.33 & -0.36 & -0.01 & \B -0.00 & 0.01 & 0.01 & -0.01 & 0.01 & 0.01 & -0.01 & \B 0.01 & 0.01 & -0.01 & -0.03 & 0.06 & -0.09 & 0.14 \\
B3 & -0.16 & 0.12 & -0.29 & -0.30 & 0.04 & -0.02 & 0.01 & \B 0.00 & -0.01 & 0.01 & 0.01 & -0.01 & \B 0.01 & 0.01 & \B 0.00 & -0.03 & 0.05 & -0.09 & 0.27 \\
B4 & -0.18 & 0.12 & -0.31 & 0.77 & -0.14 & 0.06 & -0.01 & \B 0.00 & -0.01 & \B -0.00 & 0.02 & -0.01 & \B 0.01 & 0.01 & 0.01 & -0.02 & 0.05 & -0.08 & 0.33 \\
C & -0.07 & 0.10 & -0.14 & -0.09 & -0.11 & -0.03 & -0.01 & -0.02 & -0.04 & -0.03 & -0.02 & -0.05 & -0.04 & -0.05 & -0.09 & -0.08 & 0.06 & -0.09 & -0.07 \\[.35em]
D0.1 & 0.07 & 0.05 & 0.07 & 0.24 & -0.07 & -0.04 & -0.05 & -0.05 & -0.07 & -0.07 & -0.07 & -0.10 & -0.10 & -0.12 & -0.16 & -0.15 & 0.03 & -0.08 & -0.04 \\
D0.5 & \B 0.03 & 0.17 & 0.25 & 0.29 & -0.05 & -0.05 & -0.06 & -0.07 & -0.09 & -0.10 & -0.10 & -0.13 & -0.14 & -0.17 & -0.20 & -0.19 & -0.08 & \B 0.03 & 0.26 \\
D0.9 & -0.04 & 0.29 & 0.45 & 0.34 & -0.02 & -0.05 & -0.09 & -0.09 & -0.10 & -0.13 & -0.13 & -0.15 & -0.19 & -0.20 & -0.24 & -0.24 & -0.20 & 0.14 & 0.56 \\
D1 & -0.05 & 0.32 & 0.50 & 0.36 & \B -0.00 & -0.05 & -0.10 & -0.10 & -0.10 & -0.15 & -0.15 & -0.15 & -0.20 & -0.20 & -0.25 & -0.25 & -0.23 & 0.16 & 0.64 \\
E & -0.08 & 0.08 & \B -0.03 & \B -0.03 & -0.09 & -0.02 & -0.01 & -0.02 & -0.04 & -0.03 & -0.01 & -0.05 & -0.04 & -0.05 & -0.09 & -0.08 & 0.06 & -0.09 & -0.06 \\[.35em]
F & -0.08 & 0.07 & 0.12 & 0.08 & -0.06 & -0.02 & -0.01 & -0.01 & -0.03 & -0.02 & -0.01 & -0.04 & -0.03 & -0.04 & -0.06 & -0.07 & 0.05 & -0.08 & -0.02 \\
G & -0.18 & 0.05 & 0.08 & 0.08 & -0.08 & -0.02 & \B 0.00 & \B 0.00 & \B 0.00 & 0.01 & 0.02 & \B -0.00 & \B 0.01 & \B 0.00 & -0.02 & -0.03 & 0.06 & -0.08 & \B -0.01 \\
H & -0.06 & 0.07 & 0.07 & 0.14 & -0.02 & -0.03 & -0.02 & -0.02 & -0.03 & -0.03 & -0.03 & -0.06 & -0.05 & -0.06 & -0.09 & -0.09 & 0.05 & -0.09 & 0.05 \\
I & -0.23 & 0.13 & -0.30 & -0.29 & 0.03 & -0.02 & 0.01 & \B 0.00 & -0.01 & \B 0.00 & 0.01 & -0.01 & \B 0.01 & 0.01 & \B 0.00 & -0.03 & 0.06 & -0.08 & 0.05 \\
J & -0.23 & 0.11 & -0.10 & -0.15 & 0.02 & -0.02 & 0.01 & \B 0.00 & -0.01 & \B 0.00 & 0.01 & -0.01 & \B 0.01 & 0.01 & \B 0.00 & -0.03 & 0.06 & -0.08 & 0.05 \\[.35em]
K & -0.23 & 0.11 & -0.10 & -0.15 & 0.02 & -0.02 & 0.01 & \B 0.00 & -0.01 & \B 0.00 & 0.01 & -0.01 & \B 0.01 & 0.01 & \B 0.00 & -0.03 & 0.06 & -0.08 & 0.05 \\
L & -0.14 & \B 0.01 & -0.10 & -0.10 & -0.14 & -0.02 & \B 0.00 & \B 0.00 & -0.02 & -0.01 & \B 0.00 & -0.03 & -0.02 & -0.04 & -0.05 & -0.06 & 0.11 & -0.13 & -0.11 \\
M & -0.06 & 0.06 & 0.10 & 0.11 & -0.07 & -0.03 & -0.02 & -0.02 & -0.04 & -0.03 & -0.02 & -0.05 & -0.04 & -0.06 & -0.08 & -0.09 & 0.05 & -0.09 & \B 0.01 \\
N & \B -0.03 & 0.06 & 0.25 & 0.27 & -0.03 & -0.03 & -0.02 & -0.03 & -0.04 & -0.03 & -0.03 & -0.06 & -0.05 & -0.07 & -0.08 & -0.09 & 0.04 & -0.08 & 0.07 \\
O & -0.06 & 0.06 & 0.15 & 0.17 & -0.05 & -0.03 & -0.02 & -0.02 & -0.04 & -0.03 & -0.02 & -0.05 & -0.04 & -0.06 & -0.08 & -0.09 & 0.05 & -0.09 & 0.03 \\[.35em]
P & -0.07 & 0.07 & 0.12 & 0.08 & -0.06 & -0.02 & -0.01 & -0.01 & -0.03 & -0.02 & -0.01 & -0.04 & -0.03 & -0.04 & -0.06 & -0.07 & 0.05 & -0.08 & 0.04 \\
Q & -0.23 & 0.13 & -0.30 & -0.29 & 0.03 & -0.02 & 0.01 & \B 0.00 & -0.01 & \B 0.00 & 0.01 & -0.01 & \B 0.01 & 0.01 & \B 0.00 & -0.03 & 0.06 & -0.08 & 0.05 \\
R & -0.18 & 0.17 & -0.30 & -0.40 & -0.14 & -0.02 & \B 0.00 & 0.01 & \B 0.00 & 0.01 & 0.02 & \B -0.00 & 0.02 & \B 0.00 & -0.02 & -0.02 & 0.06 & -0.08 & 0.06 \\
S & -0.05 & 0.07 & 0.11 & 0.11 & -0.08 & -0.03 & -0.02 & -0.02 & -0.04 & -0.03 & -0.02 & -0.05 & -0.05 & -0.06 & -0.09 & -0.09 & 0.05 & -0.09 & \B 0.01 \\
T & -0.14 & 0.22 & -0.39 & -0.95 & -0.31 & -0.03 & -0.03 & -0.03 & -0.03 & \B -0.00 & \B -0.00 & -0.03 & -0.03 & -0.03 & -0.03 & \B -0.00 & 0.07 & -0.09 & 0.13
\end{tabular*}
\end{adjustbox}
\caption{Median difference [cm] between estimated - observed diameter.}
\label{tab:difMeadInterpol}
\end{table*}

As the arithmetic mean could be strong influenced by outliers it was calculated
using only values which are between the 2.5\,\% and 97.5\,\% quantile. It is
shown in tab.~\ref{tab:difMeanInterpol}. G, J and K belong 10~times and B2, B3,
B4, I and Q belong 9~times to the methods with the lowest differences. Methods J
and K have with $0.09$\,cm the lowest maximum difference. Up to a height of 3\,m
method G has 2~times the lowest difference.

\begin{table*}[htbp]
  \begin{adjustbox}{width=1.95\columnwidth}
  \small
  \begin{tabular*}{\linewidth}{l |@{~~~} *{19}{S[detect-weight,mode=text,table-format=1.2]}}
h & {0.65} & {1} & {1.3} & {3} & {5} & {7} & {9} & {11} & {13} & {15} & {17} & {19} & {21} & {23} & {25} & {27} & {Coarse} & {d=7} & {DTop}\\
    \cline{1-20}
A & 0.78 & 0.48 & 0.49 & 0.54 & 0.01 & -0.07 & -0.09 & -0.10 & -0.12 & -0.13 & -0.15 & -0.17 & -0.19 & -0.24 & -0.26 & -0.26 & -0.02 & -0.13 & -0.15 \\
B2 & -0.07 & 0.14 & -0.30 & -0.62 & -0.01 & -0.01 & \B 0.00 & 0.01 & \B -0.00 & \B 0.00 & \B 0.01 & \B -0.00 & \B 0.01 & \B -0.00 & \B -0.00 & \B -0.02 & 0.05 & -0.07 & 0.09 \\
B3 & -0.08 & 0.12 & -0.24 & -0.55 & 0.07 & -0.03 & 0.01 & \B 0.00 & \B -0.00 & \B 0.00 & \B 0.01 & \B -0.00 & \B 0.01 & \B -0.00 & \B 0.00 & \B -0.02 & 0.05 & -0.07 & 0.13 \\
B4 & -0.17 & 0.12 & -0.27 & 0.54 & 0.04 & \B 0.00 & \B 0.00 & \B 0.00 & \B -0.00 & \B 0.00 & \B 0.01 & \B -0.00 & \B 0.01 & 0.01 & 0.02 & \B -0.02 & 0.05 & -0.07 & 0.09 \\
C & 0.16 & 0.08 & -0.06 & -0.10 & -0.14 & -0.03 & -0.02 & -0.02 & -0.03 & -0.03 & -0.03 & -0.04 & -0.05 & -0.07 & -0.07 & -0.08 & 0.04 & -0.08 & -0.08 \\[.35em]
D0.1 & 0.33 & \B 0.02 & 0.22 & 0.43 & -0.07 & -0.05 & -0.05 & -0.06 & -0.07 & -0.08 & -0.08 & -0.10 & -0.11 & -0.14 & -0.15 & -0.16 & \B 0.01 & -0.06 & -0.05 \\
D0.5 & 0.21 & 0.18 & 0.45 & 0.50 & -0.04 & -0.06 & -0.07 & -0.08 & -0.09 & -0.10 & -0.11 & -0.13 & -0.15 & -0.18 & -0.20 & -0.21 & -0.12 & \B 0.05 & 0.25 \\
D0.9 & 0.08 & 0.35 & 0.69 & 0.58 & \B -0.00 & -0.07 & -0.08 & -0.10 & -0.12 & -0.13 & -0.14 & -0.16 & -0.18 & -0.23 & -0.25 & -0.25 & -0.25 & 0.16 & 0.56 \\
D1 & 0.05 & 0.39 & 0.75 & 0.60 & 0.01 & -0.07 & -0.09 & -0.10 & -0.12 & -0.13 & -0.15 & -0.17 & -0.19 & -0.24 & -0.26 & -0.26 & -0.28 & 0.19 & 0.64 \\
E & 0.15 & 0.05 & 0.05 & \B 0.04 & -0.11 & -0.03 & -0.02 & -0.02 & -0.03 & -0.03 & -0.03 & -0.04 & -0.04 & -0.07 & -0.07 & -0.08 & 0.04 & -0.08 & -0.07 \\[.35em]
F & 0.09 & 0.04 & 0.24 & 0.17 & -0.07 & -0.03 & -0.01 & -0.01 & -0.03 & -0.02 & -0.02 & -0.04 & -0.03 & -0.05 & -0.05 & -0.07 & 0.03 & -0.07 & -0.02 \\
G & \B -0.02 & \B 0.02 & 0.18 & 0.16 & -0.10 & -0.02 & \B 0.00 & \B 0.00 & \B -0.00 & \B 0.00 & \B 0.01 & \B -0.00 & \B 0.01 & -0.01 & -0.01 & \B -0.02 & 0.05 & -0.07 & -0.03 \\
H & 0.08 & 0.03 & 0.18 & 0.21 & -0.01 & -0.03 & -0.02 & -0.02 & -0.04 & -0.03 & -0.04 & -0.05 & -0.05 & -0.07 & -0.08 & -0.09 & 0.03 & -0.08 & 0.03 \\
I & -0.10 & 0.12 & -0.25 & -0.55 & 0.07 & -0.03 & 0.01 & \B 0.00 & \B -0.00 & \B 0.00 & \B 0.01 & \B -0.00 & \B 0.01 & \B -0.00 & \B 0.00 & \B -0.02 & 0.05 & -0.07 & 0.02 \\
J & -0.07 & 0.09 & \B -0.04 & -0.08 & 0.02 & -0.03 & 0.01 & \B 0.00 & \B -0.00 & \B 0.00 & \B 0.01 & \B -0.00 & \B 0.01 & \B -0.00 & \B 0.00 & \B -0.02 & 0.05 & -0.07 & 0.02 \\[.35em]
K & -0.07 & 0.09 & \B -0.04 & -0.09 & 0.02 & -0.03 & 0.01 & \B 0.00 & \B -0.00 & \B 0.00 & \B 0.01 & \B -0.00 & \B 0.01 & \B -0.00 & \B 0.00 & \B -0.02 & 0.05 & -0.07 & 0.02 \\
L & 0.10 & -0.05 & -0.06 & -0.05 & -0.17 & -0.03 & -0.01 & -0.01 & -0.02 & -0.01 & \B -0.01 & -0.02 & -0.02 & -0.04 & -0.05 & -0.06 & 0.09 & -0.12 & -0.14 \\
M & 0.13 & 0.03 & 0.22 & 0.21 & -0.07 & -0.03 & -0.02 & -0.02 & -0.03 & -0.03 & -0.03 & -0.05 & -0.05 & -0.07 & -0.07 & -0.09 & 0.04 & -0.07 & 0.01 \\
N & 0.14 & \B 0.02 & 0.41 & 0.40 & -0.02 & -0.03 & -0.02 & -0.03 & -0.04 & -0.04 & -0.04 & -0.05 & -0.05 & -0.08 & -0.08 & -0.09 & 0.02 & -0.06 & 0.08 \\
O & 0.13 & \B 0.02 & 0.29 & 0.28 & -0.05 & -0.03 & -0.02 & -0.02 & -0.04 & -0.03 & -0.04 & -0.05 & -0.05 & -0.07 & -0.08 & -0.09 & 0.03 & -0.07 & 0.03 \\[.35em]
P & 0.09 & 0.04 & 0.24 & 0.17 & -0.07 & -0.03 & -0.01 & -0.01 & -0.03 & -0.02 & -0.02 & -0.04 & -0.03 & -0.05 & -0.05 & -0.07 & 0.03 & -0.07 & 0.04 \\
Q & -0.10 & 0.12 & -0.25 & -0.55 & 0.07 & -0.03 & 0.01 & \B 0.00 & \B -0.00 & \B 0.00 & \B 0.01 & \B -0.00 & \B 0.01 & \B -0.00 & \B 0.00 & \B -0.02 & 0.05 & -0.07 & 0.02 \\
R & -0.04 & 0.18 & -0.28 & -0.61 & -0.20 & -0.02 & \B 0.00 & 0.01 & \B -0.00 & \B 0.00 & \B 0.01 & \B -0.00 & \B 0.01 & -0.01 & -0.01 & \B -0.02 & 0.05 & -0.07 & 0.02 \\
S & 0.13 & 0.03 & 0.25 & 0.22 & -0.08 & -0.03 & -0.02 & -0.02 & -0.04 & -0.04 & -0.04 & -0.05 & -0.05 & -0.07 & -0.08 & -0.09 & 0.03 & -0.07 & \B -0.00 \\
T & -0.12 & 0.25 & -0.39 & -1.44 & -0.45 & -0.06 & -0.02 & -0.02 & -0.02 & -0.02 & -0.02 & -0.03 & -0.03 & -0.05 & -0.03 & \B -0.02 & 0.06 & -0.07 & 0.08
  \end{tabular*}
\end{adjustbox}
\caption{Mean difference [cm] between estimated - observed diameter skipping upper and lower 2.5\,\% of observations.}
\label{tab:difMeanInterpol}
\end{table*}

Like for the arithmetic mean also for calculating the standard deviation only
values which are between the 2.5\,\% and 97.5\,\% quantile have been used. The
results are shown in tab.~\ref{tab:stdDifInterpol}. The lowest deviations show
the linear interpolation (A) with 13 times having the lowest value followed by
D1 with 12 and D0.9 with 11 times. Methods D1 has with $0.80$\,cm the lowest
maximum deviance. A similar result is given by comparing the distance between
2.5\,\% and 97.5\.\% quantile (tab.~\ref{tab:quntRangeInterpol}) and between
25\,\% and 75\,\% quantile (tab.~\ref{tab:iqrInterpol}).

\begin{table*}[htbp]
  \begin{adjustbox}{width=1.95\columnwidth}
  \small
  \begin{tabular*}{\linewidth}{l |@{~~~} *{19}{S[detect-weight,mode=text,table-format=1.2]}}
h & {0.65} & {1} & {1.3} & {3} & {5} & {7} & {9} & {11} & {13} & {15} & {17} & {19} & {21} & {23} & {25} & {27} & {Coarse} & {d=7} & {DTop}\\
    \cline{1-20}
A & 1.16 & 0.62 & 0.61 & 0.74 & 0.29 & \B 0.26 & \B 0.28 & \B 0.29 & \B 0.30 & \B 0.33 & \B 0.35 & 0.38 & \B 0.40 & \B 0.43 & \B 0.46 & \B 0.48 & \B 0.29 & \B 0.31 & \B 0.42 \\
B2 & 0.90 & 0.56 & 0.60 & 1.27 & 0.45 & 0.31 & 0.32 & 0.34 & 0.36 & 0.39 & 0.42 & 0.45 & 0.48 & 0.51 & 0.56 & 0.58 & 0.32 & 0.33 & 0.77 \\
B3 & 0.89 & 0.56 & 0.59 & 1.32 & 0.59 & 0.35 & 0.34 & 0.35 & 0.38 & 0.41 & 0.44 & 0.47 & 0.50 & 0.54 & 0.59 & 0.62 & 0.31 & 0.34 & 1.68 \\
B4 & 0.91 & 0.56 & 0.63 & 2.91 & 1.73 & 0.80 & 0.49 & 0.44 & 0.45 & 0.49 & 0.53 & 0.57 & 0.62 & 0.68 & 0.75 & 0.84 & 0.33 & 0.36 & 2.63 \\
C & 1.03 & \B 0.55 & 0.55 & 0.65 & 0.31 & 0.28 & 0.29 & 0.31 & 0.32 & 0.35 & 0.38 & 0.41 & 0.43 & 0.46 & 0.49 & 0.51 & 0.30 & 0.32 & 0.47 \\[.35em]
D0.1 & 1.09 & 0.56 & 0.59 & 0.64 & \B 0.27 & 0.27 & \B 0.28 & 0.30 & 0.31 & 0.34 & 0.36 & 0.39 & 0.42 & 0.44 & 0.47 & 0.49 & 0.30 & 0.32 & 0.44 \\
D0.5 & 0.94 & 0.56 & 0.65 & 0.70 & \B 0.27 & \B 0.26 & \B 0.28 & \B 0.29 & 0.31 & \B 0.33 & 0.36 & 0.38 & 0.41 & 0.44 & \B 0.46 & \B 0.48 & 0.30 & 0.32 & 0.43 \\
D0.9 & 0.82 & 0.59 & 0.73 & 0.75 & 0.29 & \B 0.26 & \B 0.28 & \B 0.29 & \B 0.30 & \B 0.33 & \B 0.35 & \B 0.37 & \B 0.40 & \B 0.43 & \B 0.46 & \B 0.48 & 0.31 & 0.33 & 0.43 \\
D1 & \B 0.80 & 0.60 & 0.76 & 0.77 & 0.29 & \B 0.26 & \B 0.28 & \B 0.29 & \B 0.30 & \B 0.33 & \B 0.35 & 0.38 & \B 0.40 & \B 0.43 & \B 0.46 & \B 0.48 & 0.31 & 0.33 & \B 0.42 \\
E & 1.06 & 0.57 & 0.49 & 0.54 & 0.31 & 0.28 & 0.29 & 0.31 & 0.33 & 0.35 & 0.38 & 0.41 & 0.43 & 0.46 & 0.49 & 0.51 & 0.30 & 0.32 & 0.48 \\[.35em]
F & 1.02 & 0.57 & 0.51 & 0.50 & 0.28 & 0.28 & 0.29 & 0.31 & 0.33 & 0.35 & 0.38 & 0.41 & 0.43 & 0.46 & 0.50 & 0.52 & 0.30 & 0.33 & 0.50 \\
G & 1.03 & 0.58 & 0.48 & \B 0.48 & 0.29 & 0.28 & 0.30 & 0.32 & 0.33 & 0.36 & 0.39 & 0.42 & 0.45 & 0.48 & 0.51 & 0.53 & 0.31 & 0.32 & 0.52 \\
H & 1.00 & 0.58 & 0.57 & 0.54 & 0.29 & 0.27 & 0.29 & 0.31 & 0.32 & 0.35 & 0.38 & 0.40 & 0.43 & 0.46 & 0.49 & 0.51 & 0.30 & 0.32 & 0.58 \\
I & 0.99 & 0.58 & 0.60 & 1.33 & 0.59 & 0.35 & 0.34 & 0.36 & 0.38 & 0.41 & 0.44 & 0.48 & 0.50 & 0.54 & 0.59 & 0.62 & 0.32 & 0.33 & 0.66 \\
J & 1.08 & 0.59 & 0.49 & 0.66 & 0.44 & 0.34 & 0.34 & 0.35 & 0.38 & 0.41 & 0.44 & 0.47 & 0.50 & 0.54 & 0.59 & 0.62 & 0.32 & 0.33 & 0.66 \\[.35em]
K & 1.08 & 0.59 & 0.49 & 0.64 & 0.44 & 0.34 & 0.34 & 0.35 & 0.38 & 0.41 & 0.44 & 0.47 & 0.50 & 0.54 & 0.59 & 0.62 & 0.32 & 0.33 & 0.66 \\
L & 1.15 & 0.63 & \B 0.45 & 0.49 & 0.33 & 0.29 & 0.30 & 0.32 & 0.34 & 0.37 & 0.39 & 0.43 & 0.46 & 0.48 & 0.52 & 0.54 & 0.32 & 0.32 & 0.58 \\
M & 1.04 & 0.57 & 0.51 & 0.52 & 0.28 & 0.27 & 0.29 & 0.31 & 0.32 & 0.35 & 0.38 & 0.40 & 0.43 & 0.46 & 0.49 & 0.51 & 0.30 & 0.32 & 0.54 \\
N & 1.03 & 0.58 & 0.56 & 0.57 & 0.28 & 0.27 & 0.29 & 0.31 & 0.32 & 0.35 & 0.38 & 0.40 & 0.43 & 0.46 & 0.50 & 0.52 & \B 0.29 & 0.33 & 0.55 \\
O & 1.03 & 0.58 & 0.52 & 0.53 & 0.28 & 0.27 & 0.29 & 0.31 & 0.32 & 0.35 & 0.38 & 0.40 & 0.43 & 0.46 & 0.49 & 0.51 & 0.30 & 0.32 & 0.54 \\[.35em]
P & 1.02 & 0.58 & 0.51 & 0.52 & 0.28 & 0.28 & 0.29 & 0.31 & 0.33 & 0.35 & 0.38 & 0.41 & 0.43 & 0.46 & 0.50 & 0.52 & 0.30 & 0.33 & 0.57 \\
Q & 0.99 & 0.58 & 0.60 & 1.33 & 0.59 & 0.35 & 0.34 & 0.36 & 0.38 & 0.41 & 0.44 & 0.48 & 0.50 & 0.54 & 0.59 & 0.62 & 0.32 & 0.33 & 0.66 \\
R & 0.96 & \B 0.55 & 0.58 & 1.04 & 0.36 & 0.29 & 0.30 & 0.32 & 0.34 & 0.36 & 0.39 & 0.42 & 0.45 & 0.48 & 0.52 & 0.54 & 0.31 & 0.33 & 0.61 \\
S & 1.03 & 0.58 & 0.54 & 0.55 & 0.28 & 0.27 & 0.29 & 0.31 & 0.32 & 0.35 & 0.38 & 0.40 & 0.43 & 0.46 & 0.49 & 0.51 & 0.30 & 0.32 & 0.52 \\
T & 0.82 & 0.58 & 0.62 & 1.93 & 0.57 & 0.32 & 0.32 & 0.33 & 0.35 & 0.37 & 0.40 & 0.43 & 0.46 & 0.50 & 0.53 & 0.56 & 0.32 & 0.33 & 0.71
\end{tabular*}
\end{adjustbox}
\caption{Standard deviation [cm] between estimated - observed diameter skipping upper and lower 2.5\,\% of observations.}
\label{tab:stdDifInterpol}
\end{table*}

\begin{table*}[htbp]
  \begin{adjustbox}{width=1.95\columnwidth}
  \small
  \begin{tabular*}{\linewidth}{l |@{~~~} *{19}{S[detect-weight,mode=text,table-format=1.2]}}
h & {0.65} & {1} & {1.3} & {3} & {5} & {7} & {9} & {11} & {13} & {15} & {17} & {19} & {21} & {23} & {25} & {27} & {Coarse} & {d=7} & {DTop}\\
    \cline{1-20}
A & 5.81 & 3.17 & 3.02 & 3.79 & 1.60 & \B 1.40 & \B 1.45 & \B 1.45 & \B 1.55 & \B 1.65 & \B 1.80 & \B 1.90 & \B 2.00 & 2.19 & \B 2.27 & \B 2.35 & \B 1.47 & \B 1.59 & \B 1.90 \\
B2 & 4.89 & 2.93 & 3.58 & 7.24 & 2.61 & 1.69 & 1.69 & 1.77 & 1.86 & 1.97 & 2.13 & 2.29 & 2.37 & 2.55 & 2.80 & 2.88 & 1.67 & 1.70 & 3.79 \\
B3 & 4.81 & 2.97 & 3.52 & 7.62 & 3.46 & 1.92 & 1.79 & 1.86 & 1.94 & 2.07 & 2.24 & 2.40 & 2.50 & 2.65 & 2.93 & 3.12 & 1.65 & 1.71 & 9.37 \\
B4 & 4.84 & 3.00 & 3.83 & 19.01 & 10.82 & 4.85 & 2.71 & 2.27 & 2.33 & 2.52 & 2.76 & 2.97 & 3.08 & 3.31 & 3.63 & 3.98 & 1.70 & 1.83 & 16.49 \\
C & 5.59 & 2.95 & 3.16 & 4.02 & 1.70 & 1.49 & 1.55 & 1.60 & 1.67 & 1.78 & 1.91 & 2.05 & 2.18 & 2.32 & 2.49 & 2.57 & 1.57 & 1.64 & 2.22 \\[.35em]
D0.1 & 5.97 & 2.97 & 3.11 & 3.40 & \B 1.53 & 1.44 & 1.51 & 1.55 & 1.58 & 1.72 & 1.84 & 1.96 & 2.07 & 2.22 & 2.40 & 2.42 & 1.54 & 1.64 & 2.05 \\
D0.5 & 5.23 & 2.99 & 3.25 & 3.61 & \B 1.53 & 1.42 & 1.49 & 1.52 & \B 1.55 & 1.70 & 1.81 & 1.92 & 2.03 & 2.18 & 2.35 & 2.39 & 1.53 & 1.66 & 2.01 \\
D0.9 & 4.56 & 3.04 & 3.48 & 3.87 & 1.60 & 1.41 & 1.46 & 1.47 & \B 1.55 & 1.66 & \B 1.80 & \B 1.90 & 2.01 & \B 2.16 & 2.28 & 2.36 & 1.55 & 1.69 & 1.92 \\
D1 & 4.38 & 3.13 & 3.55 & 3.94 & 1.60 & \B 1.40 & \B 1.45 & \B 1.45 & \B 1.55 & \B 1.65 & \B 1.80 & \B 1.90 & \B 2.00 & 2.19 & \B 2.27 & \B 2.35 & 1.56 & 1.70 & \B 1.90 \\
E & 6.25 & 3.00 & 2.94 & 3.13 & 1.67 & 1.51 & 1.56 & 1.63 & 1.69 & 1.80 & 1.94 & 2.07 & 2.18 & 2.32 & 2.54 & 2.57 & 1.57 & 1.64 & 2.28 \\[.35em]
F & 5.82 & 3.04 & 2.77 & 2.83 & 1.57 & 1.50 & 1.56 & 1.62 & 1.69 & 1.79 & 1.93 & 2.05 & 2.17 & 2.35 & 2.60 & 2.54 & 1.56 & 1.67 & 2.33 \\
G & 5.93 & 3.07 & \B 2.74 & \B 2.72 & 1.57 & 1.52 & 1.58 & 1.65 & 1.73 & 1.85 & 1.98 & 2.10 & 2.25 & 2.38 & 2.66 & 2.63 & 1.61 & 1.66 & 2.44 \\
H & 5.31 & 3.06 & 3.20 & 3.37 & 1.59 & 1.47 & 1.53 & 1.56 & 1.64 & 1.77 & 1.89 & 2.01 & 2.12 & 2.32 & 2.49 & 2.51 & 1.55 & 1.64 & 2.74 \\
I & 5.47 & 3.08 & 3.56 & 7.69 & 3.47 & 1.93 & 1.80 & 1.86 & 1.94 & 2.07 & 2.25 & 2.40 & 2.51 & 2.65 & 2.93 & 3.14 & 1.68 & 1.70 & 3.23 \\
J & 6.33 & 3.10 & 3.09 & 3.69 & 2.29 & 1.82 & 1.77 & 1.83 & 1.92 & 2.07 & 2.22 & 2.39 & 2.49 & 2.64 & 2.91 & 3.14 & 1.67 & 1.70 & 3.22 \\[.35em]
K & 6.33 & 3.10 & 3.09 & 3.56 & 2.29 & 1.82 & 1.77 & 1.84 & 1.92 & 2.07 & 2.22 & 2.39 & 2.49 & 2.64 & 2.92 & 3.14 & 1.67 & 1.70 & 3.20 \\
L & 7.18 & 3.28 & 2.85 & 2.84 & 1.73 & 1.56 & 1.64 & 1.66 & 1.74 & 1.88 & 2.01 & 2.18 & 2.30 & 2.38 & 2.59 & 2.70 & 1.66 & 1.65 & 2.97 \\
M & 5.91 & 3.02 & 2.83 & 2.90 & 1.56 & 1.49 & 1.55 & 1.61 & 1.67 & 1.77 & 1.92 & 2.03 & 2.14 & 2.31 & 2.57 & 2.52 & 1.56 & 1.66 & 2.54 \\
N & 5.76 & 3.10 & 2.87 & 3.00 & 1.57 & 1.48 & 1.55 & 1.61 & 1.66 & 1.80 & 1.92 & 2.03 & 2.16 & 2.34 & 2.57 & 2.54 & 1.55 & 1.67 & 2.58 \\
O & 5.88 & 3.05 & 2.82 & 2.87 & 1.55 & 1.49 & 1.54 & 1.61 & 1.67 & 1.78 & 1.92 & 2.02 & 2.15 & 2.34 & 2.57 & 2.53 & 1.55 & 1.66 & 2.55 \\[.35em]
P & 5.81 & 3.04 & 2.78 & 2.96 & 1.57 & 1.51 & 1.56 & 1.62 & 1.70 & 1.80 & 1.94 & 2.06 & 2.17 & 2.35 & 2.60 & 2.54 & 1.56 & 1.67 & 2.72 \\
Q & 5.47 & 3.08 & 3.56 & 7.68 & 3.47 & 1.93 & 1.80 & 1.86 & 1.94 & 2.07 & 2.25 & 2.40 & 2.51 & 2.65 & 2.93 & 3.14 & 1.68 & 1.70 & 3.23 \\
R & 5.28 & \B 2.92 & 3.46 & 5.87 & 1.96 & 1.54 & 1.60 & 1.67 & 1.75 & 1.85 & 2.00 & 2.15 & 2.25 & 2.40 & 2.65 & 2.67 & 1.62 & 1.66 & 2.99 \\
S & 5.86 & 3.05 & 2.84 & 3.01 & 1.54 & 1.48 & 1.53 & 1.60 & 1.66 & 1.77 & 1.92 & 2.01 & 2.14 & 2.31 & 2.57 & 2.52 & 1.56 & 1.65 & 2.45 \\
T & \B 4.33 & 3.04 & 3.71 & 10.17 & 3.02 & 1.73 & 1.70 & 1.77 & 1.80 & 1.93 & 2.13 & 2.24 & 2.32 & 2.53 & 2.67 & 2.77 & 1.70 & 1.70 & 3.52
\end{tabular*}
\end{adjustbox}
\caption{Range between 2.5\,\% and 97.5\,\% quantile [cm] between estimated - observed diameter.}
\label{tab:quntRangeInterpol}
\end{table*}

\begin{table*}[htbp]
  \begin{adjustbox}{width=1.95\columnwidth}
  \small
  \begin{tabular*}{\linewidth}{l |@{~~~} *{19}{S[detect-weight,mode=text,table-format=1.2]}}
h & {0.65} & {1} & {1.3} & {3} & {5} & {7} & {9} & {11} & {13} & {15} & {17} & {19} & {21} & {23} & {25} & {27} & {Coarse} & {d=7} & {DTop}\\
    \cline{1-20}
A & 1.58 & 0.84 & 0.77 & 0.88 & 0.40 & \B 0.35 & 0.40 & \B 0.40 & 0.45 & \B 0.45 & \B 0.50 & \B 0.55 & 0.60 & 0.65 & \B 0.60 & \B 0.65 & \B 0.38 & \B 0.44 & \B 0.60 \\
B2 & 1.11 & 0.72 & 0.69 & 1.55 & 0.57 & 0.42 & 0.44 & 0.48 & 0.52 & 0.55 & 0.60 & 0.65 & 0.69 & 0.74 & 0.80 & 0.86 & 0.43 & 0.47 & 1.13 \\
B3 & 1.14 & 0.73 & 0.67 & 1.59 & 0.72 & 0.48 & 0.47 & 0.50 & 0.54 & 0.58 & 0.63 & 0.69 & 0.73 & 0.78 & 0.85 & 0.93 & 0.42 & 0.47 & 2.29 \\
B4 & 1.20 & 0.73 & 0.70 & 3.00 & 1.90 & 0.97 & 0.66 & 0.61 & 0.65 & 0.69 & 0.75 & 0.84 & 0.90 & 0.96 & 1.06 & 1.24 & 0.44 & 0.51 & 3.27 \\
C & 0.98 & \B 0.71 & 0.63 & 0.75 & 0.41 & 0.36 & 0.40 & 0.44 & 0.46 & 0.49 & 0.54 & 0.59 & 0.62 & 0.65 & 0.70 & 0.75 & 0.41 & 0.45 & 0.71 \\[.35em]
D0.1 & 0.95 & 0.72 & 0.71 & 0.76 & \B 0.36 & 0.36 & 0.39 & 0.42 & 0.44 & 0.47 & 0.51 & 0.57 & 0.60 & 0.63 & 0.66 & 0.70 & 0.41 & \B 0.44 & 0.66 \\
D0.5 & 0.97 & 0.73 & 0.81 & 0.83 & \B 0.36 & \B 0.35 & \B 0.38 & 0.41 & \B 0.43 & 0.46 & 0.51 & \B 0.55 & \B 0.58 & \B 0.62 & 0.64 & 0.69 & 0.41 & \B 0.44 & 0.64 \\
D0.9 & 1.03 & 0.78 & 0.95 & 0.90 & 0.38 & \B 0.35 & 0.39 & \B 0.40 & 0.44 & \B 0.45 & \B 0.50 & \B 0.55 & \B 0.58 & 0.64 & 0.62 & 0.67 & 0.43 & 0.45 & 0.61 \\
D1 & 1.03 & 0.81 & 0.99 & 0.92 & 0.40 & \B 0.35 & 0.40 & \B 0.40 & 0.45 & \B 0.45 & \B 0.50 & \B 0.55 & 0.60 & 0.65 & \B 0.60 & \B 0.65 & 0.43 & 0.45 & \B 0.60 \\
E & 0.98 & 0.73 & 0.55 & 0.68 & 0.41 & 0.37 & 0.40 & 0.44 & 0.46 & 0.50 & 0.54 & 0.59 & 0.62 & 0.65 & 0.70 & 0.75 & 0.41 & 0.45 & 0.73 \\[.35em]
F & 1.11 & 0.75 & 0.61 & 0.62 & 0.38 & 0.37 & 0.40 & 0.43 & 0.46 & 0.50 & 0.54 & 0.59 & 0.62 & 0.67 & 0.70 & 0.75 & 0.40 & 0.46 & 0.76 \\
G & 1.07 & 0.76 & 0.57 & 0.62 & 0.40 & 0.39 & 0.40 & 0.45 & 0.47 & 0.52 & 0.55 & 0.60 & 0.65 & 0.69 & 0.73 & 0.78 & 0.42 & 0.46 & 0.80 \\
H & 1.20 & 0.76 & 0.68 & \B 0.61 & 0.38 & 0.36 & 0.40 & 0.44 & 0.46 & 0.49 & 0.54 & 0.59 & 0.63 & 0.66 & 0.70 & 0.73 & 0.41 & 0.46 & 0.86 \\
I & 1.15 & 0.76 & 0.68 & 1.61 & 0.72 & 0.48 & 0.47 & 0.50 & 0.54 & 0.58 & 0.63 & 0.69 & 0.73 & 0.78 & 0.85 & 0.94 & 0.43 & 0.47 & 0.97 \\
J & 1.15 & 0.77 & 0.53 & 0.90 & 0.61 & 0.47 & 0.47 & 0.50 & 0.54 & 0.58 & 0.63 & 0.69 & 0.73 & 0.78 & 0.85 & 0.94 & 0.43 & 0.47 & 0.98 \\[.35em]
K & 1.15 & 0.77 & 0.53 & 0.89 & 0.61 & 0.47 & 0.47 & 0.50 & 0.54 & 0.58 & 0.63 & 0.69 & 0.73 & 0.78 & 0.85 & 0.94 & 0.43 & 0.47 & 0.97 \\
L & \B 0.80 & 0.86 & \B 0.51 & 0.65 & 0.45 & 0.40 & 0.42 & 0.45 & 0.47 & 0.52 & 0.57 & 0.61 & 0.65 & 0.70 & 0.75 & 0.75 & 0.45 & 0.45 & 0.83 \\
M & 1.07 & 0.75 & 0.62 & 0.63 & 0.38 & 0.37 & 0.40 & 0.43 & 0.46 & 0.49 & 0.53 & 0.58 & 0.62 & 0.66 & 0.69 & 0.74 & 0.41 & 0.45 & 0.81 \\
N & 1.17 & 0.77 & 0.70 & 0.71 & 0.37 & 0.37 & 0.40 & 0.43 & 0.46 & 0.49 & 0.53 & 0.58 & 0.62 & 0.66 & 0.69 & 0.74 & 0.40 & 0.46 & 0.83 \\
O & 1.09 & 0.76 & 0.64 & 0.65 & 0.37 & 0.37 & 0.40 & 0.43 & 0.46 & 0.49 & 0.53 & 0.58 & 0.62 & 0.66 & 0.69 & 0.74 & 0.40 & 0.45 & 0.82 \\[.35em]
P & 1.10 & 0.75 & 0.61 & 0.63 & 0.38 & 0.37 & 0.40 & 0.43 & 0.46 & 0.50 & 0.54 & 0.59 & 0.62 & 0.67 & 0.70 & 0.75 & 0.40 & 0.46 & 0.85 \\
Q & 1.15 & 0.76 & 0.68 & 1.61 & 0.72 & 0.48 & 0.47 & 0.50 & 0.54 & 0.58 & 0.63 & 0.69 & 0.73 & 0.78 & 0.85 & 0.94 & 0.43 & 0.47 & 0.97 \\
R & 1.07 & \B 0.71 & 0.67 & 1.27 & 0.46 & 0.40 & 0.40 & 0.45 & 0.47 & 0.52 & 0.57 & 0.60 & 0.65 & 0.70 & 0.73 & 0.78 & 0.42 & 0.46 & 0.91 \\
S & 1.09 & 0.75 & 0.67 & 0.67 & 0.38 & 0.37 & 0.40 & 0.43 & 0.46 & 0.49 & 0.53 & 0.59 & 0.62 & 0.66 & 0.69 & 0.73 & 0.41 & 0.45 & 0.80 \\
T & 1.07 & 0.75 & 0.72 & 2.45 & 0.70 & 0.43 & 0.43 & 0.43 & 0.50 & 0.50 & 0.57 & 0.60 & 0.67 & 0.70 & 0.73 & 0.80 & 0.43 & 0.46 & 1.04
  \end{tabular*}
\end{adjustbox}
\caption{Inter quantile range between 25\,\% and 75\,\% quantile [cm] between estimated - observed diameter.}
\label{tab:iqrInterpol}
\end{table*}

By combining the results showing the mean difference and the deviation a method
should be found which has a low bias and also low deviation. From the bias the
ranking of the methods would be G, J, K, I, Q and R. Out of those methods G also
shows good performance according to the deviation. Overall the method of
\cite{steffen1990interpolation} (G) looks like to be recommendable to be used to
interpolate between measured diameters and integrate those to a stem volume.

Fig.~\ref{fig:vResid} shows the relations between volumes estimated by sectional
step method using the diameter in the middle of the section (Section), linear
interpolation (Linear) and cubic spline (Spline) and the volume estimated by
interpolating using the method by \cite{steffen1990interpolation}. The volume by
section wise calculation is typically lower than Steffen. Especially for small
trees this difference could be in the range of 20\,\% of the volume. The linear
interpolation is underestimating small trees and overestimating marginal for
larger trees. The spine interpolation is over a wide range consistent with
Steffen but has sometimes very large differences.

\begin{figure}[htbp]
  \centering
  \includegraphics[width=.95\columnwidth]{./pic/vResid}
  \caption{Relation of different volume estimates along tree height.}
  \label{fig:vResid}
  \footnotesize{Section .. Section wise Volume / Steffen Volume, Linear ..
  Volume by linear interpolation / Steffen Volume, Spline .. Volume by
  interpolation with cubic spline / Steffen Volume. Relations larger than 1.5
  are shown as 1.5}
\end{figure}

Fig.~\ref{fig:interpolExamples} compares the result of different interpolation
methods. If there are many measurements with a continuous pattern the methods
differ marginal (TreeID: 3). If there are only some observations, but the
pattern of the measurements looks simple the differences between the methods are
still small (TreeID: 25). In case of an abrupt break of the trend and missing
observations over wide height range the differences are getting large (TreeID:
17314, 8402). Interpolation methods like splines could exceed the range of
values between two successive points, what might be possible in reality but
could lead to unrealistic values like negative diameters (TreeID: 17314). Non
overshooting methods like Steffen do not exceed the range of values between two
successive points and look like to estimate robust results.

\begin{figure}[htbp]
  \centering
  \includegraphics[width=.95\columnwidth]{./pic/interpolExamples}
  \caption{Examples of different interpolation between measured diameters.}
  \label{fig:interpolExamples}
\end{figure}


\section{Discussion}

\cite{mueller1902holzmesskunde,müller1923holzmesskunde} report a systematic
error of 3\,\% when using the section wise volume calculation with the
cross-sectional area in the middle of the stem section.
\cite{prodan1965holzmesslehre} comes to the conclusion that the volume of
concave trees is underestimated and of convex overestimated with this method.
Using linear interpolation the volume of concave sections is overestimated,
those of convex underestimated. Monotonic non overshooting interpolations showed
good results for estimating stem shape and with this tree volumes. From those
which have been compared here the method of \cite{steffen1990interpolation}
showed good performance.

Extrapolation is uncertain. Measurements at or close to the beginning and the
end of the stem could avoid the need of extrapolation. For the used dataset it
was needed to estimate the diameter at height zero. This estimated diameter
influences the calculated volume and could therefore also influence the shown
results and the drawn conclusions. Like the diameter at height zero, a
hypothetical diameter e.\,g.\ 1\,m below the ground could help some
interpolation methods to find good shapes at the ends of the stem. If this
diameter was used it needs to be documented how it was estimated.

The definition of where the stem begins needs to be considered.
\cite{prodan1965holzmesslehre} defines that the stem does not include the stump
but gives no information how high the stump is. When a tree is felled the stump
height might have positive relation with the tree size. This would have the
consequence that parts of the stem volume of a young tree will not be included
in the old tree. To avoid such problems e.\,g.\
\cite{assmann1961waldertragskunde} defined that the stem begins at the ground.
Maybe a refinement of this definition needs to be made when trees are standing
on slopes.

Diameters at heights close together followed by regions with no measurement
should be avoided as some interpolation methods propagate this short trend to
long stem sections and could increase small measurement errors proportional to
the section length of interpolation. If measurements exist which are closer
together than x\,cm followed by a section with no measurement, the close
together measurements could be aggregated to one data point to avoid this
problem.

In regions, like the stem base, where the diameter is fast decreasing in a
non-linear way shorter distances between the measurements would help to come to
accurate volume calculations. The same is the case for small trees. A standard
measurement of diameters every 2\,m will lead on small trees to a number of less
than 2--3 measurements what will lead to insecure volume calculations.


\addcontentsline{toc}{section}{Literatur}
\bibliography{literature}


%Autor: Georg Kindermann

\end{document}