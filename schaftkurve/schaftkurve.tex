\documentclass[twocolumn,10pt]{article}

\usepackage[utf8]{inputenc}
\usepackage[T1]{fontenc}
\usepackage[british]{babel}

\usepackage{lmodern}
\usepackage[sc]{mathpazo} % or option osf
\usepackage{newpxmath}

\usepackage{amsmath}
\usepackage{graphicx}

\usepackage{adjustbox}
\usepackage[a4paper, margin=1mm, includefoot, footskip=15pt]{geometry}

\usepackage[pdftitle={Estimating stem tapper shape by using tree diameter,
 height and volume}
, pdfauthor={Georg Kindermann}
, pdfsubject={stem tapper shape of a tree}
, pdfkeywords={Foresty, stem shape, forest measuration, stem tapper curve}
, pdflang={en-GB}
, colorlinks=true
, linkcolor=blue
, urlcolor=blue
, pdfpagemode=UseNone]{hyperref}

\nonfrenchspacing
\sloppy

\title{Estimating stem tapper shape by using tree diameter, height and volume}
\author{Georg Kindermann}
%\date{10. October 2023}

\begin{document}

\twocolumn[
  \begin{@twocolumnfalse}
    \maketitle
    \begin{abstract}
      When dealing with single trees beside their species also their diameter at
      breast height and tree height is typically known. With this information it
      is common practice to estimate stem volume. Also stem taper functions use
      diameter and height to estimate diameters at arbitrary tree heights what
      is used to calculate assortments. Here is a method shown which gives stem
      diameter at arbitrary tree heights and is consistent with a given stem
      volume.
    \end{abstract}
  \end{@twocolumnfalse}
]

%\tableofcontents

%\section{Einleitung}
%\label{sec:einleitung}

When measuring trees, typical their diameter (DBH or $d_{1.3}$ - diameter at
breast height in 1.3\,m height above ground) is measured. Nowadays more and more
also the height of each single tree is measured. Alternatively the height is
calculated by using a relation between DBH and height. The volume is calculated
by using DBH and height and a form factor. The form factor could also be
estimated by using DBH and height. Sometimes additional diameters, at different
heights on the tree, are measured to estimate the stem volume and the shape of
the tapper. The tapper shape can be used to split up the stem and its volume
into different diameter classes called assortments.

Calculating the stem volume by using species, DBH and height is a standard task
which was done in the past by using precalculated tables and is done now
typically by using functions. Getting the assortments could also be done by
using functions but is still done by using precalculated tables, as the
functions, for getting the shape and the algorithm of cutting the tree into logs
is somehow complex.

Here I will show a method which is using:
$$d_x = c_0 \cdot (h - x)^{c_1}$$
one of the simplest equation to describe
tapper shape. $d_x$ is the stem diameter at height $x$, $h$ is the total height
of the tree, $x$ is the height from the ground where the diameter $d_x$ should
be estimated and $c_0$ and $c_1$ are tree specific coefficients.

Instead of the diameter $d_x$ also the crosscutting area $g_x$ could be
estimated with the same equationbut with different coefficients:
$$g_x = c_2 \cdot (h - x)^{c_3}$$
While $g_x = d^2_x \cdot \pi / 4$ follows that
$c_3 = 2 \cdot c_1$ and $c_2 = c^2_0 * \pi / 4$.

The integral of the crosscutting area along the stem is the stem volume $V$:
$$V =
\int_{x=0}^h c_2 \cdot (h - x)^{c_3} dx = \frac{c_2
\cdot h^{(c_3 + 1)}}{c_3 + 1}$$

With those relations it's possible to get $c_0$ and $c_1$ for a tree where DBH,
height and volume is known.
\begin{eqnarray*}
g_{1.3} & = & d_{1.3}^2\pi/4\\
c_3 & = & \frac{W_0\left(\frac{g_{1.3}*(h-1.3)*log(1-1.3/h)}{V}\right)}{log(1 - 1.3/h)} - 1\\
g_0 & = & g_{1.3} / (1-1.3/h)^{c_3}\\
c_2 & = & g_0 / h^{c_3}\\
g_x & = & c_2 \cdot (h - x)^{c_3}\\
g_x & = & g_0 \cdot (1 - x/h)^{c_3}
\end{eqnarray*}

Where all tree dimensions (DBH, h, V) are in m and m$^3$,
and $W_0()$ is Lamberts W function: $w = W(we^w)$.

When splitting the stem at the height of the DBH into two parts, and it is
assumed that the lower part has the shape of a cylinder, the coefficient can be
estimated with:
\begin{eqnarray*}
V_{>1.3} & = & V - g_{1.3} * 1.3\\
c_5 & = & \frac{g_{1.3} \cdot (h - 1.3)}{V_{>1.3}} - 1\\
c_4 & = & g_{1.3} / (h-1.3)^{c_5}\\
g_x & = & \begin{cases}
    c_4 \cdot (h - x)^{c_5} & \text{if } x\geq 1.3m\\
    g_{1.3}               & \text{otherwise}
    \end{cases}
\end{eqnarray*}

It is also possible to make an assumption of the stem diameter at the ground
(height $x=0$), like that it is 1.3\,cm larger than the DBH. For trees smaller
than 1.3\,m it could be set to its height divided by 100. In this case
there is no need that the tapper shape goes through the DBH.
\begin{eqnarray*}
d_{0a} & = & \begin{cases}
    DBH + 0.013 & \text{if } h\geq 1.3m\\
    h / 100     & \text{otherwise}
    \end{cases}\\
g_{0a} & = & d_{0a}^2\cdot\pi/4\\
c_7 & = & \frac{g_{0a} \cdot h}{V} - 1\\
c_6 & = & g_{0a} / h^{c_7}\\
g_x & = & c_6 \cdot (h - x)^{c_7}\\
\end{eqnarray*}

Note that all three shown methods to estimate $g_x$ can give for a specific
height different values. All give monotonic decreasing or steady diameters from
the ground up to the treetop and all come to the same stem volume. It can be
used for diameter and volume with or without bark.

\par\noindent\rule{\columnwidth}{0.5pt}
\par\noindent\rule{\columnwidth}{0.5pt}

For a practical example assume a tree with the following dimensions.
\begin{verbatim}
DBH = 0.252 m
h = 27.5 m
\end{verbatim}

The diameter of the stem was measured at height 0.5\,m, 1.5\,m, 2.5\,m, \dots,
26.5\,m and is given in m (see fig.~\ref{fig:trunk}):
\begin{verbatim}
.261, .250, .242, .245, .237, .228, .220, .218, .213,
.206, .203, .197, .190, .182, .176, .165, .157, .149,
.140, .126, .116, .112, .092, .082, .063, .046, .026
\end{verbatim}

According to Huber the stem volume can be calculated with:
$$V = \sum l \cdot d^2\pi/4 = 1(0.261^2 + 0.250^ + ... + 0.026^2)\pi/4 = 0.6902\,m^3$$

\par\noindent\rule{\columnwidth}{0.5pt}

The basal area at breast height is:
$$ g_{1.3} = \text{DBH}^2\pi/4 = 0.252^2\pi/4 = 0.04988\,m^2$$

And the coefficient $c_3$ is calculated with:
\begin{eqnarray*}
c_3 & = & \frac{W_0\left(\frac{g_{1.3}*(h-1.3)*log(1-1.3/h)}{V}\right)}{log(1 - 1.3/h)} - 1\\
 & = & \frac{W_0\left(\frac{0.04988*(27.5-1.3)*log(1-1.3/27.5)}{0.6902}\right)}{log(1 - 1.3/27.5)} - 1\\
  & = & 1.09545
\end{eqnarray*}

And the coefficient $c_2$ is calculated with:
\begin{eqnarray*}
g_0 & = & g_{1.3} / (1-1.3/h)^{c_3}\\
 & = & 0.04988 / (1-1.3/27.5)^{1.09545}\\
 & = & 0.05259\,m^2 \\
c_2 & = & g_0 / h^{c_3}\\
 & = & 0.05259 / 27.5^{1.09545}\\
 & = & 0.001394
\end{eqnarray*}

And the coefficient $c_0$ and $c_1$ are calculated with:
\begin{eqnarray*}
c_0 & = & \sqrt{c_2\cdot 4 / \pi} = \sqrt{0.001394 \cdot 4 / \pi} = 0.04213\\
c_1 & = & c_3/2 = 1.09545/2 = 0.5477
\end{eqnarray*}

The Volume would be:
\begin{eqnarray*}
V & = & \frac{c_2 \cdot h^{(c_3 + 1)}}{c_3 + 1} =
  \frac{0.001394 \cdot 27.5^{(1.09545 + 1)}}{1.09545 + 1} = 0.6902\,m^3
\end{eqnarray*}

And the diameter at height $x$ is:
\begin{eqnarray*}
d_x & = & c_0 \cdot (h - x)^{c_1} = 0.04213 \cdot (27.5 - x)^{0.5477}\\
d_{1.3} & = & 0.04213 \cdot (27.5 - 1.3)^{0.5477} = 0.252\,m\\
\end{eqnarray*}

The pattern is shown in fig.~\ref{fig:trunk} as "through DBH".

\par\noindent\rule{\columnwidth}{0.5pt}

Assuming the stem is a cylinder below the DBH the calculation will be:

\begin{eqnarray*}
V_{>1.3} & = & V - g_{1.3} \cdot 1.3 = 0.6902 - 0.04988 \cdot 1.3 = 0.6254\,m^3\\
c_5 & = & \frac{g_{1.3} \cdot (h - 1.3)}{V_{>1.3}} - 1\\
  & = & \frac{0.04988 \cdot (27.5 - 1.3)}{0.6254} - 1\\
  & = & 1.0895\\
c_4 & = & g_{1.3} / (h-1.3)^{c_5} = 0.04988 / (27.5-1.3)^{1.0895} = 0.001421\\
V & = & \frac{c_4 \cdot (h-1.3)^{(c_5 + 1)}}{c_5 + 1} + g_{1.3} \cdot 1.3\\
 & = & \frac{0.001421 \cdot h^{(1.0895 + 1)}}{1.0895 + 1} + 0.04988 \cdot 1.3\\
 & = & 0.6902\,m^3\\
g_x & = & \begin{cases}
    c_4 \cdot (h - x)^{c_5} = 0.001421 \cdot (27.5 - x)^{1.0895} & \text{if } x\geq 1.3m\\
    g_{1.3} = 0.04988              & \text{otherwise}
    \end{cases}\\
d_{1.3} & = & \sqrt{c_4 \cdot (h - 1.3)^{c_5} \cdot 4/\pi}\\
 & = & \sqrt{0.001421 \cdot (27.5 - 1.3)^{1.0895} \cdot 4/\pi}\\
 & = & 0.252\,m
\end{eqnarray*}

The pattern is shown in fig.~\ref{fig:trunk} as "Cylinder".

\par\noindent\rule{\columnwidth}{0.5pt}

Assuming diameter on the ground is 0.013\,m larger than the DBH the calculation
will be:
\begin{eqnarray*}
d_{0a} & = & DBH + 0.013 = 0.252 + 0.013 = 0.265\,m\\
g_{0a} & = & d_{0a}^2\pi/4 = 0.265^2\pi/4 = 0.05515\,m^2\\
c_7 & = & \frac{g_{0a} \cdot h}{V} - 1 = \frac{0.05515 \cdot 27.5}{0.6902} - 1 = 1.1975\\
c_6 & = & g_{0a} / h^{c_7} = 0.05515 / 27.5^{1.1975} = 0.001042\\
V & = & \frac{c_6 \cdot h^{(c_7 + 1)}}{c_7 + 1} =
\frac{0.001042 \cdot 27.5^{(1.1975 + 1)}}{1.1975 + 1} = 0.6902\,m^3\\
g_x & = & c_6 \cdot (h - x)^{c_7} = 0.001042 \cdot (h - x)^{1.1975}\\
d_0 & = & = \sqrt{0.001042 \cdot (h - 0)^{1.1975} \cdot 4/\pi} = 0.265m\\
\end{eqnarray*}

The pattern is shown in fig.~\ref{fig:trunk} as "through DBH+1.3".

\begin{figure}[htb]
    \includegraphics[width=\columnwidth]{trunk}
    \caption{Different methods to estimate tapper curve}
    \label{fig:trunk}
\end{figure}




%Autor: Georg Kindermann

\end{document}