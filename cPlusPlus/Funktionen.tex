\section{Funktionen}

Fasst Anweisungen (mit Funktionsparametern) zusammen.

\begin{lstlisting}[language=C++]
#include <iostream>

// Function name:  "f"
// Parameter list: nothing
// Returns:        nothing
// Description:    Writes "Hi" to the terminal
void f()  // void..Rückgabewert, f..Funktionsname, ()..Argumente
{  // Beginn des Funktionskörpers
  std::cout << "Hi" << '\n';
}  // Ende  des Funktionskörpers

int main() {
  f();  // Aufruf von Funktion f
}
\end{lstlisting}

Wie Variblen müssen auch Funktionen \emph{vor} ihrere Verwendung deklariert
werden.

\begin{lstlisting}[language=C++]
#include <iostream>

int main() {
//f();  //Error: f ist nicht deklariert
}

void f() { std::cout << "Hi" << '\n'; }
\end{lstlisting}

\begin{lstlisting}[language=C++]
#include <iostream>

// Deklaration und Definition
void f() { std::cout << "Hi" << '\n'; }

int main() {
  f();
}
\end{lstlisting}

\begin{lstlisting}[language=C++]
#include <iostream>

// Deklaration
void f();

int main() {
  f();
}

// Definition
void f() { std::cout << "Hi" << '\n'; }
\end{lstlisting}

\subsection{Overloading}

Verschiedene Funktionen mit gleichem Namen sind erlaubt, solange deren Parameter
verschieden sind (function overloading).

\begin{lstlisting}[language=C++]
#include <iostream>

void f() { std::cout << "A" << '\n'; }
void f(int i) { std::cout << "B" << '\n'; }
void f(double i) { std::cout << "C" << '\n'; }
// Error: Unterscheidet sich nur im Rückgabewert
//int f() { std::cout << "D" << '\n'; return 0;}

int main() {
  f();    // A
  f(0);   // B
  f(0.);  // C
}
\end{lstlisting}

\subsection{Übergabe von Funktionen}

Wenn Funktionen an Funktionen übergeben oder von ihnen zürückgegeben werden
sollen, müssen dafür Pointern oder Referenzen auf Funktionen verwendet werden.

\begin{lstlisting}[language=C++]
#include <iostream>

void f() { std::cout << "A" << '\n'; }
// Gibt Pointer auf Funktion f() zurück
void (*pf())() { std::cout << "PF: "; return &f; }
// Das Gleiche mit impliziter Umwandlung
void (*pf2())() { std::cout << "PF2: "; return f; }
// Das Gleiche mit Trailing return type
auto pf3() -> void(*)() { std::cout << "PF3: "; return &f; }
// Gibt Referenz auf Funktion zurück
void (&rf())() { std::cout << "RF: "; return f; }
// Übergabe von Pointer auf Funktion
void fp( void (*p)() ) { std::cout << "FP: "; (*p)();}
// Übergabe von Referenz auf Funktion
void fr( void (&r)() ) { std::cout << "FR: "; r();}

int main() {
  f();        // A
  (*pf())();  // PF: A
  pf()();     // PF: A
  pf2()();    // PF2: A
  pf3()();    // PF3: A
  rf()();     // RF: A
  fp(&f);     // FP: A
  fr(f);      // FR: A
}
\end{lstlisting}

\subsection{Rückgabetypherleitung --  Return type deduction}

\begin{lstlisting}[basicstyle=\ttfamily\footnotesize,language=C++]
#include <type_traits>

int x = 1;
auto           f0() { return x; }   // return type: int
auto           f1() { return(x); }  // return type: int
auto&          f2() { return x; }   // return type: int&
const auto&    f3() { return x; }   // return type: const int&
decltype(auto) f4() { return x; }   // return type: int, decltype(x)
decltype(auto) f5() { return(x); }  // return type: int&, decltype((x))

int main() {
  static_assert(std::is_same_v<std::invoke_result_t<decltype(f0)>, int>);
  static_assert(std::is_same_v<std::invoke_result_t<decltype(f1)>, int>);
  static_assert(std::is_same_v<std::invoke_result_t<decltype(f2)>, int&>);
  static_assert(std::is_same_v<std::invoke_result_t<decltype(f3)>, const int&>);
  static_assert(std::is_same_v<std::invoke_result_t<decltype(f4)>, int>);
  static_assert(std::is_same_v<std::invoke_result_t<decltype(f5)>, int&>);
}
\end{lstlisting}

\subsection{Parameter Liste}

\begin{lstlisting}[language=C++]
int f0(int);  // Parameter ohne Namen
int f1(int i);  // Benannter Parameter
int f2(int i = 7);  // Mit Name und default Wert
int f3(int = 7);  // Ohne Name, mit default Wert
int f4();  // Ohne Parameter
int f5(void);  // Wie f4
int f6(int i ...);  // Variadic, Beliebige Anzahl von Argumenten
int f7(int i, ...);  // wie f6
int f8(auto);  // Wie: template<class T> int f8(T)
\end{lstlisting}

\subsubsection{Default Arguments}

Erlaubt den Funktionsaufruf, auch wenn abschießende Argumente fehlen.

\begin{lstlisting}[language=C++]
#include <iostream>

void f0(int i = 3, int j = 4) {}

void f1(int, int = 7) {}

// Error: Nachfolgender Parameter hat keinen Defaultwert
//void f2(int = 1, int);

void f3(int, int) {}
//void f3(int = 3, int);  // Error, wie bei f2
void f3(int, int = 4);  // Fügt Defaultargument zu
void f3(int = 3, int);  // Jetzt gleich wie f0
//void f3(int = 3, int = 4);  // Error: redefinition

void f4(int = 3 ...);  // OK, ellipsis ist kein Parameter

int i{1};  // Scope auf dieses i
void f5(int j = i) { std::cout << j << '\n';}
//void f6(int i, int j = i);  // Error: i Schon verwendet
//void f6(int a, int j = a);  //Error:  a ist nicht definiert

void f7() {
  int i{0};
//void f(int = i);  // Lokale Variablen sind nicht erlaubt
};

int b;
class X {
  int a;
  static int b;
  static X y;
//void f(X* p = this);  // Error: this nicht erlaubt
//void m0(int i = a);  // Error: non-static data member
  void m1(int X::* i = &X::a);  // Geht mit Pointer
  void m2(int i = y.a);
  void m3(int i = b);
};

int main() {
  f0(1, 2);  // f0(1, 2)
  f0(1);     // f0(1, 4)
  f0();      // f0(3, 4)

  f1(1, 2);  // f1(1, 2)
  f1(1);     // f1(1, 7)
//f1();      // Error: Zu wenig Argumente

  int i{0};
  f5();  // 1
  ++::i;
  f5();  // 2
}
\end{lstlisting}

\subsubsection{Variadic arguments}

Es gibt keine Möglichkeit die Anzahl der Variadic Argumente zu bestimmen. Man
kann nur auf sie zuzugreifen wenn es vor ihnen ein benanntes Argument gibt.
Alternativen sind Variadic templates oder initializer\_list.

\begin{lstlisting}[language=C++]
#include <cstdarg>
#include <iostream>

int iSum(int n...) {
  int s = 0;
  std::va_list vArgs;  // Enthält Info von ...
  va_start(vArgs, n);  // Ermöglicht Zugriff auf ...
  for (int i = 0; i < n; ++i)
    s += va_arg(vArgs, int);  // Holt nächstes Argument
  va_end(vArgs);  // Beendet Zugriff auf ...
  return s;
}

int main() {
  std::cout << iSum(4, 1, 2, 3, 4) << '\n';  // 10
}
\end{lstlisting}

\subsection{Gelöschte Funktionen}

Können mit \lstinline|delete| definiert werden.

\begin{lstlisting}[language=C++]
int f0(int) = delete;

int main() {
//int i = f0(0);  // Error: use of deleted function
//int i = f1(0);  // Error: not declared
}
\end{lstlisting}

\subsection{Zugriff auf Funktionsnamen}

Mit \lstinline|__func__|.

\begin{lstlisting}[language=C++]
#include <iostream>

void funktionA() { std::cout << __func__ << '\n';}

int main() {
  funktionA();  // Ausgabe: funktionA
}
\end{lstlisting}

\subsection{Lambda Funktionen}

Namenlose Funtkionen, die dort wo sie verwendet auch definert werden. Ihnen kann
der Zugriff auf Variablen im selben Scope (Sichtbarkeitsbereich) ermöglicht
werden.

\begin{lstlisting}[language=C++]
#include <iostream>

int main() {
  int a[]{1,2,3,4};
  int s{0};
  for(int i : a) s += i;  // Mit For loop
  std::cout << s << '\n';  // 10
  // Lambda Funktion wird gleich aufgerufen
  [&]{for(int i : a) s += i;}();
  std::cout << s << '\n';  // 20
  // Lambda Funktion definieren
  auto f = [&]{for(int i : a) s += i;};
  f();  // und aufrufen
  std::cout << s << '\n';  // 30
}
\end{lstlisting}

Minimaler Aufbau:\\
\lstinline|[captures]{body}|\\
\begin{description}
  \item[captures] Gibt ann welche Variablen von der Funktion verwendbar sind. Kann leer sein. Bei mehreren Angaben werden diese mit \lstinline|,| getrennt.
  \begin{description}
    \item[\&] Alle mittels Referenz, an erster Position
    \item[=] Alle mittels Kopie, an erster Position
    \item[Variablenname] Einezlne Varable über Kopie
    \item[\&Variablenname] Einezlne Varable über Referenz
    \item[Variablenname...] Über Kopie mit Packexpansion (bei Template)
    \item[NameInFunction = Variablenname] Varable über neuen Namen verwenden
    \item[this] Referenz auf derzeitiges Objekt
    \item[*this] Kopie auf derzeitiges Objekt
  \end{description}
  \item[body] Die Befehle der Funktion.
\end{description}

\begin{lstlisting}[language=C++]
#include <iostream>

int main() {
  int i{0};
  int j{0};
//[]{++i;}();  // Error: i nicht bekannt
//[i]{++i;}();  // Error: i read only
  [&i]{++i;}();
  std::cout << i << '\n';  // 1
  [&]{++i;}();  // Alle per Referenz
  std::cout << i << '\n';  // 2
// Alle per Referenz ausser i
//[&,i]{++i;}();  // Error: i read only
// Alle per Kopie
//[=]{++i;}();  // Error: i read only
//[&i,=]{i=j;}();  // Error: = nicht an erster Position
  [=,&i]{i=j;}();  // Alle per Kopie, i per Referenz
  std::cout << i << '\n';  // 0
// x als Kopie auf i
//[x=i]{++x;}();  // Error: x read only
  [&x=i]{++x;}();  // x als Referenz auf i
  std::cout << i << '\n';  // 1
}
\end{lstlisting}

Allgemeiner Aufbau:\\
\lstinline|[captures] <tparams> opt1 (params) opt2 {body}|\\

\begin{description}
  \item[tparams] Template Parameter
  \item[opt1] Optionale Parameter
  \item[params] Funktionsparametern
  \item[opt2] Optionale Parameter
  \begin{description}
    \item[specs] \lstinline|mutable|, \lstinline|constexpr|,
    \lstinline|consteval| oder \lstinline|static|
    \item[Rückgabetyp] z.\,B.\ \lstinline|->int|
  \end{description}
\end{description}

\begin{lstlisting}[language=C++]
#include <iostream>

int main() {
  int i{0};
  [i] mutable {++i;}();  // Verändert die Kopie von i
  std::cout << i << '\n';  // 0
  [](int j){++j;}(i);  // Verändert die Kopie von i
  std::cout << i << '\n';  // 0
  [](int& j){++j;}(i);  // Nur für int
  std::cout << i << '\n';  // 1
  [](auto& j){++j;}(i);  // Nicht nur für int
  std::cout << i << '\n';  // 2
  []<class T>(T& j){++j;}(i);  // Über Template
  std::cout << i << '\n';  // 3

  i = [i] mutable ->int {return ++i;}(); //Return Type explizit
  std::cout << i << '\n';  // 4
  i = [i] mutable {return ++i;}();  // Return Type Automatisch
  std::cout << i << '\n';  // 5
  i = [](int j){return ++j;}(i);
  std::cout << i << '\n';  // 6
}
\end{lstlisting}

\subsection{Overload resolution}

Funktionen werden über ihren Namen gesucht. Ist dieser Überladen, wird nach
Übereinstimmung mit den Argumenten gesucht. Für Funktion Templates
(Muster/Vorlage) werden die Argumenttypen hergeleitet.

\begin{lstlisting}[language=C++]
#include <iostream>

void f(double x) {std::cout << typeid(x).name() << '\n';}
void f(int x) {std::cout << typeid(x).name() << '\n';}
void g(int x) {std::cout << typeid(x).name() << '\n';}
template<class T>
void t(T x) {std::cout << typeid(x).name() << '\n';}

int main() {
  f(0.);   // d
  f(0);    // i
  f(0.f);  // d  float -> double
//f(0.l);  // Error: ambigious  long double -> ?
//f(0l);   // Error: ambigious  long int -> ?
  g(0);    // i
  g(0.);   // i  double -> int
  g(0.f);  // i  float -> int
  g(0.l);  // i  long double -> int
  g(0l);   // i  long int -> int
  t(0);    // i
  t(0.);   // d
  t(0.f);  // f
  t(0.l);  // e
  t(0l);   // l
}
\end{lstlisting}

\subsection{Operator overloading}

Für selbst definierte Typen kann die Funktion von Operatoren wie \lstinline|+|
oder \lstinline|*| selbst definiert werden.

\begin{lstlisting}[language=C++]
#include <iostream>

struct Foo {
  int x;
  Foo operator*(const Foo& Y) {
    Foo a = *this;
    a.x *= Y.x;
    return a;
  }
  Foo operator+=(const Foo& rhs) {  // compound assignment
    x += rhs.x;
    return *this;
  }
  friend Foo operator+(Foo lhs, const Foo& rhs) {
    lhs += rhs;  // Verwende compound assignment
    return lhs;
  }
  int operator()(int y) const {  // Function call operator
    return x * y;
  }
  Foo& operator++() {  // Prefix
    ++x;
    return *this;
  }
  Foo operator++(int) {  // Postfix
    Foo old = *this;
    operator++();  // Verwende Prefix
    return old;
  }
  friend bool operator<(const Foo& l, const Foo& r) {
    return l.x < r.x;
  }
  friend bool operator>(const Foo& l, const Foo& r) {
    return r < l;  // Verwendet <
  }
};

int main() {
  Foo x{2};
  Foo y{3};
  std::cout << (x + y).x << '\n';  // 5
  std::cout << (x * y).x << '\n';  // 6
  std::cout << (x + y * x).x << '\n';  // 8
  x += y;
  std::cout << x.x << '\n';  // 5
  std::cout << x(2) << '\n';  // 10
  std::cout << (x++).x << ' ' << (++x).x << '\n';  // 5 7
  std::cout << (x < y) << ' ' << (x > y) << '\n';  // 0 1
}
\end{lstlisting}