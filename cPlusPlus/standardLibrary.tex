\section{Standard Library}

Enthält Funktionen die in standard \cpp verwendet werden können. Hier erfolgt
eine knappe, subjective Auswahl.

\subsection{Input / Output}

Bildschirmausgabe (stdout) erfolgt mit \lstinline|std::cout| und Tastatureingabe
(stdin) mit \lstinline|std::cin|. Ungepufferte (unmittelbar) Fehlerausgabe
(stderr) erfolgt mit \lstinline|std::cerr| und gepufferte mit
\lstinline|std::clog|. Der Puffer kann mit \lstinline|std::flush| gelehrt
werden. Die Funktionen sind in \lstinline|iostream| definiert.

\begin{lstlisting}[language=C++]
#include <iostream>

int main() {
  int n;
  std::cout << "Bitte Anzahl eingeben:";
  std::cin >> n;
  std::cout << "Sie haben >" << n << "< eingegeben.\n";
  std::cerr << "Ungepufferte Fehlerausgabe auf stderr\n";
  std::clog << "Gepufferte Fehlerausgabe auf stderr\n";
}
\end{lstlisting}

Für formatierte Ausgabe gibt es die Funktionen \lstinline|std::print| (ohne
\lstinline|\n|) bzw.\ \lstinline|std::println| (mit \lstinline|\n|).

\begin{lstlisting}[language=C++]
#include <print>

int main() {
  std::print("{} B {}\n", "A", "C");    // A B C
  std::println("{1} B {0}", "C", "A");  // A B C
  std::print("{0} B {0}\n", "C");       // C B C
}
\end{lstlisting}

Ein / Ausgabe von / zu Dateien über \lstinline|fstream|.

\begin{lstlisting}[language=C++]
#include <fstream>
#include <iostream>
#include <string>

int main() {
  {
    std::ofstream ostrm("tmp.txt");  // Öffnet Datei zum schreiben
    ostrm << "xy " << 42 << '\n';    // Schreibt in Datei
  }                                  // Schließt ostrm
  std::ifstream istrm("tmp.txt");    // Öffnet Datei zum lesen
  if(!istrm.is_open())  // Test ob öffnen funktioniert hat
    std::cout << "Failed to open\n";
  else {
    std::string s;
    int i;
    if(istrm >> s >> i)              // Liest von Datei
      std::cout << s << ' ' << i << '\n';
  }
}
\end{lstlisting}

Mit \lstinline|getline| können ganze Zeilen eingelesen werden. Wenn Space als
Zeilenende gesetzt wird, kann damit in einzelne Wörter / Token zerlegt werden.

\begin{lstlisting}[language=C++]
#include <fstream>
#include <iostream>
#include <string>
#include <sstream>

int main() {
  {
    std::ofstream ostrm("tmp.txt");
    ostrm <<
R"(#a b c
7 5 3
2 4 6)";
  }
  std::ifstream istrm("tmp.txt");
  if(!istrm.is_open())
    std::cout << "Failed to open\n";
  else {
    std::string line;
    while(std::getline(istrm, line)) {  // Liest Zeile
      if(line.length() > 0 & line[0] != '#') { // Skip comments
        std::string token;
        std::istringstream linestream(line);
        // Teilt Zeile in Tokens (Wörter)
        while (std::getline(linestream, token, ' ')) {
          std::cout << token << " : ";
        }
        std::cout << '\n';
      }
    }
  }
}
\end{lstlisting}

\subsection{Zeichenketten / Strings}

Zeichenketten sind in \lstinline|string| definiert. Strings können
beispielsweise verkettet werden oder in ihnen nach bestimmten Zeichen gesucht
werden.

\begin{lstlisting}[language=C++]
#include <iostream>
#include <string>

int main() {
  std::string c{'E'};       // String erzeugen
  std::string s{"MC²"};
  auto cs = c + " = " + s;  // Strings zusammenhängen
  std::cout << cs << '\n';  // E = MC²
  std::string::size_type pos = cs.find("=");  // Position von = suchen
  std::cout << cs.substr(0, pos) << '\n';     // E 
  std::cout << cs.substr(pos + 1) << '\n';    //  MC²
  cs[0] = 'X';              // Estes Zeichen austauschen
  std::cout << cs << '\n';  // X = MC²
}
\end{lstlisting}

\subsection{Regular expression}

Zur Mustersuche (pattern matching) in Zeichenketten (Strings).

\begin{lstlisting}[language=C++]
#include <iostream>
#include <string>
#include <regex>

int main() {
  std::string s{"R2-D2"};
  std::regex r("D.");  // Sucht D gefolgt von beliebigem Zeichen
  if(std::regex_search(s, r)) std::cout << "Treffer\n";
}
\end{lstlisting}

\subsection{Container}

Gängige Datenstrukturen wie Vektoren, Listen oder Stapel.

\subsubsection{Sequenziell}

Elemente sind nacheinander angeordnet. Dazu gehören \lstinline|array|
(statisches fortlaufendes Datenfeld), \lstinline|vector| (größenveränderbares
fortlaufendes Datenfeld), \lstinline|deque| (Vektor mit zwei Enden),
\lstinline|list| (nach beiden Richtungen verkettete Elemente - Zeiger auf
Nachbarelemente) und \lstinline|forward_list| (in eine Richtung verkettet).

\begin{lstlisting}[language=C++]
#include <iostream>
#include <array>
#include <vector>
#include <deque>
#include <list>
#include <iterator>
#include <forward_list>

void P(auto x) {
  for(const auto& s : x) std::cout << s << ' ';
  std::cout << '\n';
}

int main() {
  std::array<int, 3> a1{1, 2, 3};
  std::array a2{4, 5, 6};
  a1[0] = 0;  // Erstes element auf 0 setzen
  P(a1);  // 0 2 3

  std::vector<int> v = {1, 2};
  v.push_back(3);  // Element hinten anhängen
  v[0] = 0;
  P(v);  // 0 2 3

  std::deque<int> d = {7};
  d.push_front(0);
  d.push_back(3);
  d[1] = 2;
  P(d);  // 0 2 3

  std::list<int> l = {2, 6};
  l.push_front(0);
  l.push_back(8);
  {
    auto it = l.begin();
    std::advance(it, 2);
    if (it != l.end()) l.insert(it, 4);
  }
  P(l);  // 0 2 4 6 8

  std::forward_list<int> f{2, 6};
  f.push_front(0);
  {
    auto it = f.begin();
    std::advance(it, 1);
    if (it != f.end()) f.insert_after(it, 4);
  }
  P(f);  // 0 2 4 6
}
\end{lstlisting}

\subsubsection{Assoziativ}

Elemente werden über einen Schlüssel gefunden. Dazu gehören \lstinline|set|
(Sammlung eindeutiger sortierter Schlüssel) \lstinline|map| (Sammlung
eindeutiger sortierter Schlüssel-Wert Paaren). Mit den Varianten
\lstinline|multiset| und \lstinline|multimap|, mit mehr als einem Schlüssel. Für
diese gibt es die Varianten \lstinline|unorderd|, welche mit Hashfunktionen die
Einträge sucht.

\begin{lstlisting}[language=C++]
#include <iostream>
#include <set>
#include <map>
#include <unordered_set>

int main() {
  std::set<int> s{3, 1};
  s.insert(2);
  s.insert(1);
  for(const auto& x : s) std::cout << x << ' ';  // 1 2 3
  std::cout << '\n';

  std::map<int, int> m{{3,7}, {1,2}};
  m[2] = 5;
  m[1] = 3;
  for(const auto& x : m) std::cout << x.first << '=' <<
    x.second << ' ';  // 1=3 2=5 3=7
  std::cout << '\n';

  std::multiset<int> ms{3, 1};
  ms.insert(2);
  ms.insert(1);
  for(const auto& x : ms) std::cout << x << ' ';  // 1 1 2 3
  std::cout << '\n';

  std::unordered_set<int> us{3, 1};
  us.insert(2);
  us.insert(1);
  for(const auto& x : us) std::cout << x << ' ';  // 2 1 3
  std::cout << '\n';
  // Optimierung des Hash
  us.max_load_factor(.5);  // Verwaltet Elemente pro Bucket
  us.reserve(1000);  // Reserviert für n Elemente
}
\end{lstlisting}

\subsection{Parallelität / Concurrency / Threads}

Synchrone Ausführung auf mehreren Prozessorkernen mit \lstinline|std::thread|
und \lstinline|join|.

\begin{lstlisting}[language=C++]
#include <thread>
#include <iostream>
#include <vector>

void f(int n) {
  double s{0.};
  for(int j=0; j<100000; ++j) {
    for(int i=0; i<100000; ++i) s+=n/100000.;
  }
  std::cout << n << ' ' << s << '\n';
}

void spawnThreads(int n) {
  std::vector<std::thread> threads(n);
  // spawn n threads:
  for(int i = 0; i < n; ++i) {
    threads[i] = std::thread(f, i + 1);
  }
  for(auto& th : threads) {
    th.join();
  }
}

int main() {
  spawnThreads(7);
}
\end{lstlisting}

Falls die Funktion einen Wert zurückgibt, kann beispielsweise mit
\lstinline|std::async| und \lstinline|std::future| gearbeitet werden.

\begin{lstlisting}[language=C++]
#include <thread>
#include <iostream>
#include <vector>
#include <future>

double f(int n) {
  double s = 0.;
  for(int j=0; j<100000; ++j) {
    for(int i=0; i<100000; ++i) s+=n/100000.;
  }
  return s;
}

void spawnThreads(int n) {
  std::vector<std::future<double>> threads(n);
  // spawn n threads:
  for(int i = 0; i < n; ++i) {
    threads[i] = std::async(f, i + 1);
  }
  for(int i = 0; i < n; ++i) {
    std::cout << i+1 << ' ' << threads[i].get() << '\n';
  }
}

int main() {
  spawnThreads(7);
}
\end{lstlisting}

\subsection{Numerics}

Allgemeine mathematische Funktionen, Konstanten, numerische Arrays und
Zufallszahlgenerierung.

\begin{lstlisting}[language=C++]
#include <iostream>
#include <numbers>
#include <cmath>
#include <random>
#include <numeric>
#include <vector>
#include <valarray>

int main() {
  std::cout << std::numbers::pi // Double
  << ' ' << std::numbers::pi_v<float> << '\n';

  std::cout << std::pow(3, 2) << '\n';  // 9

  std::random_device rd{};
  std::mt19937 gen{rd()};  // Generator
  gen.seed(42);  // Immer gleiche Zufallszahlenfolge
  std::normal_distribution<> nd(7., 2.);
  std::cout << nd(gen) << '\n';  // 5.89953

  std::vector<int> v{2, 5};
  // Summe, Produkt
  std::cout << std::accumulate(v.begin(), v.end(), 0) <<
  std::accumulate(v.begin(), v.end(), 1, std::multiplies<int>())
  << '\n';  // 7 10

  std::valarray<int> x{2, 5};
  std::cout << x[0] << ' ' << x[1] << '\n';  // 2 5
  x *= 2;  // Multiplizieren des ganzen Vektors
  std::cout << x[0] << ' ' << x[1] << '\n';  // 4 10
}
\end{lstlisting}

\subsection{Algorithmen}

Beinhaltet Funktionen zum Suchen, Sortieren, Zählen oder Verändern von
Elementlisten.

\begin{lstlisting}[language=C++]
#include <algorithm>
#include <iostream>
#include <vector>
#include <iterator>

int main() {
  std::vector<int> v {3, 7, 1};
  // ruft Funktion für jedes Element auf
  std::for_each(v.cbegin(), v.cend(),
   [](const int &n) { std::cout << n << ' '; });  // 3 7 1
  std::cout << '\n';

  // Suche ersten Treffer
  auto idx = std::find(v.cbegin(), v.cend(), 7);
  if(idx == v.cend()) {std::cout << "nicht gefunden\n";
  } else {std::cout << "Auf position: " <<
   std::distance(v.cbegin(), idx) << '\n';}  // gibt 1 aus

  // Anzahl der Zahlen kleiner 4
  std::cout << std::count_if(v.cbegin(), v.cend(),
   [](const int &n) { return n < 4; }) << '\n';  // 2

  int a = 1, b = 2;
  std::cout << a << ' ' << b << '\n';  // 1 2
  std::swap(a, b);  // Tauscht a und b
  std::cout << a << ' ' << b << '\n';  // 2 1

  std::reverse(v.begin(), v.end());  // umgekehrte Reihenfolge
  for (auto a : v) std::cout << a << ' ';  // 1 7 3
  std::cout << '\n';

  std::sort(v.begin(), v.end());
  for (auto a : v) std::cout << a << ' ';  // 1 3 7
  std::cout << '\n';

  std::cout << *std::max_element(v.begin(), v.end()) << '\n';//7
}
\end{lstlisting}

%\subsection{General utilities}