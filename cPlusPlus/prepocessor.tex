\section{Prepocessor}

Der Präprozessor wird vor der Kompilierung ausgeführt und kann den Sourcecode
verändern. Befehle beginnen mit \lstinline|#|.

\subsection{include}

Fügt eine Datei ein.

\begin{lstlisting}[language=C++]
#include <iostream>  // Fügt einen Header ein
#include "my.h"      // Fügt die Datei my.h ein
\end{lstlisting}

Um zu verhindern, dass der Inhalt einer Datei immer wieder eingebunden wird,
werden \emph{header guards} verwendet.

\begin{lstlisting}[language=C++]
#ifndef FOO_H_INCLUDED // Eindeutiger unique Name
#define FOO_H_INCLUDED
// Inhalt der Datei
#endif
\end{lstlisting}

\subsection{Bedingungen}

Bedingtes weglassen oder einbinden von Sourcecode mit \lstinline|#if| --
\lstinline|#elif| -- \lstinline|#else| -- \lstinline|#endif|, \lstinline|#ifdef|
(ist definiert) -- \lstinline|#elifdef|, \lstinline|#ifndef| (ist nicht
definiert) -- \lstinline|#elifndef|.

\begin{lstlisting}[language=C++]
#define ABCD 2  // Definere ABDC = 2
#include <iostream>

int main() {
#ifdef ABCD  // Ist ABCD definiert: Ja
  std::cout << "1: Ja\n";  // Wird ausgegeben
#endif

#ifndef ABCD  // Ist ABCD nicht definiert: nein
  std::cout << "2: JA\n";  // Wird entfernt
#else
  std::cout << "2: Nein\n";  // Wird ausgegeben
#endif

#if ABCD == 1
    std::cout << "3: 1\n";
#elif ABCD == 2
    std::cout << "3: 2\n";  // Wird ausgegeben
#endif

// Kombination von Bedingungen
#if !defined(DCBA) && (ABCD < 2*4-3)
  std::cout << "4: Ja\n";  // Wird ausgegeben
#endif

#undef ABCD
#ifndef ABCD
  std::cout << "5: JA\n";  // Wird ausgegeben
#endif
#define ABCD
#ifdef ABCD
  std::cout << "6: Ja\n";  // Wird ausgegeben
#endif
}
\end{lstlisting}

\subsection{Ersetzen}

Platzhalter können durch (generierte) Texte ersetzt werden.

\begin{lstlisting}[language=C++]
#include <iostream>

#define P std::cout
#define O(a) std::cout << #a

int main() {
  P << "A\n";  // p wird durch std::cout ersetzt
  O(B\n);  // Gibt B\n aus, ohne Anführungszeichen
}
\end{lstlisting}

\subsection{Warnungen und Fehler}

Mit \lstinline|#error| wird die Kompilierung abgebrochen.

\begin{lstlisting}[language=C++]
#define E
#ifdef E
#error "E is defined"
#endif
\end{lstlisting}

Mit \lstinline|#warning| wird beim Kompilieren eine Warnung ausgegeben aber
weiter kompiliert.

\begin{lstlisting}[language=C++]
#define W
#ifdef W
#warning "W is defined"
#endif
\end{lstlisting}

Mit \lstinline|#line| kann die Zeilennummer und der Dateinahme verändert werden.

\begin{lstlisting}[language=C++]
#include <cassert>
int main() {
  assert(false);
}
\end{lstlisting}

Gibt: \lstinline|c.cc:3: int main(): Assertion `false' failed| aus.

\begin{lstlisting}[language=C++]
#include <cassert>
int main() {
#line 42 "datName.cc"

  assert(false);
}
\end{lstlisting}

Gibt: \lstinline| datName.cc:43: int main(): Assertion `false' failed|
aus.

