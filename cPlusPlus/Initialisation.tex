\section{Initialisation}

Setzt Anfangswert beim Erstellen.

\begin{lstlisting}[language=C++]
#include <string>

struct T1 { int i; };  // implicit default constructor
struct T2 {
    int i;
    T2() {}; // default Constructor
};
struct T3 {
  int x;
  struct T4 {
    int i;
    int j;
    int a[3];
  } b;
};

int main() {
// Default-initialization - Objekt erzeugt ohne Initializer
  int i;  // hier Wert unbestimmt
  int* a{new int[3]};  // Array, Werte unbestimmt
  delete[] a;
//const int j;  // Error: uninitialized const
//const T1 t1;  // Error: kein default constructor
  const T2 t2;  // t2.i unbestimmt
//int& r;  // Error: uninitialized Referenz

// Value-initialization - Objekt erzeugt mit leerem Initializer
  int j{};  // j == 0
  int k = int();    // k == 0
//int f();  // Würde die Funktion f deklarieren
  int* a2{new int[3]()};  // Array, Werte 0
  delete[] a2;
  const T1 t3{};  // t3.i == 0
  T2 t4{};  // t4.i ist undefiniert

// Copy-initialization
  int i1 = 3;  // Copy elision - wird direkt initialisiert
  int i2 = 3.14;  // Double nach Int
//int i2n = {3.14};  // Error: narrowing conversion

// Direct-initialization
  int i3(3);
  int i4(3.14);  // Double nach Int
  int i5{3};
//int i6{3.14};  // Error: narrowing conversion

// Aggregate initialization - für Aggregate (ohne Constructor)
  T3 s1 = {1, {2, 3, {4, 5, 6}}};
  T3 s2 = {1, 2, 3, 4, 5, 6};  // Das Gleiche nur ohne Klammern
  T3 s3{1, {2, 3, {4, 5, 6}}}; // direct-list-initialization
  T3 s4{1, 2, 3, 4, 5, 6};
  // Designated initializers, Nur in dieser Reihenfolge
  T3 s5{.x{1}, .b{.i{2}, .j{3}, .a{4, 5, 6}}};
  T3 s6{1};  // alle Restlichen sind 0
  int ar[] = {1, 2, 3};

// List-initialization - für Objekte
  int n1{1}; // direct-list-initialization
  // initializer-list constructor call
  std::string s7{'a', 'b', 'c', 'd'};

// Reference initialization
  double d{2.};
  double& rd{d};  // Referenz auf d, Direct binding
//double& rd2{2.};  // Error: Ref. auf rvalue
  const double& rd2{2.}; // Indirect binding auf temporären Wert
  double&& rd3{2.};  // rvalue Referenz
}
\end{lstlisting}