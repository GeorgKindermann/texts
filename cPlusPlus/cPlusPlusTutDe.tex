\documentclass[10pt,twocolumn]{scrartcl}

\usepackage[utf8]{inputenc}
\usepackage[T1]{fontenc}
\usepackage[ngerman]{babel}

\usepackage{lmodern}
\usepackage[sc]{mathpazo} % or option osf
\usepackage{newpxmath}
\usepackage[scaled = 0.75]{beramono}

\newcommand*\cpp{C\kern-0.3ex\raisebox{0.17ex}{\texttt{+\kern-0.3ex+}}}

\usepackage{xpatch}
\usepackage{xcolor}
\usepackage{realboxes}
\definecolor{mygray}{rgb}{0.9,0.9,0.9}
\usepackage{listings}
\lstdefinelanguage{commentonly}{ morecomment=[l]{\#} }
\lstset{literate=%
{Ö}{{\"O}}1
{Ä}{{\"A}}1
{Ü}{{\"U}}1
{ß}{{\ss}}1
{ü}{{\"u}}1
{ä}{{\"a}}1
{ö}{{\"o}}1
,basicstyle=\ttfamily
,backgroundcolor=\color{mygray}
,commentstyle=\emph
,language=commentonly
,upquote=true
}
%\makeatletter
%\xpretocmd\lstinline{\Colorbox{mygray}\bgroup\appto\lst@DeInit{\egroup}}{}{}
%\makeatother
\makeatletter
\xpretocmd\lstinline
  {%
   \bgroup\fboxsep=1.5pt
   \Colorbox{mygray}\bgroup\kern-\fboxsep\vphantom{\ttfamily\char`\\y}%
   \appto\lst@DeInit{\kern-\fboxsep\egroup\egroup}%
  }{}{}
\makeatother

\usepackage{enumitem}
\setlist{nosep}

\usepackage{adjustbox}
\usepackage[a4paper, margin=1mm, includefoot, footskip=15pt]{geometry}

\usepackage[pdftitle={Grundlagen der Programmiersprache C++}
, pdfauthor={Georg Kindermann}
, pdfsubject={C++}
, pdfkeywords={C++, Lang, Progammiersprache, Tutorial, Einführung, German,
               Deutsch}
, pdflang={de-AT-1996}
, colorlinks=true
, linkcolor=blue
, urlcolor=blue
, pdfpagemode=UseNone]{hyperref}

\nonfrenchspacing
\sloppy

\title{Grundlagen der Programmiersprache \cpp{}}
\author{Georg Kindermann}
%\date{19. Juni 2023}

\begin{document}

\maketitle

%\begin{abstract}
%  Eine kurze Einführung in die Sprache Julia.
%\end{abstract}

\tableofcontents


Die Sprache \cpp{} baut auf der Sprache C auf. Für beide gibt es eine ISO Norm
die laufend aktualisiert wird. Eine Sammlung der letzten Normentwürfe (final
working draft) ist auf
\href{https://en.cppreference.com/w/cpp/links}{Cppreference/Links} zu finden.
Inwieweit diese Norm von verschiedenen Compilern erfüllt wird, ist auf
\href{https://en.cppreference.com/w/cpp/compiler_support}{Cppreference/Compiler\_support}
zusammengestellt. Hier wird \cpp{}23 verwendet. Eine ausführliche Referenz zu
\cpp{} und C ist auf \href{https://en.cppreference.com/w/}{Cppreference} oder
auch auf \href{https://learn.microsoft.com/de-de/cpp/cpp}{MS Dokumentation zu
C++} zu finden.

Programme können mit einem beliebigen Texteditor geschrieben werden und
anschließend mit einem Compiler in ein ausführbares Programm übersetzt werden.
Integrierte Entwicklungsumgebungen (IDE) helfen beim Schreiben von Programmen
und Werkzeuge wie \lstinline|make| beim compilieren und erstellen von
Bibliotheken (libraries).

Mit \lstinline|g++| wird ein Programm mit dem Befehl \lstinline|g++ Programm.cc|
compiliert und als \lstinline|a.out| abgespeichert. Mit
\lstinline|g++ -o prg Programm.cc| wird eine Datei \lstinline|prg| erzeugt.

Mit \lstinline|clang| wird ein Programm mit dem Befehl
\lstinline|clang++ Programm.cc|
compiliert und als \lstinline|a.out| abgespeichert. Mit
\lstinline|clang++ -o prg Programm.cc| wird eine Datei \lstinline|prg| erzeugt.

Mit \lstinline|-std=c++23| wird angegeben das \cpp{}23 verwendet wird.
Mit z.\,B.\ \lstinline|-O3 -march=native -flto| wird das erzeugte Programm
optimiert, um schneller zu laufen. Mit \lstinline|-Weverything| (nur bei clang),
\lstinline|-Wall| oder \lstinline|-Wextra| gibt der Compiler zusätzliche
Warnungen aus.

Mit \lstinline|msvc| (Developer PowerShell for VS) wird ein Programm mit dem
Befehl \lstinline|cl Programm.cc| compiliert und als \lstinline|Programm.exe|
abgespeichert. Mit \lstinline|cl Programm.cc /O2| wird auf Geschwindigkeit
optimiert. Mit \lstinline|cl Programm.cc /Wall| werden alle Warnungen
ausgegeben. Mit \lstinline|cl Programm.cc /std:c++20| wird \cpp{}20\footnote{23
ist derzeit v17.8.2/19.38.33130 nicht verfügbar. Eventuell
\lstinline|/std:c++latest| oder \lstinline|/experimental| verwenden} verwendet. 

C Dateien enden mit \lstinline|.c|. \cpp{} Dateien haben üblicherweise die
Extension \lstinline|.cc| oder \lstinline|.cpp|. Header Dateien für C enden mit
\lstinline|.h|. Header Dateien für \cpp{} enden meist mit \lstinline|.h|,
\lstinline|.hh| oder \lstinline|.hpp|.

Die Quelldatei (Sourcecode) wird zunächst vom Präprozessor verarbeitet (z.\,B.\
\lstinline|#include|) und wird anschließend compiliert. Danach bindet der Linker
Bibliotheken (Libraries) zu einem ausführbaren Programm zusammen. Mit
\lstinline|g++ -c srcA.cc| wird nur compiliert und die Datei \lstinline|srcA.o|
erzeugt. Mit \lstinline|g++ srcB.cc srcB.o| wird srcB compiliert und die
Objektdatei von srcA dazugebunden.

\section{Basis}
\label{sec:einleitung}

\subsection{Grundstruktur}

Die erste Funktion die bei einem Programmstart aufgerufen wird ist
\lstinline|main|.
\begin{lstlisting}[language=C++]
int main() {}
\end{lstlisting}
\lstinline|main| gibt mit \lstinline|return| eine \lstinline|Integerzahl| an das
Betriebssystem zurück, die angibt, ob das Programm erfolgreich abgeschlossen (0)
oder durch einen Fehler (nicht 0) beendet wurde. In \lstinline|main| kann
\lstinline|return| weggelassen werden, was \lstinline|return 0;| entspricht. In
\lstinline|stdlib.h| bzw.\ \lstinline|cstdlib| sind \lstinline|EXIT_SUCCESS| und
\lstinline|EXIT_FAILURE| definiert.
\begin{lstlisting}[language=C++]
#include <cstdlib>

int main() {
  return EXIT_SUCCESS;
}
\end{lstlisting}
\lstinline|main()| (oder \lstinline|main(void)|) bedeutet, das beim
Programmaufruf keine Argumente an die Funktion übergeben werden. Mit:
\begin{lstlisting}[language=C++]
int main(int argc, char* argv[]) {}
\end{lstlisting}
können Argumente (command line arguments) von der Funktion übernommen werden.
\lstinline|argc| gibt die Anzahl an Argumenten an die in \lstinline|argv| zu
finden sind. Auf Position 0 steht der Programmname.

\subsection{Kommentare}

\begin{lstlisting}[language=C++]
/* Kommentar */

// Kommentar

#if 0
  std::cout << "Wird weder ausgeführt noch Compiliert\n";
#endif

if (false)
{
  std::cout << "Wird nicht ausgeführt\n";
}
\end{lstlisting}

\subsection{Bibliotheken -- Libraries}

Funktionen sind in Bibliotheken abgelegt. Diese müssen aktiv ausgewählt werden,
bevor sie verwendet werden können. Dies geschieht mit dem Präprozessorbefehl
\lstinline|#include|. \lstinline|#include <lib>| sucht die Headerdatei von
\lstinline|lib| im Standardsuchpfad für header Dateien.
\lstinline|#include "lib.h"| sucht im aktuellen Verzeichnis nach der Datei
\lstinline|lib.h|. Wenn diese nicht gefunden wird, wird im Standardsuchpfad für
Headerdateien weiter gesucht. Headerdateien von C enden mit \lstinline|.h|.

\subsection{Ein-- und Ausgabe}

Ein-- und Ausgabe ist in \lstinline|iostream| definiert. Es gibt die Befehle
\lstinline|cin| (Eingabe), \lstinline|cout| (Ausgabe), \lstinline|cerr|
(unmittelbare Fehlerausgabe) und \lstinline|clog| (gepufferte Fehlerausgabe).
Alle Befehle gibt es auch für wide Character mit vorangestelltem w, also z.B.
\lstinline|wcin|.

\begin{lstlisting}[language=C++]
#include <iostream>

int main() {
  int n;
  std::cout << "Anzahl n: ";
  std::cin >> n;
}
\end{lstlisting}

Eine neue Zeile kann mit \lstinline|std::cout << "\n";| oder mit
\lstinline|std::cout << std::endl;| erzeugt werden, wobei \lstinline|endl| den
buffer leert (flush) und damit in der Regel langsamer ist als \lstinline|\n|.

\subsection{Namespace}

Helfen Namenskonflikte zu vermeiden, indem Funktionen bestimmten Namensräumen
zugeordnet werden. Funktionen mit gleichem Nahmen stellen kein Problem dar,
solange sie in verschiedenen Namensräumen definiert sind. Funktionen des
Standardlibraries sind im Namensraum \lstinline|std| und können mit
\lstinline|std::FUNKTIONSNAME| aufgerufen werden. Mit
\lstinline|using namespace std;| kann die Funktion auch ohne \lstinline|std::|
aufgerufen werden. Namespaces können auch selbst erzeugt werden.

\subsection{Datentypen}

\lstinline|bool| ist der logische typ mit \lstinline|true| oder
\lstinline|false|.

Ganzzahlige Typen mit steigender Größe sind \lstinline|char|, \lstinline|short|,
\lstinline|int|, \lstinline|long| und \lstinline|long long|. Diese können mit
\lstinline|signed| und \lstinline|unsigned| also positive und negative oder nur
positive Zahl definiert werden.

Gleitkommazahlen mit steigender Größe sind \lstinline|float|,
\lstinline|double|, \lstinline|long double|.

\subsection{Deklaration, Definition, Initialisierung und Zuweisung}

\begin{lstlisting}[language=C++]
int i;     // Deklaration und Definition, i hat unbekannten Wert
i = 0;     // Zuweisung - Assignment
i = {0};   // Zuweisung - Assignment - EMPFOHLEN
//i = {0.};  // Fehler: narrowing conversion
int j = 1; // Dekl., Def. und Copy-Initialisierung
int jn = 1.5; // jn == 1
int k(2);  // Dekl., Def. und Constructor/Direct-Initialisierung
int kn(2.5);  // kn == 2
// int f();  // Würde die Funktion k deklarieren
// int f() {return 0};  // Dek. und Def. von Funktion f
// f(2);   // Würde die Funktion f mit Argument 2 aufrufen
int l{};   // Dekl., Def. und Value-Initialisierung, l == 0
int m{3};  // Dekl., Def. und Uniform/Direct-Init. - EMPFOHLEN
// int mn{3.5};  // Fehler: narrowing conversion
\end{lstlisting}

\subsection{Blöcke -- Compound / block}

Fassen mehrere Anweisungen zu einer zusammen. Sie beeinflussen die Gültigkeit
und Sichtbarkeit von Variablen.

\begin{lstlisting}[language=C++]
#include <fstream>
int i{0};  // i0, global
int main()
{ // Beginn des äußeren Blocks
  int i{1};  // i1 verdeckt globales i
  int j{3};
  {  // Beginn des inneren Blocks
    std::ofstream f("test.txt");  // Datei test.txt öffnen
    f << i << "\n";               // i1 schreiben
    int i {2};  // lokales i2, verdeckt i1, auf welches ab hier
                //   nicht mehr zugegriffen werden kann
    f << i << "\n";               // lokales i2 schreiben
    f << ::i << "\n";             // globales i0 schreiben
    f << j << "\n";               // j3 schreiben
  }  // Ende des inneren Blocks
     //   Destruktor Aufruf von f -> flush und schließen
     //   lokales i2 entfernt, i1 wider sichtbar
} // Ende des äußeren Blocks
\end{lstlisting}

\section{Kontrollstrukturen}

\subsection{Auswahlanweisungen -- Selection}

\subsubsection{if else}

Ausführen von Code, wenn Bedingung erfüllt ist.

\begin{lstlisting}[language=C++]
int i{1};

if (i > 2)
  if (i < 4)
    i = {0};
  else  // gehört zu if (i < 4)
    i = {1};
else  // gehört zu if (i > 4)
  i = {2};

if (i > 2)  // das Gleiche nur mit Klammern
{
  if (i < 4)
  {
    i = {0};
  }
  else
  {
    i = {1};
  }
}
else
{
  i = {2};
}

if (int ii = i*i; i > 0)  // mit Initialisierer
  i -= ii;
else {
  // int ii;  // error: redeclaration of ii
  i += ii;
}
// ++ii;  // ii hier nicht bekannt

// typedef declaration
if (typedef int Foo; true) { (void)Foo{}; }

// alias declaration
if (using Foo = int; true) { (void)Foo{}; }

constexpr bool TEST = true;
if constexpr (TEST) {  //Auswertung wenn Compiliert wird
  if (i == 0) i = {1};
}
i = {10 / i};
\end{lstlisting}

Mit \lstinline|if consteval| testen ob constexpr-Funktion zur Compilationszeit
ausgewertet wird. Hier müssen Blöcke (compound \lstinline|{ }|) verwendet
werden.

\begin{lstlisting}[language=C++]
#include <iostream>

constexpr int f(int x) noexcept
{
  if consteval { return 0; } else { return x; }
}

int main()
{
  const int i { f(2) };  // Auswertung zur Compilationszeit
  int j{2};             // nicht konstant
  const int k { f(j) };  // Auswertung zur Laufzeit
  std::cout << i << " " << k << "\n";
}
\end{lstlisting}

\subsubsection{?: -- dreiteilige Bedingung / Ternary conditional}

\lstinline|E1 ? E2 : E3| Wenn E1 true, dann E2 sonst E3.

\begin{lstlisting}[language=C++]
int n = {1 > 2 ? 7 : 8};  // 1 > 2 ist false, so n = 8
int m{0};
(n == m ? n : m) = {9};  // n == m ist false, so m = 9
\end{lstlisting}

\subsubsection{switch}

Springe, in Abhängigkeit eines Wertes, an eine Programmzeile. Der Wert muss
\lstinline|int| oder \lstinline|enum| oder implizit in int umwandelbar
sein.

\begin{lstlisting}[language=C++]
#include <iostream>

int main()
{
  const int k{9};
  switch (int i{k / 2}; i)  // i nur in switch bekannt
  // switch (k)  // Variante ohne Initialisierer
  {
    case 1:
      std::cout << '1';
      break;
    [[unlikely]] case 2:
      std::cout << '2';
      [[fallthrough]]; // nicht nötig, Hinweise das beabsichtigt
    [[likely]] case 3:
      std::cout << '3';
      break;
    case 4:
      //int j{0};  // Error weil man in den scope von
                    // initialisiertem i springen kann
      int j;
      j = {0};
      break;
    case 5:
    {
      int j{0};
      break;
    }
    case 6:
      // int i;  // error: redeclaration of i
    default:
      std::cout << "Default\n";
  }
}
\end{lstlisting}

\subsection{Schleifen, Wiederholungen -- Iteration}
\subsubsection{for}

Schleife mit Startwert, Bedingung (wird vor jeder Iteration geprüft) und
iteration-expression (wird nach jedem loop ausgeführt).

\begin{lstlisting}[language=C++]
#include <iostream>
#include <tuple>
#include <vector>

int main()
{
  double s{0.};
  int j{1};  // Wird in Schleife verdeckt
  for (int i{1}, j{9}; i < 10; ++i, --j)
  {
    double x{0.5};  // constructor and destructor wird bei
                    // jedem loop aufgerufen
    //double i{0.};  // Error: redeclaration von i
    s += i * j * x;
  }
  //--i;  // Error: Out of scope
  std::cout << s << "\n";

  char cstr[]{"Hallo"};
  for (int n = 0; char c = cstr[n]; ++n) std::cout << c;

  s = {0.};
  // Verschiedene Typen mittels structured binding declaration
  for (auto [i, f] = std::tuple{int{1}, float{0.1}}; i < 10 ;
       ++i, f += .1)
  {
    s += i * f;
    //int i;  // Error: redeclaration von i
  }
  std::cout << s << "\n";

  s = {0.};
  {
    int i{1};
    float f{0.1};
    for (; i < 10 ; ++i, f += .1) {
      s += i * f;
      int i;
    }
  }
  std::cout << s << "\n";
}
\end{lstlisting}

Float Zähler (counter) vermeiden.

\begin{lstlisting}[language=C++]
#include <iostream>
#include <cmath>

int main()
{
  int j{0};
  for (float x{0.}; x <= 1.; x += .01)
    std::cout << j++ << ":" << x << " ";
  std::cout << "\n";
  float start{0.}, end{1.}, inc{.01};
  long n{lround((end - start) / inc)};
  for (long i{0}; i <= n; ++i)
    std::cout << i << ":" << (start + (inc * i)) << " ";
  std::cout << "\n";
  for (long i{0}; i <= 100; ++i)  //Empfohlen
    std::cout << i << ":" << i/100.f << " ";
}
\end{lstlisting}

\subsubsection{range-for}

Schleife über einen Bereich (range).

\begin{lstlisting}[language=C++]
#include <iostream>
#include <vector>
#include <algorithm>
#include <ranges>

void o(const std::vector<int>& v, const char* s = {""}) {
  std::cout << s;
  for (const auto& i : v) std::cout << i << ' ';
  std::cout << "\n";
}

int main()
{
  std::vector<int> v = {2, 0, 1};

  // Index based. Geht nur bei sequential random access
  for (std::size_t i{0}; i < v.size(); ++i) ++v[i];
  o(v, "0: ");  // 3 1 2

  // Iterator based
  for (auto it{v.begin()}; it != v.end(); ++it) ++*it;
  o(v, "1: ");  // 4 2 3

  for (auto& e : v) ++e;  // Über Reference
  o(v, "2: ");  // 5 3 4

  for (auto e : v) ++e;  // Kopie
  o(v, "3: ");  // 5 3 4

  for (auto&& e : v) ++e; // forwarding reference
  o(v, "4: ");  // 6 4 5

  std::ranges::for_each(v, [](auto& e) { ++e; });
  o(v, "5: ");  // 7 5 6

  std::for_each(v.begin(), v.end(), [](auto& e) { ++e; });
  o(v, "6: ");  // 8 6 7

  for (int i{0}; const auto& e : v) // Mit init-statement
    std::cout << i++ << ':' << e << ' ';
  std::cout << '\n';

  for (const auto& [i, e] : std::views::enumerate(v))
    std::cout << i << ':' << e << ' ';
  std::cout << '\n';

  for (int i : {3, 1, 2}) std::cout << i; // braced-init-list
  std::cout << '\n';

  // Variable nicht verwendet
  for ([[maybe_unused]] int i : v) std::cout << 'X';
}
\end{lstlisting}

\subsubsection{while}

Führt eine Anweisung so lange aus, bis die Bedingung \lstinline|false| wird.
Geprüft wird \emph{vor} jeder Iteration.

\begin{lstlisting}[language=C++]
#include <iostream>

int main()
{
  int i{0};
  while (i < 3) ++i;
  std::cout << i << '\n';

  while (i < 9)
  {
    std::cout << i << '\n';
    i += 2;
  }

  char cstr[]{"Hallo"};
  i = {0};
  while (char c{cstr[i++]})
    std::cout << c;
}
\end{lstlisting}

\subsubsection{do-while}

Führt eine Anweisung so lange aus, bis die Bedingung \lstinline|false| wird.
Geprüft wird \emph{nach} jeder Iteration.

\begin{lstlisting}[language=C++]
#include <iostream>

int main()
{
  int i{0};
  do
  {
    i += 2;
    std::cout << i << '\n';
  }
  while (i < 9);
}
\end{lstlisting}

\subsection{Sprunganweisungen -- Jump}
\subsubsection{continue}

Überspringt den verbleibenden Teil des umschließenden \lstinline|for| oder
\lstinline|while| Schleifenkörpers.

\begin{lstlisting}[language=C++]
#include <iostream>

int main()
{
  for (int i{0}; i < 3; ++i)
  {
    for (int j{0}; j < 3; ++j)
    {
      if (j == 1) continue;  // wirkt bei j for loop
      std::cout << i << ' ' << j << '\n';
    }
  }
}
\end{lstlisting}

\subsubsection{break}

Beendet umschließendes \lstinline|for|, \lstinline|while| oder
\lstinline|switch|.

\begin{lstlisting}[language=C++]
  #include <iostream>

int main()
{
  for (int i{0}; i < 3; ++i)
  {
    for (int j{0}; j < 3; ++j)
    {
      if (j == 1) break;  // wirkt bei j for loop
      std::cout << i << ' ' << j << '\n';
    }
  }
}
\end{lstlisting}

\subsubsection{goto}

Setzt Programm an anderer Stelle fort.

\begin{lstlisting}[language=C++]
#include <iostream>

struct nt {  // non-trivial destructor
  ~nt() { std::cout << "X\n";}
};

int main()
{
  for (int i{0}; i < 3; ++i)
  {
    for (int j{0}; j < 3; ++j)
    {
      if (j == 1) goto endloop; // Ausstieg aus beiden for loops
      std::cout << i << ' ' << j << '\n';
    }
  }
endloop:  // Sprungmarke label
  goto label2; // Sprung in den scope von n
//int n{0};  //error: jump bypasses variable initialization
  [[maybe_unused]] int n; // no initializer
  n = {7};
//nt o;  // error: bypas variable with non-trivial destructor
label2:
  std::cout << n;  // n hat Zufallswert
}
\end{lstlisting}

\subsubsection{return}

Beendet Funktion und gibt Wert (falls vorhanden) an den Aufrufer zurück.

\begin{lstlisting}[language=C++]
#include <iostream>
#include <utility>

void fa(int i) {
  if (i == 1) return;
  std::cout << "fa("<< i << ")\n";
} // impliziertes return;

int fb(int i) {
  if (i > 4) return 4;
  std::cout << "fb(" << i << ")\n";
  return 2; }

std::pair<int, float> fc(int i, float x) {
  return {i, x}; }

int main()
{
  fa(0);  // Gibt fa(0) aus
  fa(1);  // Nichts
  int i{fb(5)};  // i == 4
  i = fb(i);     // gibt fb(4) aus, i == 2
  std::cout << fc(0, 1.).first << " " << fc(2, 3.).second; //0 3
}
\end{lstlisting}

\section{Ausdruck -- Expression}

Operatoren (z.\,B.\ +,-,*,/,\dots) führen etwas mit Operanden aus. Bei der
Auswertung (evaluation) kann ein Ergebnis (z.\,B.\ 4 bei \lstinline|2+2|) und
oder Nebenwirkungen (side-effects) (z.\,B.\ gibt
\lstinline|std::printf("\%d", 4)| 4 aus) entstehen.

\subsection{Wertkategorien -- Value categories}

\lstinline|lvalue = prvalue;|

\begin{description}
  \item[lvalue] Steht auf der linken Seite einer Zuweisung.
  \begin{itemize}
    \item Adresse eines Ausdrucks kann übernommen werden.
    \item Typ eines Ausdrucks ist eine L-Wert-Referenz (z.\,B.\ \lstinline|int&|
    oder \lstinline|const int&|)
  \end{itemize}
  \item[prvalue] Steht auf der rechten Seite einer Zuweisung. R-Werte
  entsprechen temporären Objekten, wie sie beispielsweise von Funktionen
  zurückgegeben oder durch implizite Typkonvertierungen erstellt werden. Die
  meisten Literale (z.\,B.\ \lstinline|7| oder \lstinline|7.8|) sind ebenfalls
  R-Werte.
  \item[xvalue] Verschwindende (expiring) lvalue.
  \item[glvalue] lvalue oder xvalue
  \item[rvalue] prvalue oder xvalue
\end{description}

\subsection{Operator}

Diese werden nach definierten Reihenfolgen und Richtungen evaluiert.

\begin{description}
  \item[assignment] \lstinline|a = b| :
    \lstinline"= += -= *= /= \%= \&= |= ^= <<= >>="
  \item[increment / decrement] \lstinline|++a --a a++ a--|
  \item[arithmetic]
    \lstinline|+a| \lstinline|-a| \lstinline|~a|;
    \lstinline|a + b| :
    \lstinline"+ - * / \% & | ^ << >>"
  \item[logical]
    \lstinline|!a|
    \lstinline|a && b|
    \lstinline"a || b"
  \item[comparison] \lstinline|a == b| :
    \lstinline|== != < > <= >= <=>|
  \item[member access] \lstinline|a[b] a[...] *a &a a->b a.b a->*b a.*b|
  \item[function call] \lstinline|a(a1, a2)|
  \item[comma] \lstinline|a, b|
  \item[conditional] \lstinline|a ? b : c|
  \item[Conversions] \lstinline|const_cast| \lstinline|static_cast|
    \lstinline|dynamic_cast| \lstinline|reinterpret_cast|
  \item[Memory allocation] \lstinline|new delete|
  \item[Other] \lstinline|sizeof alignof typeid throw-expression|
\end{description}

\subsection{Constant}

\begin{description}
  \item[const] Wert kann nicht geändert werden.
  \item[constexpr] Wert liegt beim Kompilieren vor.
\end{description}

\begin{lstlisting}[language=C++]
#include <iostream>
#include <array>

int f0 ( int a, int b ) { return a + b; }
constexpr int f1 ( int a, int b ) { return a + b; }

int main()
{
  int n{2};
//std::array<int, n> a0;  // Error: Argument not const expr
  const int cn{2};
  std::array<int, cn> a1;
  constexpr int cn2{2};
  std::array<int, cn2> a2;
//std::array<int, f0(1, 1)> a3; // Error: non-constexpr function
  std::array<int, f1(1, 1)> a4;
  ++n;
//++cn;   // Error: declared const
//++cn2;  // Error: declared const
}
\end{lstlisting}

\subsection{Operator overloading}

Operatoren (z.\,B.\ +) für benutzerdefinierte Typen definieren. Overloaded operators sind Funktionen mit speziellen Funktionsnamen.

\begin{lstlisting}[language=C++]
#include <iostream>

struct Foo {
  int a;
  double b;
  Foo& operator++()  // prefix increment
  {
    ++a;
    ++b;
    return *this; // return new value by reference
  }
};

std::ostream& operator<<(std::ostream& out, Foo const& f) {
  return out << f.a << " " << f.b;
}

int main() {
  Foo x = {0, 1.5};
  std::cout << x << "\n";  // 0 1.5
  ++x;
  std::cout << x << "\n";  // 1 2.5
}
\end{lstlisting}

\subsection{Conversions}

\subsubsection{Implicit}

\begin{lstlisting}[language=C++]
float f{ 0 };  // int nach float
f = 1;         // int nach float
if (2) {}      // int nach bool
\end{lstlisting}

\subsubsection{Explicit}

\begin{lstlisting}[language=C++]
int i;
i = 0.5;  // Implicit
i = (int) 1.5;
i = int(2.5);
//i = int{3.5};  // Error: narrowing conversion
i = auto(2.5);
i = auto{3.5};
\end{lstlisting}

\subsubsection{User-defined}

\begin{lstlisting}[language=C++]
#include <iostream>

struct foo {
  int i{2};
  // implicit conversion
  operator int() const { return 1; }
  // explicit conversion
  explicit operator int*() const { return const_cast<int*>(&i);}
};

int main() {
  foo x;
  int i = x;
  std::cout << i << ' ' << (int)x << ' ' << int(x) << ' ' <<
    int{x} << ' ' << static_cast<int>(x) << '\n';  // immer 1
//int* j = x;  // Error: no implicit conversion
  int* j = (int*)x;
  std::cout << *j << ' ' << *((int*)x) << ' ' <<
    *static_cast<int*>(x) << '\n';  // immer 2
}
\end{lstlisting}

\subsubsection{Usual arithmetic conversions}

\begin{lstlisting}[language=C++]
#include <iostream>
#include <typeinfo>

int main() {
  int i = 2;
  long l = 5l;
  auto n = i + l;
  std::cout << typeid(n).name() << '\n';  // l
  float x = 1.f;
  double y = 2.;
  auto z = x + y;
  std::cout << typeid(z).name() << '\n';  // d
}
\end{lstlisting}

\subsubsection{static\_cast}

\lstinline|static_cast<int>(2.)| macht das gleiche wie \lstinline|int(2.)| oder
\lstinline|(int)2.|, ist aber bei einer Suche leichter zu finden.

\begin{lstlisting}[language=C++]
long l = 5l;
//int i{l};  // narrowing conversion
int i{static_cast<int>(l)};
int j{int(l)};
int k{(int)l};
\end{lstlisting}

\subsubsection{dynamic\_cast}

Bei virtual Vererbung für Zeiger oder Referenz auf eine Klasse.

\begin{lstlisting}[language=C++]
#include <iostream>
struct Animal   { virtual ~Animal() = default; int i{0}; };
struct Creature { virtual ~Creature() = default;
                  int i{1}; int j{2}; };
struct Bird : public Animal, Creature { };

int main() {
  Bird *bird = new Bird();
  Creature *creature = dynamic_cast<Creature*>(bird);
  Animal *animal = dynamic_cast<Animal*>(creature);
  Creature *creature1 = new Creature();
//Bird *bird1 = dynamic_cast<Animal*>(creature1);  // invalid conversion
  Animal *animal1 = dynamic_cast<Animal*>(creature1);
//std::cout << bird->i << '\n';  // Error: ambiguous
  std::cout << bird->Animal::i << '\n';
  std::cout << bird->Creature::i << ' ' << bird->j << '\n';
  std::cout << creature->i << ' ' << creature->j << '\n';
  std::cout << animal->i /*<< ' ' << animal->j*/ << '\n';
  std::cout << creature1->i << ' ' << creature1->j << '\n';
//std::cout << animal1->i << '\n'; //Seg fault
  delete bird;
  delete creature1;
}
\end{lstlisting}

\subsubsection{const\_cast}

Erzeugt eine nicht konstante Referenz oder Zeiger auf ein konstantes Objekt, das
damit verändert werden kann und damit undefiniertes Verhalten verursachen kann.

\begin{lstlisting}[language=C++]
#include <iostream>

int main() {
  int i = 0;
  const int& rci = i;
//rci = 1;  /Error: read-only reference
  const_cast<int&>(rci) = 2;  // OK
  std::cout << i << ' ' << rci << '\n';  // 2 2

  const int j = 3;
  int* pj = const_cast<int*>(&j);
  *pj = 4;      // undefined behavior
  std::cout << j << ' ' << *pj << '\n';  // ? ?
}
\end{lstlisting}

\subsubsection{reinterpret\_cast}

Konvertiert zwischen Typen durch Neuinterpretation des zugrunde liegenden
Bitmusters.

\begin{lstlisting}[language=C++]
#include <iostream>

int main() {
  signed char i = -1;
  std::cout << +i << ' ' <<                       // -1
   +reinterpret_cast<unsigned char&>(i) << '\n';  // 255
}
\end{lstlisting}

\subsection{Literals}

Im Quellcode eingebettete konstante Werte.
\begin{description}
  \item[boolean] \lstinline|true|, \lstinline|false|
  \item[integer] ~
  \begin{description}
    \item[integer-suffix] ~
    \begin{description}
      \item[u U] Unsigned \lstinline|0u|
      \item[l L] Long \lstinline|0l|
      \item[ll LL] Long long \lstinline|0ll|
      \item[z Z] size \lstinline|std::size_t i{0uz};|
    \end{description}
    \item[decimal-literal] \lstinline|42|
    \item[octal-literal] \lstinline|052|
    \item[hex-literal] \lstinline|0x2a| \lstinline|0X2A|
    \item[binary-literal] \lstinline|0b101010| \lstinline|0B101010|
  \end{description}
  \item[floating] ~
  \begin{description}
    \item[3.1f] float
    \item[3.1] double
    \item[3.1l] long double
    \item[3.1f16] float16; f32, f64, f128, bf16
    \item[4e2] double
    \item[3.14'15'92] double, single quotes ignored
  \end{description}
  \item[character] ~
  \begin{description}
    \item['c'] Gewöhnliches Zeichenliteral
    \item[u8'c'] UTF8
    \item[u'c'] UTF16
    \item[U'c'] UTF32
    \item[L'c'] Wide character
    \item['char'] Literal mit mehreren Zeichen
  \end{description}
  \item[string] ~
  \begin{description}
    \item[\dq\dq] Gewöhnliches Stringliteral
    \item[\dq{}abc\dq] Gewöhnliches Stringliteral
    \item[R\dq{}(abc)\dq] Raw string literals
    \item[R\dq{}any(abc)any\dq] Raw string literals
    \item[u8\dq{}abc\dq] UTF8
    \item[u\dq{}abc\dq] UTF16
    \item[U\dq{}abc\dq] UFT32
    \item[L\dq{}abc\dq] Wide
  \end{description}
  \item[Escape sequences] Für spezielle Zeichen
  \begin{description}
    \item[\textbackslash{}'] single quote
    \item[\textbackslash{}"] double quote
    \item[\textbackslash{}?] question mark
    \item[\textbackslash{}\textbackslash{}] backslash
    \item[\textbackslash{}a] audible bell
    \item[\textbackslash{}b] backspace
    \item[\textbackslash{}f] form feed - new page
    \item[\textbackslash{}n] line feed - new line
    \item[\textbackslash{}r] carriage return
    \item[\textbackslash{}t] horizontal tab
    \item[\textbackslash{}v] vertical tab
    \item[\textbackslash{}nnn] Octal code 1-3 stellig
    \item[\textbackslash{}o\{n\dots\}] Octal code
    \item[\textbackslash{}xn\dots] Hex Code
    \item[\textbackslash{}x\{n\dots\}] Hex Code
    \item[\textbackslash{}unnnn] Unicode Wert 4 stellig
    \item[\textbackslash{}Unnnnnnnn] Unicode Wert 8 stellig
    \item[\textbackslash{}u\{n\dots\}] Unicode Wert
    \item[\textbackslash{}N\{Name\}] Unicode Name
    \item[\textbackslash{}c] Conditional escape sequence
  \end{description}
  \item[nullptr] Nullpointer literal
  \item[user-defined] Literals können auch selber definiert werden
\end{description}

\section{Deklaration}

\begin{description}
  \item[Deklaration] Bekanntgabe von Namen
  \item[Definition] Deklaration die ausreicht um sie verwenden zu können
\end{description}

\begin{lstlisting}[language=C++]
int f(int); // deklariert Funktion f, aber definiert sie nicht
int g(int x) {  // definert Funktion g
  return x + 1;
}
\end{lstlisting}

Eine Deklaration kann keine bereits bestehende Deklaration im selben Namensraum
(Namepace) überschreiben (Conflicting declarations).

\begin{lstlisting}[language=C++]
int i;
//int i; // Error:  redeclaration
{
  int i;  // OK: Im Block
}
int f(float x);
int f(int x);   // OK Function Overloading
\end{lstlisting}

\subsection{Storage class specifiers}

\begin{description}
  \item[~] Nichts: automatische Speicherdauer. Dauer: Vom Beginn bis zum Ende des Blocks.
  \item[static] Nur bei Deklaration von Objekten, Funktionen oder anonymen
  unions. Greift bei jeder Instanz auf die gleiche Variable zurück. Dauer: Vom Beginn bis zum Ende des Programms.

\begin{lstlisting}[language=C++]
#include <iostream>

class A {
  static int N;
  public:
  A() { ++N; }
  int n() {return N;}
};
int A::N{0};

int main() {
  A x;
  std::cout << x.n() << '\n';  // 1
  A y;
  std::cout << x.n() << ' ' << y.n() << '\n';  // 2 2
}
\end{lstlisting}

  \item[extern] Für Zugriff auf variablen oder Funktionen, die in anderen souce-Dateien definiert sind.

\begin{lstlisting}[language=C++]
//Datei1
int a{1};
const int b{2};  // Gilt nur in Datei1
extern const int c{3};
\end{lstlisting}
\begin{lstlisting}[language=C++]
//Datei2
#include <iostream>
extern int a;
extern const int b;  // Kommt nicht von Datei1
extern const int c;

int main() {
  std::cout << a << '\n';  // 1
//std::cout << b << '\n';  // undefined reference to `b'
  std::cout << c << '\n';  // 3
}
\end{lstlisting}

  \item[thread\_local] Jeder thread hat eine eigen Kopie der Variable. Dauer: Vom Beginn bis zum Ende des threads.
  \item[mutable] Erlaubt die Veränderung eines Klassenmitglieds, wenn das Objekt als const deklariert ist.
\end{description}

\subsection{Translation-unit-local}

Verhindert, dass lokale Instanzen in anderen Dateien verwendet werden

\begin{lstlisting}[language=C++]
//Datei1
int a{1};  // Kann mit extern in anderem Source verwendet werden
static int b{2};  // Lokal verfügbar
namespace {
  int c{3};      // Lokal verfügbar
}
\end{lstlisting}
\begin{lstlisting}[language=C++]
//Datei2
#include <iostream>
extern int a;
extern int b;  // Kommt nicht von Datei1
extern int c;  // Kommt nicht von Datei1

int main() {
  std::cout << a << '\n';  // 1
//std::cout << b << '\n';  // undefined reference to `b'
//std::cout << c << '\n';  // undefined reference to `c'
}
\end{lstlisting}

\subsection{Language linkage}

\lstinline|extern| zur Verknüpfung von Funktionen, die in einer anderen
Programmiersprache geschrieben wurden.

\subsection{Namespace declaration}

Zur Vermeidung von Namenskonflikten.
Bezeichner im Unnamed namespace sind nur in der Übersetzungseinheit sichtbar.

\begin{lstlisting}[language=C++]
#include <iostream>

int i{0};  // global namespace
int j{1};

namespace {  // Unnamed namespace
  int i{2};  // Nicht erreichbar wegen globalem i
  int k{3};
}

namespace A {
  int i{4};
  int n{12};
  namespace B {
    int i{5};
    extern int j;
    extern int k;
  }
//int B::i{6};  // Error: Redefinition von i
  int B::j{7};
//int B::l{8};  // Error: B::l nicht deklariert
  inline namespace C {  // In C und A
    int m{9};
    int n{13};  // Verdeck A::n
  }
  namespace {  // In A
    int o{14};
  }
}
namespace D {
//int A::B::k{10};  // Error: Does not enclose B
}
int A::B::k{11};

namespace A::B {  // nested namespace definition
  int p{15};
}

namespace A::inline D {
  int q{16};
}

namespace X = A::B;  // Namespace alias

int main() {
//std::cout << i << '\n';  // ambiguous
  std::cout << ::i << '\n';  // 0
  std::cout << j << ' ' << ::j << '\n';  // 1 1
  std::cout << k << ' ' << ::k << '\n';  // 3 3
  std::cout << A::i << ' ' << ::A::i << '\n';  // 4 4
  std::cout << A::B::i << '\n';  // 5
  std::cout << A::B::j << '\n';  // 7
  std::cout << A::B::k << '\n';  // 11
  std::cout << A::m << ' ' << A::C::m << '\n';  // 9 9
//std::cout << A::n << '\n';  // ambiguous
  std::cout << A::C::n << '\n';  // 13
  std::cout << A::o << '\n';  // 14
  std::cout << A::B::p << '\n';  // 15
  std::cout << A::q << ' ' << A::D::q << '\n';  // 16 16
  std::cout <<X::i <<' ' <<X::j <<' ' <<X::k <<'\n'; // 5 7 11
}
\end{lstlisting}

Mit \lstinline|using| können Inhalte von Namensräume sichtbar gemacht werden. In
Header Dateien \lstinline|using namespace| vermeiden, da jeder der den Heder
verwendet auch diesen namespace mitverwendet.

\begin{lstlisting}[language=C++]
#include <iostream>

int i{0};

namespace A {
  int i{1};
  int j{2};
}

namespace B {
  int i{3};
  int k{4};
}

namespace C {
//using ::i, A::i;  // Error: Name conflict
  using ::i, A::j; // Werden in C sichtbar
}

int main() {
  using namespace A;
  using namespace B;
//std::cout << i << '\n';  // ambiguous
  std::cout << A::i << ' ' << B::i << '\n';  // 1 3
  std::cout << j << ' ' << k << '\n';  // 2 4
  std::cout << C::i << ' ' << C::j << '\n';  // 0 2
}
\end{lstlisting}

\subsection{References}

Mit \lstinline|&| (lvalue) bzw.\ \lstinline|&&| (rvalue) wird ein zusätzlicher
Namen für ein bereits vorhandenes Objekt vergeben.

\begin{lstlisting}[language=C++]
#include <iostream>

void add(int& x) { ++x; }  // Direkt Zugriff auf übergebenes x
int& set(int& x) { return x; }  // lvalue expression
int& bad() {  // Dangling reference
  int i{0};
  return i;  // return reference to local variable
}  // Aufruf Destruktor von i

int main() {
  int i{0};
  int& r = i;
  const int& cr = i;
  ++r;
  std::cout << i << '\n';  // 1
//++cr;  // Error: cr ist const
  add(i);
  std::cout << i << '\n';  // 2
  add(r);
  std::cout << i << ' ' << r << ' ' << cr << '\n';  // 3 3 3
  set(i) = 5;
  std::cout << i << '\n';  // 5

//int&& r1 = i;  // Error: can't bind to lvalue
  int&& r2 = i + 0;
  std::cout << i << ' ' << r2 << '\n';  // 5 5
  r2 += 5;
  std::cout << i << ' ' << r2 << '\n';  // 5 10
  i += 5;
  std::cout << i << ' ' << r2 << '\n';  // 10 10

  int& j = bad();
//std::cout << j << '\n';  // Undef. bis Seg. fault
}
\end{lstlisting}

\subsection{Pointer -- Zeiger}

Zeiger enthalten üblicherweise die Anfangsadresse eines Objekts. Bei der
Deklaration wird \lstinline|*| verwendet. Die Adresse eines Objekts erhält man
durch voranstellen von \lstinline|&| vor dem Namen. Zeiger werden mit
\lstinline|*| vor dem Zeigernamen dereferenziert um auf deren Inhalt und nicht
auf ihre Adresse zuzugreifen.

\begin{lstlisting}[language=C++]
#include <iostream>

float f0(const int& i) {return i + .5;}
float f1(const int& i) {return i - .5;}

int main() {
  int i{0};
  int* p{&i};  // Pointer auf i
  int& r{i};   // Referenz auf i
//int&* pr;  // Error: Pointer auf Referenz
  int& rd{*p};  // Referenz auf dereferenzierten Pointer
  int*& rp{p};  // Referenz auf Pointer
  std::cout << i << ' ' << *p << ' ' << r << ' ' << rd
            << ' ' << *rp << '\n';  // 0 0 0 0 0

  int j{0};
  const int* q{&i};  // Kann i nicht verändern
//++*q;
  q = &j;
  int const* q2{&i};  // Kann i nicht verändern
//++*q2;
  q2 = &j;
  int* const s{&i};  // Kann die Pointeradresse nicht verändern
  ++*s;
//s = &j;
  const int* const t{&i};
//++*t;
//t = &j;
  int const* const t2{&i};
//++*t2;
//t2 = &j;
  int** pp{&p};  // Pointer auf Pointer
  int* const* cp{&p}; // Const P. auf P.
  std::cout << **pp << ' ' << **cp << '\n'; // 1 1

  p = nullptr;  // Empfohlen Null Pointer
  p = NULL;
  p = 0;

  int a[]{0,1};  // Array
  std::cout << a[0] << ' ' << a[1] << '\n';  // 0 1
  int (*ap)[2]{&a};  // Pointer auf Array
  std::cout << (*ap)[0] << ' ' << (*ap)[1] << '\n';  // 0 1
  int* a0{a};  // int Pointer auf erstes Element
  std::cout << a0[0] << ' ' << a0[1] << '\n';  // 0 1
  std::cout << *a0 << ' ' << *(a0+1) << '\n';  // 0 1

  void* pv{&i};  // Void kann auf alles Zeigen
  std::cout << i << ' ' << *static_cast<int*>(pv) << '\n';  // 1 1

  float (*pf0)(const int&){&f0};
  float (*pf1)(const int&){f1};  // Implizite Umwandlung in &f1
  float (*af[2])(const int&);  // Pointer zu Funktion Array
  af[0] = &f0;
  af[1] = &f1;
  using F = float(const int&);  // named type alias
  F* af2[]{f0, f1};              // vereinfacht Deklaration
  std::cout << f0(1) << ' ' << f1(1) << '\n';  // 1.5 0.5
  std::cout << pf0(1) << ' ' << (*pf1)(1) << '\n';  // 1.5 0.5
  std::cout << af[0](1) << ' ' << af[1](1) << '\n';  // 1.5 0.5
  std::cout << af2[0](1) << ' ' << af2[1](1) << '\n';  // 1.5 0.5

  struct K { int m; };
  int K::* pK{&K::m};          // Pointer auf m der Klasse K
  K k{7};
  std::cout << k.m << ' ' << k.*pK << '\n';   // 7 7
  K* kp{&k};
  kp->m = 10;
  std::cout << k.m << ' ' << kp->m << ' ' << (*kp).m <<
     ' ' << kp->*pK << '\n'; // 10 10 10 10

  struct Base { int m; };
  struct Derived : Base { int n; };
  Derived d;
  d.m = 1;
  int Base::* bpm = &Base::m;
  int Derived::* dpm = bpm;
  std::cout << d.*dpm << ' ' << d.*bpm << '\n';  // 1 1
  Base* bp{&d};  // implicit conversion
  std::cout << bp->m << ' ' << (*bp).m << '\n';  // 1 1
  std::cout << (*bp).*bpm << ' ' << bp->*bpm << '\n';  // 1 1
  d.n = 7;
  int Derived::* dpn = &Derived::n;
  int Base::* bpn = static_cast<int Base::*>(dpn);
  std::cout << d.*bpn << '\n';  // 7
  std::cout << bp->*bpn << '\n';  // 7
  Base B;
//std::cout << B.*bpn << '\n';  // Undefined

  struct E { void f(int n) { std::cout << n << '\n'; } };
  void (E::* pf)(int) = &E::f; // Pointer auf Funktion f in E
  E e;
  e.f(4);  // 4
  (e.*pf)(5);  // 5
  E* ep{&e};
  (ep->*pf)(6);  // 6
}
\end{lstlisting}

%Arrays
%Structured bindings
%Enumerations and enumerators
%inline specifier
%Inline assembly
%const/volatile
%constexpr
%consteval
%constinit
%decltype
%auto
%typedef
%Type alias
%Elaborated type specifiers
%Attributes
%alignas
%static\_assert


%\section{Initialisation}

%\section{Functions}

%\section{Classes}

%\section{Templates}

%\section{Exceptions}

%\section{Prepocessor}


%Autor: Georg Kindermann

\end{document}
