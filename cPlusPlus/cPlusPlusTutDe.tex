\documentclass[10pt,twocolumn]{scrartcl}

\usepackage[utf8]{inputenc}
\usepackage[T1]{fontenc}
\usepackage[ngerman]{babel}

\usepackage{lmodern}
\usepackage[sc]{mathpazo} % or option osf
\usepackage{newpxmath}
\usepackage[scaled = 0.75]{beramono}

\newcommand*\cpp{C\kern-0.3ex\raisebox{0.17ex}{\texttt{+\kern-0.3ex+}}}

\usepackage{xpatch}
\usepackage{xcolor}
\usepackage{realboxes}
\definecolor{mygray}{rgb}{0.9,0.9,0.9}
\usepackage{listings}
\lstdefinelanguage{commentonly}{ morecomment=[l]{\#} }
\lstset{literate=%
{Ö}{{\"O}}1
{Ä}{{\"A}}1
{Ü}{{\"U}}1
{ß}{{\ss}}1
{ü}{{\"u}}1
{ä}{{\"a}}1
{ö}{{\"o}}1
,basicstyle=\ttfamily
,backgroundcolor=\color{mygray}
,commentstyle=\emph
,language=commentonly
,upquote=true
}
%\makeatletter
%\xpretocmd\lstinline{\Colorbox{mygray}\bgroup\appto\lst@DeInit{\egroup}}{}{}
%\makeatother
\makeatletter
\xpretocmd\lstinline
  {%
   \bgroup\fboxsep=1.5pt
   \Colorbox{mygray}\bgroup\kern-\fboxsep\vphantom{\ttfamily\char`\\y}%
   \appto\lst@DeInit{\kern-\fboxsep\egroup\egroup}%
  }{}{}
\makeatother

\usepackage{enumitem}

\usepackage{adjustbox}
\usepackage[a4paper, margin=1mm, includefoot, footskip=15pt]{geometry}

\usepackage[pdftitle={Grundlagen der Programmiersprache C++}
, pdfauthor={Georg Kindermann}
, pdfsubject={C++}
, pdfkeywords={C++, Lang, Progammiersprache, Tutorial, Einführung, German,
               Deutsch}
, pdflang={de-AT-1996}
, colorlinks=true
, linkcolor=blue
, urlcolor=blue
, pdfpagemode=UseNone]{hyperref}

\nonfrenchspacing
\sloppy

\title{Grundlagen der Programmiersprache \cpp{}}
\author{Georg Kindermann}
%\date{19. Juni 2023}

\begin{document}

\maketitle

%\begin{abstract}
%  Eine kurze Einführung in die Sprache Julia.
%\end{abstract}

\tableofcontents


Die Sprache \cpp{} baut auf der Sprache C auf. Für beide gibt es eine ISO Norm
die laufend aktualisiert wird. Eine Sammlung der letzten Normentwürfe (final
working draft) ist auf
\href{https://en.cppreference.com/w/cpp/links}{Cppreference/Links} zu finden.
Inwieweit diese Norm von verschiedenen Compilern erfüllt wird, ist auf
\href{https://en.cppreference.com/w/cpp/compiler_support}{Cppreference/Compiler\_support}
zusammengestellt. Hier wird \cpp{}23 verwendet. Eine ausführliche Referenz zu
\cpp{} und C ist auf \href{https://en.cppreference.com/w/}{Cppreference} oder
auch auf \href{https://learn.microsoft.com/de-de/cpp/cpp}{MS Dokumentation zu
C++} zu finden.

Programme können mit einem beliebigen Texteditor geschrieben werden und
anschließend mit einem Compiler in ein ausführbares Programm übersetzt werden.
Integrierte Entwicklungsumgebungen (IDE) helfen beim Schreiben von Programmen
und Werkzeuge wie \lstinline|make| beim compilieren und erstellen von
Bibliotheken (libraries).

Mit \lstinline|g++| wird ein Programm mit dem Befehl \lstinline|g++ Programm.cc|
compiliert und als \lstinline|a.out| abgespeichert. Mit
\lstinline|g++ -o prg Programm.cc| wird eine Datei \lstinline|prg| erzeugt.

Mit \lstinline|clang| wird ein Programm mit dem Befehl
\lstinline|clang++ Programm.cc|
compiliert und als \lstinline|a.out| abgespeichert. Mit
\lstinline|clang++ -o prg Programm.cc| wird eine Datei \lstinline|prg| erzeugt.

Mit \lstinline|-std=c++23| wird angegeben das \cpp{}23 verwendet wird.
Mit z.\,B.\ \lstinline|-O3 -march=native -flto| wird das erzeugte Programm
optimiert, um schneller zu laufen. Mit \lstinline|-Weverything| (nur bei clang),
\lstinline|-Wall| oder \lstinline|-Wextra| gibt der Compiler zusätzliche
Warnungen aus.

Mit \lstinline|msvc| (Developer PowerShell for VS) wird ein Programm mit dem
Befehl \lstinline|cl Programm.cc| compiliert und als \lstinline|Programm.exe|
abgespeichert. Mit \lstinline|cl Programm.cc /O2| wird auf Geschwindigkeit
optimiert. Mit \lstinline|cl Programm.cc /Wall| werden alle Warnungen
ausgegeben. Mit \lstinline|cl Programm.cc /std:c++20| wird \cpp{}20\footnote{23
ist derzeit v17.8.2/19.38.33130 nicht verfügbar. Eventuell
\lstinline|/std:c++latest| oder \lstinline|/experimental| verwenden} verwendet. 

C Dateien enden mit \lstinline|.c|. \cpp{} Dateien haben üblicherweise die
Extension \lstinline|.cc| oder \lstinline|.cpp|. Header Dateien für C enden mit
\lstinline|.h|. Header Dateien für \cpp{} enden meist mit \lstinline|.h|,
\lstinline|.hh| oder \lstinline|.hpp|.

Die Quelldatei (Sourcecode) wird zunächst vom Präprozessor verarbeitet (z.\,B.\
\lstinline|#include|) und wird anschließend compiliert. Danach bindet der Linker
Bibliotheken (Libraries) zu einem ausführbaren Programm zusammen.

\section{Basis}
\label{sec:einleitung}

\subsection{Grundstruktur}

Die erste Funktion die bei einem Programmstart aufgerufen wird ist
\lstinline|main|.
\begin{lstlisting}[language=C++]
int main() {}
\end{lstlisting}
\lstinline|main| gibt mit \lstinline|return| eine \lstinline|Integerzahl| an das
Betriebssystem zurück, die angibt, ob das Programm erfolgreich abgeschlossen (0)
oder durch einen Fehler (nicht 0) beendet wurde. In \lstinline|main| kann
\lstinline|return| weggelassen werden, was \lstinline|return 0;| entspricht. In
\lstinline|stdlib.h| bzw.\ \lstinline|cstdlib| sind \lstinline|EXIT_SUCCESS| und
\lstinline|EXIT_FAILURE| definiert.
\begin{lstlisting}[language=C++]
#include <cstdlib>

int main() {
  return EXIT_SUCCESS;
}
\end{lstlisting}
\lstinline|main()| (oder \lstinline|main(void)|) bedeutet, das beim
Programmaufruf keine Argumente an die Funktion übergeben werden. Mit:
\begin{lstlisting}[language=C++]
int main(int argc, char* argv[]) {}
\end{lstlisting}
können Argumente (command line arguments) von der Funktion übernommen werden.
\lstinline|argc| gibt die Anzahl an Argumenten an die in \lstinline|argv| zu
finden sind. Auf Position 0 steht der Programmname.

\subsection{Kommentare}

\begin{lstlisting}[language=C++]
/* Kommentar */

// Kommentar

#if 0
  std::cout << "Wird weder ausgeführt noch Compiliert\n";
#endif

if (false)
{
  std::cout << "Wird nicht ausgeführt\n";
}
\end{lstlisting}

\subsection{Bibliotheken -- Libraries}

Funktionen sind in Bibliotheken abgelegt. Diese müssen aktiv ausgewählt werden,
bevor sie verwendet werden können. Dies geschieht mit dem Präprozessorbefehl
\lstinline|#include|. \lstinline|#include <lib>| sucht die Headerdatei von
\lstinline|lib| im Standardsuchpfad für header Dateien.
\lstinline|#include "lib.h"| sucht im aktuellen Verzeichnis nach der Datei
\lstinline|lib.h|. Wenn diese nicht gefunden wird, wird im Standardsuchpfad für
Headerdateien weiter gesucht. Headerdateien von C enden mit \lstinline|.h|.

\subsection{Ein-- und Ausgabe}

Ein-- und Ausgabe ist in \lstinline|iostream| definiert. Es gibt die Befehle
\lstinline|cin| (Eingabe), \lstinline|cout| (Ausgabe), \lstinline|cerr|
(unmittelbare Fehlerausgabe) und \lstinline|clog| (gepufferte Fehlerausgabe).
Alle Befehle gibt es auch für wide Character mit vorangestelltem w, also z.B.
\lstinline|wcin|.

\begin{lstlisting}[language=C++]
#include <iostream>

int main() {
  int n;
  std::cout << "Anzahl n: ";
  std::cin >> n;
}
\end{lstlisting}

Eine neue Zeile kann mit \lstinline|std::cout << "\n";| oder mit
\lstinline|std::cout << std::endl;| erzeugt werden, wobei \lstinline|endl| den
buffer leert (flush) und damit in der Regel langsamer ist als \lstinline|\n|.

\subsection{Namespace}

Helfen Namenskonflikte zu vermeiden, indem Funktionen bestimmten Namensräumen
zugeordnet werden. Funktionen mit gleichem Nahmen stellen kein Problem dar,
solange sie in verschiedenen Namensräumen definiert sind. Funktionen des
Standardlibraries sind im Namensraum \lstinline|std| und können mit
\lstinline|std::FUNKTIONSNAME| aufgerufen werden. Mit
\lstinline|using namespace std;| kann die Funktion auch ohne \lstinline|std::|
aufgerufen werden. Namespaces können auch selbst erzeugt werden.

\subsection{Datentypen}

\lstinline|bool| ist der logische typ mit \lstinline|true| oder
\lstinline|false|.

Ganzzahlige Typen mit steigender Größe sind \lstinline|char|, \lstinline|short|,
\lstinline|int|, \lstinline|long| und \lstinline|long long|. Diese können mit
\lstinline|signed| und \lstinline|unsigned| also positive und negative oder nur
positive Zahl definiert werden.

Gleitkommazahlen mit steigender Größe sind \lstinline|float|,
\lstinline|double|, \lstinline|long double|.

\subsection{Deklaration, Definition, Initialisierung und Zuweisung}

\begin{lstlisting}[language=C++]
int i;     // Deklaration und Definition, i hat unbekannten Wert
i = 0;     // Zuweisung - Assignment
i = {0};   // Zuweisung - Assignment - EMPFOHLEN
//i = {0.};  // Fehler: narrowing conversion
int j = 1; // Dekl., Def. und Copy-Initialisierung
int jn = 1.5; // jn == 1
int k(2);  // Dekl., Def. und Constructor/Direct-Initialisierung
int kn(2.5);  // kn == 2
// int f();  // Würde die Funktion k deklarieren
// int f() {return 0};  // Dek. und Def. von Funktion f
// f(2);   // Würde die Funktion f mit Argument 2 aufrufen
int l{};   // Dekl., Def. und Value-Initialisierung, l == 0
int m{3};  // Dekl., Def. und Uniform/Direct-Init. - EMPFOHLEN
// int mn{3.5};  // Fehler: narrowing conversion
\end{lstlisting}

\subsection{Blöcke -- Compound / block}

Fassen mehrere Anweisungen zu einer zusammen. Sie beeinflussen die Gültigkeit
und Sichtbarkeit von Variablen.

\begin{lstlisting}[language=C++]
#include <fstream>
int i{0};  // i0, global
int main()
{ // Beginn des äußeren Blocks
  int i{1};  // i1 verdeckt globales i
  int j{3};
  {  // Beginn des inneren Blocks
    std::ofstream f("test.txt");  // Datei test.txt öffnen
    f << i << "\n";               // i1 schreiben
    int i {2};  // lokales i2, verdeckt i1, auf welches ab hier
                //   nicht mehr zugegriffen werden kann
    f << i << "\n";               // lokales i2 schreiben
    f << ::i << "\n";             // globales i0 schreiben
    f << j << "\n";               // j3 schreiben
  }  // Ende des inneren Blocks
     //   Destruktor Aufruf von f -> flush und schließen
     //   lokales i2 entfernt, i1 wider sichtbar
} // Ende des äußeren Blocks
\end{lstlisting}

\section{Kontrollstrukturen}

\subsection{Auswahlanweisungen -- Selection}

\subsubsection{if else}

Ausführen von Code, wenn Bedingung erfüllt ist.

\begin{lstlisting}[language=C++]
int i{1};

if (i > 2)
  if (i < 4)
    i = {0};
  else  // gehört zu if (i < 4)
    i = {1};
else  // gehört zu if (i > 4)
  i = {2};

if (i > 2)  // das Gleiche nur mit Klammern
{
  if (i < 4)
  {
    i = {0};
  }
  else
  {
    i = {1};
  }
}
else
{
  i = {2};
}

if (int ii = i*i; i > 0)  // mit Initialisierer
  i -= ii;
else {
  // int ii;  // error: redeclaration of ii
  i += ii;
}
// ++ii;  // ii hier nicht bekannt

// typedef declaration
if (typedef int Foo; true) { (void)Foo{}; }

// alias declaration
if (using Foo = int; true) { (void)Foo{}; }

constexpr bool TEST = true;
if constexpr (TEST) {  //Auswertung wenn Compiliert wird
  if (i == 0) i = {1};
}
i = {10 / i};
\end{lstlisting}

Mit \lstinline|if consteval| testen ob constexpr-Funktion zur Compilationszeit
ausgewertet wird. Hier müssen Blöcke (compound \lstinline|{ }|) verwendet
werden.

\begin{lstlisting}[language=C++]
#include <iostream>

constexpr int f(int x) noexcept
{
  if consteval { return 0; } else { return x; }
}

int main()
{
  const int i { f(2) };  // Auswertung zur Compilationszeit
  int j{2};             // nicht konstant
  const int k { f(j) };  // Auswertung zur Laufzeit
  std::cout << i << " " << k << "\n";
}
\end{lstlisting}

\subsubsection{?: -- dreiteilige Bedingung / Ternary conditional}

\lstinline|E1 ? E2 : E3| Wenn E1 true, dann E2 sonst E3.

\begin{lstlisting}[language=C++]
int n = {1 > 2 ? 7 : 8};  // 1 > 2 ist false, so n = 8
int m{0};
(n == m ? n : m) = {9};  // n == m ist false, so m = 9
\end{lstlisting}

\subsubsection{switch}

Springe, in Abhängigkeit eines Wertes, an eine Programmzeile. Der Wert muss
\lstinline|int| oder \lstinline|enum| oder implizit in int umwandelbar
sein.

\begin{lstlisting}[language=C++]
#include <iostream>

int main()
{
  const int k{9};
  switch (int i{k / 2}; i)  // i nur in switch bekannt
  // switch (k)  // Variante ohne Initialisierer
  {
    case 1:
      std::cout << '1';
      break;
    [[unlikely]] case 2:
      std::cout << '2';
      [[fallthrough]]; // nicht nötig, Hinweise das beabsichtigt
    [[likely]] case 3:
      std::cout << '3';
      break;
    case 4:
      //int j{0};  // Error weil man in den scope von
                    // initialisiertem i springen kann
      int j;
      j = {0};
      break;
    case 5:
    {
      int j{0};
      break;
    }
    case 6:
      // int i;  // error: redeclaration of i
    default:
      std::cout << "Default\n";
  }
}
\end{lstlisting}

\subsection{Schleifen, Wiederholungen -- Iteration}
\subsubsection{for}

Schleife mit Startwert, Bedingung (wird vor jeder Iteration geprüft) und
iteration-expression (wird nach jedem loop ausgeführt).

\begin{lstlisting}[language=C++]
#include <iostream>
#include <tuple>
#include <vector>

int main()
{
  double s{0.};
  int j{1};  // Wird in Schleife verdeckt
  for (int i{1}, j{9}; i < 10; ++i, --j)
  {
    double x{0.5};  // constructor and destructor wird bei
                    // jedem loop aufgerufen
    //double i{0.};  // Error: redeclaration von i
    s += i * j * x;
  }
  //--i;  // Error: Out of scope
  std::cout << s << "\n";

  char cstr[]{"Hallo"};
  for (int n = 0; char c = cstr[n]; ++n) std::cout << c;

  s = {0.};
  // Verschiedene Typen mittels structured binding declaration
  for (auto [i, f] = std::tuple{int{1}, float{0.1}}; i < 10 ;
       ++i, f += .1)
  {
    s += i * f;
    //int i;  // Error: redeclaration von i
  }
  std::cout << s << "\n";

  s = {0.};
  {
    int i{1};
    float f{0.1};
    for (; i < 10 ; ++i, f += .1) {
      s += i * f;
      int i;
    }
  }
  std::cout << s << "\n";
}
\end{lstlisting}

\subsubsection{range-for}

Schleife über einen Bereich (range).

\begin{lstlisting}[language=C++]
#include <iostream>
#include <vector>
#include <algorithm>
#include <ranges>

void o(const std::vector<int>& v, const char* s = {""}) {
  std::cout << s;
  for (const auto& i : v) std::cout << i << ' ';
  std::cout << "\n";
}

int main()
{
  std::vector<int> v = {2, 0, 1};

  // Index based. Geht nur bei sequential random access
  for (std::size_t i{0}; i < v.size(); ++i) ++v[i];
  o(v, "0: ");  // 3 1 2

  // Iterator based
  for (auto it{v.begin()}; it != v.end(); ++it) ++*it;
  o(v, "1: ");  // 4 2 3

  for (auto& e : v) ++e;  // Über Reference
  o(v, "2: ");  // 5 3 4

  for (auto e : v) ++e;  // Kopie
  o(v, "3: ");  // 5 3 4

  for (auto&& e : v) ++e; // forwarding reference
  o(v, "4: ");  // 6 4 5

  std::ranges::for_each(v, [](auto& e) { ++e; });
  o(v, "5: ");  // 7 5 6

  std::for_each(v.begin(), v.end(), [](auto& e) { ++e; });
  o(v, "6: ");  // 8 6 7

  for (int i{0}; const auto& e : v) // Mit init-statement
    std::cout << i++ << ':' << e << ' ';
  std::cout << '\n';

  for (const auto& [i, e] : std::views::enumerate(v))
    std::cout << i << ':' << e << ' ';
  std::cout << '\n';

  for (int i : {3, 1, 2}) std::cout << i; // braced-init-list
  std::cout << '\n';

  // Variable nicht verwendet
  for ([[maybe_unused]] int i : v) std::cout << 'X';
}
\end{lstlisting}

\subsubsection{while}

Führt eine Anweisung so lange aus, bis die Bedingung \lstinline|false| wird.
Geprüft wird \emph{vor} jeder Iteration.

\begin{lstlisting}[language=C++]
#include <iostream>

int main()
{
  int i{0};
  while (i < 3) ++i;
  std::cout << i << '\n';

  while (i < 9)
  {
    std::cout << i << '\n';
    i += 2;
  }

  char cstr[]{"Hallo"};
  i = {0};
  while (char c{cstr[i++]})
    std::cout << c;
}
\end{lstlisting}

\subsubsection{do-while}

Führt eine Anweisung so lange aus, bis die Bedingung \lstinline|false| wird.
Geprüft wird \emph{nach} jeder Iteration.

\begin{lstlisting}[language=C++]
#include <iostream>

int main()
{
  int i{0};
  do
  {
    i += 2;
    std::cout << i << '\n';
  }
  while (i < 9);
}
\end{lstlisting}

\subsection{Sprunganweisungen -- Jump}
\subsubsection{continue}

Überspringt den verbleibenden Teil des umschließenden \lstinline|for| oder
\lstinline|while| Schleifenkörpers.

\begin{lstlisting}[language=C++]
#include <iostream>

int main()
{
  for (int i{0}; i < 3; ++i)
  {
    for (int j{0}; j < 3; ++j)
    {
      if (j == 1) continue;  // wirkt bei j for loop
      std::cout << i << ' ' << j << '\n';
    }
  }
}
\end{lstlisting}

\subsubsection{break}

Beendet umschließendes \lstinline|for|, \lstinline|while| oder
\lstinline|switch|.

\begin{lstlisting}[language=C++]
  #include <iostream>

int main()
{
  for (int i{0}; i < 3; ++i)
  {
    for (int j{0}; j < 3; ++j)
    {
      if (j == 1) break;  // wirkt bei j for loop
      std::cout << i << ' ' << j << '\n';
    }
  }
}
\end{lstlisting}

\subsubsection{goto}

Setzt Programm an anderer Stelle fort.

\begin{lstlisting}[language=C++]
#include <iostream>

struct nt {  // non-trivial destructor
  ~nt() { std::cout << "X\n";}
};

int main()
{
  for (int i{0}; i < 3; ++i)
  {
    for (int j{0}; j < 3; ++j)
    {
      if (j == 1) goto endloop; // Ausstieg aus beiden for loops
      std::cout << i << ' ' << j << '\n';
    }
  }
endloop:  // Sprungmarke label
  goto label2; // Sprung in den scope von n
//int n{0};  //error: jump bypasses variable initialization
  [[maybe_unused]] int n; // no initializer
  n = {7};
//nt o;  // error: bypas variable with non-trivial destructor
label2:
  std::cout << n;  // n hat Zufallswert
}
\end{lstlisting}

\subsubsection{return}

Beendet Funktion und gibt Wert (falls vorhanden) an den Aufrufer zurück.

\begin{lstlisting}[language=C++]
#include <iostream>
#include <utility>

void fa(int i) {
  if (i == 1) return;
  std::cout << "fa("<< i << ")\n";
} // impliziertes return;

int fb(int i) {
  if (i > 4) return 4;
  std::cout << "fb(" << i << ")\n";
  return 2; }

std::pair<int, float> fc(int i, float x) {
  return {i, x}; }

int main()
{
  fa(0);  // Gibt fa(0) aus
  fa(1);  // Nichts
  int i{fb(5)};  // i == 4
  i = fb(i);     // gibt fb(4) aus, i == 2
  std::cout << fc(0, 1.).first << " " << fc(2, 3.).second; //0 3
}
\end{lstlisting}


%Autor: Georg Kindermann

\end{document}
