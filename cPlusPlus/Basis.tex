\section{Basis}
\label{sec:einleitung}

\subsection{Grundstruktur}

Die erste Funktion die bei einem Programmstart aufgerufen wird ist
\lstinline|main|.
\begin{lstlisting}[language=C++]
int main() {}
\end{lstlisting}
\lstinline|main| gibt mit \lstinline|return| eine \lstinline|Integerzahl| an das
Betriebssystem zurück, die angibt, ob das Programm erfolgreich abgeschlossen (0)
oder durch einen Fehler (nicht 0) beendet wurde. In \lstinline|main| kann
\lstinline|return| weggelassen werden, was \lstinline|return 0;| entspricht. In
\lstinline|stdlib.h| bzw.\ \lstinline|cstdlib| sind \lstinline|EXIT_SUCCESS| und
\lstinline|EXIT_FAILURE| definiert.
\begin{lstlisting}[language=C++]
#include <cstdlib>

int main() {
  return EXIT_SUCCESS;
}
\end{lstlisting}
\lstinline|main()| (oder \lstinline|main(void)|) bedeutet, das beim
Programmaufruf keine Argumente an die Funktion übergeben werden. Mit:
\begin{lstlisting}[language=C++]
int main(int argc, char* argv[]) {}
\end{lstlisting}
können Argumente (command line arguments) von der Funktion übernommen werden.
\lstinline|argc| gibt die Anzahl an Argumenten an die in \lstinline|argv| zu
finden sind. Auf Position 0 steht der Programmname.

\subsection{Kommentare}

\begin{lstlisting}[language=C++]
/* Kommentar */

// Kommentar

#if 0
  std::cout << "Wird weder ausgeführt noch Compiliert\n";
#endif

if (false)
{
  std::cout << "Wird nicht ausgeführt\n";
}
\end{lstlisting}

\subsection{Bibliotheken -- Libraries}

Funktionen sind in Bibliotheken abgelegt. Diese müssen aktiv ausgewählt werden,
bevor sie verwendet werden können. Dies geschieht mit dem Präprozessorbefehl
\lstinline|#include|. \lstinline|#include <lib>| sucht die Headerdatei von
\lstinline|lib| im Standardsuchpfad für header Dateien.
\lstinline|#include "lib.h"| sucht im aktuellen Verzeichnis nach der Datei
\lstinline|lib.h|. Wenn diese nicht gefunden wird, wird im Standardsuchpfad für
Headerdateien weiter gesucht. Headerdateien von C enden mit \lstinline|.h|.

\subsection{Ein-- und Ausgabe}

Ein-- und Ausgabe ist in \lstinline|iostream| definiert. Es gibt die Befehle
\lstinline|cin| (Eingabe), \lstinline|cout| (Ausgabe), \lstinline|cerr|
(unmittelbare Fehlerausgabe) und \lstinline|clog| (gepufferte Fehlerausgabe).
Alle Befehle gibt es auch für wide Character mit vorangestelltem w, also z.B.
\lstinline|wcin|.

\begin{lstlisting}[language=C++]
#include <iostream>

int main() {
  int n;
  std::cout << "Anzahl n: ";
  std::cin >> n;
}
\end{lstlisting}

Eine neue Zeile kann mit \lstinline|std::cout << "\n";| oder mit
\lstinline|std::cout << std::endl;| erzeugt werden, wobei \lstinline|endl| den
buffer leert (flush) und damit in der Regel langsamer ist als \lstinline|\n|.

\subsection{Namespace}

Helfen Namenskonflikte zu vermeiden, indem Funktionen bestimmten Namensräumen
zugeordnet werden. Funktionen mit gleichem Nahmen stellen kein Problem dar,
solange sie in verschiedenen Namensräumen definiert sind. Funktionen der
Standardbibliothek (Standard Library) sind im Namensraum \lstinline|std| und
können mit \lstinline|std::FUNKTIONSNAME| aufgerufen werden. Mit
\lstinline|using std::FUNKTIONSNAME;| kann die \emph{einzelne} Funktion auch
ohne \lstinline|std::| aufgerufen werden. Mit \lstinline|using namespace std;|
können \emph{alle} Funktionen der Standardbibliothek auch ohne \lstinline|std::|
aufgerufen werden. Namespaces können auch selbst erzeugt werden.

\subsection{Datentypen}

\lstinline|bool| ist der logische typ mit \lstinline|true| oder
\lstinline|false|.

Ganzzahlige Typen mit steigender Größe sind \lstinline|char|, \lstinline|short|,
\lstinline|int|, \lstinline|long| und \lstinline|long long|. Diese können mit
\lstinline|signed| und \lstinline|unsigned| also positive und negative oder nur
positive Zahl definiert werden.

Gleitkommazahlen mit steigender Größe sind \lstinline|float|,
\lstinline|double|, \lstinline|long double|.

\subsection{Deklaration, Definition, Initialisierung und Zuweisung}

\begin{lstlisting}[language=C++]
int i;     // Deklaration und Definition, i hat unbekannten Wert
i = 0;     // Zuweisung - Assignment
i = {0};   // Zuweisung - Assignment - EMPFOHLEN
//i = {0.};  // Fehler: narrowing conversion
int j = 1; // Dekl., Def. und Copy-Initialisierung
int jn = 1.5; // jn == 1
int k(2);  // Dekl., Def. und Constructor/Direct-Initialisierung
int kn(2.5);  // kn == 2
// int f();  // Würde die Funktion f deklarieren
// int f() {return 0};  // Dek. und Def. von Funktion f
// f(2);   // Würde die Funktion f mit Argument 2 aufrufen
int l{};   // Dekl., Def. und Value-Initialisierung, l == 0
int m{3};  // Dekl., Def. und Uniform/Direct-Init. - EMPFOHLEN
// int mn{3.5};  // Fehler: narrowing conversion
\end{lstlisting}

\subsection{Blöcke -- Compound / block}

Fassen mehrere Anweisungen zu einer zusammen. Sie beeinflussen die Gültigkeit
und Sichtbarkeit von Variablen.

\begin{lstlisting}[language=C++]
#include <fstream>
int i{0};  // i0, global
int main()
{ // Beginn des äußeren Blocks
  int i{1};  // i1 verdeckt globales i
  int j{3};
  {  // Beginn des inneren Blocks
    std::ofstream f("test.txt");  // Datei test.txt öffnen
    f << i << "\n";               // i1 schreiben
    int i {2};  // lokales i2, verdeckt i1, auf welches ab hier
                //   nicht mehr zugegriffen werden kann
    f << i << "\n";               // lokales i2 schreiben
    f << ::i << "\n";             // globales i0 schreiben
    f << j << "\n";               // j3 schreiben
  }  // Ende des inneren Blocks
     //   Destruktor Aufruf von f -> flush und schließen
     //   lokales i2 entfernt, i1 wider sichtbar
} // Ende des äußeren Blocks
\end{lstlisting}