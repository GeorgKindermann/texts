\section{Introduction}

Increasing climate warming poses major challenges for forestry. Even for tree species that are fundamentally capable of coping with difficult site conditions, their natural regeneration via seed or reforestation can present significant problems. In such cases, vegetative regeneration through coppicing and especially root suckering can play an important role in stand regeneration.

However, vegetative regeneration can only contribute if tree species capable of vegetative reproduction are already present in the stand. If such species are not sufficiently represented, they should be introduced under the currently still relatively favorable site conditions, as their successful establishment is likely to become increasingly difficult with continued climate change.

In this context, black locust (\emph{Robinia pseudoacacia} L.) is considered a promising tree species. It combines high drought resistance with a strong capacity for vegetative propagation, particularly through root suckers. This enables it to establish efficiently and extensively even on sites that are unfavorable for natural regeneration. Unlike coppicing, root suckering largely preserves both the repeated regeneration capacity and the growth performance over multiple generations. This significantly reduces dependence on generative regeneration.

Additionally, Robinia is capable of fixing atmospheric nitrogen, thereby increasing both the stand’s productivity and the soil’s humus content at a wide range of sites. Its intensive root system helps retain nutrients by preventing leaching and makes nutrients that are less available accessible. Moreover, it promotes earthworm density, which, in combination with its rooting activity, leads to soil loosening. A high humus content and porous soil improve water retention capacity, which is especially advantageous in dry regions, not only for Robinia itself, but also for other tree species growing in mixture with it.

Many of the Robinia trees growing in Central Europe have a crooked form, limiting their use mostly to firewood. However, reports of very tall and straight-growing Robinia in its native range date back to the late 18\textsuperscript{th} century. For about 100 years, selective breeding for straight growth has been practiced. Some of these cultivated varieties also have few or very small thorns. Today, varieties with absolutely straight growth exist. Since Robinia wood is the most durable among tree species growing in Central European forests, high-quality Robinia timber can serve as a sustainable alternative to tropical hardwoods for demanding outdoor applications.


\section{Climate-Adaptive Forest Management}

\summary{Black locust is considered a drought-tolerant tree species that, due to its ability to produce root suckers, can regenerate well even on challenging sites. In coppice forestry, it can be used with longer rotation periods compared to regeneration from stool shoots. From an ecological perspective, it requires a differentiated view: on the one hand, it improves nutrient-poor sites through nitrogen fixation; on the other hand, it can outcompete weaker species. Its use should therefore be targeted and adapted to the surrounding habitats.}

In light of changing climate conditions, the question of which tree species can reliably establish and maintain stands under increasingly dry conditions is gaining importance. Particularly valuable are species that can tolerate drought and also possess alternative regeneration strategies. In this context, black locust appears promising. Although not native, it is a tree species that has successfully established in many regions, combining pronounced drought resistance with a high capacity for vegetative propagation via root suckers.

\subsection{Vegetative Regeneration}

On sites where precipitation is already a limiting factor, increasing drought and heat are progressively restricting both the range of economically viable silvicultural options and the selection of suitable tree species.

Among the native principal stand-forming tree species, Scots pine (\emph{Pinus sylvestris}), black pine (\emph{Pinus nigra}), and various oak species are most commonly considered. The native oaks can generally be ranked in terms of drought resistance as follows: pedunculate oak (\emph{Quercus robur}), sessile oak (\emph{Quercus petraea}), Turkey oak (\emph{Quercus cerris}), and downy oak (\emph{Quercus pubescens}). Unfortunately, timber quality tends to decrease along this ranking. The wood of Turkey oak, for instance, is not suitable for barrel production due to its large pores, which make it non-watertight; additionally, its sapwood is wider. High-quality downy oak stems are rare. However, recent breeding efforts and hybridizations with other oak species have produced trees that are both drought-tolerant and capable of developing straight trunks.

As long as there are regional tree species from lower elevations available, it seems reasonable, given climate change, to use these in higher elevation zones. However, in the lowest elevation zones, some tend to consider tree species from more southern or equatorial regions. Although temperatures there are indeed higher, factors such as solar radiation and day length differ significantly. Plants may have adapted to these conditions, for example in terms of bud break timing \citep{phillips1941tageslaenge}, which can influence growth \citep{jester1939zuwachsUndTageslaenge}. At the same latitude, higher summer temperatures can be found in more continental regions.


\subsection{Pioneer and Sprouting Tree Species}

On marginal sites where natural regeneration is difficult, vegetative regeneration methods are becoming increasingly important compared to generative ones. In high-altitude areas, spruce can regenerate via layering. According to \citet{boring1984robinie}, nitrogen-fixing species play a key role in primary succession. In alpine regions, sea buckthorn (\emph{Hippophae rhamnoides}) and green alder (\emph{Alnus alnobetula} or \emph{Alnus viridis}) are examples of this. In softwood floodplain forests, poplar and willow frequently regenerate via root suckers.

As drought increases, regeneration becomes more difficult, potentially causing a shift in silvicultural systems from high forest to coppice-with-standards and eventually to coppice. This transition favors tree species with strong sprouting ability. From a forestry perspective, in addition to regeneration via stool shoots, the ability to form root suckers on sites unfavorable to regeneration is highly desirable, as the shoots are not restricted to a quality-reducing stump and are typically distributed more densely and evenly across the site.

According to \citet{nicolescu2019robinie}, black locust can produce up to 50\,000 root suckers per hectare, enabling the regeneration area to be fully covered again within 1–2 years.

\subsection{Comparison of Root Suckering and Coppicing from Stumps}

Coppicing from stumps (stool shoots) shows a significant decline in growth performance and sprouting ability after about three rotations. Therefore, even in silvicultural systems that rely on vegetative regeneration, generative regeneration via seed remains essential.

Over the same time span, more generative regeneration may actually be required in coppice forests than in high forests. Assuming that both silvicultural systems require the same initial number of stems during the regeneration phase, and that the rotation period for high forest is 180 years while that of coppice is 30 years, the following emerges: if coppice is regenerated generatively after three rotations (i.e., after 90 years), then over a 180-year period, the high forest requires one generative regeneration, while the coppice forest requires two. From this perspective, coppicing may even exacerbate the already difficult regeneration situation.

Its advantage, however, lies in the fact that if generative regeneration fails after harvesting, the vegetative shoots alone can restore canopy closure. Generative regeneration can then be postponed until the next or even the subsequent rotation.

While growth performance and resprouting capacity decline with each successive harvest in coppice from stumps, root suckers show little to none of these negative effects. The ability to produce root suckers typically remains intact for a long time, depending on the species.

Among native tree species, those with sufficient root suckering ability include aspen (\emph{Populus tremula}), grey alder (\emph{Alnus incana}), elm (\emph{Ulmus} spp.), field maple (\emph{Acer campestre}), wild cherry (\emph{Prunus avium}), and other wild fruit species. Among non-native species, black locust and tree of heaven (\emph{Ailanthus altissima}) are notable for their prolific root suckering.

\subsection{Black Locust in Coppice Forestry}

In black locust, effective regeneration via root suckers is possible even at an advanced stand age. Unlike coppicing from stumps, where new shoots emerge only from the stump, root suckers can develop along horizontal roots at a considerable distance from the parent tree.

In long rotation periods, the number of stems in the overstory decreases significantly. If regeneration were to rely solely on stump sprouting, the resulting shoots would be widely spaced, delaying canopy closure. Among other reasons, this is why a short rotation period is usually chosen when regeneration occurs through coppice shoots. In contrast, regeneration via root suckers allows for dense and evenly distributed natural regeneration, even with a low number of overstory stems. This makes black locust well suited for vegetative regeneration in stands with long rotation periods.

According to \citet{iski2019robinie}, the stem quality of repeated coppice shoots tends to decline over time. In contrast, \citet{redei2011robinieWaldbau} found no differences in either stem quality or growth performance between regeneration from seed and from root suckers. The growth of black locust stands established from root suckers shows multiple peaks: the first at ages 3–5, another at 9–12, and a third around 15 years, the latter due to increased mortality.

After harvesting, canopy closure in coppice stands is typically restored much faster than in high forest systems. This is sometimes interpreted as evidence of higher productivity in coppice systems. However, this overlooks the fact that coppice areas are cleared more frequently and at shorter intervals than high forests. Thus, a higher productivity of coppice compared to high forest should not be expected.

\subsection{Spread, Site Dynamics, and Ecological Interactions}

The ability to reproduce via root suckers is also one of the reasons why non-native tree species are often considered to have invasive potential. This trait allows them to colonize habitats such as dry grasslands more readily than many native species. Interestingly, even on dry grasslands, the durability of black locust wood appears to be appreciated, within the national park, for instance, it is used as rustic posts (Fig.~\ref{fig:feldmannstreu}).

This raises the legitimate question of how such dry grassland communities originated. At times, it will become evident that they were initiated by past deforestation, occasionally involving slash-and-burn techniques. What followed was erosion, sod-cutting, and grazing, for example by sheep and goats.

After land use was abandoned, natural succession typically began, initially involving native shrubs such as blackthorn (\emph{Prunus spinosa}), hawthorn (\emph{Crataegus monogyna}), dog rose (\emph{Rosa canina}), buckthorn (\emph{Rhamnus cathartica}), privet (\emph{Ligustrum vulgare}), barberry (\emph{Berberis vulgaris}), fly honeysuckle (\emph{Lonicera xylosteum}), broom (\emph{Cytisus scoparius}), bilberry (\emph{Vaccinium myrtillus}), cornelian cherry (\emph{Cornus mas}), or dogwood (\emph{Cornus sanguinea}). Eventually, these sites would develop into forests dominated by sessile oak, pedunculate oak, Turkey oak, downy oak, hornbeam (\emph{Carpinus betulus}), field maple, birch (\emph{Betula pendula}), Scots pine, black pine, or small-leaved lime (\emph{Tilia cordata}), a natural progression.

This can be illustrated by an area on the small Perchtoldsdorf Heath, which was fenced off as a natural monument in 1940 and has since been virtually ungrazed, left to develop naturally \citep{rosenkranz1953heide}. In 1952, a steppe fire occurred, after which, alongside dry grassland species, pines also established \citep{rosenkranz1953heideBrand}. I have known this site since the 1980s, when it had already largely transformed into a forest of black pine and oak. That everything is constantly changing has long been understood. Already in antiquity, this was aptly expressed with the well-known phrase \enquote{Pánta rheî} (everything flows). 

Currently, there is growing evidence that dry habitats are becoming increasingly common in Austria as well. Forests, particularly in the Pannonian-Illyrian region, are expected to experience significant declines in growth performance and timber volume, along with a marked increase in regeneration challenges.

% Fig

An additional aspect of black locust is its symbiosis with nitrogen-fixing bacteria. This enables it to enrich the soil with nitrogen, which generally enhances site productivity. The improved conditions facilitate the establishment of further plant species, which can gradually displace the less competitive species that are typical of dry grassland communities.

Native species such as bird's-foot trefoil (\emph{Lotus corniculatus}), kidney vetch (\emph{Anthyllis vulneraria}), horseshoe vetch (\emph{Hippocrepis comosa}), sickle alfalfa (\emph{Medicago falcata}), black medick (\emph{Medicago lupulina}), or meadow vetchling (\emph{Lathyrus pratensis}) are typical companion species of dry grasslands. Like black locust, they are capable of fixing atmospheric nitrogen and thus contribute significantly to the nitrogen supply of nutrient-poor sites.

Other native legumes, such as common broom (\emph{Cytisus scoparius}), spring vetchling (\emph{Lathyrus vernus}), or sainfoin (\emph{Onobrychis viciifolia}), typically appear at later stages of natural succession and may, in the long term, displace the characteristic dry grassland communities.

\section{Non-native Tree Species}

\summary{The legal classification of non-native tree species significantly influences their forestry use. Species such as the tree of heaven are currently strictly regulated within the EU. Legal reforms may alter existing management options and, in some cases, entail mandatory control measures. To enable long-term and reliable forestry planning, the use of established stands should, at minimum, be legally secured until the final harvest age.}

Although some dry grasslands have no natural origin, their preservation, particularly in regions with few habitats unaffected by humans, as well as that of gravel pits, brick ponds, or quarries, whose anthropogenic origins are evident, is considered justified due to their unique site conditions.

Against this background, the following section addresses the management of non-native tree species, especially black locust, taking into account legal and ecological aspects.


\subsection{Legal Status of Non-native Species}% in the EU

Several non-native species are already subject to strict legal restrictions in Austria, including the tree of heaven (\emph{Ailanthus altissima}), which is listed in the EU Regulation on invasive alien species \citep{eu2019verordnungListeInvasiverArten,eu2014verordnungInvasiverArten}. This list is intended to be reviewed and updated at least every six years. Following the amendment of the Austrian Forestry Act, the tree of heaven was removed from the list of permitted woody plants in the annex. As a result, it no longer formally meets the legal criteria to be considered forest under the Forestry Act. In mixed stands, the tree of heaven is not regarded as a tree within the meaning of the forestry law definition of forest, since it is no longer recognized as a legally approved woody species. Consequently, a legally required minimum canopy cover by the other tree species listed in the annex must be fulfilled.

\subsection{Legal Protection of Individual Tree Species}

Conversely, there is also the possibility that tree species are legally protected, as is the case for the yew (\emph{Taxus baccata}) in Lower Austria \citep{niederoesterreich2000Naturschutzgesetz,niederoesterreich2005artenschutzverordnung}. There, the yew is classified as vulnerable to picking. Contemporary and sustainable commercial, agricultural, and forestry use is not prohibited. Nevertheless, this can lead to uncertainty, which might cause some to hesitate in promoting and utilizing the yew.

\subsection{Criteria for Listing Invasive Species}

Listed species must: be non-native; be capable of spreading; have adverse impacts on biodiversity, ecosystem services, human health, or the economy; it must be demonstrated that coordinated measures are necessary to prevent their introduction, establishment, or spread; and it must be likely that the adverse impacts can actually be prevented, minimized, or mitigated.

Listed species shall not be intentionally introduced, kept, or bred within the territory of the Union; transported into, out of, or within the Union; placed on the market; or released into the environment; nor used or exchanged.

Additionally, appropriate management measures are developed, aiming to minimize the impacts of these species on biodiversity and associated ecosystem services as well as, where applicable, on human health or the economy. Furthermore, affected, damaged, or destroyed ecosystems should be restored.

Exemptions apply to species in regions at the outermost limits of Union-wide relevance, for which these species still must be listed. Such exemptions mainly apply to geographically isolated outermost regions of the EU, which are unlikely to affect Austria. Furthermore, exemptions can be granted for facilities conducting research and ex-situ conservation. In exceptional cases, authorizations may be issued. Additionally, Member States may list species significant for them and adopt differing measures.

Legally mandated measures may require landowners to control species at their own expense, even if they never introduced them, as is the case with ragweed in Burgenland \citep{burgenland2021ragweed}.

\subsection{Example of a Legal Change}% in the area of agroforestry

In Germany, the use of black locust within agroforestry systems has been excluded from receiving CAP direct payments (Common Agricultural Policy of the European Union) since 2022. The use of black locust for newly established short-rotation coppice stands has also been prohibited since January 1, 2022. Short-rotation coppice stands established with black locust before this cutoff date are not affected by this regulation \citep{deLaw2022gap}. Their existence remains unaffected, and there is neither cause nor obligation to remove or convert these stands. The restriction applies exclusively to areas for which CAP direct payments are claimed. Other legal requirements may apply to non-subsidized areas.

\subsection{Strategies for Risk Minimization}

Legal frameworks can change. Those who want to minimize their risk from this perspective should preferably introduce selected black locust varieties only where black locust already exists. On the other hand, in the face of massive site changes, such as the already ongoing and expected future temperature increases, the establishment of currently non-native species must be expected.

Commercially used tree species can be restricted in their future use not only by site changes but also by changes in legal frameworks. Therefore, tree species lists guaranteeing by law that their use remains legally secured until harvest maturity would be helpful for forestry planning and decision-making. If control measures are still necessary, they should be supported as much as any potentially required regeneration with suitable alternative tree species.

\subsection{Control of Abundance and Distribution}

Should the legal situation change or for other reasons the desire arise to reduce black locust in forests, this could be achieved through shading within the stand closure. As a strongly light-demanding tree species, black locust reacts sensitively to shading.

An ecologically based approach to regulating invasive plant species consists of the targeted use of species with allelopathic effects. \citet{jung2010robinie} describes that Japanese walnut (\emph{Juglans ailanthifolia}) significantly inhibits the growth of black locust seedlings by releasing juglone through its root system. The walnut (\emph{Juglans regia}), which is not native but naturalized in Central Europe, also possesses similar properties. \citet{dordevic2022nussAllelopathi} conclude that their extracts could potentially serve as substitutes for synthetic herbicides. Since it colonizes similar sites as black locust and also produces strong shading, its use for containment appears particularly suitable.

In the long term, natural succession can also contribute to the regulation of black locust. In forest sites in Germany, it is generally inferior to native tree species. In black locust forest communities, it is often observed that under the canopy of black locust a dense regeneration of Norway maple (\emph{Acer platanoides}) develops. If this development is left to itself, black locust would soon be so strongly impaired in its light regime that it would wither or even die \citep{kohler1963robinie}.

Besides competition from other tree species, site factors also influence the spread of black locust. According to \citet[p.~134]{landeck2022robinie}, there is no risk of invasion on dry grasslands if these are more than 500\,m apart, regardless of the intervening vegetation structures. Nevertheless, proximity to potential natural and anthropogenic dispersal vectors such as roads and rivers should be considered \citep{skowronek2020robinieNaturschutz}. Soil disturbances within an approximately 100\,m wide buffer zone around seed-producing trees should be avoided. Furthermore, black locust should not be planted on steep slopes near streams, as spread from these locations is more likely \citep{morimoto2009robinie}.

In intensively used agricultural areas and urban spaces, however, the risk of uncontrolled spread of black locust is minimal. In these habitats, it contributes to the diversity of landscape structures and mosaics and provides a welcome habitat for many organisms \citep{vitkova2018robinie}.

Once established, black locust is particularly difficult to remove in unmanaged dry grasslands but also in open forests due to its sprouting ability and root suckers. The timing of felling influences the formation of stump shoots and can be used deliberately to control black locust. As a light-demanding pioneer species, its regeneration is strongly suppressed by dense undergrowth or shade-tolerant tree species. While it can quickly resprout on clear-cuts, due to its light-permeable crown it is hardly able to permanently overshadow other tree species.

An efficient strategy to curb the spread of black locust consists of avoiding disturbances that could promote its establishment and waiting for natural displacement of the species by other trees \citep{motta2009robinieBekaempfung}.



\section{Statements on Black Locust and Their Original Sources}

\summary{Statements about black locust are analyzed and contrasted with
  their respective original sources. It becomes apparent that both critical
  and positive evaluations are sometimes not supported by the underlying literature
  or are strongly simplified.}

There exist partly contradictory statements regarding the effects of black locust
on soils, forest ecosystems, and the natural regeneration of native tree species.
Such statements are occasionally heavily condensed or generalized, yet they can still
influence professional and public debates as well as concrete management decisions.

Some statements lack traceable sources, while others are based on citable scientific literature,
whose content, however, sometimes proves to be considerably more nuanced than the statements that refer to them.

Most sources and studies are quoted in their original English wording. German-language quotes have been translated, with all translations clearly indicated to avoid distortion. The original German texts are provided in the German section of this article. Between the quotations, I have inserted explanatory notes
and highlighted specific terms. The selection of quotes might be subject to a certain
selection bias, for example due to limited accessibility or focus on particular lines of argument.

A brochure contains the following statement:

\hyperlink{german:oebf2019aliensAusDemGarten}{\enquote{Even if the black locust is removed, it has enriched the soil with nitrogen and toxic excretions from roots and leaves, and has permanently changed the plant community at the site. It is not browsed by wild animals (thorny and toxic). \dots
  It suppresses almost every other plant and is therefore poorly suited for garden design.%
%Due to numerous runners, difficult to control. Beware of injuries from
%the toxic thorns!
}} \citep[own translation]{oebf2019aliensAusDemGarten}.

Unfortunately, the brochure lacks literature references that would allow verification
of these claims. Without the mentions of nitrogen enrichment and thorns,
I would rather attribute the description to walnut and even consider it exaggerated for that species.
In my view, it contributes to the false impression that black locust poisons the soil
and thereby harms the forest.

When quotations are available, the original statements can be consulted.
In \citet{szyp2023robinieGenetik} the following sentence regarding black locust is found:

\enquote{Due to its negative impact on biodiversity and forest
  ecosystem functioning \citep{langmaier2020alienPlants}, there are no
  plans to increase its role in forests or expand the range of this
  species.}

In \citet{langmaier2020alienPlants} it states:

\enquote{Another aspect of chemical impacts is the fact that the
  chemical composition of plant litter from alien plants such as
  Robinia pseudoacacia can cause high levels of nitrogen in the upper
  soil horizons, thereby \emph{exerting an effect} on regeneration
  \citep{rahmonov2009robinieLitter}).}

\enquote{The regeneration of Scots pine (Pinus sylvestris) was
  negatively influenced by Prunus serotina and Robinia pseudoacacia
  \citep{sebert2007invasive,rahmonov2009robinieLitter}.}

\enquote{Robinia pseudoacacia originating from North America is a good
  example of an invading plant species that uses different impact
  mechanisms at different stages of invasion. It increases nitrogen
  availability, changes light conditions, creates plant communities
  and is also associated with allopathic activity
  \citep{rahmonov2009robinieLitter,campagnaro2018alien}.
  R. pseudoacacia can be a \emph{desirable and beneficial} species for forest
  management on degraded, sandy, urban, and initial soil, while other
  studies report its negative impacts on riparian forests, Pannonian
  mixed forests and Western European broadleaf forests. Its effects
  particularly \emph{affect} the natural regeneration of Ulmus laevis,
  Ulmus minor, Quercus pubescens, Quercus petrea, Quercus robur,
  Populus sp., Crataegus monogyna, Betula pendula, Pinus sylvestris,
  and Fraxinus angustifolia
  \citep{rahmonov2009robinieLitter,maringer2012robinePostFire,petrasova2013neophyten,radtke2013robinieNiederwald,terwei2013nonNative}.}

In \citet{rahmonov2009robinieLitter} it says:

\enquote{The \emph{positive influence} of R. pseudoacacia on a habitat
  is primarily connected with the chemical composition of plant
  litter, as well as with the biology of the species.}

\enquote{Pines can sow under the canopy of R. pseudoacacia, but they
  extinct very quickly.}

Would this last statement not also apply to almost any other tree species?
When aiming for natural regeneration, it is not uncommon to first thin the stand.
Once regeneration has established, further thinning or even removal of the
overstory is performed, because otherwise the regeneration of light-demanding
species such as pine would suffer or die.

The following statement is probably not universally valid, but in this case
it might be true:

\hyperlink{german:erteld1952robinieErtrag}{\enquote{Where black locust appeared to adversely affect the development of pine or oak in coexistence within our study area, it always turned out that the reason lay in insufficient attention by the forest manager.}} \citep[own translation, p.~90]{erteld1952robinieErtrag}

In \cite{campagnaro2018alien} it is stated:

\enquote{For example, with respect to impacts on vascular plant
  species by R. pseudoacacia, Sitzia et al. (2012) \emph{did not find
  changes in diversity} when comparing several recent secondary
  \emph{forest types} in a rural context, but Trentanovi et al. (2013)
  found a strong reduction in diversity compared to \emph{native}
  birch forests \emph{within the city of Berlin}.}

\enquote{\dots whereas, Agonimia allobata, a rare and vulnerable
  lichen species (Nascimbene, Nimis, \& Ravera 2013), was found in
  R. pseudoacacia but not in oak stands in Italy (Nascimbene \& Marini
  2010).}

\citet{sebert2007invasive} deals with the invasive cherry species
(\emph{Prunus serotina}). Therein, with reference to \citet{lee2004robinie},
a description is found stating that black locust in the overstory holds a
regeneration in waiting, which can quickly restore canopy closure when
released. How this affects pine regeneration is not described.

In \citet{lee2004robinie} it is said:

\enquote{Although native oaks (Quercus spp.) are usually
  \emph{succeeding} black locust colonies, changes to native
  vegetation are often interrupted by frequent disturbance by
  \emph{human activities}, as evident from persistent sprouts and
  expansions of black locust suckers in the disturbed areas, such as
  urban center and rural areas.}

In \citet{maringer2012robinePostFire} it says:

\enquote{Due to the different ecological requirements of indigenous
  and alien tree seedlings, \emph{not any interaction} between the two
  groups was detected.}

In \citet{petrasova2013neophyten}:

\enquote{Even though the black locust \emph{does not have influence}
  on the presence of native and typical forest species, it has ability
  to homogenize tree composition of spontaneously growing forest
  patches.}

In \citet{radtke2013robinieNiederwald}:

\enquote{Due to the relatively \emph{short cycle of coppice} forests,
  the time of forest development is \emph{too short} for native
  species to \emph{out-shade} the light-demanding non-natives.}

And in \citet{terwei2013nonNative}:

\enquote{Presence of R. pseudoacacia in the canopy \emph{promoted} the
  regeneration of Q. robur.}

Conversely, there are also authors who attribute exclusively positive
properties to black locust and consider it a solution to various problems.
Friedrich Casimir Medicus was probably among them. From 1794 to 1803, he
published his own journal about black locust. He also allowed critical voices,
for example:

\hyperlink{german:medicus1794ffRobinie}{\enquote{In the Forestry and Hunting Calendar 1796 of Prof. Leonhardi,
  p. 297, the following passage appears: “Besides, I wish, and ask Mr. M.
  in the name of several forestry enthusiasts, to for now stop writing
  about the false acacia tree and only after a few years again
  communicate all experiences, because otherwise the good cause might suffer."}} \citep[own translation, Vol. 2, No. 2, p. 3]{medicus1794ffRobinie}

In his journal, he holds the opinion that:

\hyperlink{german:medicus1794ffRobinieB}{\enquote{\dots through the general cultivation of the false acacia tree,
  not only is the future shortage of firewood prevented, but also minds
  are already freed from fear,\dots}}
\citep[own translation, Vol. 1, No. 3, p. 186]{medicus1794ffRobinie}.

\citet{hartig1798robinie}, who is quite positive towards black locust,
replies to Medicus that a shortage of firewood can be addressed not
by cultivating black locust but by more efficient stoves, and presents
a corresponding stove design.

Thus, in \citet{hartig1798robinie} it reads:

\hyperlink{german:hartig1798robinieA}{\enquote{Many treatises have been written in praise of this certainly
  valuable timber species,\dots}} (own translation, p. 12),

\hyperlink{german:hartig1798robinieB}{\enquote{Although I admit with sincere conviction that the growth of acacia
  is very strong and surpasses all our good broadleaf firewood species,\dots}} (own translation, p. 27),

\hyperlink{german:hartig1798robinieC}{\enquote{However, do not believe that the advantages of acacia are unknown
  or indifferent to me, or that I want to work against the cultivation
  of this useful timber species. No, not at all! On the contrary,
  I plant it here and there myself and recommend its cultivation in general,\dots}} (own translation, pp. 35--36),

\hyperlink{german:hartig1798robinieD}{\enquote{Furthermore, I again solemnly protest against the accusation
  that I want to work against the cultivation of acacia: for this is
  certainly not my intention at all. On the contrary, I recommend this
  useful timber species very strongly everywhere; but not to cover the
  current or imminent shortage of firewood with it; rather to increase
  the number of our very useful timber species.}}
(own translation, p. 83).

\hyperlink{german:erteld1952robinieErtragB}{\enquote{It is assumed that only the title of Hartig's valuable work
  has always been read and noted and that the high authority of the author
  sufficed to conclude from the denial of the possibility that black locust
  could cover firewood demand to its general rejection. Subsequently,
  black locust initially disappeared from the interest of experts.}}
\citep[own translation]{erteld1952robinieErtrag}.


\section{Biodiversity and Mixed Forests}

\summary{Through nitrogen fixation and a more open canopy, the black locust can improve site conditions. It promotes the growth and natural regeneration of many tree species that thrive better under or beside it than without it. The dense herb and shrub layer beneath black locust stands influences the species composition and plays an important role in succession.}

The black locust affects both biodiversity and site conditions in the habitats it colonizes. It influences the light conditions within the stand, the composition of flora and fauna, as well as successional developments in forest ecosystems. At the same time, it plays a significant role in mixed forests and exhibits diverse interactions with other tree species that are important for forestry.

\subsection{Light Ecology and Vegetation Structure}

The black locust is characterized by a comparatively short leafing period. Its leaves appear late in spring (May) and begin to fall early, usually during the summer drought (August) \citep{vitkova2017robinie}. This results in high light availability, allowing light-demanding species such as spring geophytes or a dense shrub layer to survive. However, this dense herb and shrub vegetation is simultaneously unfavorable for the regeneration of shade-intolerant native tree species as well as for the black locust itself. It is typical for pioneer plants that they prepare a site for subsequent plants as first colonizers, but are eventually displaced by them. \citet{baker1949schattentoleranz} classifies the black locust as very intolerant to shading.

Studies in Central Europe show that black locust stands are more light-permeable throughout the entire growing season than native forests, which is reflected in a denser shrub and herb layer \citep{hanzelka2015robinie}. For example, shading in black locust stands was 57\,\%, compared to 72\,\% in an oak forest. Shrub cover amounted to 57\,\% versus 11\,\% in the oak forest, and the herb layer 53\,\% instead of 5\,\%.

\subsection{Species Diversity and Succession}

The black locust exhibits a differentiated pattern of effects regarding species diversity and successional processes. While it can contribute to diversity in certain contexts, there are indications in other cases of promoting homogeneous species compositions. Its role as a pioneer species as well as its impact on soil and light conditions significantly shape the development of vegetation structures and species compositions.

\citet{kraftova2017robinieVoegel} showed that at a medium proportion of black locust, the species diversity of forest birds is highest. Similarly, \citet{vitkova2010robinieVegTyp} describe several vegetation types in Central Europe dominated by the naturalized black locust, which are distinguished by floristic characterization and generally exhibit high species diversity.

In contrast, \citet{trentanovi2013robinie} point out that black locust can reduce species diversity in urban forests without significantly altering site differences. While urban structures mainly contribute to the homogenization of native flora, the black locust particularly promotes the homogenization of non-native species.

Regarding ground plant diversity, \citet{sitzia2012robinie} found no reduction caused by black locust. However, under its influence, a ruderal flora inhibiting regeneration develops (see Fig.~\ref{fig:glaswein}).

%%%

An example of only minor ecological impacts of black locust is provided by a study from North America. \citet{deneau2013robinie} compared black locust stands with local deciduous forests at nine site pairs in Sleeping Bear Dunes National Lakeshore (USA). No differences were found in ground vegetation and only minor ones in soil nutrients. The results suggest that black locust has only minor effects in this ecosystem.

According to \citet{vitkova2017robinie}, changes in species composition under black locust are caused more by altered soil nutrient availability and light conditions than by allelopathic effects. \citet{csiszar2009allelopathy} investigated allelopathic effects of 15 non-native tree species and found that, unlike other species such as false indigo (\emph{Amorpha fruticosa}), tree of heaven, or black walnut (\emph{Juglans nigra}), black locust shows no special allelopathic properties.

Studies on woody plant succession on ruderal rubble sites in Berlin by \citet{kowarik1990robinie} showed that spontaneous black locust stands are surprisingly rich in woody species. In the recorded material, 77 woody species were documented, including 38 tree species (e.g., rowan, pedunculate and sessile oak, winter lime, field and mountain elm, ash, birch, yew), 35 shrub species, and 4 woody climbers. Maple species have invaded black locust stands during succession, while birches die off due to the shading pressure of black locust. Also, escaped fruit trees such as apple, pear, walnut, or vine occur alongside black locust. The black elder (\emph{Sambucus nigra}) emerges as a dominant shrub species, found in almost every older stand and reaching cover values above 75\,\%. Its importance for woody succession lies in its shade suppressing grasses and tall herbs, thereby promoting the growth of shade-tolerant woody plants. The invasion of black locust into closed mesophilic forests, where it has not previously occurred, is considered unlikely.

\citet{evans2013robinie} demonstrate that black locust on former open-cast mining sites contributes to carbon sequestration and soil development through biomass production and nitrogen fixation. Thus, it supports species diversity and the restoration of ecosystem services and promotes forest establishment as a pioneer species on highly degraded sites.

Finally, grasses can also represent competition for black locust. For example, erosion control measures showed that black locusts on untilled, grass-covered areas beside erosion gullies exhibit higher mortality and slower growth than those on less fertile but bare slopes of the gullies \citep{meginnis1934robinie, cooper1950blacklocust}.

Overall, it is evident that black locust can have both promoting and inhibiting effects on other species in successional processes, which interact with it and among each other. Such interactions are typical for the interplay of species in dynamic ecosystems.


\subsection{Interactions with Other Tree Species}

The black locust engages in diverse interactions with other tree species, ranging from site-dependent compatibilities to different growth and competitive relationships. Predominantly, these effects are beneficial. In many cases, the black locust promotes the growth of neighboring species, particularly shade-tolerant trees. Inhibitory effects mainly occur with distinctly light-demanding tree species.

The compatibility of black locust with other tree species varies depending on the species. For example, \citet[p.~138]{krauss1986sauenerWald} reports that spruce and oak thrive well under black locust stands, whereas Scots pine is less compatible. \citet{kellog1934robinieMischbestand} describes that black locust is not compatible with pine and larch species.

The highly light-demanding birch is likely among the few tree species that can indeed compete with black locust in terms of light, although this does not exclude mixtures with it. \citet{gaier2009robinieVorverjuengung} report a black locust–birch mixed stand in which successful pre-regeneration under the canopy of black locust was carried out at the age of 60 years. At a stocking degree of 0.8, winter lime and Norway maple developed better, whereas at 0.3, pedunculate oak and hornbeam performed somewhat better. The black locust formed root suckers, but no negative effects on the pre-crop could be observed.

In the studies by \citet[pp.~150--160]{scamoni1952robinieWurzeln}, the root growth of neighboring maples as well as neighboring black locusts was not disturbed by the roots of other black locusts. Similarly, \citet{bencat1992robinie} states that root system development is not influenced by neighboring trees. In contrast, according to \citet[p.~53]{kolesnikov1971wurzeln}, apple tree roots avoid the root zone of other apple trees but freely spread among the roots of cherry and apricot.

\citet{ferguson1922robinie,mcintyre1932robinie,chapman1935robinie} report on an adjacent planting of black locust and tree of heaven. The tree of heaven were consistently twice as tall and twice as strong at the stand edge at ages 13 and 23, with approximately eight times the volume compared to trees 20 m behind the stand edge (Fig.~\ref{fig:bestandesrand}). Similarly, \citet{mcintyre1932robinie} observed growth increases in tree of heaven, and \citet{chapman1935robinie} noted similar growth boosts in white ash, tulip tree, dyer’s oak, and chestnut oak. However, such growth increases of this magnitude are to be expected only on sites with severe nitrogen deficiency. For example, \citet[p.~107]{krauss1986sauenerWald} presents a diagram indicating that Scots pine mixed with black locust had a diameter at breast height of about 20 cm at age 70, compared to about 16 cm without black locust, though this difference could also be caused by differing stand densities. \citet{ashby1986robinieWuchssteigerung} studied height growth of 16-year-old trees planted under black locust and shortleaf pine (\emph{Pinus echinata}), where some species were approximately twice as tall under black locust compared to pine (Tab.~\ref{tab:zuwachsUnterbau}).

%%%
%%%

\citet{chapman1951robinieSchirm} investigated the mortality rates and growth performance of tulip tree, green ash, cherry, as well as red and white oak when planted on (1) a fallow site with natural herbaceous vegetation, grasses, and small shrubs under (2) sassafras (\emph{Sassafras albidum}), (3) black locust, or (4) pine. It was found that black locust acts as a light competitor against red oak, resulting in increased mortality. Regarding height growth, all tree species examined, except green ash, were at least twice as tall under black locust compared to the other tree species or the fallow site (Tab.~\ref{tab:zuwachsUnterbau2}). As expected, light-demanding tree species are not suitable for underplanting. Eastern white pine (\emph{Pinus strobus}) survived but exhibited low growth. Shortleaf pine (\emph{Pinus echinata}) and pitch pine (\emph{Pinus rigida}) completely failed as underplantings, even under black locust.

%%%

Similar observations were made by \citet{limstrom1951schuttaufforstung}. After two years, black locust showed a survival rate of 82\,\% in afforestation on mining spoil, ranking second highest after ash with 86\,\% among 18 compared tree species. After two years, black locust was significantly taller at 5.7 feet (1.74\,m) on leveled and 4.8 feet (1.46\,m) on unleveled spoil sites compared to the 16 tree species it was compared with. The next tallest species was poplar with 1.7 feet (0.52\,m) on leveled and 2.0 feet (0.61\,m) on unleveled sites. On a plot where underplanting was also compared, black locust had the highest survival rate and clearly the greatest height in the open area, and increased the survival rate of black walnut from 67\,\% in the open area to 97\,\% under black locust underplanting. The light-demanding tree species poplar and sweetgum had a survival rate of 14\,\% under black locust (Tab.~\ref{tab:zuwachsUnterbau3}).

%%%

In black locust stands, planted deciduous trees were three times as tall after three years as nine-year-old trees of the same species on sites without prior black locust planting. Natural regeneration of deciduous trees only occurs under black locust, whereas it is absent on bare slopes or in pine forests \citep{chapman1947robinie}.

According to \citet[p.~19]{erteld1952robinieErtrag}, natural pollination with oak and valuable hardwoods under the canopy of black locust often succeeds excellently; however, due to the multilayered structure and lush ground flora, it is difficult to underplant black locust with pine. Underplanting with spruce and Douglas fir is more feasible with appropriate management \citep[p.~94]{erteld1952robinieErtrag}.

Contrary to the widespread belief that black locust does not tolerate other tree species beside it, it has been shown that it is also suitable as a mixed tree species. Therefore, it is often recommended to introduce other tree species into pure black locust stands through pre- or underplanting \citep{ewald2001klone}.

According to \citet[pp.~90--92]{erteld1952robinieErtrag}, black locust significantly improves the soil, allowing a species-rich ground flora to develop. Within this ground flora, oak, ash, lime, Norway maple, sycamore maple, and bird cherry have established and show good growth. In contrast, these species are absent in adjacent pine stands without black locust. Especially on poorer soils, a gradual introduction of black locust into pure pine stands is recommended.

\citet{beck1974robinieTulpenbaum} found that tulip trees released from black locust shading exhibit higher growth than those growing under its crown. The black locust itself was noticeably larger than the released tulip trees (Tab.~\ref{tab:tulpenbaumRobinie}).

%%%

According to \citet{dickmann1985robinieMischbestand}, a pure black locust stand produced a biomass yield of 4.6\,t/ha/year after 4 years. A pure sycamore stand (\emph{Platanus occidentalis}) showed the same yield. However, a mixed stand of both tree species reached only 3.6\,t/ha/year. In this case, the nitrogen-fixation effect of black locust was apparently neutralized by two applications of 135\,kg ammonium nitrate/ha, two applications of 900\,kg, and one application of 560\,kg 10-10-10 (N-P-K)/ha fertilizer.

\citet[pp.~467, 586]{ashton2018silviculture} describes an agroforestry silvopastoral system, where black locust, due to its nitrogen-fixing capacity and its light canopy, promotes the growth of grasses underneath.

In a survey of a railway fallow site, \citet{kowarik1996robinie} found that within five years, 10 oaks and five maples established themselves within the closed black locust stand, whereas only one oak established outside it.

Overall, these studies illustrate that black locust does not generally act as a competitor in mixed stands but usually has a positive effect on neighboring tree species. Limitations exist mainly for light-demanding species. The results emphasize the importance of site-specific planning, where the positive interactions of black locust can be purposefully utilized.


\section{Influence on Site Factors}

\summary{Black locust can improve soil quality primarily through nitrogen fixation, the promotion of a loose and well-aerated soil structure, deep rooting, and the accumulation of humus. However, nitrogen fixation is reduced or ceases when sufficient nitrogen is already present in the soil. Furthermore, black locust is able to access nutrients from deeper soil layers and to increase the soil’s water-holding capacity, which facilitates the establishment of other tree species, particularly in dry regions. Its dense root system can also reduce nutrient losses through leaching.}

Black locust affects various site factors, particularly nutrient balance, humus content, soil reaction, and the physical properties of the soil.

\subsection{Nutrient Balance and Soil Reaction}

Black locust is a nitrogen-fixing tree species whose influence on nutrient balance and soil reaction has been extensively studied. The following presents the known mechanisms of nitrogen fixation, their effects on the ecosystem, and comparable processes in other species.

Black locust can fix atmospheric nitrogen.
\citet{boring1984robinieN} report that 67\,\% of the biomass of nitrogen-fixing nodules is located in the upper 15\,cm of the soil and that a four-year-old black locust stand fixes 30\,kgN/ha/year.
\citet{noh2009robinieN} observed fixation rates of 23–112\,kgN/ha/year, \citet{danso1995robinieN} 220\,kgN/ha/year, and \citet{marron2018robinieN} 5.7–12.5\,kgN/ha/year in black locust stands.
According to \citet{moshki2011robinieN}, there are differences in nitrogen fixation capacity among black locust provenances. Biomass production can serve as an indirect indicator for selecting black locust clones with high nitrogen fixation.

\citet{boring1984robinie} found that nitrogen fixation rates initially increase with age and then decline at older ages. For example, a 4-year-old black locust stand fixed 191\,kgN/ha, a 17-year-old 1272\,kgN/ha, and a 38-year-old 1231\,kgN/ha since afforestation. Accordingly, nitrogen does not increase at older ages. This pattern of nitrogen accumulation resembles that of other woody, nitrogen-fixing species on secondary succession sites.
For example, \citet{cleve1971grauerle} describe for grey alder that soil nitrogen reserves increase strongly up to age 5, slightly until age 15, and then may even decrease somewhat. The highest fixation rates thus occur in early to mid-successional stages and decline or cease in later phases.

In older black locust stands, a net loss of nitrogen in forest soil and litter layer may occur, as large amounts of nitrogen are taken up and stored in the biomass. However, the decline in soil nitrogen could be the result of long-term nutrient redistribution into biomass after clear-cut management \citep{boring1984robinie}.
Black locust did not lead to a significant accumulation of stable nitrogen-containing material in the upper mineral soil \citep{auten1945robinieN}.

Nitrogen fixation by trees is rarely encountered. Deposition compensates or slightly exceeds the losses during harvest. In nitrogen-saturated soils, leaching is to be expected, which should be avoided especially in spring water protection areas. At sites where tree growth can be increased by nitrogen additions, such as formerly hay meadow forests, nitrogen-fixing plants are generally welcome. For example, \citet{wiedemann1951ertragskunde} recommends cultivating perennial lupine to improve the site and increase growth performance. The amount of nitrogen fixed by black locust is comparable in magnitude to that fixed by other legumes such as lupine or clover, which are used in organic farming.

Among native tree species, alders can fix nitrogen. This property is associated with soil improvement. \hyperlink{german:schuett2014alnusIncarna}{\enquote{The basis for the repeatedly emphasized soil-improving effect of grey alder is its ability to bind atmospheric nitrogen via an actinorhiza}} \citep[own translation]{schuett2014alnusIncarna}.
\citet{schuett2014alnusIncarna} also provide a literature overview of nitrogen fixation rates of grey alder ranging from 43–72\,kgN/ha/year.
\citet{cleve1971grauerle} report an average nitrogen fixation rate of 156\,kgN/ha/year for a 20-year-old grey alder stand, with 362\,kgN/ha/year fixed between ages 0 and 5.
Among shrub species, for example, golden rain tree and sea buckthorn can fix nitrogen.

Legumes are also used in agriculture. \citet{kolbe2008stickstoff} state nitrogen fixation amounts of 105–309\,kgN/ha by legumes in organic farming.

According to \citet{zheng2023nFixierung}, artificial nitrogen fertilization inhibits natural N fixation, though higher soil carbon content mitigates this inhibition.

The net nutrient fertilizer sales in Austria amount to about 100\,000 t N/year \citep{ama2024duengemittel}. According to Statistics Austria, 1\,322\,900 ha were arable land in 2020, with a total agricultural area of 2\,602\,700 ha. This results in 75\,kgN/ha if fertilizer is applied only on arable land, or 38\,kgN/ha when considering the total agricultural area.

\citet{uba1998deposition} report that wet total nitrogen deposition can reach more than 20\,kg/ha/year.
\citet{raspe2018stickstoff} also report nitrogen inputs exceeding 20\,kg/ha/year, with outputs of 6\,kg/ha/year. Depending on tree species and site, 4\,kg/ha/year (pine) to 16\,kg/ha/year (oak) nitrogen is stored in wood and bark and removed during harvest. Total nitrogen uptake by trees is significantly higher, and nitrogen returned via litterfall amounts to 50\,kg/ha/year.
\citet{raspe2018stickstoff} report an average nitrogen accumulation through deposition of about 6\,kg/ha/year over the last 25 years.

\citet{emberger1965stickstoff} found nitrogen reserves in forest soils down to 1 m depth ranging from 1905\,kg/ha to 15\,929\,kg/ha.

According to \citet{mueller1991robinie}, in the first years after black locust introduction, a \emph{temporary} increased withdrawal of main nutrients nitrogen, phosphorus, and potassium is expected due to nutrient uptake by the intensive black locust root system. This \emph{temporary} nutrient immobilization has given black locust the reputation of being a \enquote{nutrient robber}.
This rapid nutrient uptake by black locust can help minimize nutrient losses through leaching on temporarily bare sites \citep{boring1984robinie}.

\citet{lazzaro2018robinie} found in black locust stands compared to oak forests a decrease in soil pH and increased nitrogen contents. This was accompanied by increased bacterial diversity but reduced diversity of microarthropods and nematodes, likely due to soil acidification and secondary plant compounds.

According to \citet{lee2010robinie}, the rapid decomposition of black locust flower and leaf litter significantly increases soil phosphate content. However, this is not an altruistic behavior but an indirect interaction within the ecosystem. Through interception and dry deposition, trees and forests accumulate higher amounts of pollutants and nutrients compared to open fields. The efficiency of these deposition processes varies considerably between tree species, depending on factors such as leaf structure, needle density, or crown shape. Moreover, substance losses due to erosion and leaching are generally much lower in forested areas than in open land.

\citet{garman1938robinie} analyzed soil nutrient content based on samples taken after removal of the topsoil and litter from a depth of 2–5 inches. They found that all examined nutrients were present in higher concentrations under black locust than in an adjacent area not influenced by its leaf litter (Tab.~\ref{tab:naehrstoffeUnterRob}).

%%%

According to \citet{kou2016robinieBoden}, after afforestation of arable land with black locust, the phosphorus content increased in the top 5\,cm, soil carbon content in the upper 60\,cm, and nitrogen content up to the maximum investigated depth of 100\,cm.

Overall, the studies presented show that black locust and other nitrogen-fixing plant species can contribute significantly to the nutrient balance, but also bring ecological and site-specific risks. Nitrogen fixation from the atmosphere does not proceed continuously but is subject to ecological regulation. Depending on the age of the stand, soil nitrogen content, and external influences such as fertilization or deposition, nitrogen fixation can be markedly reduced or completely cease. Therefore, the assessment of these effects should always be carried out in the context of the respective site conditions and management objectives.


\subsection{Humus or Soil Carbon}

The formation and stabilization of humus are central components of soil fertility. Nitrogen-fixing tree species such as black locust contribute significantly to humus accumulation and carbon sequestration through litter input, root activity, and nutrient enrichment.

According to \citet{papaioannou2016robinieBoden}, after planting black locust on degraded arable soils, the organic soil matter increased by 1.3 to 3 times within 20 years. Nitrogen content also rose by 1.2 to 2.5 times. Additionally, the highest concentrations of phosphorus and potassium were often measured, most likely due to fertilization during previous agricultural use. Tree species that intensely root deep soil horizons can mobilize nutrients from these layers and make them available to plants, especially regeneration, which has shallower root systems.

\citet{gustafson1935robinie} showed that black locust significantly contributes to the accumulation of organic matter and nitrogen on nutrient-poor sandy soils. Under black locust trees, a nutrient-rich leaf layer accumulated within a few years, which not only improved soil fertility but also enabled the growth of plants like meadow fescue, which otherwise would not thrive on this sandy soil. Furthermore, black locust effectively prevented sand drifting, suggesting their use as windbreaks or for soil stabilization.

\citet{kastler2013robinieBoden} determined a soil carbon content of 5.28\,\% in a black locust stand in the floodplain near Orth, while an adjacent pedunculate oak-hornbeam stand showed 5.03\,\%.

\citet{kanzler2021robinieBodenc} demonstrated that soil carbon accumulates faster under black locust vegetation on mining spoil sites than under cereal cultivation.

\citet{gurlevik2016longterm} found that afforestation on dry sandy soils improves soil fertility through the accumulation of organic carbon and nutrients. Only black locust significantly increased nitrogen availability. In contrast to black locust, pines primarily store nitrogen in the soil, which increases the risk of nitrate leaching.

\citet{resh2002nFixUndC} report that forests with nitrogen-fixing trees typically accumulate more carbon in the soil than comparable stands without such species. Both more new carbon is fixed and decomposition of older, already present soil carbon is reduced.

Compared to non-nitrogen-fixing species, planting nitrogen fixers increases organic soil carbon content by an average of 16\,\%. For every additional gram of nitrogen, on average 7.8\,g of carbon is stored. Larger soil carbon increases by nitrogen-fixing plants are especially expected in warm and dry regions \citep{sun2025nFixiUndC}.

According to \citet{ye2024nFixCstab}, nitrogen-fixing tree species promote the chemical stability of organic soil carbon.

These studies show that black locust, through its ability to fix nitrogen, to root deeply, and enrich organic matter, can substantially contribute to humus formation and the improvement of soil fertility. Especially on degraded or nutrient-poor sites, they make an important contribution to long-term carbon storage and the ecological stabilization of soils.


\subsection{Soil Physical Changes}

Soil physical properties such as porosity, water retention capacity, soil density, and soil structure significantly influence plant growth as well as biological activity in the soil. Several studies show that black locust has a positive effect on these parameters.

According to \citet[pp.~474, 483]{ramann1898influssBodendeckeAufPhysikalischeBodeneigenschaften}, soil loosening occurs under black locust. The black locust appears to have a mild but beneficial effect on the soil.

\citet{albert1926robinieBodenlockerheit,penschuck1931robinieBodenphysik} state that black locust possesses the ability to loosen forest soil and develop particularly favorable soil structure, thereby creating optimal physical conditions for plant growth. This soil loosening facilitates the gas exchange with the air required by the nodulating bacteria. At the same time, soil water capacity is increased.

\citet{nemec1925waldboden} investigated soil porosity, water, and air capacity of forest soils, finding that soils under black locust were the most porous and exhibited the highest water retention capacity (see Tab.~\ref{tab:bodenPorositaet}).

%%%

\citet{ashby1986robinieWuchssteigerung} found a lower soil bulk density under black locust compared to shortleaf pine (Pinus echinata).

Compact, wet, and waterlogged horizons cannot be penetrated by roots \citep{auten1933robineStandort,mueller1991robinie} and thus are hardly loosened.

This soil loosening could be attributed to a combination of intensive root activity, a beneficial soil fauna, and favorable litter decomposition with crumb formation. Such physical changes and the associated improved aeration are especially advantageous on heavy soils. On light, sandy soils, however, humus content and the related nutrient and water retention capacity play a decisive role.

According to \citet{bluemke1955robinie}, the greatest numbers of earthworms in forest soils are found under black locust and elderberry, which contribute to soil loosening.

\citet{vaupel2023robinie} describes that windbreak strips promote earthworms. Compared to poplar, both earthworm density and biomass were higher under black locust.

These results show that black locust can improve the physical properties of soil at many sites through a combination of intensive root activity, favorable litter decomposition, and support of soil biological processes such as earthworm activity.


\section{Browsing}

\summary{Black locust trees are sometimes heavily browsed by wildlife, are toxic to certain animal species, but can also be used as fodder.}

Black locust interacts with animals in various ways. These interactions include not only browsing damage but also use as a forage plant.

According to \citet{landgraf2014immenser}, browsing by wildlife causes a yield reduction in black locust short rotation coppice plantations ranging from 1.3 to 6.8 tons of dry matter per hectare and year. Under very high browsing pressure, complete dieback of entire stands can even occur in extreme cases \citep{landgraf2024robinie1,landgraf2024robinie2}.

Black locust is browsed by roe deer, and its bark is stripped by hares and mice, which particularly complicates afforestation on grasslands \citep{barta2023robinieReh}. According to \citet{berner2018robinie}, black locust is not browsed by cattle. It is toxic to horses \citep{grosche2008robiniePferd}.

However, black locust leaves are suitable fodder for sheep \citep{ganai2009robnieSchaf}, goats \citep{papachristou1999robinieZiege}, and hares \citep{singh2010robnieHasennahrung}, with the latter even experiencing improved immune functions \citep{yang2017robinieHasen}. Additionally, black locust provides shade for animals.

Overall, these examples show that black locust is both threatened by browsing and usable as a forage plant.


\section{Distribution}

\summary{Black locust originates from eastern North America but is now distributed worldwide. In Central Europe, it shows great forestry potential due to its high growth performance and drought tolerance, even under future climate conditions. Its water use efficiency varies considerably between clones and environmental conditions, making targeted selection of varieties crucial. Under drought stress, it adapts through leaf movement and increased root growth, while under sufficient water supply it achieves high growth increments.}

Its natural distribution area is located in the mid-eastern part of North America (Fig.~\ref{fig:verbreitungNatuerlich}), although an extension of the original natural distribution area by indigenous peoples seems possible. Vienna's temperature profile is similar to that of some provenances within the natural distribution range, but the precipitation amount in Vienna is significantly lower (Fig.~\ref{fig:wetter}). Even though the average monthly precipitation is considerably higher, longer periods without rainfall, especially on soils with low water retention capacity, can still cause drought stress in these regions.

%%%

Over time, black locust was introduced to new regions and is now distributed across Europe and globally (Figs.~\ref{fig:verbreitungGlob}, \ref{fig:verbreitungEuJetzt}). According to \citet{bouteiller2019robinie}, the black locust populations in Europe descend from four populations originating in North America and exhibit low genetic diversity. The first specimens were likely introduced from Virginia in the early 17th century, with further introductions from Pennsylvania and West Virginia during the 17th and 18th centuries. These appear to be the progenitor plants of most black locusts currently found in Europe.

%%%

\subsection{Current and Future Distribution in Europe}

Figure~\ref{fig:verbreitungEuPot} shows the current site suitability for black locust. Almost all lowland areas in Austria are suitable, which is also supported by recorded observations of black locust (Fig.~\ref{fig:verbreitungEur}). For forestry planning, however, not only the current but also the future site suitability must be considered, especially in times of climate change. Figure~\ref{fig:verbreitungEuPotZukunft} depicts the expected site suitability of black locust under the climate change scenario with very strong temperature increase (RCP 8.5) for the year 2095. Accordingly, most sites currently suitable for black locust in Austria are expected to remain suitable in the future.

%%%

Of course, such estimates are associated with uncertainties; however, the risk that a site loses its suitability over time should be lower for tree species that allow relatively short rotation periods than for those with very long ones, since only limited changes can occur at a site over a few years. According to forestry law, most tree species are only ready for harvest after 60 years. Black locust is included in the regulation for fast-growing tree species, where the upper limit of its immaturity for harvest is set at 10 years. Rotation periods of 15–30 years are possible with black locust on average sites and can, if necessary, be extended up to 100 years.


\subsection{Drought Tolerance and Water Use} % Water Use Efficiency

A sought-after trait of black locust is its relatively high tolerance to drought. Black locust can adapt to prolonged drought by closing its stomata and reducing both the size of individual leaves and the total leaf area. It can also fold its paired leaflets almost completely upwards, thereby reducing the leaf surface exposed to the sun \citep{schildknecht1984blattbewegung}. However, as long as there is no water stress, it does not reduce its transpiration and is therefore not considered a water-saving tree species.

Under moderate drought stress, black locust increases root growth at the expense of aboveground growth, which enhances its drought resistance in the long term \citep{yang2018robinie}.

The aboveground biomass production per liter of water (Water Use Efficiency, WUE) of black locust is approximately 2.31~g/L, referring to aboveground wood 1.63~g/L. This value remains largely stable under water stress. Thus, it does not belong to the particularly water-efficient plants that can produce a lot of biomass with very little water \citep{mantovani2014robinieWasser}. \citet{lindroth1994wasserverbrauchWeide} report a WUE of 4.1--5.5~g/L for basket willow (\emph{Salix viminalis}). \citet{lacher2001OekophysiologieDerPflanzen} states a WUE of 3--5~gDM/L for deciduous trees of the temperate zone and conifers.

\citet{ombodi2022robinieWasserverbrauch} give a WUE of 1.87~g/L for black locust under sufficient water supply, but only 0.26~g/L under high water stress. According to \citet{veste2013robinie}, WUE increases under drought stress and is temperature-dependent. Further WUE values reported are: \citet{raper1992robinieWasserverbrauch} with 0.045~g/L, \citet{mantovani2014robinieWue} with 2.57~g/L (constant, but with drought memory). Black locusts that had previously experienced drought showed adjusted growth and higher drought tolerance compared to plants without such experience. According to \citet{wang2007robinie}, WUE increased with decreasing soil water potential, except in young black locusts, where it remained approximately constant at a low level.

The wide range of WUE values measured by different authors for the same tree species makes comparisons with values from other tree species, which also come from different authors, uncertain. Since different clones exhibit varying responses to drought stress \citep{mapelli2019robinieTrockenstress}, specifying which clone was investigated would be useful.

\citet{abri2022robinieTrokenresistenz,abri2023robiieWasser,abri2023robinieDroughtTolerance} examined net assimilation and transpiration for individual clones and calculated WUE (Tab.~\ref{tab:WUErobinie}). \citet{abri2025robinieTrockenheit} determined a WUE of 3.64~g/L for PL040 in 2024. Thus, at least certain varieties are very efficient in terms of growth per unit of water consumed.

Clones with high WUE are expected to show higher growth performance, at least on water-limited sites, which is tentatively confirmed by the observed diameter and height growth increments (Tab.~\ref{tab:WUErobinie}).

%%%

\subsection{Impact on Water Balance}

Besides water consumption, interception also plays a role, i.e., the proportion of rainfall that is retained in the tree canopy and does not reach the soil.

According to \citet{gemeinhardt1959robinie}, the average soil moisture values in areas improved by black locust are higher down to a depth of about 10\,cm compared to control plots. No differences were found in deeper soil layers.

\citet{kou2016robinieBoden} compared a fallow field with spontaneous vegetation to a newly afforested area with black locust. Under the black locust trees, soil moisture was higher in the top 5\,cm, approximately the same in the 5--60\,cm layer, and lower at depths of 60--100\,cm.

According to \citet{yan2009robinieVergleich}, black locust showed an early response to reduced leaf water potentials by reducing the stomatal conductance of its leaves. However, its water use characteristics and specific leaf area indicated that it consumes more water and is possibly less drought tolerant than the broad-leaved lilac (\emph{Syringa oblata}) and the Liaotung oak (\emph{Quercus liaotungensis}).

Especially in dry regions, black locust competes with other plants for water \citep{halasz2021robinieAlsTierutter}. For example, \citet[p.~96]{donaubauer1974kiefernsterben} reported that Scots pine mixed with black locust presumably suffered greater damage from the rust fungus \emph{Cenangium ferruginosum} due to water competition. Also, Marter in \citet[p.~93]{erteld1952robinieErtrag} made the rare observation that pines near black locust developed red needles and subsequently died.

\subsection{Photosynthesis and Light Requirements}

As long as water is not limiting, the efficiency of CO\textsubscript{2} uptake largely determines growth performance. \citet{lubimenko1906npp} investigated the amounts of CO\textsubscript{2} uptake per \SI{1}{\gram} of leaf mass in eight tree species at different temperatures and angles of solar radiation (Tab.~\ref{tab:nppTempWinkel}). Black locust consistently showed about twice as high values as the second-best tree species and operated very efficiently at high temperatures. However, this superiority is relativized by the comparatively small leaf mass of black locust.

According to \citet{lyr1963beschattung,lyr1964beschattung}, the growth of black locust is more strongly restricted by shading than that of Scots pine. At 15\,\% ambient light, black locust reduces its root growth by 85\,\%, black alder by 79\,\%, Scots pine by 75\,\%, birch by 70\,\%, Douglas fir by 62\,\%, and red oak by 6\,\%. At 15\,\% ambient light, black locust reduces its total biomass by 79\,\%, black alder by 73\,\%, birch by 58\,\%, Scots pine by 55\,\%, Douglas fir by 48\,\%, while red oak was even able to increase growth slightly. From 70 to 55\,\% ambient light, inhibition of symbiotic nitrogen fixation occurs.

There are also differences between individual clones. According to \citet{abri2024lightResponse,abri2024lightResponseB}, it is likely that clones PL251 and NK2 have better shade tolerance than the cultivar Üllöi and also utilize more intense light conditions more efficiently.

Black locust shows remarkable adaptability to different environmental conditions both in its native habitat and in Europe. Its ability to tolerate drought, grow rapidly, and efficiently utilize CO\textsubscript{2} makes it an interesting tree species for forestry use, especially with regard to climate change. However, the differences between individual clones demonstrate the importance of considering cultivar-specific traits in silvicultural planning.


\section{History}

\summary{During the ice ages, warm-loving species in Europe went extinct, including black locust. The selection of cold-tolerant species may have reduced the range of native options for climate-adapted forest management.

The black locust was first mentioned in writing at the beginning of the 17th century and was presumably first grown from seed in Europe before 1634 by John Tradescant. The earliest records of its presence in Vienna date back to the late 17th century, while its introduction in Hungary is suspected around 1710–1720, although exact dates remain uncertain.

Breeding and performance testing of various black locust cultivars are presented. The focus is on especially vigorous, straight-stemmed, and resistant clones. Targeted breeding began around the 1930s, roughly coinciding with the description of the ship mast locust. Subsequently, over 100 provenances were selected and compared in America, with the Appalachia cultivar being one of the best known today. In Hungary, targeted breeding has been ongoing for almost 100 years, continuously incorporating breeding material from other countries. Today, a wide selection of excellent clones is available.}

\subsection{Until the Ice Age}

In the Oligocene (33.9–23.03 mya; Million Years Ago), deposits of two black locust species were detected in Europe. In the Miocene (23.03–5.333 mya), more than a dozen deposits were identified as black locust species. One from the late Miocene resembles today’s North American black locust so closely that its describer was inclined to consider both identical. This form was also widespread in the European region during the Pliocene (5.333–2.588 mya). At the same time, a second black locust species was widespread along the Mediterranean coast of southern Europe. There is no evidence that these black locust species survived the first ice age in Europe \citep{berry1918robinie}.

During the ice ages, the Alps and the Mediterranean Sea formed migration barriers for many plant species in Europe. This presumably favored especially cold-tolerant species, while warm-loving and drought-resistant species were displaced or became extinct.

The impoverished native species spectrum could limit the scope of action in times of strong climate warming and complicate adequate management.

\subsection{First Written Records}

One of the first, if not the very first, written references to black locust (locust tree), or at least to a similar species, comes from Strachey, who recorded this between 1610–1612 \citep{strachey1610-1612historie}: \enquote{The bowes are of some young plant, eyther of the \emph{locust-tree} or of weech, \dots} (weech witch-hazel -- Hamamelis virginiana)) and \enquote{By the dwellings of the salvages are bay-trees, wild roses, and a kind of low tree, which bears a cod like to the peas, but nothing so big: we take yt to be \emph{locust}.}

It was brought to Europe for the first time in the early 17th century and is now widespread there. However, it is currently unknown exactly how and when this happened. According to current knowledge, it was first listed in 1634 in England by John Tradescant the Elder under the name \emph{Locusta virginiana arbor} \citep[p.~339]{gunther1922botanists}. John Tradescant the Younger brought several plants from a journey to Virginia and may have been the first to bring black locust seeds to Europe. Tradescant the Elder was in Paris in mid-1625, where an exchange may have taken place \citep{ginter2022robinieGeschichte}.

\citet[pp.~171--173]{cornuti1635robinie} described a plant similar to black locust and named it \emph{Acacia Americana Robini}. His description reads: \enquote{Succedunt semina Lenticulae similia, quae singula singulis nucleis duris admodum, \& ex omni parte echinatis clauduntur.} (The lentil-like seeds are each enclosed in separate, very hard, and spiny capsules.) However, black locust does not have seeds individually enclosed in spiny capsules.

\citet[p.~28]{deLaBrosse1636robinie} mentioned an \emph{Acatia Indica}, which was allegedly planted by Vespasien Robin and was still growing in Paris at the time.

\citet[p.~370]{gunther1922botanists} lists seeds from Virginia, received on March 18, 1636, which Parkinson apparently obtained from Mr. Morrice, with the following entry: \enquote{A yellow wood called \emph{Locust} long flat blackish browne pods, black round flatt seede kidney like.}

\citet[p.~1550]{parkinson1640theatrumBotanicum} describes the \emph{Acacia Americana Robini} from \citet{cornuti1635robinie} and gives it the name \emph{Pseudoacacia Americana Robini} to distinguish it from the black locust, which he refers to as \emph{Arbor siliquosa Virginensis spinosa, Locus nostratibus dicta, Virgin Locus tree}. He writes: \enquote{A very like tree hereunto hath beene sent and brought us out of \emph{Virginia}, growing to be a very great tree, and of an exceeding height with Masters \emph{Tradescant}, \dots} and \enquote{We have not seene the tree to bear any flowers with us as yet nor fruite, but the cods that came to us, were small, long, and somewhat flat \dots, containing small grayish flat and round seede.} and \enquote{\dots is called Locus by our Nation resident in Virginia.}

There are differing accounts regarding when the first black locust arrived in Vienna. According to \citet[p.~147]{loudon1838arboretum1}, the first black locust in Vienna was planted in 1696 at what is now Palais Pallavicini on Josefsplatz. \citet[pp.~15--16]{jacquin1825univGarten} mentions this tree as well as a second black locust of roughly the same age, which was allegedly planted by Emperor Leopold I in the Neue Favorita in Wieden, today the Theresian Academy on Favoritenstraße, possibly around 1690, following the reconstruction of the palace.

\citet{jagr1949robinie} reports a 300-year-old black locust in the Gatterhölzl area of Meidling, which would suggest it germinated around 1649. \citet{anonymNatLand1949robinie} likely refers to the same tree, describing it as \enquote{approximately 300 years old} and located in the park of the Springervilla in Vienna’s 12th district, Tivoligasse 73. This age estimate is thus only approximate, and the derived germination year is uncertain. \citet[p.~1395]{hegi1924band43} cites the second half of the 17th century as the period when black locust arrived in Austria.

More than 100 years later, \citet[p.~339]{Feistmantel1835dieForstwissenschaft} wrote about black locust: \hyperlink{german:Feistmantel1835dieForstwissenschaft}{\enquote{Although it is already widespread throughout Austria, one does not see it forming actual forest stands.}} (own translation). \citet{hofmann1851baumloseEbenen} recommended the use of black locust among many other tree species for establishing windbreak hedges, as was already being done at that time in Schönau (Lower Austria) and Altenburg (then in Hungary).

Ten years later, \citet{hofmann1861waldbaumCulturWarchfelde} lamented that, with few exceptions, no one had followed his recommendation, and that the Marchfeld region had suffered major losses due to wind damage. It is noteworthy that ten years later, significantly fewer species were recommended, namely black locust, tree of heaven, balsam poplar, oak, and black and Scots pine, with the comment: \hyperlink{german:hofmann1861waldbaumCulturWarchfelde}{\enquote{In the Marchfeld, everything that grows is valuable and can be used profitably}} (own translation).

According to data from the Austrian Forest Inventory \citep{bfw2025waldinventurWeb}, black locust currently makes up about 12\,\% of the growing stock in the district forest inspections of Gänserndorf and Mistelbach (Tab.~\ref{tab:waldinventur}).

%%%

\citet[p.~3]{vadas1911robinie} concluded from old trees that black locust was first planted in Hungary around 1710--1720, a claim questioned by \citet[p.~179]{ernyey1926robinie}. He noted that János György (Johann Georg Heinrich) Kramer recommended black locust for afforestation in Hungary in 1735, but the plan was not implemented, as Prince Eugene of Savoy, who supported it, died in 1736.

According to the Hungarian Forest Inventory \citep{waldinventur20152019ungarn} for the period 2015–2019, black locust covers 421\,066 hectares (equal to 19.2\,\% of Hungary's forest area) and accounts for a standing volume of 60\,987\,300\,m³ (12.5\,\% of the total volume of all tree species). It is thus the leading species in terms of area and shares the top position with Turkey oak in terms of volume.

\citet{bund1899robinie,gaskil1906robinie} report that between five and six million black locust seedlings were distributed free of charge each year for planting, which likely contributed significantly to its widespread establishment in Hungary.


\subsection{Selection and Breeding}

Even though the genetic diversity of black locust in Europe is said to be limited, there are still notable variations. \citet[pp.~259--260]{Michaux1813arbres} reported on a thornless black locust discovered by M. Descemet from Saint-Denis around 1803–1805, which is particularly well suited for coppice forests. In addition, it is said to grow faster than other types.

It was also observed that offspring grown from the seeds of this thornless tree often developed thorns again. However, \citet{Michaux1813arbres} suspected that the seeds from these thorny offspring would once again produce thornless individuals.

\citet[p.~173]{quatrefages1861robinie} described that this variety was propagated vegetatively and has since become widespread around the world.


\subsubsection{Propagation}

Black locust can be relatively easily propagated vegetatively using \emph{root cuttings}. Since vegetative propagation has been practiced for a long time, it is possible that later selections from different locations are genetically identical \citep{liesebach2012robinie}.

The diameter of root cuttings should be between 1/4 and 1 inch (0.6–2.5\,cm) and about 3–5 inches (8–13\,cm) in length. The yield from young trees, especially those dug up in nurseries for transplanting, is significantly better than from older trees (up to 80\,\% of cuttings from young plants sprout). Up to 50\,\% of cuttings from older trees sprouted when the roots had a thickness of 2–4\,cm.

Root cutting and planting should take place before bud break in spring. Roots must be protected from drying out and from frost. Horizontally planted root segments had a higher success rate than vertically planted ones, possibly due to incorrect orientation when placed vertically. Vertically oriented roots tend to produce better seedlings but may require marking to ensure correct planting direction. On the other hand, many roots naturally grow horizontally, especially those that produce root suckers in forest settings. From a practical standpoint, it is more efficient to lay root pieces horizontally and cover them with soil than to plant them vertically. Vertically planted cuttings should be covered with no more than 1\,cm of soil. In the case of root seeding, 3–5\,cm long cuttings are placed horizontally and covered with no more than 4\,cm of soil.

For nursery-grown trees, short root stumps of 3–4.5\,cm with fine roots are sufficient, as long as they remain in the nursery. Trees planted in forests should have a fully developed root system that has neither been shortened nor harvested for propagation. Light soils such as sandy loam are suitable for propagation, but they must not dry out. Adequate irrigation is necessary until a height of 10–15\,cm is reached. Pruning the shoots during the growing phase is detrimental. Cuttings from both herbaceous and woody young shoots rarely develop roots \citep{swingle1937robinie,redei2001robinieVermehrung,redei2005robinieVermehrung}.

When propagated generatively by seed, germination rates can be improved through various pre-treatments. These include soaking the seeds in water for one to two days, brief heating (e.g., by pouring hot water over them for a few seconds), or nicking/scarifying the seed coat. On mining spoil heaps, the germination rate of untreated seeds was only between 3\,\% and 17\,\%, and the survival rate of resulting seedlings ranged between 23\,\% and 78\,\% \citep{limstrom1949robinie}.

According to \citet{draghici2024robinie}, both the substrate and the watering method influence seedling establishment. Slightly acidic, nutrient-poor, loose soils performed best under moderate to high irrigation, while alkaline, nutrient-rich soils showed poorer results. However, these differences were not statistically significant.

Vegetatively propagated plants have low genetic variability. To counteract this, some varieties are offered as clonal mixtures. Even seed-propagated plants can exhibit low genetic variability if they originate from clonal seed stands. To obtain genetically diverse seed material, the harvested stands must be heterogeneous \citep{pakull2024robinieKlon}. However, in contrast to vegetative propagation, it is uncertain whether the resulting trees will still express the desired traits.

Hungarian stands show high genetic variation within populations and only minor differences between them, which is attributed to seed propagation and seed movement across the country. In Germany, by contrast, predominantly vegetative reproduction from a few local trees has resulted in low genetic diversity within stands, but high variation between them \citep{liesebach2011robinie}.

\subsubsection{Honey Production}

\citet{keresztesi1983robinie} classified selected black locust cultivars into three categories:  
(1) Timber production (for high-quality sites),  
(2) Poles and posts (for medium-quality sites), and  
(3) Beekeeping and ornamental planting.  
Some cultivars are considered suitable both for forestry and honey production.

For honey harvesting, particular interest lies in early-flowering types  
(\emph{var.\ praecox}), late-flowering types (\emph{var.\ galiana}), and continuous or multiple-flowering types (\emph{var.\ semperflorens} Carrière).  
The nectar potential of different melliferous plants varies significantly and is shown in Table~\ref{tab:honigertraege}.  
Black locust exhibits a high honey yield potential, ranging from 48 to 1600\,kg/ha.

%%%

The black locust is also important for pollinating insects.  
A total of 20 wild bee species have been recorded on black locust trees in urban environments.  
59\% of these species collected both nectar and pollen, making black locust a valuable resource for wild bees \citep{hausmann2016bienen}.


\subsubsection{Early Beginnings}

Black locust cultivars were described by \citet{vilmos1908robiniensorten}, who also discussed the possibilities of breeding. He addressed, among other things, the external form of the tree as well as the placement and growth of its branches.

Breeding of black locust began in Hungary in 1930, initiated by R.~Fleischmann \citep{keresztesi1983robinie}. \citet{fleischmann1933robinie} noted that breeding for drought resistance would be a promising endeavor. For this purpose, he collected seeds from local Hungarian black locust trees but also obtained seeds from American provenances (Washington State Forest; Asheville, North Carolina; Jarfield, Ohio; East Lansing, Michigan). Not only height and diameter growth, but also differences in thorn length among the various provenances were observed.

According to \citet{bloes1992robinie,mebrahtu1989robinie}, fast-growing cultivars tend to have larger thorns. However, through combined selection, it should be possible to achieve both high growth performance and reduced thorn length.

\paragraph{Drought Resistance and Rooting Depth}

According to \citet{guse2011robinie}, there is a positive correlation between drought tolerance and biomass production, suggesting that selecting vigorously growing varieties can simultaneously lead to the selection of drought-resistant ones.

The root form (shallow or taproot) is also considered an important selection criterion in black locust breeding for arid regions. According to \citet{bunger1938robinieWurzeltiefe}, black locust roots can reach depths of up to 26 feet (7.9\,m), and according to \citet[p.~424]{harlow2000dendrology}, 20--25 feet (6.1--7.2\,m). According to \citet{lyr1967wurzel}, they can already reach depths of 1.5--2\,m in the first growing season, which helps them survive periods of drought. \citet[p.~38]{bluemke1955robinie} reports that a one-year-old seedling can reach a root depth of 2.32\,m.

Based on the literature compiled by \citet{stone1991wurzel} on root extension, black locust is among the tree species with the deepest-reaching root systems. However, this is always dependent on site conditions. For example, spruce, often characterized as a shallow-rooted species, has been measured with sinker root depths of up to 6\,m on suitable soils. For Douglas fir, depths of 10\,m have been reported, 9\,m for pedunculate oak, and 8.2\,m for Siberian elm.

\subsubsection{Shipmast Locust}

With reference to the Shipmast Locust \citep{raber1936shipmast}, it was mentioned that the Hungarian breeding program also includes wood quality and resistance breeding against pests.

\citet{mihalyi1937robinie} reported that he first heard about robust, straight-growing locusts from America at the IX IUFRO World Congress in 1936 in Budapest. As a result, Hungary requested seeds from a certified locust stand in America, and he received a package with root cuttings of the Shipmast Locust.

Although the first publications I found that explicitly state the goal of breeding straight-stemmed locusts date from the 1930s, some illustrations in \citet{vadas1911robinie} already show straight-stemmed locusts that were likely planted around 1850. \cite{gaskil1906robinie} also notes that individual trees can grow quite straight.

\citet[p.~67]{wangenheim1781nordamericanischeHolzarten}, who was in North America from 1777, writes about the locust tree: \hyperlink{german:wangenheim1781nordamericanischeHolzartenA}{\enquote{This tree grows fairly fast, \emph{very tall and straight}, and also reaches a considerable thickness. \dots It is therefore used solely for utility wood.}} (own translation), However, \citet[pp.~22--23]{wangenheim1781nordamericanischeHolzarten} also reports: \hyperlink{german:wangenheim1781nordamericanischeHolzartenB}{\enquote{The seeds were obtained through certain persons in America who had not the slightest knowledge of their selection and ripening and pursued this trade solely for profit. \dots Germany has so far been largely planted with such degenerated, nursery-bred stems produced in the English schools.}} (own translation)

\citet[p.~249]{Michaux1813arbres} differentiates between locusts whose heartwood is red (best quality), green (moderate), or white (lowest quality), and speculates that the variations in color result from the different soils on which the trees grew. He also describes that the forests on Long Island were largely destroyed during the War of Independence, after which locusts were planted there.

\citet{cobbett1825woodlands} also described various types of locusts (Yellow, Sweet, Water), each yielding different wood qualities, and thus emphasizes the importance of the variety. According to him, the best varieties come from near Harrisburg, Pennsylvania, from where he obtained his seeds. Cobbett was on Long Island from 1817 to 1819 and reported on very durable locust wood. In 1819, he brought locust seeds to England and is said to have sold more than a million trees.

\citet{hicks1883robinie} reported on black, yellow, and white locusts on Long Island, with only the yellow variety being of high value. The yellow type reportedly forms coarse bark, is more difficult to propagate by seed, and can reach heights of 90 feet (27\,m). This locust was reportedly used for reforestation in \emph{Austria and Hungary}.

\citet{hopp1941robinie} classifies locusts into the primary classes \emph{determinative} and \emph{diffusive}, the latter appearing only in solitary trees. The determinative type is further subdivided by the height of the wind attack point (form point) into \emph{pinnate}, \emph{spreading}, and \emph{palmate}.
\begin{description}
\item[determinative:] clearly defined stem, with either no or consistent branching and curvature
  \begin{description}
    \item[pinnate:] feathered, determinative, low form point -- A-shaped crown
    \item[spreading:] spreading, determinative, high form point -- umbrella-shaped crown
    \item[palmate:] hand-shaped, diffusive
  \end{description}
\item[diffusive:] many small branches, but no easily recognizable main stem
\end{description}
The \emph{best provenances} come from the pinnate class, which was found \emph{in the natural range only in Elkins, West Virginia, at high elevations}. Spreading types mostly form crooked stems. In dense stands, the stem form of palmate types is also quite good, whereas pinnate types tolerate dense stands less well.

\citet{cope1929robinie} differentiates between black, yellow, and white locusts and suggests that these differences are not site-related but are instead due to varietal differences, as the types occur side by side on Long Island. The wood of the black variety is considered more durable, and the tree reportedly grows very straight. However, this variety is not exclusive to Long Island.

According to \citet{detwiler1937robinie}, \emph{Charles F.\ Swingle} proposed in 1934 to name the yellow locust, due to its long, straight trunk, the \emph{Shipmast Locust}. This name was first published by \citet{raber1936shipmast}, without mentioning that the idea did not originate from him. He gave the variety the name \emph{Robinia pseudoacacia var.\ rectissima}. It reportedly grows straight even in open stands and can reach heights of up to 100 feet (30\,m). It is said to occur in New Jersey, New York, Long Island, and Massachusetts. Its wood is said to be more durable, its crown narrower, its bark coarser, its growth more vigorous, and its insect resistance better. It produces few flowers, its pollen has low germination capacity, and the tree has little to no seed production, so it is mostly propagated vegetatively.

According to \citet[p.~321]{usForestService1948seedManual}, it is clearly inferior in the central states to the native locust there, more susceptible to locust borers, but does produce many seeds in that region.
%This may suggest that the Shipmast Locust is possibly a hybrid or a sterile clone.

\citet{minckler1948robinie} reported on a reforestation in Illinois in 1935, where seeds from the best local natural stand and root cuttings from the Shipmast Locust from Long Island were used. A comparison of the two in 1948 showed no significant differences in form or growth.

According to \citet{hopp1941robinieUnterschied}, the Shipmast Locust has smaller thorns and broader leaves. On sites with poor growth, the common locust shows better growth performance than the Shipmast Locust, which tends to die back at the top, halting height growth.

\citet{hopp1947robinie} stated that up to the age of 50, the common locust outperforms the Shipmast Locust in growth. After that, the growth rate of the common locust declines, while the Shipmast Locust maintains high growth rates.

\paragraph{Resistance to Wood-Destroying Insects and Fungi}

\citet{hirt1938robinie} and \citet{toole1938robinie} compared the resistance of common black locust and the Shipmast Locust to wood-decaying fungi. The Shipmast Locust showed greater resistance in two out of the four fungi species tested.

\citet{scheffer1949robinieDecay} compared the black locust variants Flowerfield (a previously undescribed clonal cultivar), Shipmast, and ordinary seed-propagated locusts with regard to their resistance to wood-decaying fungi. Flowerfield proved to be the most resistant, followed by Shipmast. The ordinary locust was significantly more susceptible. Resistance also varied within individual trees depending on the location of the wood. The outer heartwood at the base of the trunk showed the highest resistance, while the inner heartwood had the lowest.

According to \citet{duenisch2009robineHolzJungAlt}, the first 10–20 growth rings adjacent to the pith show significantly lower resistance to fungi than the rest of the heartwood.

The heartwood formed during the juvenile phase of the tree lost an average of 17.0\% of its wood mass within 16 weeks, while heartwood formed at a more mature age decreased by only 1.7\%. The decay rate in the wood near the pith varies considerably and is likely influenced by site conditions as well as the genotype of the tree \citep{brischke2024robineDauerhaftigkeit}.

\citet{szczepkowski2025robinePilze} also found that the resistance of black locust wood increases significantly with age.

The mechanical strength of the wood is also reduced in the innermost core from the first 7–11 years, whereas wood density shows little difference \citep{adamopoulos2007jungesUndAltesRobinienholz,bijak2021robinienholz}.

According to \citet{stringer1987robinieHolzdichte}, basic density increased from 0.57\,g/cm³ near the pith to 0.68\,g/cm³ near the cambium. Fiber length also increased radially from 0.75\,mm (pith) to 1.06\,mm (cambium).

\paragraph{Resistance to Insects}

\citet{hall1937robinie,cummings1947robinie} describe that a large proportion of the black locust trees planted in the United States were grown from seeds harvested in Europe. When seed was collected in America, even small, low-quality trees, often infested by the locust borer, were harvested. In 1935, the resistance of the Shipmast Locust was compared to that of the common black locust, showing that the Shipmast Locust was less susceptible to the locust borer, and that susceptibility decreased with increasing site quality.

\citet{wollerman1968robinieBorer} observed that clones appearing resistant in the first year were infested in the following year. Among the ten clones compared, apparently including Appalachia, Allegheny, and Algonquin, Clone~4193 (likely BN-4193 = HC-4148, which, like Allegheny, originates from Barton \citep{santamour1960robinie,steinergroup1987robinie}) showed the lowest infestation on a heavily affected site, with an average of 10 locust borers per tree. The next-best clone (No.~8450) had 22 borers per tree, while Appalachia was the most heavily infested with 38 borers per tree.

\citet{genys1990robinie,bridgen1988robinie,mebrahtu1989robinie} also demonstrated that different provenances vary in terms of growth performance and pest resistance. However, reliable conclusions require long-term and multi-site observations.

\paragraph{Provenance}

There is general agreement that the Shipmast locusts described on Long Island were planted and had clearly been previously selected somewhere for propagation. Some authors speculate that the variety originated in Virginia (\citet{hicks1883robinie} over 100 years ago, \citet{raber1936shipmast} around 1700, \citet{detwiler1937robinie} in 1683), while others state that its origin remains unclear (\citet{raber1938robinie}).

\citet{detwiler1937robinie} notes that even better black locust trees continue to be selected and that, as of 1934, a state nursery in North Carolina was actively propagating Shipmast locusts.

%\cite{raber1936shipmast} also writes that John Sands brought these locusts to Long Island.
%\cite{raber1938robinie} mentions the possibility that Captain John Smith introduced black locusts to Sand Point around 1700, from where they spread further. He concludes that it is unclear where this particular black locust variety originated, when it was introduced to Long Island, or by whom. He also states that the variety is now widely used throughout the United States to reduce erosion.

\citet{cope1938robinie} describes that Shipmast locusts can be found throughout the Hudson Valley, though their tops were killed back during a cold winter, except for apparently frost-hardy variants found in Saratoga and Washington, which he considers more suitable for propagation.

\subsubsection{Selection in North America}

According to \citet{santamour1970robinie,steinergroup1987robinie}, over 100 black locust clones were collected between 1938 and 1943 that exhibited high growth rates, good stem form, and low infestation by the locust borer. Most of the stands examined were located outside the species' natural range and had developed from earlier plantations \citep{hopp1941robinie}.

The collected clones were tested between 1943 and 1950 in Beltsville, Maryland. Of these, five were selected, including the \emph{Shipmast Locust} from Long Island. However, this clone performed worse than the others in terms of growth, resistance, self-pruning, and stem straightness \citep{santamour1960robinie}.

In 1950, 15 additional clones from native stands were added to the program. Between 1948 and 1965, 46 test plantations of the selected clones were established. Many of these plantations contained fewer than ten clones, and only the Cape May site included all 20 clones. Fifteen of these sites were relocated and measured by Ruffner in 1985 \citep{bongarten1992robinie}. From this evaluation, the three clones \emph{Appalachia}, \emph{Allegheny}, and \emph{Algonquin} were selected. These three were part of the original five clones selected in 1950.

These three clones are also mentioned in \citet{bridgen1988robinie}, where, on one test site, \emph{Algonquin} achieved a height of 157\,cm after two years, followed by \emph{Appalachia} with 138\,cm, making them the two fastest-growing sources among 26 tested. These three clones were propagated vegetatively, whereas the other provenances were propagated generatively via seed. In the same study, the provenance from Caryville Campbell (No.~716, 36°18'N 84°13'W), which was planted on different test plots from the three clones, also showed significantly greater height growth compared to the remaining 55 provenances.

\citet{bongarten1992robineSelektion} compared 24 provenances on a test site in Georgia (33.3°N; 83.5°W; elevation 130\,m), where FamilyNr.~716 was again among the more vigorous. However, after two years, it was outperformed by five families, and after three years by nine families in terms of biomass yield. No.~716 produced 10.5\,t biomass/ha after three years, while the best family (No.~704) produced 15.0\,t/ha. Family~704 had already outperformed the others in the first year. The plants were also sorted by size each year, and it was observed that in one year, the larger plants grew faster, while in the following year the smaller ones had greater growth. The greatest gain in growth performance is expected through selection of individual trees followed by vegetative propagation.

According to \citet{cummings1947robinie}, seedlings grown from fast-growing parent trees also exhibit better growth than those from slow-growing trees, provided the seed trees are younger than 25 years.

\begin{description}
  \item[Appalachia:] (HC-4138; BN-4191; NA-4913; 9030613) was discovered between Blackwood and Appalachia in Virginia (approx.~36.91°N; 82.70°W) and named in 1956. It exhibits excellent growth and form. This clone has a straight, cylindrical stem, thin and well-branched limbs, good natural pruning, and is more susceptible to browsing. It yields 85\,\% saw timber \citep{steinergroup1987robinie,zsombor1980robinie,kapusi1995robinie}.
  
  \item[Allegheny:] (HC-4146; BN-4192; NA-4914; 9030614) originates from near Bartow in West Virginia (approx.~38.54°N; 79.78°W), was named in 1987, and displays excellent vigor, straight and unbranched stems, and an above-average diameter at breast height (DBH) both in youth and maturity \citep{steinergroup1987robinie}.
  
  \item[Algonquin:] (HC-4149; BN-4194; NA-4916; 9030615) originates from near Thornwood in West Virginia (approx.~38.56°N; 79.74°W), was also named in 1987, and shows the best growth performance as well as above-average resistance to the locust borer \citep{steinergroup1987robinie}.
\end{description}

The discovery sites of Allegheny and Algonquin are estimated to be about 2\,km apart and are located approximately 50\,km from Elkins, which, according to \citet{hopp1941robinie}, is home to the best provenance within the natural range of black locust.

To ensure a certain level of genetic heterogeneity, it is recommended to plant all three clones together. Due to its superior growth performance, Algonquin should constitute 50--80\,\% of the planting. This clone mixture is referred to as the \emph{Steiner Group}, named after the breeder Wilmer W.~Steiner.

The samples of the clones Allegheny and Algonquin examined by \citet{liesebach2012robinie} were genetically identical and also matched an earlier delivery of Appalachia. However, a more recent delivery of Appalachia showed genetic differences from the others. However, this matching may also result from the fact that the microsatellite markers only capture very small sections of the core genome.

In \citet{liesebach2021robinie}, a distinction is made between Appalachia--4138 and Appalachia--4191. According to \citet{steinergroup1987robinie}, there is an original SCS (Soil Conservation Service) Hillculture (HC) number and a later Beltsville (BN) designation, and for Appalachia, these are HC-4138 (also written HC-41-38) and BN-4191, all referring to the same clone. In addition, there is an NRCS (Natural Resources Conservation Service) number for Appalachia, which is 9030613, a NA number (presumably referring to the National Arboretum) as NA-4913, and another code from the Morris Arboretum: 50-308. For some numbers, the first two digits likely indicate the year.

\subsubsection{Modern Breeding in Hungary}

Straight-growing black locusts have been known for a long time, and there was an exchange of selected propagation material (seeds and root cuttings). According to \citep{keresztesi1983robinie}, the experiments conducted by R.~Fleischmann were lost during World War II; however, on page 224, he shows an image of 44-year-old rectissima black locusts in the Gödöllö Arboretum, which likely originated from root cuttings received by \citet{mihalyi1937robinie} from America. Ongoing measurements were published by \citet{bujtas1984robinie}.

Breeding was restarted in 1951 in Budapest by F.~Tuskò and B.~Keresztesi and intensified in 1955 by F.~Kopecky. The aim was, on the one hand, to select fast-growing, straight-stemmed, and frost-resistant trees, and on the other hand, to breed varieties with an extended flowering period and increased nectar production \citep{kopecky1965robinienzuechtung,redei2007robinieSelektion,csiha2016robinie}.

In 1958, Antal Kisrómai discovered stands of acacia in several parts of the country whose stem and crown forms closely resembled those of the Shipmast Locust. On the advice of Keresztesi and the experimental station, he propagated a considerable quantity of planting material from this stock by means of root cuttings and grafting. Keresztesi observed that the straight-stemmed acacia form discovered by Kisrómai occurred in many acacia stands throughout the country and subsequently began to systematically expand its selection and propagation \citep{kopecky1965robinienzuechtung}.

To achieve the breeding objectives, trees with good stem form were crossed with those exhibiting good growth. To reduce thorn length, the variety \enquote{inermis} was introduced into the breeding process \citep{kopecky1965robinienzuechtung}.

A clonal bank was established to carry out clone evaluations and crossing experiments. Additionally, straight-stemmed black locusts were sought in all black locust forests across Hungary. Cuttings were tested in Gödöllö and the best individuals were propagated vegetatively. In 1964, 134 of these straight-growing varieties were planted and compared. Currently, Gödöllö hosts 210 clones and cultivars on an area of 50 hectares \citep{redei2005robinieVermehrung,csiha2016robinie}.

\citet{keresztesi1974robinie} presents an evaluation of seven cultivars regarding growth performance, with Nyirségi, Röjtökmuzsaji, and Üllöi standing out in particular. Additionally, a graphic illustrates 32 black locust cultivars according to four quality classes (very good, good timber, vineyard stakes, firewood). The best-performing cultivars are listed in Table~\ref{tab:robinienherkuenfte}. Further interesting provenances were also compared, some of which are shown in Table~\ref{tab:robinienherkuenfte2}. Among them, Appalachia, HC--4146 (=Allegheny), HC--4148, and HC--4149 (=Algonquin) are American cultivars, while Pénzesdombi is a Romanian variety.

%%%

A summary of this can also be found in \citet{keresztesi1983robinie}, showing that selections from America and Romania arrived in Hungary almost simultaneously and were tested there.

\citet{bluemke1955robinie} reports that clone HG--4138 from Blackwood (likely a typographical error, actually HC--4138, i.e., \emph{Appalachia}) was approved for commercial use in the Netherlands. This demonstrates that this selection also spread to other countries.

The following cultivars were registered in Hungary in the early 1980s:

\begin{description}
  \item[Zalai:] Registered in Hungary in 1973 \citep{keresztesi1983robinie}
  \item[Nyirségi:] Registered in Hungary in 1973, 6 clones, vigorous crown, requires dense spacing due to tendency to branch, pruning necessary at wide spacing, thorn length 13\,mm, low seed production \citep{keresztesi1983robinie,kapusi1995robinie,abri2024dis}
  \item[Rozsaszin~AC:] Registered in Hungary in 1973, 6 clones, pink flowers, honey-producing \citep{keresztesi1983robinie,kapusi1995robinie}
  \item[Jászkiséri:] Registered in Hungary in 1979, 1 clone, vigorous growth, large crown, thick branches, thorn length 19\,mm, tendency to branch => requires dense spacing, low lignin content \citep{keresztesi1983robinie,zsombor1980robinie,kapusi1995robinie,abri2024dis}
  \item[Kiskunsági:] Registered in Hungary in 1979, straight cylindrical trunk, thin branches, large thorns, 41\,\% saw timber yield \citep{keresztesi1983robinie,zsombor1980robinie}
  \item[Pénzesdombi:] Registered in Hungary in 1979, from Romania, very fast-growing, 66\,\% saw timber yield, few small thorns \citep{zsombor1980robinie}
  \item[Csázátötélési:] Registered in Hungary in 1979, straight trunk, thin-branched, large thorns, early bud burst, 37\,\% saw timber yield \citep{zsombor1980robinie}
  \item[Appalachia:] Registered in Hungary in 1979, from the United States
  \item[Üllöi:] Registered in Hungary in 1982, from Üllő Forest Department 10D by J.~Fila, 3 clones, straight trunk, 44\,\% more wood mass of highest quality, thorn length 10\,mm, low seed production \citep{bach1983robinie,kapusi1995robinie,redei2020ulloi,abri2024dis}
  \item[Debreceni-2:] Candidate in Hungary in 1979 \citep{keresztesi1983robinie}
\end{description}

For saw timber production, Nyirségi, Kiskunsági, Jászkiséri, Pénzesdombi, \emph{Röjtökmuzsaji}, \emph{Góri}, and Appalachia are considered well-suited, with Kiskunság also producing a significant amount of honey. Besides growth form and growth performance, selections were also made based on resistance to late frost, viruses, and insects. The new clones are also tested for susceptibility to insects and fungi \citep{abri2024robinieResistenz}.

In the 1980s, Imre Kapusi selected approximately 50\,000 particularly large one-year-old seedlings, from which the best 125 were selected at the age of 8–12 years. From these 125 parent trees, 8 were chosen whose generatively propagated offspring exhibited the highest wood production in progeny tests and form a seed plantation from which \emph{Turbo} seedlings are derived. In a further step, from the seedlings of these 125 parent trees, after another 17 years, 70 plus trees (OBE01–OBE70?) were selected, which form the basis of the cultivar \emph{Turbo Obelisk}, with OBE26, OBE34, OBE53, OBE54, and OBE69 being registered \citep{szabo2014turbo,nemeth2022robinie}.
%Of these, OBE26, OBE34, and OBE53 are the straightest and have high growth rates (personal communication Viktor Jósa, Feb 15, 2024).

Comparisons between some of the originally selected 125 clones (Group A) and their progeny (Group B) were conducted by \citet{barna2009robinieTurbo}. Clones from Group A took the top three places in height growth. Regarding diameter increment, a clone from Group B showed the best result, followed by two clones from Group A.

\begin{description}
  \item[Turbo:] Seedling, fast-growing \citep{nemeth2022robinie}
  \item[Turbo Obelisk:] 3 clones (OBE26, 34, 53), 5 clones (OBE26, 34, 53, 54, 69), fast-growing, straight-stemmed \citep{nemeth2022robinie}
\end{description}

According to \citet{redei2008robinieImprovement}, good quality can only be produced on sites with sufficient moisture and well-aerated, preferably light, nutrient- and humus-rich soils. To expand the range of suitable sites, drought resistance was selected for in the 1990s under the direction of Károly Rédei, with the following cultivars:

\begin{description}
  \item[Vacsi:] PV~201E~2/1, from Pusztavacs, straight trunk, medium vigor, fine-branched, small thorns
  \item[Szálas:] drought-resistant
  \item[Oszlopos:] PV~233A/1, from Pusztavacs, straight trunk, medium vigor, thorn length 1–3\,mm
  \item[Homoki:] MB~17D~3/4, from Mikebuda, slightly curved trunk, vigorous growth, thorn length 5–8\,mm
  \item[Bácska:] KH~56A~2/5, from Kéleshalom, tends to fork, vigorous growth, thorn length 6–12\,mm
\end{description}

Among these, Vacsi, Homoki, Bácska, and PV201E2/4 are particularly promising regarding quality and growth performance \citep{redei2018robinieImprovement,keserue2021robinie,abri2023robinieUngarn,abri2024dis}.

In the late 2010s, selection continued focusing on drought resistance combined with rapid juvenile growth and high stem quality, using the following cultivars \citep{abri2023robinieUngarn,abri2024dis}:

\begin{description}
  \item[PL251 -- Püspökladányi:] straight cylindrical trunk, good growth, short thorns
  \item[PL040 -- Farkasszigeti:] straight cylindrical trunk, fine-branched, medium vigor, large thorns
  \item[NK1 -- Laposi:] nearly straight trunk, moderate height growth and strong diameter growth, small thorns
  \item[NK2 -- Napkori:] straight cylindrical trunk, vigorous growth, fine-branched, small thorns
\end{description}

These cultivars outperformed Üllöi in diameter and height increment during the juvenile stage.

\subsubsection{Selection in Other Countries}

\citet{schroeck1953robine} attributes the low appreciation of black locust
to its crooked growth, which, however, can be remedied through breeding.
In the teaching and experimental districts Sauen, Schlepke, and in the
farmers' forest Hasenholz, clones with completely straight shafts were found.

\citet{erteld1952robinieErtrag} mentions, besides the district Sauen, also
the district forestry office Spitzberg with straight-shafted black locusts.
In Sauen, Privy Councillor Bier and his son Prof. August Bier are said to have
established single-stem seedlings from particularly good mother trees.

In addition to clones from the Buckow district, \citet{naujoks2005robinie}
also lists clones from the Sauen district already mentioned by
\citet{schroeck1953robine}.

\citet{naujoks2005robinie,hofmann2020robinie,lange2021robinie,lange2022robinie}
report on a cultivation trial where 12 pre-selected black locust clones,
three stand-origin seedlings, and two plantation-origin seedlings were planted
on four experimental plots in the winter season 2013/2014. For some of these,
it was known that they originated from Hungary, but not their variety, which
could partly be reassigned through genetic analyses. The variety Kiskunsági
was propagated by seed rather than vegetatively.

In the 1990s, 33 vigorous and straight-shafted black locust plus trees,
with few or no large branches, were selected in Brandenburg, six of which
were established as tissue culture clones and included in the trial
(Robert, Roger, Romy, Rowena, Roy). The varieties Bendida and Tangra from the
company Lignum from Bulgaria and Cuci from Romania were also used. Differences
in vitality at age 5 were recorded, with Appalachia, Nyirségi, and Fraport3
(=Zalai?) showing the highest proportions of fully vital trees on two test sites.

Regarding steep branches, Bendida (45.4\,\%), Fraport3 (46.3\,\%),
Romy (50.0\,\%), and Cuci (51.9\,\%) performed best; regarding terminal
shaft formation, Fraport3, Appalachia, Nyirségi, Bendida, Romy were top,
and in terms of straightness, Fraport3, Jászkiséri, Appalachia, Bendida,
and Nyirségi were the best.

Regarding height growth over 6 years, Fraport3 (7.54\,m), Roger (6.93\,m),
Rowena (6.84\,m), and Appalachia (6.76\,m) showed the strongest growth.
In the dry year 2018, Appalachia (0.86\,m), Roy (0.73\,m), Romy (0.72\,m),
Rowena (0.70\,m), and Fraport3 (0.68\,m) had the greatest height increments.

The highest biomass gains over 6 years were recorded for Fraport3 (6.9 t/ha),
Roger (6.1 t/ha), and Romy (6.1 t/ha). According to \citet{loeffler2017fastwood},
the clones Fra3, Rowena, and Roy performed best in terms of growth performance.

\citet{guse2015robinie} compared 55 provenances from 7 countries with
respect to growth performance. It was shown that high yields in the greenhouse
do not automatically correlate with good performance in the field.
The provenances Nessebar (Bulgaria) and Waldsieversdorf III (Germany)
showed above-average yield performance.

According to \citet{dimitrova2024robinie}, seedlings of the Bulgarian clones
Pordim-10, Pordim-13, Obretenik-1, Obretenik-6, and Ryahovo-1, as well as
the Hungarian clones Szajki and Nyirségi, exhibit high growth performance.

According to \citet{diniPapanastasi2004robinie}, the clones A-7B(6),
A-B(3), and B-2B(8) showed the highest growth performances, are among
the clones with many leaves, and consistently had longer thorns than the
weaker-growing clones with fewer leaves.

\citet{budaeu2023robinie} compared the Romanian clones S1, S2, S3, and S4,
where S4 performed best both in terms of the number and the height of shoots.

In Poland, 28 straight-shafted clones from 7 stands were collected in 2014.
The stands are: Krosno--90b, Krosno--232i, Cybinka--98y, Wołów--194f,
Pińczów--426f, Strzelce--150Am, Mieszkowice--210j, and Wyszanów
\citep{wojda2015robiniePolen}.

Large areas in Korea have been reforested with black locust to combat soil erosion.
Seventy percent of honey production comes from black locust. Straight-shafted
selections include Hapcheon and Daegu \citep{lee2007robinieKorea}.

\subsubsection{Comparative Plantations}

Comparative plantations with non-native tree species were published in Austria by \citet{cieslar1901fremdlaendischeHolzarten}. Black locust was not investigated in these studies. Likewise, in Germany, black locust was not examined in comparisons of non-native tree species \citep{schwappach1901fremdlaendischeHolzarten}.

\hyperlink{german:rannert1979fremdlaendischeBaumarten}{\enquote{Based on the positive experiences made so far with Weymouth pines, black locust, American black walnut, red oak, and Douglas fir planted in native forests, Cieslar advocates conducting cultivation trials with exotic species and announces planned exotic cultivation experiments from the research institute. \dots Black locust and various poplar species were excluded from the exotic inventory because they could already be considered "native" due to their wide distribution in eastern Austria}}
\citep[own translation]{rannert1979fremdlaendischeBaumarten}.

As long as the aim is to check whether a tree species can grow at all in a new region, this restriction may be justified.
However, if the goal is to find the species with the best performance, all promising species, both native and non-native, must be planted and compared, as was already demanded by \citet[p.~300]{reaumur1721ertragstafel}:

\enquote{Il ne faudroit commencer qu’à défricher de très-petits cantons, \& à les planter de différentes sortes d'Arbres, pour voir ceux qui y réussiraient mieux.} (One should initially clear only very small plots and plant them with different kinds of trees to find out which grow best there.)

\enquote{Enfin il faudroit tâcher de reconnoître les terrains les plus  propres à chaque espéce d’Arbres, \& ne leur donner que Les espéces d’Arbres qui leur sont propres.} (Finally, one should try to identify the soils most suitable for each tree species and give them only the tree species that are appropriate for them.)

\enquote{Notre attention ne devroit-elle pas aller jusques à chercher si les pays étrangers n'ont point des Arbres qui nous seroient utiles, qui croîtroient aisément dans le Royaume?} (Should our attention not even extend to searching whether there are trees from foreign countries that would be useful to us and would easily grow in the kingdom?)

Also, \citet[p.~253]{carlowitz1713sylvicultura} considered using non-native tree species if they had good properties:

%and that the European species do not equal them in quality / and thus a greater benefit / perhaps through propagation of those foreign wild trees / than through delicate exotic plants / could arise / also the wild wood / because it is more durable and stronger by nature / is more suitable than those / especially since it can be propagated by seed. \dots that the useful American or other foreign wood could also be cultivated in Europe.
\hyperlink{german:carlowitz1713sylvicultura}{\enquote{\dots and that European [wood] is not equal to those [foreign types] in quality; and thus greater benefit might perhaps arise from the propagation of foreign wild trees rather than from delicate exotic plants. Moreover, wild wood, being of a more durable and stronger nature, could be propagated more easily than the latter, especially by means of seed. \dots to cultivate useful Indian or other foreign wood also in Europe.}} (own translation) (\dots and since European wood is not equal in quality to foreign varieties, greater benefit might be gained from cultivating foreign wild trees rather than delicate ornamental species. Wild trees, being more durable and robust by nature, may also be more easily propagated, especially by seed. \dots with the goal of establishing useful American or other foreign timber species in Europe.)

\subsubsection{Austria}

In Austria, in the 1980s, Ferdinand Müller selected the variety \emph{Tulln}
for biomass production in short rotation. Growth form was not a selection criterion.

\citet{mueller1999robinie} aimed to create a clone mixture, a multi-clone variety
composed of about 30 clones, referring to \citet{huehn1986klonanazahl}, who recommends
20--40 components depending on the expected risk of infection.

\hyperlink{german:mueller1999robinie}{\enquote{Currently, the goal is to propagate the ten most vigorous clones in larger numbers
for broader cultivation trials and to select additional clones to increase the number
of clones. Only when 30 clones tested at multiple sites are available will the use
of the planting material be recommended. Even then, a continuous change in clone composition
through the removal of less suitable clones in favor of more productive selections
will ensure variation in the genotype composition of future short rotation plantations.}}
\citep[own translation]{mueller1999robinie}.

Comparative plantations were established in 1982 in Karlwald near Halbturn and Rüsterwald
near Neusiedel, in 1985 in St. Margarethen, and in 1988 in Riedenthal near Mistelbach.
Short rotation areas were established in 1984 in Wasserburg near St. Pölten and in 1985 in Bruckneudorf,
biomass trials in 1987 in Neckenmarkt and in 1988 in Nickelsdorf.

An evaluation of a biomass trial plot newly established in 1985 in the Tulln experimental garden
showed, for a 4-year rotation period, an average dry matter production over all Austrian selections
of 7.44 t/ha/year, 10.07 t/ha/year for the ten most vigorous clones, 5.93 t/ha/year for Appalachia,
and 4.07 t/ha/year for Nyirségi \citep{mueller1991robinie}. According to \citet{redei2005robinieEnergieholz},
the yield in the first rotation is lower.

The evaluation of the Riedenthal plot was also published, as it had to be abandoned due to
the construction of a motorway. The varieties compared were Tulln~81/29, Tulln~81/55,
Tulln~81/62, Tulln~81/66, Tulln~81/83, Tulln~83/09, Tulln~83/10, Zalai, Nyirségi, Jászkiséri,
and Appalachia.

Appalachia had about 55\,\%, Nyirségi and Jászkiséri about 40\,\%, and Tulln~83/10 20\,\%
straight-shafted stems. Including slightly curved stems in these varieties, the total proportion
amounts to around 70\,\%.

Regarding diameter, Jászkiséri is among the strongest and Nyirségi among the weakest varieties \citep{schueler2006robinie}.

However, these diameter differences are likely due to site differences, as \citet{heinze2014robinie}
found that the two clones identified as Jászkiséri and Nyirségi on the trial plot are genetically identical.
This is even more surprising because Jászkiséri consists of one clone, whereas Nyirségi consists of six clones.

High-quality black locust varieties have been reported in Austria, for example, by \citet{mueller1991robinie},
\citet{iby1998robinie}, and \citet{demel2004robinie}. All three articles include photos of straight-growing black locusts.

However, there is the impression that at that time the focus was on stand conversion and mixed growth management,
especially to reduce black locust.

Black locust is quite common on trial plots in Austria, but only in the form of stakes marking the plot boundaries or as fence posts.

In 2001, Werner Ruhm established an experimental plot with locust
trees in Glaswein, partly to save remaining plants from the nursery
from destruction.

The varieties compared there are Tulln~81/29, Tulln~81/62, Appalachia, Nyirségi, and Jászkiséri.

Since the planting material for this trial plot was obtained from the same nursery as for the Riedenthal plot,
the varieties Nyirségi and Jászkiséri might also be identical here.

Particularly convincing in quality is Appalachia (Fig.~\ref{fig:glaswein2}), which was initially inferior
to the other varieties in growth performance but almost caught up by the age of 25 years.

In 2008, a trial plot with the variety Tulln was established in Laa an der Thaya. In 2011, additional plots with the varieties Tulln, Nyirségi, and Zainet as well as another plot with the variety Üllöi were established in Groß Harras. These plantations were set up by the Chamber of Agriculture of Lower Austria within the framework of the project \enquote{Bewirtschaftung von Energieholzplantagen} (Management of Energy Wood Plantations) \citep{huber2018kurzumtrieb}.

%%%

Older stands of straight-growing black locust also exist in Austria.
For example, the seed source stand Rob1~(8.1/ko) at Weichselberg near Oberwinden,
Gutenbrunner Forest, Herzogenburg in Lower Austria, whose plants germinated
around 1934 (Fig.~\ref{fig:hezogenburg}).

\citet{heinze2014robinie} compared this stand using 14 microsatellite markers
with Appalachia, Jászkiséri, Nyirségi, Tulln~81/29, Tulln~81/62, and three
black locust trees from Mariabrunn, but could not find any match.

As mentioned earlier, there was likely a mix-up at the nursery from which the clones
called Jászkiséri and Nyirségi were obtained, since they were genetically identical.

A 20-year-old stand near Oberwinden had a top height of 23 m and a DBH of 25–30 cm.
An 80-year-old stand had a top height of 31 m and an average DBH of 43 cm.

%%%

\subsubsection{Selection of Varieties}

\citet{bloes1992robinie} observed that the performance of individual families is relatively stable across different environments and suggests that selecting varieties for specific sites may not be necessary.

However, I do not expect that a single variety will be the best in all situations. Therefore, it is encouraging that there is now a large number of black locust varieties available, even when limiting the selection to those with the straightest growth.

Black locust varieties can be propagated relatively easily by root cuttings, whereby breeder’s rights must be observed.

In Austria, there are hardly any comparative plantations, and if so, only for older varieties. Additionally, there have been cases of variety mix-ups.

For a sound decision-making process, demonstration plots would be desirable to enable practical experience and provide a realistic impression of variety performance.

Among the varieties described in the literature as high quality, the following appear particularly promising:

\begin{description}
  \item[North America:] Appalachia, Allegheny, Algonquin
  \item[Hungary:] Zalai, Kiskunsági, Csázátötélési, Üllöi, Zajki, Röjtökmuzsaji, Góri, Turbo, Turbo Obelisk, Vacsi, Oszlopos, Püspökladányi (PL251), Farkasszigeti (PL040), Napkori (NK2)
  \item[Romania:] Pénzesdombi, Oltenica
  \item[Bulgaria:] Bendida, Tangra, Pordim-10, Pordim-13, Obretenik-1, Obretenik-6, Ryahovo-1
  \item[Poland:] Krosno--90b, Krosno--232i, Cybinka--98y, Wołów--194f, Pińczów--426f, Strzelce--150Am, Mieszkowice--210j, Wyszanów
  \item[China:] Lüman Qingshan, Miyuan~1
  \item[Korea:] Hapcheon, Daegu
  \item[Germany:] Romy
  \item[Austria:] Oberwinden
\end{description}

Due to previous mix-ups of individual clones \citep{heinze2014robinie,liesebach2012robinie}, an investigation would be useful to clarify whether these varieties actually differ.

Early and late frosts or drought can cause the crown to die back, resulting in crooked stems. If these influences occur in one region but not in another, a clone may grow straight-stemmed at one site but crooked at another.

According to \citet[p.~48]{erteld1952robinieErtrag}, black locust trees are considerably better formed on fertile sites.

Differences in juvenile growth between varieties might be due to certain varieties allocating a larger proportion of their growth resources to root development than others. This could in turn explain their varying resistance to drought.

For experimental plots, it is advisable to include a commonly used single-clone variety as a reference. Appalachia would be suitable for this purpose, but it should first be clarified which clone is actually Appalachia.

Initially, it makes sense to compare black locust varieties among themselves and, if resources permit, to additionally compare them with other tree species, as \citet{Gruenewald2009robinie} did, for example.

If sufficient capacity is available, mixtures of tree species as well as different treatment variants could also be investigated. Since there are countless possible combinations, a careful preselection of the most promising variants is recommended.

\subsubsection{Breeding Objectives}

In summary, the breeding objectives can be grouped into several directions that are not necessarily mutually exclusive:

\begin{itemize}
  \item good wood quality and straightness of the stem
  \item high overall growth performance
  \item high juvenile growth rates
  \item low decline in growth performance with age to enable long rotation periods and achieve large dimensions in diameter and height
  \item high growth efficiency per liter of water consumed
  \item for urban trees: stability, longevity, and limited maximum size
  \item drought resistance
  \item frost resistance
  \item tree bark avoided by hares
  \item deep root system
  \item resistance to pests and diseases
  \item high natural durability of the wood, especially of the innermost heartwood
  \item low to no thorniness
  \item fine branching
  \item good natural branch shedding
  \item cylindrical stem form
  \item shade tolerance
  \item high nectar production for beekeeping
  \item abundant root suckering for sites with difficult sexual regeneration
  \item low to no root suckering and few to no viable seeds for forests near nature reserves
  \item improved wood processability (e.g., low warping during drying)
  \item lower tannin or phenol content for specific wood uses
  \item site improvement such as nitrogen fixation or soil loosening
  \item traits optimized for agroforestry, such as light absorption, water use, or nitrogen fixation rate
\end{itemize}

\section{Comparative Planting}

\summary{Various black locust clones are compared in terms of growth performance, quality, and suitability for different sites. Newer clones are compared with already available ones. The main trial site is located in the Weinviertel region, supplemented by single planting rows in Vienna and near Schneeberg.

In addition to general forestry aspects such as stand density, pruning, mixed growth, planting technique, and site selection, the economic importance of black locust timber and its role in CO$_2$ sequestration are discussed. As a durable, regional timber, black locust offers an attractive alternative to tropical hardwoods.

The implementation was carried out using simple, practical methods. Particular attention was paid to practical handling of planting material (e.g., pruning, root development) as well as management of competing vegetation and browsing damage.

Literature, expertise, and plants were gathered thanks to support from experts, forestry institutes, and nurseries in Austria and Hungary. The project is intended as a practice-oriented contribution to better assess the suitability and quality of various black locust clones.}

Although the black locust clones used in the Austrian trial sites were current at the time of planting, there is a lack of local experience with the newer clones introduced since then. To provide a small impetus toward closing this knowledge gap, I decided to plant a few black locust clones side-by-side on a small scale.

This endeavor almost failed at the very first steps because initially I could only obtain the clone Appalachia from the desired varieties. To at least have something to compare with, I also took clones available from two nurseries in eastern Austria.

In hindsight, such a comparison is quite interesting, since it is also possible that these clones may likewise prove to be of good quality.

By chance, I came into conversation with Gyula Kovács on this topic, who facilitated contacts for obtaining further clones.

On the one hand, this was the cultivar Turbo, for which he gave me the breeder's contact information; on the other hand, the clones Üllői, Vacsi, Napkori (NK2), and Püspökladányi (PL251), for which he connected me to Director Dr.~Attila Borovics.

He forwarded my request to Dr.~Zsolt Keserű, who got in touch with two nurseries, obtained their approval for me to receive the new clones, procured the plants, and also organized transport for the plants from Debrecen to Sopron.

Dr.~Attila Benke attended a meeting in Debrecen where he received the plants from Dr.~Zsolt Keserű and brought them to Sopron for me.
%, where he was kind enough to invite me for coffee, since I had no Forints.}

I am very grateful for their support.

\subsection{Experimental Plots and Initial Situation}

In a small forest area in the Weinviertel near Ernstbrunn, a Scots pine stand on the verge of collapse grows, interspersed with individual sessile oaks, walnut trees, black locusts, ashes, field elms, winter linden, pear trees, and alder buckthorn, predominantly in the regeneration layer, as well as climbing bindweed, blackthorn, rose, hawthorn, and dogwood.

Scots pine hardly regenerates naturally but shows very good establishment performance with few losses when planted.

Since further climate warming is expected to make regeneration increasingly difficult, it seems advantageous to me to switch predominantly to tree species that produce root suckers or at least have good sprouting ability, already now, while this is still reasonably possible without significant effort.

Some larger black locusts can be found in neighboring stands; however, these show poor stem quality, especially at forest edges (Fig.~\ref{fig:robLokalErnstbrunn}).

%%%

Even though the concerns raised by \citet{bouteiller2019robinie} regarding the introduction of new black locust selections are justified, at least in regions where black locust is already widespread, the potential uses of high-quality timber should be weighed against firewood.

On a two-year-old clearcut near Ernstbrunn, black locusts about 3–5 m tall have meanwhile established themselves on approximately half of the area (Fig.~\ref{fig:ernstbrunnSchlagflaeche}).

%%%

At first glance, it may easily appear that black locust displaces other tree species there. For light-demanding pioneer species such as birch, which was not present in the old stand and is therefore missing there, this may indeed be true. However, oak or Scots pine are not displaced here by shading. It rather seems that black locust acts as a natural browsing shield for other tree species when larger game animals are present.


\subsection{Wood Utilization}

Whether wood is used or not can influence the CO$_2$ concentration in the atmosphere. According to \citet{boiger2024schnittholzBrennholz}, under current conditions, the climate impact hardly depends on whether the wood is used as firewood or sawn timber.

Nevertheless, there is usually a price difference between these assortments, which creates economic incentives to produce high-quality roundwood.

Freshly harvested black locust wood has a relatively low moisture content due to its narrow sapwood. Ring-porosity, strong tylosis formation, and high wood density lead to a slow burning rate.

Although the bark of black locust reaches an energy content almost equivalent to its wood, it should preferably remain in the forest due to its high nutrient content, in contrast to the wood.

Short rotation plantations with whole-tree harvesting therefore require fertilization on nutrient-poor sites, except for nitrogen, or should not be established on such sites, as otherwise growth performance is likely to decline over time.

With rotation periods of about 50 years, the greatest average growth is achieved. Shorter rotation periods can be chosen for special assortments (e.g., vine stakes) \citep[p.~136]{erteld1952robinieErtrag}.

If the goal is to produce high-quality sawn timber, the final harvest should take place at an older age, as long as there is no risk of heart rot and the stand quality is good.

Among the tree species growing in native forests, black locust provides the most durable wood, suitable for outdoor use even without wood preservatives. In terms of natural durability, it is only slightly surpassed by a few tropical woods.

By utilizing black locust, the supply of durable wood can be supplemented with a regionally growing tree species, as long as the dimensions allow it \citep{sari2005robinienholz,benthien2020robinieTropenholz}.

According to \citet{itto2024biennial}, about 1\,080\,000 m³ of roundwood, 760\,000 m³ of sawn timber, 257\,000 m³ of veneer, and 561\,000 m³ of plywood from tropical woods were consumed in the EU in 2024.

\subsection{Mixed Stands and Compatibility}

Although pure stands on experimental plots are significantly easier to evaluate than mixed stands, the Glaswein experimental plot, with its extremely lush undergrowth, gave me the impression that a combination with a serving shade tree species could better exploit the site potential.

\citet{redei2006robiniePappel} describe mixtures of black locust with white poplar, which have the same rotation periods.

Mixtures with oak seem appealing, especially since black locust is said to allegedly infiltrate and subsequently displace oak stands.

However, \citet{kallina1888robinie} reported that the growth performance and vitality of pedunculate oak benefit from the shade of black locust, and that the black locust should be harvested at the time when the oak reaches the height of the black locust.

Also, \citet{feher2024robinie} found no inhibitory effects of black locust on the development of oaks. On the contrary: in black locust stands, sessile oak even showed increased height growth. \citet{foeldes1903robinie} attributes this to the soil-improving properties of black locust.

\citet[p.~174ff]{pfeil1829deutscheWaldbaeume}, who was very skeptical of non-native tree species and attributed their positive descriptions to the interests of seed merchants, only discusses black locust and the white pine (\emph{Pinus strobus}). \hyperlink{german:pfeil1860holzzucht}{\enquote{Of all the numerous other foreign wood species cultivated and sold in plantations, not a single one is suitable for further cultivation in our German forests}} \citep[own translation]{pfeil1860holzzucht}.

\citet{pfeil1850robinie} recommends planting black locust in mixture with birch to achieve better stem quality.

\citet[p.~88–96]{erteld1952robinieErtrag} describes a certain incompatibility between black locust and birch, with an unfavorable interaction of both wood species on each other, so that both suffer under the presence of the other. A mixture with beech also seems somewhat problematic. In contrast, mixtures with sessile oak, linden, maple, and elm are mutually beneficial and even show very vigorous growth under the canopy of black locust. Establishing pure stands of black locust is not recommended. In pine forests, black locust can serve as fire protection wood, similar to birch.

At the planned site for black locust planting, sessile oak is able to prevent black locust regeneration under its canopy through shading, possibly also in combination with root competition.

In the end, it will depend on the site whether oak and black locust promote and complement each other or whether one suppresses the other.

I expect that black locust will be beneficial for oak and walnut, will make efficient use of the site even with wide spacing, can form an early nurse stand, will reach harvest maturity quickly, and will ultimately produce high-quality timber.

An overview of the planned planting scheme is shown in Fig.~\ref{fig:versuchsflaeche}.

%%%

The planting spacing for oak and walnut is very wide, and valuable oak timber will hardly be achievable. A dense oak planting seems too expensive to me.

The current planting aims to provide seed trees for the next generation in order to obtain a sufficiently dense regeneration.

Additionally, hornbeam is already being introduced as a serving tree species to support stem cleaning.

In addition to these, walnut, black walnut, and field elm are planted, which have germinated from seeds found along the roadside.

Roots of the mixed tree species are shown in Fig.~\ref{fig:wurzelAndere}.

Pedunculate oak and especially field elm had well-developed roots. The sessile oak shown was one of the better-developed ones. Unfortunately, some roots of the sessile oak were very similar to those of the pictured red oak and will probably have lower chances of survival. For these, it would have been better if I had left them at least one more year in the intermediate planting in the garden, with watering possibilities, although some already died there as well.

%%%

To test the varieties on other sites, tree rows are being established in gardens in Vienna–Laaerberg and at the foot of Schneeberg.

Although the forestry significance may be somewhat limited, the growth form can certainly be observed this way.

The interest in free-standing black locusts and avenue trees is also demonstrated by the yield table of \citet{fekete1931robinieErtragstafel}, which describes their growth behavior.

The site in Vienna is drier than the one in Ernstbrunn, which can be attributed on the one hand to the more permeable soil and on the other hand to the warmer urban climate. Hornbeams there regularly lose large amounts of their leaves during summer.

At the foot of Schneeberg at 650 m, cold and storms could cause difficulties, although black locusts can occasionally be found in surrounding gardens. Additionally, the trees are protected from trampling by livestock behind a sparse fence, but not from browsing by cows.

\subsection{Expected Findings}

Results regarding growth performance here must be limited to the height and diameter of individual trees. Ultimately, detailed information on the average annual increments related to area, depending on rotation period and stand density, as well as broken down by quality and diameter classes, would be desirable.

A comparison of different varieties under schematically identical treatment will not be meaningful, as straight-growing varieties often have smaller crowns \citep{bujtas1984robinie} and therefore require higher stem numbers \citep{keresztesi1974robinie} to exploit the growing space in the same way as widely spreading varieties.

Since varieties can differ significantly, there are specific treatment recommendations for some well-studied ones, such as Üllői in \citet{redei2020ulloi}.

It becomes significantly more complex with uneven-aged mixtures with temporally varying mixture proportions, which can usually only be exemplarily realized experimentally in trial plantations with few variants.

\subsection{Pruning at Planting}

When planting black locust, occasional shoot pruning is recommended. This is justified by the fact that during lifting of the young plants, roots may be damaged or lost, and the original contact between soil and root is no longer completely intact. Pruning is intended to rebalance the shifted shoot-to-root ratio.

Occasionally, pruning close to the ground, or if possible even slightly below ground level, is recommended.

\citet{bier1958robinie} recommends covering the cut surface with 1–2 mm of soil to prevent unnecessary water evaporation from the fresh wounds.

\citet[p.~78]{erteld1952robinieErtrag} recommends stump pruning in the form of a cut about hand height above the ground at planting, as the shoots are supposed to be stronger and more vigorous than the main shoots. In some cases, especially with weak and less vigorous plants, pruning should be done twice, once in the year of planting and again after 1 or 2 years.

\citet{bund1899robinie} recommends shortening the plants 5 to 10 cm above the root collar during repotting and cutting off damaged roots.

\citet{meginnis1940robinieRueckschnitt} investigated the effect of pruning. A pruning to about 20 cm at lifting or planting slightly increased survival chances. Pruning at ground level at planting reduced survival chances. Ground-level pruning in the following year showed the greatest height increment but could not match the total height of unpruned trees after 2 years:\\
Pruning to 0 cm height in the second year + height growth in the second year = 72 cm\\
versus: initial height + height growth in the first year + height growth in the second year = 92 cm.

Stumping lowers planting costs and can reduce planting shock, especially on dry sites with trees having unfavorable root-to-shoot ratios \citep{south1996toppruning,south1998toppruning,south2016toppruning}.

Originally, I intended to prune part of the plants immediately at planting to ground level, to prune another group one year after planting, and to leave a third group unpruned.

Based on experience from the first half-year with the three varieties I received in 2023/24, small plants (about under 1.2 m height) should remain in the nursery for another year to reduce their risk of failure. Well-developed roots, as shown in Figure~\ref{fig:wurzelRobinie}, are essential.

%%%

On more favorable sites, one-year-old seedlings might be the best choice, as recommended by \citet{ciuvat2022robinieRumaenien}. \citet[p.~51]{fekete1931robinieErtragstafel} recommends 2–4-year-old saplings. According to \citet{fuehrer2005robinie}, plants should not exceed 3 m in height at planting.

Handling larger plants is generally much more labor-intensive. Lifting them without damaging the roots becomes increasingly difficult. This leads to the correct but somewhat vague recommendation: \enquote{As large as necessary, as small as possible}.

Pruning the plants to about 80 cm total length when lifting them facilitated transport and can have positive effects on establishment, as the shoot-to-root ratio is shifted in favor of the roots.

This pruning, which is done well above ground level, is likely less favorable for the quality of the future stem. However, a close-to-ground pruning can compensate for this. The new shoots often grow over a meter long and are initially nearly free of branches.

The trees that I pruned close to the ground when planting and that subsequently sprouted again actually appear more vigorous than those that were pruned lightly. However, of the ones pruned at ground level, only half sprouted, whereas almost all of those pruned at about 80\,cm survived.

Only with Appalachia were there more losses. It was smaller at planting than the others and would have been better off spending another year in the acclimatization nursery.

For the other trees, depending on their development, I plan a close-to-ground pruning either in early spring of the following year, two years after planting, or if the quality is already satisfactory, not at all. In doing so, the influence of possibly present dense ground vegetation must be taken into account.

The trees for this quality-promoting pruning should not be too large to allow the pruning wound to be overgrown within one year.

Occasionally, multiple shoots develop after pruning, which are often desired in short-rotation coppice plantations. When producing high-quality timber, these shoots should be removed as early as possible during the growing season to prevent new sprouts and to direct the growth vigor to the remaining leader shoot.

In early summer, when the thorns are still soft, these excess shoots can be broken off by hand at the sprouting point.

Subjectively, it appears that the new shoots after pruning are more heavily covered with thorns.

In somewhat older trees, however, I observed that while the gradually disappearing thorns are still clearly visible on the trunk, the branches above approximately two meters are thornless.

When somewhat larger black locusts are coppiced, root suckers can also form in addition to stump shoots.

I did not perform any root pruning at planting.

\subsection{Treatment}

Root suckers can also be utilized in afforestation by keeping the initial 
stock number low. This makes the afforestation cost-effective, and the 
initially open spaces are later closed by the formation of root suckers 
\citep{larsen1935robinieWurzelbrut}.

Usually, regeneration originating from root suckers is high in stem number, 
and thus \citet{redei2012robinieVerj} recommend reducing the stem number to 
below 5000 trees/ha at an age of 3--6 years.

When coppice shoots and root suckers emerge after harvesting, I would 
recommend removing the coppice shoots. This would be one of the first 
possible tending measures in the form of negative selection.

Since coppice shoots generally have lower quality than root suckers, they 
can ideally be manually removed in early summer.

The negative selection does not need to be limited to coppice shoots but can 
include all \emph{pre-mature} poorly formed trees.

In the early phase, a reduction of black locust in favor of desired, very 
light-demanding mixed tree species such as Scots pine, larch, and birch is 
also advisable. For tree species with somewhat higher shade tolerance, such 
as oak, this does not seem necessary.

Once the stand has differentiated somewhat, positive selections can follow. 
In this process, particularly promising trees are freed from their strongest 
competitors to give them more growing space and thus direct growth 
specifically towards future valuable trees.

\citet[p.~47]{bluemke1955robinie} writes that in 25--30\,m tall black locust 
with a high HD value (ratio of tree height to diameter, i.e., thin tall 
trees), wind movement and the associated breaking off of branches create 1--2\,m 
wide \enquote{wind shafts} or light shafts between tree crowns, which allow much 
light to reach the forest floor or ground vegetation. In black locust with a 
low HD value, these gaps are at most 0.5\,m wide or the crowns may touch or 
even overlap.

With the planting distance of 16.5 feet (5\,m) recommended by 
\cite{jessup1791robinie}, the trees will have a low HD value.

At low stand densities, the remaining trees show greater diameter growth than 
at high stand densities. Height growth is influenced to a much lesser extent 
by stand density. However, especially deciduous trees often show lower height 
growth at low stand densities.

According to \citet[p.~55]{bluemke1955robinie}, completely free-standing and 
too closely growing black locust have the lowest height growth. According to 
\citet[p.~7]{roach1958robinie}, black locust responds to release with 
increased height growth. According to \citet{roberts1939robinieHoehenzuwachs}, 
there is a close relationship between soil depth (thickness of the A-horizon) 
and height growth.

Planting distances can also be tied to technical conditions. For example, in 
Hungary, row spacings of 1.2\,m and later 2.5\,m were used in afforestation. 
This change occurred around 1970 when the narrow Bulgarian tractor TL-30 was 
no longer available, and agricultural tractors were used instead 
\citep{keresztesi1988robinieLw}.

It would be interesting to know whether black locust varieties that show little 
tendency to grow towards light also react little to stand density in terms of 
height growth, similar to many conifer species.

At low stand densities, strong individual trees with large crowns and more 
woody stems develop, which leads to lower yield in the sawmill due to stem 
rejuvenation. For the greatest volume increment per area, there is an optimal 
stand density that changes with age and varies depending on site, tree species, 
and variety.

For free-standing valuable trees, pruning is recommended as long as natural 
branch shedding does not seem sufficient to obtain a high-quality, branch-free 
stem. At the same time, any forks forming close to the ground can be removed if there is reason to believe that they were caused by ground frost.

The recommended time for pruning is between late winter and the beginning of 
bud break.

The observation by \citet[pp.~69, 86]{erteld1952robinieErtrag} that black 
locust in early youth (8--10 years) does not like direct sunlight but is 
quite grateful for radiation protection can be confirmed on dry sites. For 
example, regeneration on dry sites is more likely to succeed at the northern 
edge or under slight shade than on the southern edge.

On the other hand, on cooler sites without drought stress, full sunlight is 
generally preferred. Also, slight shading can provide some protection against 
late frost.

\subsection{Costs}

An overview of the black locust used is given in Table~\ref{tab:preisuebersicht}.

%%%

Dr.~Zsolt Keserű kindly gifted me Üllői and Vacsi. For both, there was an invoice in forints among the documents, which was converted to euros using a factor of 408.

Plants from Hungary have a tax rate of 27\,\%, those from Austria 13\,\%. The shipping costs for 100 Turbo plants amounted to 25 + 27\,\% = 31.75~euros.

Appalachia had the size 30/50\,cm and was therefore cheaper than Nagybudméri or Ramocsaháza~2E with 50/80. For the others, there was no size differentiation for me.

For larger orders (over 500 or 1000 plants), lower prices are offered.

For plants priced around 1.20\,euros, exactly the specified quantities were delivered. For the cheaper ones, however, more than 10 extra pieces were often included.

From 100 pieces onwards, Turbo--Obelisk had a price of 7.5\,euros + 27\,\% = 9.525~euros. This price was also offered to me for a quantity of 25 pieces. Considering that during the same period one could pick out and individually purchase grafted fruit trees about 0.8\,m tall for 4.99~euros, and about 1.8\,m tall for 6.99~euros at grocery discounters, I did not make use of this option.

On the other hand, \citet{keresztesi1988robinieLw,redei2005robinieVermehrung} indicate that vegetative propagation by root cuttings is 5--8 times, by cuttings 20 times, and by micropropagation 100 times more expensive than generative propagation.

Micropropagation is particularly useful in the initial stage to quickly produce a large number of plants of a selected clone with genetic purity, while for the subsequent large-scale propagation, root cuttings should preferably be used for cost reasons.

All costs were privately financed by me and all work was done during leisure time. As a result, cost-intensive methods (mulching on a large scale, irrigation, fencing, large-scale mechanical mowing, \dots) as well as disproportionately expensive planting material were not used.

On low-growth sites, it is questionable anyway whether cost-intensive methods are economically justified.

The trees were protected against browsing by painter’s tape, and different provenances were marked with short strips of differently colored insulating tape.

\subsection{Planting Procedure}

\citet{ciuvat2022robinieRumaenien} specify a pit depth of 40\,cm. \citet[pp.~164--165, 173]{vadas1911robinie} recommend planting the trees as deep as in the nursery. On sandy or windy sites, they advise planting 5--10\,cm deeper, usually about 40\,cm deep, so that the roots are not exposed by sand drift and the trees are stable enough.

The planting was carried out by me using an ordinary spade, with the planting hole mostly dug to one and a half spade depths (approx. 35\,cm). This is somewhat deeper than in the nursery, in the hope that water available to the plants may be provided longer in the deeper soil layers during dry periods.

None of the black locust trees were planted immediately in the forest after purchase; instead, they spent the first year in the acclimatization nursery, where they were watered during extended dry spells.

Some had already been shortened to an overall length of about 50\,cm at purchase, so that after planting they protruded about 25\,cm above the ground. Those that had not yet been shortened were also cut back to about 25\,cm after planting. By the end of August, the trees had reached a height of about 1.0--1.2\,m, with some even reaching 1.5\,m (Fig.~\ref{fig:robinieEinJahr}). No mortality occurred during the first year in the acclimatization nursery.

%%%

The plants were transported by train, bus, and bicycle (Fig.~\ref{fig:fahrradPflanzung}).

%%%

For site preparation, undesirable understory was mainly cut back with a billhook, and thicker stems were cut to the stump using a handsaw. If needed, freeing the young plants is done using a sickle, with the painter’s tape used for browsing protection being helpful in locating the plants.

\section*{Acknowledgments and Outlook}
\addcontentsline{toc}{section}{Acknowledgments and Outlook}

Mostly freely accessible literature was used. I would like to thank the libraries, institutes, publishers, authors, and sponsors who make this possible and have also digitized and freely provided many older documents.

Many of the works I cited were written in languages I am not familiar with. Freely accessible online translation tools helped to access these sources. In particular, the English translation of this text was largely created using machine translation services.

Countless other aids were used in the preparation of this work in the form of text editors, typesetting systems, reference management, literature search, spell checking, writing assistants, image editing, drawing programs, programming languages, operating systems, \dots all of which were exclusively freely available. I would like to thank all those who created and made these valuable tools accessible.

I would like to sincerely thank Dr.~Zsolt Keserű, Dr.~Gyula Kovács, Dr.~Attila Borovics, and Dr.~Attila Benke for their support in providing new black locust varieties.

If anyone is interested in establishing comparative plantations with black locust or other tree species, or has already carried out such plantations, I would be happy to exchange ideas.
