\section{Templates}

Um nicht den gleichen Code für verschiedene Datentypen schreiben zu müssen, wird
der Datentyp als Parameter übergeben (generische Programmierung).

\subsection{Function template}

Definiert eine Funktionsfamilie.

\begin{lstlisting}[language=C++]
#include <iostream>
#include <typeinfo>

int minI(const int& lhs, const int& rhs) {
  return lhs < rhs ? lhs : rhs;
}

auto minA(const auto& lhs, const auto& rhs) {
  return lhs < rhs ? lhs : rhs;
}

template<typename T>
T minT(const T& lhs, const T& rhs) {
  return lhs < rhs ? lhs : rhs;
}

int main() {
  std::cout << minI(3, 7) << '\n';    // 3
  std::cout << minI(3.6, 7) << '\n';  // 3
  std::cout << minA(3, 7) << '\n';    // 3
  std::cout << minA(3.6, 7) << '\n';  // 3.6
  std::cout << minT(3, 7) << '\n';    // 3
//std::cout << minT(3.6, 7) << '\n';  // Error: double int
  std::cout << minT<double>(3.6, 7) << '\n';  // 3.6
  std::cout << minT<int>(3.6, 7) << '\n';     // 3
  std::cout << typeid(minA(3, 7)).name() << '\n';    // int
  std::cout << typeid(minA(3.6, 7)).name() << '\n';  // double
  std::cout << typeid(minA(3, 7.2)).name() << '\n';  // double
}
\end{lstlisting}

Funktionen können auf bestimmte Typen spezialisiert werden.

\begin{lstlisting}[language=C++]
#include <iostream>

template<typename T>
void f(const T& x) {
  std::cout << "T: " << x << '\n';
}

template<>  // Spezialisierung auf f(int)
void f<int>(const int& x) {
  std::cout << "Int: " << x << '\n';
}

int main() {
  f('a');        // T: a
  f(1);          // Int: 1
  f<int>('a');   // Int: 97
//void *pf = f;  // Error: Overloaded Function
  void (*pfi)(const int&) = f;
  pfi(1);        // Int: 1
  pfi('a');      // Int: 97
  void (*pfc)(const char&) = f;
  pfc(98);       // Int: b
  pfc('a');      // Int: a
}
\end{lstlisting}

\subsection{Class template}

Definiert eine Klassenfamilie.

\begin{lstlisting}[language=C++]
#include <iostream>

template<class T>
struct S {
  T x;
  S(T x) : x(x) {};
  void f() {std::cout << x << '\n';}
};

int main() {
  S<int> i{0};
  i.f();  // 0
  S j{1};
  j.f();  // 1
  S d{4.2};
  d.f();  // 4.2
}
\end{lstlisting}

%Variable template
%Template parameters and arguments
%Class member template
%Template argument deduction: function – class
%Explicit specialization
%Partial specialization
%Parameter packs
%sizeof...
%Fold expressions
%Dependent names
%SFINAE
%Constraints and concepts
%Requires expression