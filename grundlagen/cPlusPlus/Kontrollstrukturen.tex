\section{Kontrollstrukturen}

\subsection{Auswahlanweisungen -- Selection}

\subsubsection{if else}

Ausführen von Code, wenn Bedingung erfüllt ist.

\begin{lstlisting}[language=C++]
int i{1};

if (i > 2)
  if (i < 4)
    i = {0};
  else  // gehört zu if (i < 4)
    i = {1};
else  // gehört zu if (i > 4)
  i = {2};

if (i > 2)  // das Gleiche nur mit Klammern
{
  if (i < 4)
  {
    i = {0};
  }
  else
  {
    i = {1};
  }
}
else
{
  i = {2};
}

if (int ii = i*i; i > 0)  // mit Initialisierer
  i -= ii;
else {
  // int ii;  // error: redeclaration of ii
  i += ii;
}
// ++ii;  // ii hier nicht bekannt

// typedef declaration
if (typedef int Foo; true) { (void)Foo{}; }

// alias declaration
if (using Foo = int; true) { (void)Foo{}; }

constexpr bool TEST = true;
if constexpr (TEST) {  //Auswertung wenn Compiliert wird
  if (i == 0) i = {1};
}
i = {10 / i};
\end{lstlisting}

Mit \lstinline|if consteval| testen ob constexpr-Funktion zur Compilationszeit
ausgewertet wird. Hier müssen Blöcke (compound \lstinline|{ }|) verwendet
werden.

\begin{lstlisting}[language=C++]
#include <iostream>

constexpr int f(int x) noexcept {
  if consteval { return 0; } else { return x; }
}

int main() {
  const int i { f(2) };  // Auswertung zur Compilationszeit
  int j{2};             // nicht konstant
  const int k { f(j) };  // Auswertung zur Laufzeit
  std::cout << i << " " << k << "\n";
}
\end{lstlisting}

\subsubsection{?: -- dreiteilige Bedingung / Ternary conditional}

\lstinline|E1 ? E2 : E3| Wenn E1 true, dann E2 sonst E3.

\begin{lstlisting}[language=C++]
int n = {1 > 2 ? 7 : 8};  // 1 > 2 ist false, so n = 8
int m{0};
(n == m ? n : m) = {9};  // n == m ist false, so m = 9
\end{lstlisting}

\subsubsection{switch}

Springe, in Abhängigkeit eines Wertes, an eine Programmzeile. Der Wert muss
\lstinline|int| oder \lstinline|enum| oder implizit in int umwandelbar
sein.

\begin{lstlisting}[language=C++]
#include <iostream>

int main()
{
  const int k{9};
  switch (int i{k / 2}; i)  // i nur in switch bekannt
  // switch (k)  // Variante ohne Initialisierer
  {
    case 1:
      std::cout << '1';
      break;
    [[unlikely]] case 2:
      std::cout << '2';
      [[fallthrough]]; // nicht nötig, Hinweise das beabsichtigt
    [[likely]] case 3:
      std::cout << '3';
      break;
    case 4:
      //int j{0};  // Error weil man in den scope von
                    // initialisiertem i springen kann
      int j;
      j = {0};
      break;
    case 5:
    {
      int j{0};
      break;
    }
    case 6:
      // int i;  // error: redeclaration of i
    default:
      std::cout << "Default\n";
  }
}
\end{lstlisting}

\subsection{Schleifen, Wiederholungen -- Iteration}
\subsubsection{for}

Schleife mit Startwert, Bedingung (wird vor jeder Iteration geprüft) und
iteration-expression (wird nach jedem loop ausgeführt).

\begin{lstlisting}[language=C++]
#include <iostream>
#include <tuple>
#include <vector>

int main()
{
  double s{0.};
  int j{1};  // Wird in Schleife verdeckt
  for (int i{1}, j{9}; i < 10; ++i, --j)
  {
    double x{0.5};  // constructor and destructor wird bei
                    // jedem loop aufgerufen
    //double i{0.};  // Error: redeclaration von i
    s += i * j * x;
  }
  //--i;  // Error: Out of scope
  std::cout << s << "\n";

  char cstr[]{"Hallo"};
  for (int n = 0; char c = cstr[n]; ++n) std::cout << c;

  s = {0.};
  // Verschiedene Typen mittels structured binding declaration
  for (auto [i, f] = std::tuple{int{1}, float{0.1}}; i < 10 ;
       ++i, f += .1)
  {
    s += i * f;
    //int i;  // Error: redeclaration von i
  }
  std::cout << s << "\n";

  s = {0.};
  {
    int i{1};
    float f{0.1};
    for (; i < 10 ; ++i, f += .1) {
      s += i * f;
      int i;
    }
  }
  std::cout << s << "\n";
}
\end{lstlisting}

Float Zähler (counter) vermeiden.

\begin{lstlisting}[language=C++]
#include <iostream>
#include <cmath>

int main()
{
  int j{0};
  for (float x{0.}; x <= 1.; x += .01)
    std::cout << j++ << ":" << x << " ";
  std::cout << "\n";
  float start{0.}, end{1.}, inc{.01};
  long n{lround((end - start) / inc)};
  for (long i{0}; i <= n; ++i)
    std::cout << i << ":" << (start + (inc * i)) << " ";
  std::cout << "\n";
  for (long i{0}; i <= 100; ++i)  //Empfohlen
    std::cout << i << ":" << i/100.f << " ";
}
\end{lstlisting}

\subsubsection{range-for}

Schleife über einen Bereich (range).

\begin{lstlisting}[language=C++]
#include <iostream>
#include <vector>
#include <algorithm>
#include <ranges>

void o(const std::vector<int>& v, const char* s = {""}) {
  std::cout << s;
  for (const auto& i : v) std::cout << i << ' ';
  std::cout << "\n";
}

int main()
{
  std::vector<int> v = {2, 0, 1};

  // Index based. Geht nur bei sequential random access
  for (std::size_t i{0}; i < v.size(); ++i) ++v[i];
  o(v, "0: ");  // 3 1 2

  // Iterator based
  for (auto it{v.begin()}; it != v.end(); ++it) ++*it;
  o(v, "1: ");  // 4 2 3

  for (auto& e : v) ++e;  // Über Reference
  o(v, "2: ");  // 5 3 4

  for (auto e : v) ++e;  // Kopie
  o(v, "3: ");  // 5 3 4

  for (auto&& e : v) ++e; // forwarding reference
  o(v, "4: ");  // 6 4 5

  std::ranges::for_each(v, [](auto& e) { ++e; });
  o(v, "5: ");  // 7 5 6

  std::for_each(v.begin(), v.end(), [](auto& e) { ++e; });
  o(v, "6: ");  // 8 6 7

  for (int i{0}; const auto& e : v) // Mit init-statement
    std::cout << i++ << ':' << e << ' ';
  std::cout << '\n';

  for (const auto& [i, e] : std::views::enumerate(v))
    std::cout << i << ':' << e << ' ';
  std::cout << '\n';

  for (int i : {3, 1, 2}) std::cout << i; // braced-init-list
  std::cout << '\n';

  // Variable nicht verwendet
  for ([[maybe_unused]] int i : v) std::cout << 'X';
}
\end{lstlisting}

\subsubsection{while}

Führt eine Anweisung so lange aus, bis die Bedingung \lstinline|false| wird.
Geprüft wird \emph{vor} jeder Iteration.

\begin{lstlisting}[language=C++]
#include <iostream>

int main()
{
  int i{0};
  while (i < 3) ++i;
  std::cout << i << '\n';

  while (i < 9)
  {
    std::cout << i << '\n';
    i += 2;
  }

  char cstr[]{"Hallo"};
  i = {0};
  while (char c{cstr[i++]})
    std::cout << c;
}
\end{lstlisting}

\subsubsection{do-while}

Führt eine Anweisung so lange aus, bis die Bedingung \lstinline|false| wird.
Geprüft wird \emph{nach} jeder Iteration.

\begin{lstlisting}[language=C++]
#include <iostream>

int main()
{
  int i{0};
  do
  {
    i += 2;
    std::cout << i << '\n';
  }
  while (i < 9);
}
\end{lstlisting}

\subsection{Sprunganweisungen -- Jump}
\subsubsection{continue}

Überspringt den verbleibenden Teil des umschließenden \lstinline|for| oder
\lstinline|while| Schleifenkörpers.

\begin{lstlisting}[language=C++]
#include <iostream>

int main()
{
  for (int i{0}; i < 3; ++i)
  {
    for (int j{0}; j < 3; ++j)
    {
      if (j == 1) continue;  // wirkt bei j for loop
      std::cout << i << ' ' << j << '\n';
    }
  }
}
\end{lstlisting}

\subsubsection{break}

Beendet umschließendes \lstinline|for|, \lstinline|while| oder
\lstinline|switch|.

\begin{lstlisting}[language=C++]
  #include <iostream>

int main()
{
  for (int i{0}; i < 3; ++i)
  {
    for (int j{0}; j < 3; ++j)
    {
      if (j == 1) break;  // wirkt bei j for loop
      std::cout << i << ' ' << j << '\n';
    }
  }
}
\end{lstlisting}

\subsubsection{goto}

Setzt Programm an anderer Stelle fort.

\begin{lstlisting}[language=C++]
#include <iostream>

struct nt {  // non-trivial destructor
  ~nt() { std::cout << "X\n";}
};

int main()
{
  for (int i{0}; i < 3; ++i)
  {
    for (int j{0}; j < 3; ++j)
    {
      if (j == 1) goto endloop; // Ausstieg aus beiden for loops
      std::cout << i << ' ' << j << '\n';
    }
  }
endloop:  // Sprungmarke label
  goto label2; // Sprung in den scope von n
//int n{0};  //error: jump bypasses variable initialization
  [[maybe_unused]] int n; // no initializer
  n = {7};
//nt o;  // error: bypas variable with non-trivial destructor
label2:
  std::cout << n;  // n hat Zufallswert
}
\end{lstlisting}

\subsubsection{return}

Beendet Funktion und gibt Wert (falls vorhanden) an den Aufrufer zurück.

\begin{lstlisting}[language=C++]
#include <iostream>
#include <utility>

void fa(int i) {
  if (i == 1) return;
  std::cout << "fa("<< i << ")\n";
} // impliziertes return;

int fb(int i) {
  if (i > 4) return 4;
  std::cout << "fb(" << i << ")\n";
  return 2; }

std::pair<int, float> fc(int i, float x) {
  return {i, x}; }

int main()
{
  fa(0);  // Gibt fa(0) aus
  fa(1);  // Nichts
  int i{fb(5)};  // i == 4
  i = fb(i);     // gibt fb(4) aus, i == 2
  std::cout << fc(0, 1.).first << " " << fc(2, 3.).second; //0 3
}
\end{lstlisting}