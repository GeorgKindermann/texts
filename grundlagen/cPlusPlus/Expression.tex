\section{Ausdruck -- Expression}

Operatoren (z.\,B.\ +,-,*,/,\dots) führen etwas mit Operanden aus. Bei der
Auswertung (evaluation) kann ein Ergebnis (z.\,B.\ 4 bei \lstinline|2+2|) und
oder Nebenwirkungen (side-effects) (z.\,B.\ gibt
\lstinline|std::printf("\%d", 4)| 4 aus) entstehen.

\subsection{Wertkategorien -- Value categories}

\lstinline|lvalue = prvalue;|

\begin{description}
  \item[lvalue] Steht auf der linken Seite einer Zuweisung.
  \begin{itemize}
    \item Adresse eines Ausdrucks kann übernommen werden.
    \item Typ eines Ausdrucks ist eine L-Wert-Referenz (z.\,B.\ \lstinline|int&|
    oder \lstinline|const int&|)
  \end{itemize}
  \item[prvalue] Steht auf der rechten Seite einer Zuweisung. R-Werte
  entsprechen temporären Objekten, wie sie beispielsweise von Funktionen
  zurückgegeben oder durch implizite Typkonvertierungen erstellt werden. Die
  meisten Literale (z.\,B.\ \lstinline|7| oder \lstinline|7.8|) sind ebenfalls
  R-Werte.
  \item[xvalue] Verschwindende (expiring) lvalue.
  \item[glvalue] lvalue oder xvalue
  \item[rvalue] prvalue oder xvalue
\end{description}

\subsection{Operator}

Diese werden nach definierten Reihenfolgen und Richtungen evaluiert.

\begin{description}
  \item[assignment] \lstinline|a = b| :
    \lstinline"= += -= *= /= \%= \&= |= ^= <<= >>="
  \item[increment / decrement] \lstinline|++a --a a++ a--|
  \item[arithmetic]
    \lstinline|+a| \lstinline|-a| \lstinline|~a|;
    \lstinline|a + b| :
    \lstinline"+ - * / \% & | ^ << >>"
  \item[logical]
    \lstinline|!a|
    \lstinline|a && b|
    \lstinline"a || b"
  \item[comparison] \lstinline|a == b| :
    \lstinline|== != < > <= >= <=>|
  \item[member access] \lstinline|a[b] a[...] *a &a a->b a.b a->*b a.*b|
  \item[function call] \lstinline|a(a1, a2)|
  \item[comma] \lstinline|a, b|
  \item[conditional] \lstinline|a ? b : c|
  \item[Conversions] \lstinline|const_cast| \lstinline|static_cast|
    \lstinline|dynamic_cast| \lstinline|reinterpret_cast|
  \item[Memory allocation] \lstinline|new delete|
  \item[Other] \lstinline|sizeof alignof typeid throw-expression|
\end{description}

\subsection{Constant}
\label{sec:Ausdruck:constant}

\begin{description}
  \item[const] Kann nicht geändert werden.
  \item[constexpr] Wert \emph{kann} beim Kompilieren vorliegen.
  \item[consteval] Wert \emph{muss} beim Kompilieren vorliegen. Nur für Funktionen.
\end{description}

\begin{lstlisting}[language=C++]
#include <array>

int f0 ( int a, int b ) { return a + b; }
constexpr int f1 ( int a, int b ) { return a + b; }
consteval int f2 ( int a, int b ) { return a + b; }

int main() {
  int n{2};
//std::array<int, n> a0;  // Error: Argument not const expr
  const int cn{2};
  std::array<int, cn> a1;
  constexpr int cn2{2};
  std::array<int, cn2> a2;
//std::array<int, f0(1, 1)> a3; // Error: non-constexpr function
  std::array<int, f1(1, 1)> a4;
//std::array<int, f1(n, n)> a5; // Error: nicht const
  std::array<int, f1(cn, cn)> a6;
  f1(n, n);
//f2(n, n);  //Error: nicht constant
  f2(cn, cn);
  ++n;
//++cn;   // Error: declared const
//++cn2;  // Error: declared const
}
\end{lstlisting}

\subsection{Operator overloading}

Operatoren (z.\,B.\ +) für benutzerdefinierte Typen definieren. Overloaded operators sind Funktionen mit speziellen Funktionsnamen.

\begin{lstlisting}[language=C++]
#include <iostream>

struct Foo {
  int a;
  double b;
  Foo& operator++()  // prefix increment
  {
    ++a;
    ++b;
    return *this; // return new value by reference
  }
};

std::ostream& operator<<(std::ostream& out, Foo const& f) {
  return out << f.a << " " << f.b;
}

int main() {
  Foo x = {0, 1.5};
  std::cout << x << "\n";  // 0 1.5
  ++x;
  std::cout << x << "\n";  // 1 2.5
}
\end{lstlisting}

\subsection{Conversions}

\subsubsection{Implicit}

\begin{lstlisting}[language=C++]
  float f{ 0 };  // int nach float
  f = 1;         // int nach float
  if (2) {}      // int nach bool
//int i{0.};     // Error: float nach int narrowing conversion
  int i = 0.;    // float nach int
//i = { 1. };    // Error: float nach int narrowing conversion
  i = 1.;        // float nach int
\end{lstlisting}

\subsubsection{Explicit}

\begin{lstlisting}[language=C++]
int i;
i = 0.5;  // Implicit
i = (int) 1.5;
i = int(2.5);
//i = int{3.5};  // Error: narrowing conversion
i = auto(2.5);
i = auto{3.5};
\end{lstlisting}

\subsubsection{User-defined}

\begin{lstlisting}[language=C++]
#include <iostream>

struct foo {
  int i{2};
  // implicit conversion
  operator int() const { return 1; }
  // explicit conversion
  explicit operator int*() const { return const_cast<int*>(&i);}
};

int main() {
  foo x;
  int i = x;
  std::cout << i << ' ' << (int)x << ' ' << int(x) << ' ' <<
    int{x} << ' ' << static_cast<int>(x) << '\n';  // immer 1
//int* j = x;  // Error: no implicit conversion
  int* j = (int*)x;
  std::cout << *j << ' ' << *((int*)x) << ' ' <<
    *static_cast<int*>(x) << '\n';  // immer 2
}
\end{lstlisting}

\subsubsection{Usual arithmetic conversions}

\begin{lstlisting}[language=C++]
#include <iostream>
#include <typeinfo>

int main() {
  int i = 2;
  long l = 5l;
  auto n = i + l;
  std::cout << typeid(n).name() << '\n';  // l
  float x = 1.f;
  double y = 2.;
  auto z = x + y;
  std::cout << typeid(z).name() << '\n';  // d
}
\end{lstlisting}

\subsubsection{static\_cast}

\lstinline|static_cast<int>(2.)| macht das gleiche wie \lstinline|int(2.)| oder
\lstinline|(int)2.|, ist aber bei einer Suche leichter zu finden.

\begin{lstlisting}[language=C++]
long l = 5l;
//int i{l};  // narrowing conversion
int i{static_cast<int>(l)};
int j{int(l)};
int k{(int)l};
\end{lstlisting}

\subsubsection{dynamic\_cast}

Bei virtual Vererbung für Zeiger oder Referenz auf eine Klasse.

\begin{lstlisting}[language=C++]
#include <iostream>
struct Animal   { virtual ~Animal() = default; int i{0}; };
struct Creature { virtual ~Creature() = default;
                  int i{1}; int j{2}; };
struct Bird : public Animal, Creature { };

int main() {
  Bird *bird = new Bird();
  Creature *creature = dynamic_cast<Creature*>(bird);
  Animal *animal = dynamic_cast<Animal*>(creature);
  Creature *creature1 = new Creature();
//Bird *bird1 = dynamic_cast<Animal*>(creature1);  // invalid conversion
  Animal *animal1 = dynamic_cast<Animal*>(creature1);
//std::cout << bird->i << '\n';  // Error: ambiguous
  std::cout << bird->Animal::i << '\n';
  std::cout << bird->Creature::i << ' ' << bird->j << '\n';
  std::cout << creature->i << ' ' << creature->j << '\n';
  std::cout << animal->i /*<< ' ' << animal->j*/ << '\n';
  std::cout << creature1->i << ' ' << creature1->j << '\n';
//std::cout << animal1->i << '\n'; //Seg fault
  delete bird;
  delete creature1;
}
\end{lstlisting}

\subsubsection{const\_cast}
\label{sec:Conversion:constCast}

Erzeugt eine nicht konstante Referenz oder Zeiger auf ein konstantes Objekt, das
damit verändert werden kann und damit undefiniertes Verhalten verursachen kann.

\begin{lstlisting}[language=C++]
#include <iostream>

int main() {
  int i = 0;
  const int& rci = i;
//rci = 1;  /Error: read-only reference
  const_cast<int&>(rci) = 2;  // OK
  std::cout << i << ' ' << rci << '\n';  // 2 2

  const int j = 3;
  int* pj = const_cast<int*>(&j);
  *pj = 4;      // undefined behavior
  std::cout << j << ' ' << *pj << '\n';  // ? ?
}
\end{lstlisting}

\subsubsection{reinterpret\_cast}

Konvertiert zwischen Typen durch Neuinterpretation des zugrunde liegenden
Bitmusters.

\begin{lstlisting}[language=C++]
#include <iostream>

int main() {
  signed char i = -1;
  std::cout << +i << ' ' <<                       // -1
   +reinterpret_cast<unsigned char&>(i) << '\n';  // 255
}
\end{lstlisting}

\subsection{Literals}

Im Quellcode eingebettete konstante Werte.
\begin{description}
  \item[boolean] \lstinline|true|, \lstinline|false|
  \item[integer] ~
  \begin{description}
    \item[integer-suffix] ~
    \begin{description}
      \item[u U] Unsigned \lstinline|0u|
      \item[l L] Long \lstinline|0l|
      \item[ll LL] Long long \lstinline|0ll|
      \item[z Z] size \lstinline|std::size_t i{0uz};|
    \end{description}
    \item[decimal-literal] \lstinline|42|
    \item[octal-literal] \lstinline|052|
    \item[hex-literal] \lstinline|0x2a| \lstinline|0X2A|
    \item[binary-literal] \lstinline|0b101010| \lstinline|0B101010|
  \end{description}
  \item[floating] ~
  \begin{description}
    \item[3.1f] float
    \item[3.1] double
    \item[3.1l] long double
    \item[3.1f16] float16; f32, f64, f128, bf16
    \item[4e2] double
    \item[3.14'15'92] double, single quotes ignored
  \end{description}
  \item[character] ~
  \begin{description}
    \item['c'] Gewöhnliches Zeichenliteral
    \item[u8'c'] UTF8
    \item[u'c'] UTF16
    \item[U'c'] UTF32
    \item[L'c'] Wide character
    \item['char'] Literal mit mehreren Zeichen
  \end{description}
  \item[string] ~
  \begin{description}
    \item[\dq\dq] Gewöhnliches Stringliteral
    \item[\dq{}abc\dq] Gewöhnliches Stringliteral
    \item[R\dq{}(abc)\dq] Raw string literals
    \item[R\dq{}any(abc)any\dq] Raw string literals
    \item[u8\dq{}abc\dq] UTF8
    \item[u\dq{}abc\dq] UTF16
    \item[U\dq{}abc\dq] UFT32
    \item[L\dq{}abc\dq] Wide
  \end{description}
  \item[Escape sequences] Für spezielle Zeichen
  \begin{description}
    \item[\textbackslash{}'] single quote
    \item[\textbackslash{}"] double quote
    \item[\textbackslash{}?] question mark
    \item[\textbackslash{}\textbackslash{}] backslash
    \item[\textbackslash{}a] audible bell
    \item[\textbackslash{}b] backspace
    \item[\textbackslash{}f] form feed - new page
    \item[\textbackslash{}n] line feed - new line
    \item[\textbackslash{}r] carriage return
    \item[\textbackslash{}t] horizontal tab
    \item[\textbackslash{}v] vertical tab
    \item[\textbackslash{}nnn] Octal code 1-3 stellig
    \item[\textbackslash{}o\{n\dots\}] Octal code
    \item[\textbackslash{}xn\dots] Hex Code
    \item[\textbackslash{}x\{n\dots\}] Hex Code
    \item[\textbackslash{}unnnn] Unicode Wert 4 stellig
    \item[\textbackslash{}Unnnnnnnn] Unicode Wert 8 stellig
    \item[\textbackslash{}u\{n\dots\}] Unicode Wert
    \item[\textbackslash{}N\{Name\}] Unicode Name
    \item[\textbackslash{}c] Conditional escape sequence
  \end{description}
  \item[nullptr] Nullpointer literal
  \item[user-defined] Literals können auch selber definiert werden
\end{description}