\section{Deklaration}

\begin{description}
  \item[Deklaration] Bekanntgabe von Namen
  \item[Definition] Deklaration die ausreicht um sie verwenden zu können
\end{description}

\begin{lstlisting}[language=C++]
int f(int); // deklariert Funktion f, aber definiert sie nicht
int g(int x) {  // definert Funktion g
  return x + 1;
}
\end{lstlisting}

Eine Deklaration kann keine bereits bestehende Deklaration im selben Namensraum
(Namepace) überschreiben (Conflicting declarations).

\begin{lstlisting}[language=C++]
int i;
//int i; // Error:  redeclaration
{
  int i;  // OK: Im Block
}
int f(float x);
int f(int x);   // OK Function Overloading
\end{lstlisting}

\subsection{Storage class specifiers}

\begin{description}
  \item[~] Nichts: automatische Speicherdauer. Dauer: Vom Beginn bis zum Ende des Blocks.
  \item[static] Nur bei Deklaration von Objekten, Funktionen oder anonymen
  unions. Greift bei jeder Instanz auf die gleiche Variable zurück. Dauer: Vom Beginn bis zum Ende des Programms.

\begin{lstlisting}[language=C++]
#include <iostream>

class A {
  static int N;
  public:
  A() { ++N; }
  int n() {return N;}
};
int A::N{0};  // Initialisierung von static N

int main() {
  A x;
  std::cout << x.n() << '\n';  // 1
  A y;
  std::cout << x.n() << ' ' << y.n() << '\n';  // 2 2
}
\end{lstlisting}

  \item[extern] Für Zugriff auf variablen oder Funktionen, die in anderen souce-Dateien definiert sind.

\begin{lstlisting}[language=C++]
//Datei1
int a{1};
const int b{2};  // Gilt nur in Datei1
extern const int c{3};
\end{lstlisting}
\begin{lstlisting}[language=C++]
//Datei2
#include <iostream>
extern int a;
extern const int b;  // Kommt nicht von Datei1
extern const int c;

int main() {
  std::cout << a << '\n';  // 1
//std::cout << b << '\n';  // undefined reference to `b'
  std::cout << c << '\n';  // 3
}
\end{lstlisting}

  \item[thread\_local] Jeder thread hat eine eigen Kopie der Variable. Dauer: Vom Beginn bis zum Ende des threads.
  \item[mutable] Erlaubt die Veränderung eines Klassenmitglieds, wenn das Objekt als const deklariert ist.
\end{description}

\subsection{Translation-unit-local}

Verhindert, dass lokale Instanzen in anderen Dateien verwendet werden

\begin{lstlisting}[language=C++]
//Datei1
int a{1};  // Kann mit extern in anderem Source verwendet werden
static int b{2};  // Lokal verfügbar
namespace {
  int c{3};      // Lokal verfügbar
}
\end{lstlisting}
\begin{lstlisting}[language=C++]
//Datei2
#include <iostream>
extern int a;
extern int b;  // Kommt nicht von Datei1
extern int c;  // Kommt nicht von Datei1

int main() {
  std::cout << a << '\n';  // 1
//std::cout << b << '\n';  // undefined reference to `b'
//std::cout << c << '\n';  // undefined reference to `c'
}
\end{lstlisting}

\subsection{Language linkage}

\lstinline|extern| zur Verknüpfung von Funktionen, die in einer anderen
Programmiersprache geschrieben wurden.

\begin{lstlisting}[language=fortran]
! Fortran compiliert z.B.: gfortran -c fortran.f90
integer FUNCTION fun1(i,j)
    fun1 = i + j
END FUNCTION fun1

MODULE fort
contains
integer FUNCTION fun2(i,j)
    fun2 = i - j
END FUNCTION fun2
END MODULE fort
\end{lstlisting}

\begin{lstlisting}[language=C++]
// C++ compiliert z.B.: g++ c.cc fortran.o
#include <iostream>

extern"C" {
  int fun1_(int*, int*);
  int __fort_MOD_fun2(int*, int*);
}

int main() {
  int i{1};
  std::cout << fun1_(&i, &i) << ' ' <<  // 2
    __fort_MOD_fun2(&i, &i) << '\n';    // 0
}
\end{lstlisting}

\subsection{Namespace declaration}

Zur Vermeidung von Namenskonflikten.
Bezeichner im Unnamed namespace sind nur in der Übersetzungseinheit sichtbar.

\begin{lstlisting}[language=C++]
#include <iostream>

int i{0};  // global namespace
int j{1};

namespace {  // Unnamed namespace
  int i{2};  // Nicht erreichbar wegen globalem i
  int k{3};
}

namespace A {
  int i{4};
  int n{12};
  namespace B {
    int i{5};
    extern int j;
    extern int k;
  }
//int B::i{6};  // Error: Redefinition von i
  int B::j{7};
//int B::l{8};  // Error: B::l nicht deklariert
  inline namespace C {  // In C und A
    int m{9};
    int n{13};  // Verdeck A::n
  }
  namespace {  // In A
    int o{14};
  }
}
namespace D {
//int A::B::k{10};  // Error: Does not enclose B
}
int A::B::k{11};

namespace A::B {  // nested namespace definition
  int p{15};
}

namespace A::inline D {
  int q{16};
}

namespace X = A::B;  // Namespace alias

int main() {
//std::cout << i << '\n';  // ambiguous
  std::cout << ::i << '\n';  // 0
  std::cout << j << ' ' << ::j << '\n';  // 1 1
  std::cout << k << ' ' << ::k << '\n';  // 3 3
  std::cout << A::i << ' ' << ::A::i << '\n';  // 4 4
  std::cout << A::B::i << '\n';  // 5
  std::cout << A::B::j << '\n';  // 7
  std::cout << A::B::k << '\n';  // 11
  std::cout << A::m << ' ' << A::C::m << '\n';  // 9 9
//std::cout << A::n << '\n';  // ambiguous
  std::cout << A::C::n << '\n';  // 13
  std::cout << A::o << '\n';  // 14
  std::cout << A::B::p << '\n';  // 15
  std::cout << A::q << ' ' << A::D::q << '\n';  // 16 16
  std::cout <<X::i <<' ' <<X::j <<' ' <<X::k <<'\n'; // 5 7 11
}
\end{lstlisting}

Mit \lstinline|using| können Inhalte von Namensräume sichtbar gemacht werden. In
Header Dateien \lstinline|using namespace| vermeiden, da jeder der den Heder
verwendet auch diesen namespace mitverwendet.

\begin{lstlisting}[language=C++]
#include <iostream>

int i{0};

namespace A {
  int i{1};
  int j{2};
}

namespace B {
  int i{3};
  int k{4};
}

namespace C {
//using ::i, A::i;  // Error: Name conflict
  using ::i, A::j; // Werden in C sichtbar
}

int main() {
  using namespace A;
  using namespace B;
//std::cout << i << '\n';  // ambiguous
  std::cout << A::i << ' ' << B::i << '\n';  // 1 3
  std::cout << j << ' ' << k << '\n';  // 2 4
  std::cout << C::i << ' ' << C::j << '\n';  // 0 2
}
\end{lstlisting}

\subsection{References}

Mit \lstinline|&| (lvalue) bzw.\ \lstinline|&&| (rvalue) wird ein zusätzlicher
Namen für ein bereits vorhandenes Objekt vergeben.

\begin{lstlisting}[language=C++]
#include <iostream>

void add(int& x) { ++x; }  // Direkt Zugriff auf übergebenes x
int& set(int& x) { return x; }  // lvalue expression
int& bad() {  // Dangling reference
  int i{0};
  return i;  // return reference to local variable
}  // Aufruf Destruktor von i

int main() {
  int i{0};
  int& r = i;
  const int& cr = i;
  ++r;
  std::cout << i << '\n';  // 1
//++cr;  // Error: cr ist const
  add(i);
  std::cout << i << '\n';  // 2
  add(r);
  std::cout << i << ' ' << r << ' ' << cr << '\n';  // 3 3 3
  set(i) = 5;
  std::cout << i << '\n';  // 5

//int&& r1 = i;  // Error: can't bind to lvalue
  int&& r2 = i + 0;
  std::cout << i << ' ' << r2 << '\n';  // 5 5
  r2 += 5;
  std::cout << i << ' ' << r2 << '\n';  // 5 10
  i += 5;
  std::cout << i << ' ' << r2 << '\n';  // 10 10

  int& j = bad();
//std::cout << j << '\n';  // Undef. bis Seg. fault
}
\end{lstlisting}

\subsection{Pointer -- Zeiger}

Zeiger enthalten üblicherweise die Anfangsadresse eines Objekts. Bei der
Deklaration wird \lstinline|*| verwendet. Die Adresse eines Objekts erhält man
durch voranstellen von \lstinline|&| vor dem Namen. Zeiger werden mit
\lstinline|*| vor dem Zeigernamen dereferenziert um auf deren Inhalt und nicht
auf ihre Adresse zuzugreifen. Bei Zeiger auf Klassen kan mit \lstinline|->| auf
Klassenelemente zugegriffen werden.

\begin{lstlisting}[language=C++]
#include <iostream>
#include <iterator>
#include <ranges>

float f0(const int& i) {return i + .5;}
float f1(const int& i) {return i - .5;}

int main() {
  int i{0};
  int* p{&i};  // Pointer auf i
  int& r{i};   // Referenz auf i
//int&* pr;  // Error: Pointer auf Referenz
  int& rd{*p};  // Referenz auf dereferenzierten Pointer
  int*& rp{p};  // Referenz auf Pointer
  std::cout << i << ' ' << *p << ' ' << r << ' ' << rd
            << ' ' << *rp << '\n';  // 0 0 0 0 0

  int j{0};
  const int* q{&i};  // Kann i nicht verändern
//++*q;
  q = &j;
  int const* q2{&i};  // Kann i nicht verändern
//++*q2;
  q2 = &j;
  int* const s{&i};  // Kann die Pointeradresse nicht verändern
  ++*s;
//s = &j;
  const int* const t{&i};
//++*t;
//t = &j;
  int const* const t2{&i};
//++*t2;
//t2 = &j;
  int** pp{&p};  // Pointer auf Pointer
  int* const* cp{&p}; // Const P. auf P.
  std::cout << **pp << ' ' << **cp << '\n'; // 1 1

  p = nullptr;  // Empfohlen Null Pointer
  p = NULL;
  p = 0;

  int a[]{0,1};  // Array
  std::cout << a[0] << ' ' << a[1] << '\n';  // 0 1
  int (*ap)[2]{&a};  // Pointer auf Array
  std::cout << (*ap)[0] << ' ' << (*ap)[1] << '\n';  // 0 1
  int (*ap2)[std::size(a)]{&a};
  std::cout << (*ap2)[0] << ' ' << (*ap2)[1] << '\n';  // 0 1
  int (*ap3)[std::ranges::size(a)]{&a};
  std::cout << (*ap3)[0] << ' ' << (*ap3)[1] << '\n';  // 0 1
  int* a0{a};  // int Pointer auf erstes Element
  std::cout << a0[0] << ' ' << a0[1] << '\n';  // 0 1
  std::cout << *a0 << ' ' << *(a0+1) << '\n';  // 0 1

  void* pv{&i};  // Void kann auf alles Zeigen
  std::cout << i << ' ' << *static_cast<int*>(pv) << '\n';  // 1 1

  float (*pf0)(const int&){&f0};
  float (*pf1)(const int&){f1};  // Implizite Umwandlung in &f1
  float (*af[2])(const int&);  // Pointer zu Funktion Array
  af[0] = &f0;
  af[1] = &f1;
  using F = float(const int&);  // named type alias
  F* af2[]{f0, f1};              // vereinfacht Deklaration
  std::cout << f0(1) << ' ' << f1(1) << '\n';  // 1.5 0.5
  std::cout << pf0(1) << ' ' << (*pf1)(1) << '\n';  // 1.5 0.5
  std::cout << af[0](1) << ' ' << af[1](1) << '\n';  // 1.5 0.5
  std::cout << af2[0](1) << ' ' << af2[1](1) << '\n';  // 1.5 0.5

  struct K { int m; };
  int K::* pK{&K::m};          // Pointer auf m der Klasse K
  K k{7};
  std::cout << k.m << ' ' << k.*pK << '\n';   // 7 7
  K* kp{&k};
  kp->m = 10;
  std::cout << k.m << ' ' << kp->m << ' ' << (*kp).m <<
     ' ' << kp->*pK << '\n'; // 10 10 10 10

  struct Base { int m; };
  struct Derived : Base { int n; };
  Derived d;
  d.m = 1;
  int Base::* bpm = &Base::m;
  int Derived::* dpm = bpm;
  std::cout << d.*dpm << ' ' << d.*bpm << '\n';  // 1 1
  Base* bp{&d};  // implicit conversion
  std::cout << bp->m << ' ' << (*bp).m << '\n';  // 1 1
  std::cout << (*bp).*bpm << ' ' << bp->*bpm << '\n';  // 1 1
  d.n = 7;
  int Derived::* dpn = &Derived::n;
  int Base::* bpn = static_cast<int Base::*>(dpn);
  std::cout << d.*bpn << '\n';  // 7
  std::cout << bp->*bpn << '\n';  // 7
  Base B;
//std::cout << B.*bpn << '\n';  // Undefined

  struct E { void f(int n) { std::cout << n << '\n'; } };
  void (E::* pf)(int) = &E::f; // Pointer auf Funktion f in E
  E e;
  e.f(4);  // 4
  (e.*pf)(5);  // 5
  E* ep{&e};
  (ep->*pf)(6);  // 6
}
\end{lstlisting}

\subsection{Arrays}

Arrays (Felder) beinhalten mehrere Elemente des gleichen Typs. Bei bekannter
Größe wird \lstinline|std::array| empfohlen. \lstinline|std::vector| wenn Größe
verändert werden soll. \lstinline|std::valarray| bietet einfache
Rechenoperatoren für alle Elemente.

\begin{lstlisting}[language=C++]
#include <iostream>
#include <span>
#include <array>
#include <vector>
#include <valarray>

int main() {
  int a[2];  // Array mit 2 Elementen uninitialisiert
  std::cout << a[0] << ' ' << a[1] << '\n';  // ? ?
  std::fill_n(a, std::size(a), 7);  // Alle auf 7 setzen
  std::cout << a[0] << ' ' << a[1] << '\n';  // 7 7
  int b[2]{};  // Array mit 2 Elementen default initialisiert
  std::cout << b[0] << ' ' << b[1] << '\n';  // 0 0
  int c[]{3,4}; // 2 Elemente mit 3 4 initialisiert
  std::cout << c[0] << ' ' << c[1] << '\n';  // 3 4
  for (const int& i : c) std::cout << i << '\n';  // 3 4
  int d[2]{3};  // 2 Elemente mit 3 0 initialisiert
  std::cout << d[0] << ' ' << d[1] << '\n';  // 3 0

  int* p{c};  // Pointer auf erstes Element
  std::cout << p[0] << ' ' << p[1] << '\n';  // 3 4
// Error: no viable 'begin'
//for (const int& i : p) std::cout << i << '\n';
  int (*ap)[2]{&c};  // Pointer auf Array
  std::cout << (*ap)[0] << ' ' << (*ap)[1] << '\n';  // 3 4
  for (const int& i : *ap) std::cout << i << '\n';  // 3 4
  int (&r)[] = c;  // Referenz
  std::cout << r[0] << ' ' << r[1] << '\n';  // 3 4
// Error: incomplete type 'int[]'
//for (const int& i : r) std::cout << i << '\n';
  int (&s)[2] = c;  // Referenz
  for (const int& i : s) std::cout << i << '\n';  // 3 4

  int e[2][3]{{1,2,3},{4,5,6}};  // 2*3 Array
  std::cout << e[0][0] << e[0][1] << e[0][2] <<
               e[1][0] << e[1][1] << e[1][2] << '\n';  // 123456
  int* q{*e};  // Pointer Auf erstes Element
  std::cout << q[0] << q[1] << q[2] << q[3] <<
               q[4] << q[5] << '\n';  // 123456

  int* g{new int[2]};//Dynamisch erzeugtes Array uninitialisiert
  delete[] g;  // Speicher freigeben wenn nicht mehr benötigt
  int* h{new int[2]{}};  // Default 0 initialisiert
  std::cout << h[0] << ' ' << h[1] << '\n';  // 0 0
  delete[] h;
  int* j{new int[]{7,8}};  // 7, 8 initialisiert
  std::cout << j[0] << ' ' << j[1] << '\n';  // 7 8
// Error: no viable 'begin'
//for (const int& i : j) std::cout << i << '\n';
  delete[] j;
  int (*k)[2]{reinterpret_cast<int (*)[2]>(new int[2]{})};
  for (const int& i : *k) {std::cout << i << '\n';}  // 0 0
  delete[] k;
  std::span<int> l = std::span<int>(new int[2]{}, 2);
  std::cout << l[0] << ' ' << l[1] << '\n';  // 0 0
  for (const int& i : l) {std::cout << i << '\n';}  // 0 0
  delete[] l.data();

  std::array<int, 2> m{};  // Default 0 initialisiert
  std::cout << m[0] << ' ' << m[1] << '\n';  // 0 0
  for (const int& i : m) {std::cout << i << '\n';}  // 0 0
  std::array<int, 2> m1;  // Uninitialisiert
  m1.fill(8);  // Alle auf 8 setzen
  std::array<int, 2> m2{4,7};  // 4 7 initialisiert

  std::vector<int> v(2);  // default 0 initialisiert
  std::cout << v[0] << ' ' << v[1] << '\n';  // 0 0
  for (const int& i : v) std::cout << i << '\n';  // 0 0
  std::vector<int> v1(2, 5);  // 5 initialisiert

  std::valarray<int> w(2);  // default 0 initialisiert
  std::cout << w[0] << ' ' << w[1] << '\n';  // 0 0
  for (const int& i : w) std::cout << i << '\n';  // 0 0
  std::valarray<int> w1(7, 2);  // 7 initialisiert
}
\end{lstlisting}

Wenn die Größe das Arrays zur Kompilierungszeit nicht bekannt ist, bieten sich
folgende Möglichkeiten an, das Array zu erzeugen. Für uninitialisierte Arrays
zeigt die Variante \lstinline|new| mit \lstinline|span| Vorteile.

\begin{lstlisting}[language=C++]
#include <iostream>
#include <array>
#include <vector>
#include <span>
#include <valarray>
#include <memory>

int main() {
  std::size_t n;
  std::cin >> n;
//int a[n];  // C99 feature: variable length arrays

  int* p{new int[n]};  // uninitialisiert
  std::cout << p[0] << '\n';
// Error: no viable 'begin' function
//for (int i : p) std::cout << i << '\n';
  delete[] p;

  int* p1{new int[n]{}};  // default 0 initialisiert

// C99 feature: variable length arrays
//int (*q)[n]{reinterpret_cast<int (*)[n]>(new int[n])};
//std::cout << (*q)[0] << '\n';
//for (const int& i : *q) {std::cout << i << '\n';}
//delete[] q;

  std::span<int> e = std::span<int>(new int[n], n);
  std::cout << e[0] << '\n';
  for (const int& i : e) {std::cout << i << '\n';}
  delete[] e.data();

  std::unique_ptr<int[]> f = std::make_unique_for_overwrite<int[]>(n);
  std::span<int> g = std::span<int>(f.get(), n);
  for (const int& i : g) {std::cout << i << '\n';}

// Error: argument is not a constant expression
//std::array<int, n> b;

  std::vector<int> v(n);  // default 0 initialisiert
  std::cout << v[0] << '\n';
  for (const int& i : v) std::cout << i << '\n';
  std::vector<int> v1(n, 5);  // 5 initialisiert

  std::valarray<int> w(n);  // default 0 initialisiert
  std::cout << w[0] << '\n';
  for (const int& i : w) std::cout << i << '\n';
  std::valarray<int> w1(7, n);  // 7 initialisiert
}
\end{lstlisting}

\subsection{Structured bindings}

Zugriff auf Strukturelemente mittels selbst vergebener Namen.

\begin{lstlisting}[language=C++]
#include <iostream>

int main() {
  int a[2]{1, 2};

  auto [x, y] = a;  // Kopie
  ++x; ++y;
  std::cout << a[0] << a[1] << ' ' << x << y << '\n';  // 12 23

  auto& [xr, yr] = a;  // Referenz
  ++a[0]; ++yr;
  std::cout << a[0] << a[1] << ' ' << xr << yr << '\n'; //23 23

  const auto& [cxr, cyr] = a;  // Const Referenz
  ++a[0];
//++cyr;  // Error: const
  std::cout << a[0] << a[1] << ' ' << cxr << cyr << '\n';//33 33
}
\end{lstlisting}

\subsection{Aufzählung -- Enumerations und enumerators}

\begin{lstlisting}[language=C++]
#include <iostream>

enum Color1 { red, blue };
//enum Color2 { red };  // Error: Redefiniotion von red

// Typsicher, keine implizite Umwandlung möglich
enum class Color3 { red, blue };

enum Foo { a, b, c = 10, d, e = 1, f, g = f + c };
//a = 0, b = 1, c = 10, d = 11, e = 1, f = 2, g = 12

// Mit Typangabe
enum class byte : unsigned char { a, b, c };

int main() {
  Color1 x = red;
  std::cout << x << '\n';  // 0
//Color3 y = red;  // Error: red ist type Color1
  Color3 z = Color3::red;
//std::cout << z << '\n';  // Error: Keine Methode
  std::cout << static_cast<int>(z) << '\n';  // 0
}
\end{lstlisting}

\subsection{inline}

Hat die Bedeutung: Mehrfachdefinitionen sind zulässig. Wenn Funktionen im header
definiert sind. Für Variable mit externer Verknüpfung.

\begin{lstlisting}[language=C++]
// example.h
#ifndef EXAMPLE_H
#define EXAMPLE_H

inline int sum(int a, int b) {
  return a + b;
}

inline std::atomic<int> counter(0);

#endif
\end{lstlisting}

\begin{lstlisting}[language=C++]
// Source 1
#include "example.h"

int a() {
  ++counter;
  return sum(1, 2);
}
\end{lstlisting}

\begin{lstlisting}[language=C++]
// Source 2
#include "example.h"

int b() {
  ++counter;
  return sum(3, 4);
}
\end{lstlisting}

\subsection{Inline assembly}

Mit \lstinline|asm| kann Assembler--Quellcode in einem \cpp{}--Programm
geschrieben werden.

\subsection{const, mutable und volatile}

Siehe \nameref{sec:Conversion:constCast} und \nameref{sec:Ausdruck:constant}.

\begin{description}
  \item[const] Wert kann nicht geändert werden.
  \item[mutable] Erlaubt die Veränderung eines Klassenmitglieds in einem const Objekt.
  \item[volatile] Kann von aussen abgefragt und verändert werden. Schränkt Optimierung ein.
\end{description}

\begin{lstlisting}[language=C++]
int main() {
  int n1 = 0;          // non-const object
  const int n2 = 0;    // const object
  int const n3 = 0;    // const object wie n2
  volatile int n4 = 0; // volatile object

  const struct {
    int n1;
    mutable int n2;
  } x = {0, 0};        // const object mit mutable

  n1 = 1;   // OK
//n2 = 2;   // error: const
  n4 = 3; n4 = 4;  // Wird nicht optimiert als nur n4=4;
//x.n1 = 4; // error: member of a const object is const
  x.n2 = 4; // OK: mutable member of a const object isn't const
}
\end{lstlisting}

\subsection{constexpr}

Siehe \nameref{sec:Ausdruck:constant}. Wert liegt bei Kompilierzeit vor bzw.\
Funktionsaufruf \emph{kann}, muss aber nicht, eine Konstante zur
Kompilierungszeit erzeugen.

\subsection{consteval}

Siehe \nameref{sec:Ausdruck:constant}. Funktionsaufruf \emph{muss} eine Konstante zur Kompilierungszeit erzeugen.

\subsection{constinit}

Variable wird beim compilieren initialisiert, und nicht erst bei Ausführen wenn
es über die Deklaration läuft.

\begin{lstlisting}[language=C++]
constinit int i{0};
constinit const int j{0};

int main() {
  ++i;
//++j;  // Error: Const
}
\end{lstlisting}

\subsection{decltype}

Ermittelt den Typ einer Deklaration.

\begin{lstlisting}[language=C++]
#include <iostream>

int main() {
  int i{7};
  decltype(i) j{i + 0};   // int : Kopie von i + 0
//decltype((i)) k{i + 0}; // Error: braucht lvalue
  decltype((i)) k{i};     // int& : Referenz auf i;
  ++i;
  std::cout << i << ' ' << j << ' ' << k <<'\n';  // 8 7 8
  ++j;
  std::cout << i << ' ' << j << ' ' << k <<'\n';  // 8 8 8
  ++k;
  std::cout << i << ' ' << j << ' ' << k <<'\n';  // 9 8 9
}
\end{lstlisting}

\subsection{auto}

Typ (int, double, \dots) wird automatisch abgeleitet.

\begin{lstlisting}[language=C++]
#include <iostream>
#include <typeinfo>

int main() {
  auto i{7};  // int
  std::cout << typeid(i).name() << '\n';  // int

  auto c0{i};  // int, Kopie von i
  auto& c1{i};  // int&, Referenz auf i

  decltype(auto) c2{i};   // int, Kopie von i
  decltype(auto) c3{(i)};  // int&, Referenz auf i

//auto a[]{1, 2};  // Error: Geht nur mit einem Element
  int a[]{1, 2};
  auto [x, y]{a};  // Kopie: x=a[0]; y=a[1];
  auto& [xr, yr]{a};  // Referenz
}
\end{lstlisting}

\subsection{typedef}

Alias für einen (komplexen) Typnamen.

\begin{lstlisting}[language=C++]
typedef unsigned long ul;
unsigned long l1; // unsigned long
ul l2;            // unsigned long
\end{lstlisting}

\subsection{Type alias}

Alias für einen bereits definierten Typ.

\begin{lstlisting}[language=C++]
template<class T>  // alias template
using ptr = T*;  // 'ptr<T>' als alias für T*
ptr<int> x;  // int* x

int main() {
  float (*af[2])(const int&);  // Pointer zu Funktion Array
  using F = float(const int&);  // named type alias
  F* af2[2];  // Das gleiche wie af
}
\end{lstlisting}

\subsection{Elaborated type specifiers}

Werden verwendet, um auf einen zuvor deklarierten Klassennamen zu verweisen,
selbst wenn diser verdeckt wurde, oder um einen neunen Klassennamen zu
deklarieren.

\begin{lstlisting}[language=C++]
struct X { int y; };

int main() {
  int X;  // verdeckt Klasse X
//X z;    // Error: Verwendet Variable X
  class X z;
}
\end{lstlisting}

\subsection{Attributes}

Geben Hinweise.

\begin{lstlisting}[language=C++]
#include <iostream>

// Kehrt nicht zu aufrufender Funktion zurück
[[noreturn]] void f0() { throw "error"; }

[[deprecated]] int f1() {return 1;}
[[deprecated("Use f0 instead")]] int f2() {return 2;}

[[nodiscard]] int f3() {return 1;}

int f4(int n) {
  int i;
  switch (n) {
    case 1: [[fallthrough]];  // Keine Warnung
    case 2: i=1; break;
    default: i=0;
  }
  return i;
}

int f5(int i) {
  if (i < 1) [[likely]]  // Hilft beim Optimieren
    return 0;
  else [[unlikely]]
    return 1;
}

int f6(int x, int y) {
    [[assume(y == 1)]];  // Wenn y!=1 Ergebnis undefiniert
    return x / y;
}

int main() {
  f1();  // Warnung depreciated beim compilieren
  f2();  // Warnung depreciated beim compilieren

  f3();  // Warnung das Rückgabewert nicht verwendet
  int i;
  i = f3();

  // Verhindert Warnung wenn nicht verwendet
  [[maybe_unused]] int j{0};

  std::cout << f6(4, 2) << '\n';
}
\end{lstlisting}

\subsection{alignas}

Gibt die Bytegrenzen eines Types an.

\begin{lstlisting}[language=C++]
#include <iostream>

struct alignas(32) Bar {  // Angereiht alle 32 byte
  int i;       // 4 bytes
  alignas(16) char arr[5];  // Angereiht alle 16 byte
  short s;     // 2 bytes
};

int main() {
  Bar x;
  std::cout << alignof(Bar) << std::endl;    // 32
  std::cout << sizeof(Bar) << std::endl;     // 32
  std::cout << x.arr - (char*)&x.i << '\n';  // 16
  std::cout << (char*)&x.s - x.arr << '\n';  // 6
}
\end{lstlisting}

\subsection{static\_assert}

Überprüfung von Annahmen wärend der Compilierung.

\begin{lstlisting}[language=C++]
int main() {
  const int x{3};
//static_assert(x / 2 > 1); // error: static assertion failed
  static_assert(x / 2 > 0, "x/2 > 0?");
  static_assert(sizeof(int) == 4, "Int ist nicht 4 Byte");
}
\end{lstlisting}
